


\chapter{\ProgramName 程序理论推导及开发}

\ProgramName (\ProgramFullName)是本文开发的基于GPU加速的、三维2群扩散细网有限差分、稳态/时空动力学求解程序。


\section{中子扩散时空动力学理论推导}

\subsection{稳态公式推导}

多群扩散时空动力学方程为
\begin{align}
  \newcommand{\para}{\pb{\bm{x},t}}
  \left\{
  \begin{aligned}
    \frac{1}{v_g}\frac{\partial \phi_g\para}{\partial t}
    &=\nabla\cdot D_g\para \nabla\phi_g\para 
      -\Sigma_{t,g}\para \phi_g\para
      +\sum_{i=1}^I \chi_{i,g}\para \lambda_i C_i\para \\
          & \hspace{1cm}
      +\sum_{g'=1}^G\pb[B]{\chi_g\para \pb{1-\beta}\nu\Sigma_{f,g'}\para
                            +\Sigma_{g'\rightarrow g}\para}\phi_{g'}\para \\
    \frac{\partial C_i\para}{\partial t}
     &=\beta_i \sum_{g'=1}^G \nu\Sigma_{f,g'}\para \phi_{g'}\para
        -\lambda_i C_i\para \qquad i=1,2,\cdots,I
  \end{aligned}
  \right.
  \titlelabel{equ:pro.diff.equ}{多群扩散时空动力学方程}
\end{align}

如果初始条件为稳态则有
\begin{align}
  \newcommand{\para}{\pb{\bm{x},t}}
  \left\{
  \begin{aligned}
    \frac{\partial \phi_g\para}{\partial t}\Big|_{t=0} &=0 \\
    \frac{\partial C_i\para}{\partial t}\Big|_{t=0} &=0
  \end{aligned}
  \right.
  \label{equ:pro.diff.init.equ}
\end{align}

联立\aeqref{equ:pro.diff.equ}及\aeqref{equ:pro.diff.init.equ},
消去$C_i\pb{\bm{x},0}$可得
\begin{align}
  \newcommand{\para}{\pb{\bm{x},0}}
  \begin{aligned}
  &\nabla\cdot D_g\para \nabla\phi_g\para 
   -\Sigma_{t,g}\para \phi_g\para \\
  & \hspace{3cm}
   +\sum_{g'=1}^G\pb[B]{\chi_g\para \nu\Sigma_{f,g'}\para
                        +\Sigma_{g'\rightarrow g}\para}\phi_{g'}\para =0
  \end{aligned}
\end{align}

此为$k_\mathrm{eff}=1$时的稳态扩散方程,
由于问题的已知条件一般仅有$k_\mathrm{eff}\approx 1$,
所以先求解普通的临界扩散方程。
\begin{align}
  \newcommand{\para}{\pb{\bm{x},0}}
  \begin{aligned}
  &\nabla\cdot D_g\para \nabla\phi_g\para 
   -\Sigma_{t,g}\para \phi_g\para \\
  & \hspace{3cm}
   +\sum_{g'=1}^G\pb[B]{\frac{1}{k_\mathrm{eff}}\chi_g\para \nu\Sigma_{f,g'}\para
                        +\Sigma_{g'\rightarrow g}\para}\phi_{g'}\para =0
  \end{aligned}
  \titlelabel{equ:pro.diff.init.diff.equ1}{扩散时空动力学问题的初始通量方程}
\end{align}
然后对裂变截面进行修正
\begin{align}
  \newcommand{\para}{\pb{\bm{x},t}}
  \nu\Sigma'_{f,g'}\para = \frac{1}{k_\mathrm{eff}}\nu\Sigma_{f,g'}\para
\end{align}
修正后的问题的$k_\mathrm{eff}=1$,可以适用\aeqref{equ:pro.diff.init.equ},
求得到初始条件的$C_i\pb{\bm{x},0}$
\begin{align}
  \newcommand{\para}{\pb{\bm{x},0}}
  C_i\para = \frac{\beta_i}{\lambda_i}\sum_{g'=1}^G \nu\Sigma_{f,g'}\para\phi_{g'}\para
  \label{equ:pro.diff.init.c}
\end{align}

边界条件为反射边界条件时
\begin{align}
  \bm{n}\cdot\nabla\phi_g\pb{\bm{x},t} = 0
  \qquad \bm{x} \in \partial \mathcal{D}
  \titlelabel{equ:pro.diff.boundary.equ}{扩散方程反射边界条件}
\end{align}
其中$\partial \mathcal{D}$是待求解问题区域的边界面,
$\bm{n}$是边界面上的点$\bm{x}$在边界面上的法向量,
方向指向区域外。

边界条件为0边界条件时
\begin{align}
  \phi_g\pb{\bm{x},t} = 0
  \qquad \bm{x} \in \partial \mathcal{D}
\end{align}

\subsubsection{空间离散}

程序使用三维直角坐标网格,空间划分为$K_x\times K_y \times K_z$个结构网格,
则网格集合为
\begin{align}
  \mathcal{D}_{\bm{k}}=\big\{(k_x,k_y,k_z)\big|k_w = 0,1,\cdots,K_w-1 ; w=x,y,z\big\}
\end{align}

在$xyz$坐标系中有一般离散关系
\begin{align}
  \phi(\bm{x}) &\rightarrow \phi_{\bm{k}} &
  \Sigma(\bm{x}) &\rightarrow \Sigma_{\bm{k}} &
  C_i(\bm{x}) &\rightarrow C_{i,\bm{k}}
\end{align}

其中$\bm{k}=(k_x,k_y,k_z)$为$xyz$空间离散后的网格坐标,
为方便起见这里暂时省略能群$g$,时间步长$n$等下标上标,下同。

离散的主要问题是微分项$\nabla\cdot D(\bm{x})\nabla\phi(\bm{x})$的离散方式,
设$\nabla\cdot D(\bm{x})\nabla\phi(\bm{x})$对应的离散项记为
$\pb[b]{\nabla\cdot D\nabla\phi}_{\bm{k}}$。
本文的网格通量值取在网格中心位置,所以一个网格内$D$相同,
相邻的网格如$D_{\bm{k}}$与$D_{\bm{k}\pm\hat{\bm{x}}}$、
$D_{\bm{k}\pm\hat{\bm{y}}}$、$D_{\bm{k}\pm\hat{\bm{z}}}$
\footnote{记$\hat{\bm{x}}=(1,0,0)^T$,$\hat{\bm{y}}=(0,1,0)^T$,$\hat{\bm{z}}=(0,0,1)^T$。}
(这三项以下简记为$D_{\bm{k}\pm\hat{\bm{w}}}$其中$w=x,y,z$)
未必相同,相邻网格交界面上的$D$取为
\begin{align}
D_{\bm{k}\pm\frac{1}{2}\hat{\bm{w}}}
  =\frac{2D_{\bm{k}}D_{\bm{k}\pm\hat{\bm{w}}}}{D_{\bm{k}}+D_{\bm{k}\pm\hat{\bm{w}}}}
\end{align}
则
\begin{align*}
\pb[b]{\nabla\cdot D\nabla\phi}_{\bm{k}}
 &= \pb[bg]{ D_{\bm{k}\pm\frac{1}{2}\hat{\bm{x}}}\frac{\phi_{\bm{k}+\hat{\bm{x}}} - \phi_{\bm{k}}}{\Delta x^2}
    -D_{\bm{k}\pm\frac{1}{2}\hat{\bm{x}}}\frac{\phi_{\bm{k}+\hat{\bm{x}}} - \phi_{\bm{k}}}{\Delta x^2}
    }\\
 &\hspace{1cm}+
    \pb[bg]{ D_{\bm{k}\pm\frac{1}{2}\hat{\bm{y}}}\frac{\phi_{\bm{k}+\hat{\bm{y}}} - \phi_{\bm{k}}}{\Delta y^2}
       -D_{\bm{k}\pm\frac{1}{2}\hat{\bm{y}}}\frac{\phi_{\bm{k}+\hat{\bm{y}}} - \phi_{\bm{k}}}{\Delta y^2}
       }\\
 &\hspace{1cm}+
    \pb[bg]{ D_{\bm{k}\pm\frac{1}{2}\hat{\bm{z}}}\frac{\phi_{\bm{k}+\hat{\bm{z}}} - \phi_{\bm{k}}}{\Delta z^2}
       -D_{\bm{k}\pm\frac{1}{2}\hat{\bm{z}}}\frac{\phi_{\bm{k}+\hat{\bm{z}}} - \phi_{\bm{k}}}{\Delta z^2}
       }\\
 &= \sum_{w=x,y,z} \pb[bg]{ D_{\bm{k}\pm\frac{1}{2}\hat{\bm{w}}}\frac{\phi_{\bm{k}+\hat{\bm{w}}} - \phi_{\bm{k}}}{\Delta w^2}
   -D_{\bm{k}\pm\frac{1}{2}\hat{\bm{w}}}\frac{\phi_{\bm{k}+\hat{\bm{w}}} - \phi_{\bm{k}}}{\Delta w^2}
   }
\end{align*}
即
\begin{align}
  \pb[b]{\nabla\cdot D\nabla\phi}_{\bm{k}}
    &=\sum_{w=x,y,z} \Sb[bg]{
           \frac{2D_{\bm{k}}D_{\bm{k}+\hat{\bm{w}}}\pb{\phi_{\bm{k}+\hat{\bm{w}}} - \phi_{\bm{k}}}}
                {\Delta w^2\pb{D_{\bm{k}}+D_{\bm{k}+\hat{\bm{w}}}}}
          -\frac{2D_{\bm{k}}D_{\bm{k}-\hat{\bm{w}}}\pb{\phi_{\bm{k}} - \phi_{\bm{k}-\hat{\bm{w}}}}}
                {\Delta w^2\pb{D_{\bm{k}}+D_{\bm{k}-\hat{\bm{w}}}}}
          }
  \label{equ:dnabla2.equ0}
\end{align}
其中$\Delta x,\Delta y,\Delta z$是$x,y,z$方向上的网格尺寸。

则\aeqref{equ:pro.diff.init.diff.equ1}的离散形式为
\begin{align}
  \pb[b]{\nabla\cdot D_g^{(0)} \nabla\phi_g^{(0)}}_{\bm{k}}
   -\Sigma_{t,g,\bm{k}}^{(0)} \phi_{g,\bm{k}}^{(0)}
   +\sum_{g'=1}^G\pb[B]{\frac{1}{k_\mathrm{eff}^{(0)}}\chi_{g,\bm{k}}^{(0)} \nu\Sigma_{f,g',\bm{k}}^{(0)}
                        +\Sigma_{g'\rightarrow g,\bm{k}}^{(0)}}\phi_{g',\bm{k}}^{(0)} =0
  \quad \bm{k} \in \mathcal{D}_{\bm{k}}
\end{align}

\aeqref{equ:pro.diff.init.c}的离散形式为
\begin{align}
  C_{i,\bm{k}}^{(0)} = \frac{\beta_i}{\lambda_i}
    \sum_{g'=1}^G \nu\Sigma_{f,g',\bm{k}}^{(0)}\phi_{g',\bm{k}}^{(0)}
  \qquad \bm{k} \in \mathcal{D}_{\bm{k}}
\end{align}

\subsection{动力学公式推导}

将\aeqref{equ:pro.diff.equ}对时间$t$采用全隐式向后差分格式进行离散得
\begin{align}
  \newcommand{\para}[1][n]{\pb{\bm{x}}^{(#1)}}
  \left\{
  \begin{aligned}
    \frac{1}{v_g}\frac{\phi_g\para - \phi_g\para[n-1]}{\Delta t}
    &=\nabla\cdot D_g\para \nabla\phi_g\para 
      -\Sigma_{t,g}\para \phi_g\para \\
    & \hspace{1cm}
      +\sum_{g'=1}^G\pb[B]{\chi_g\para \pb{1-\beta}\nu\Sigma_{f,g'}\para
                           +\Sigma_{g'\rightarrow g}\para}\phi_{g'}\para \\
    &\hspace{1cm}
      +\sum_{i=1}^I \chi_{i,g}\para \lambda_i C_i\para \\
    \frac{C_i\para - C_i\para[n-1]}{\Delta t}
     &=\beta_i \sum_{g'=1}^G \nu\Sigma_{f,g'}\para \phi_{g'}\para
        -\lambda_i C_i\para
  \end{aligned}
  \right.
  \label{equ:pro.diff.dt.equ0}
\end{align}

解出$C_i\pb{\bm{x}}^{(n)}$可得
\begin{align}
  \newcommand{\para}[1][n]{\pb{\bm{x}}^{(#1)}}
  C_i\para = \frac{1}{1+\lambda_i\Delta t}
    \pb[B]{C_i\para[n-1]
    + \beta_i \Delta t \sum_{g'=1}^G \nu\Sigma_{f,g'}\para \phi_{g'}\para}
  \label{equ:pro.diff.dt.c}
\end{align}

代回\aeqref{equ:pro.diff.dt.equ0}得
\begin{align}
  \newcommand{\para}[1][n]{\pb{\bm{x}}^{(#1)}}
  \begin{aligned}
    &\quad \frac{1}{v_g}\frac{\phi_g\para - \phi_g\para[n-1]}{\Delta t} \\
    &=\nabla\cdot D_g\para \nabla\phi_g\para 
      -\Sigma_{t,g}\para \phi_g\para \\
    & \hspace{1cm}
      +\sum_{g'=1}^G\pb[B]{\chi_g\para \pb{1-\beta}\nu\Sigma_{f,g'}\para
                           +\Sigma_{g'\rightarrow g}\para}\phi_{g'}\para \\
    &\hspace{1cm}
      +\sum_{i=1}^I \frac{\chi_{i,g}\para \lambda_i}{1+\lambda_i\Delta t}
          \pb[B]{C_i\para[n-1] 
      + \beta_i \Delta t \sum_{g'=1}^G \nu\Sigma_{f,g'}\para \phi_{g'}\para}
  \end{aligned}
\end{align}

取$\chi_{i,g}=\chi_g$得
\begin{align}
  \newcommand{\para}[1][n]{\pb{\bm{x}}^{(#1)}}
  \begin{aligned}
    &\quad \frac{1}{v_g}\frac{\phi_g\para - \phi_g\para[n-1]}{\Delta t} \\
    &=\nabla\cdot D_g\para \nabla\phi_g\para 
      -\Sigma_{t,g}\para \phi_g\para 
      +\sum_{i=1}^I \frac{\chi_g\para \lambda_i}{1+\lambda_i\Delta t} C_i\para[n-1]\\
    & \hspace{1cm}
      +\sum_{g'=1}^G\pb[Bg]{\chi_g\para
        \pb[bg]{1-\beta 
          + \sum_{i=1}^I \frac{\lambda_i \beta_i \Delta t }{1+\lambda_i\Delta t}}
      \nu\Sigma_{f,g'}\para \\
    &\hspace{8cm}
         +\Sigma_{g'\rightarrow g}\para}\phi_{g'}\para
  \end{aligned}
\end{align}

记
\begin{align}
  \newcommand{\para}[1][n]{(\bm{x})^{(#1)}}
  S_C\para = \sum_{i=1}^I \frac{\lambda_i}{1+\lambda_i\Delta t} C_i\para[n-1]
  \titlelabel{equ:pro.diff.dt.sc}{离散扩散时空动力学中$S_C(\bm{x})^{(n)}$定义式} \\
  %
  B = 1-\beta + \sum_{i=1}^I \frac{\lambda_i \beta_i \Delta t }{1+\lambda_i\Delta t}
  \titlelabel{equ:pro.diff.dt.B}{离散扩散时空动力学中$B(\bm{x})^{(n)}$定义式}
\end{align}

则有
\begin{align}
  \newcommand{\para}[1][n]{\pb{\bm{x}}^{(#1)}}
  \begin{aligned}
    &\quad \frac{1}{v_g}\frac{\phi_g\para - \phi_g\para[n-1]}{\Delta t} \\
    &=\nabla\cdot D_g\para \nabla\phi_g\para 
      -\Sigma_{t,g}\para \phi_g\para + \chi_g\para S_C\para\\
    & \hspace{1cm}
      +\sum_{g'=1}^G\pb[B]{\chi_g\para
        B \nu\Sigma_{f,g'}\para
         +\Sigma_{g'\rightarrow g}\para}\phi_{g'}\para
  \end{aligned}
  \titlelabel{equ:pro.diff.dt.equ1}{时间$t$隐式向后差分离散后的扩散时空动力学通量$\phi$方程}
\end{align}
此为固定源扩散问题,可通过解线性方程进行求解。

\subsubsection{空间离散}
同临界问题部分,\aeqref{equ:pro.diff.dt.equ1}的离散形式为
\begin{align}
  \begin{aligned}
    \frac{1}{v_g}\frac{\phi_{g,\bm{k}}^{(n)} - \phi_{g,\bm{k}}^{(n-1)}}{\Delta t} 
    &=\pb[b]{\nabla\cdot D_{g}^{(n)} \nabla\phi_{g}^{(n)}}_{\bm{k}}
      -\Sigma_{t,g,\bm{k}}^{(n)} \phi_{g,\bm{k}}^{(n)} + \chi_{g,\bm{k}}^{(n)} S_{C,\bm{k}}^{(n)}\\
    & \hspace{1cm}
      +\sum_{g'=1}^G\pb[B]{\chi_{g,\bm{k}}^{(n)}
        B \nu\Sigma_{f,g',\bm{k}}^{(n)}
         +\Sigma_{g'\rightarrow g,\bm{k}}^{(n)}}\phi_{g',\bm{k}}^{(n)}
  \end{aligned}
  \qquad \bm{k} \in \mathcal{D}_{\bm{k}}
  \label{equ:pro.diff.dt.dx.equ1}
\end{align}

其中
\begin{align}
  S_{C,\bm{k}}^{(n)} &= \sum_{i=1}^I \frac{\lambda_i}{1+\lambda_i\Delta t} C_{i,\bm{k}}^{(n-1)}
  \qquad \bm{k} \in \mathcal{D}_{\bm{k}}
  \titlelabel{equ:pro.diff.dt.dx.sc}{离散扩散时空动力学中$S_{C,\bm{k}}^{(n)}$定义式}
\end{align}

\aeqref{equ:pro.diff.dt.c}的离散形式为
\begin{align}
  C_{i,\bm{k}}^{(n)} = \frac{1}{1+\lambda_i\Delta t}
    \pb[B]{C_{i,\bm{k}}^{(n-1)}
    + \beta_i \Delta t \sum_{g'=1}^G \nu\Sigma_{f,g',\bm{k}}^{(n)} \phi_{g',\bm{k}}^{(n)}}
  \qquad \bm{k} \in \mathcal{D}_{\bm{k}}
\end{align}



\subsection{能群耦合}

本文不考虑向上散射,向下散射只散射到邻近的能群,
则以上离散后的固定源方程可以写成如下形式
\begin{align}
  \begin{pmatrix}
  A_{11} & D_{12} & \cdots & D_{1G}\\
  D_{21} & A_{22} & &\\
   & \ddots & \ddots &\\
   & & D_{G-1,G} & A_{GG}
  \end{pmatrix}
  \begin{pmatrix}
  \phi_1 \\ \phi_2 \\ \vdots \\ \phi_G
  \end{pmatrix}
  =
  \begin{pmatrix}
  S_1 \\ S_2 \\ \vdots \\ S_G
  \end{pmatrix}
\end{align}
其中 $A_{gg}$是7对角对称阵,$D_{g_1g_2}$ 是对角阵。

两群情况则简化为
\begin{align}
  \begin{pmatrix}
  A_{11} & D_{12} \\
  D_{21} & A_{22}
  \end{pmatrix}
  \begin{pmatrix}
  \phi_1 \\ \phi_2
  \end{pmatrix}
  =
  \begin{pmatrix}
  S_1 \\ S_2
  \end{pmatrix}
  \label{equ:program.group.g2equ.fixs}
\end{align}

对于\aeqref{equ:program.group.g2equ.fixs},
可直接使用支持非对称矩阵的迭代算法进行求解,
如Jacobi迭代、BiCGStab、GMRES等。

不过在实际计算中,如果能够更充分的利用$A_{ii}$、
$D_{ij}$的实对称矩阵性质往往可以加速计算过程。
这样就有逐群迭代求解过程
\begin{align}
  \left\{
  \begin{aligned}
    \phi_1^{(k+1)}&=A_{11}^{-1}\pb[B]{S_1-D_{12}\phi_2^{(k)}}\\
    \phi_2^{(k+1)}&=A_{22}^{-1}\pb[B]{S_2-D_{21}\phi_1^{(k+1)}}
  \end{aligned}
  \right.
\end{align}

对于临界问题,则可写成如下形式的广义特征值问题
\begin{align}
  \begin{pmatrix}
  A_{11} &  \\
   & A_{22}
  \end{pmatrix}
  \begin{pmatrix}
  \phi_1 \\ \phi_2
  \end{pmatrix}
  =\frac{1}{k_\mathrm{eff}}
    \begin{pmatrix}
    F_{11} & F_{12} \\
    S_{21} & 0
    \end{pmatrix}
  \begin{pmatrix}
    \phi_1 \\ \phi_2
  \end{pmatrix}
\end{align}
其中 $A_{gg}$是7对角对称阵,$F_{g_1g_2}$、$S_{g_1g_2}$ 是对角阵。

如果使用源迭代进行求解,则每轮外迭代的分群迭代形式为
\begin{align}
  \left\{
  \begin{aligned}
    \mathcal{S}^{(k+1)}&=\frac{1}{k_\mathrm{eff}^{(k)}}\pb[B]{F_{11}\phi_1^{(k)}+F_{12}\phi_2^{(k)}}\\
    \phi_1^{(k+1)}&=A_{11}^{-1}\mathcal{S}^{(k+1)}\\
    \phi_2^{(k+1)}&=A_{22}^{-1}S_{21}\phi_1^{(k+1)}\\
    k_\mathrm{eff}^{(k+1)}&=k_\mathrm{eff}^{(k)}\frac{\left\|\mathcal{S}^{(k+1)}\right\|}{\left\|\mathcal{S}^{(k)}\right\|}
  \end{aligned}
  \right.
  \label{equ:program.group.g2equ.keff}
\end{align}

\aeqref{equ:program.group.g2equ.fixs}和\aeqref{equ:program.group.g2equ.keff}
中的$A_{ii}^{-1}$项可以使用支持实对称正定的迭代算法进行求解,
如Jacobi、CG等。


\section{\ProgramName 程序开发}

\subsection{整体介绍}
\ProgramName 程序整体结构见\floatref{fig:program.structure.whole},
其中内核部分是程序的主要部分,结构见\floatref{fig:program.structure.core}。

\begin{figure}[h]
\centering
\begin{tikzpicture}[scale=0.8, transform shape]

\node at (1.5,2) {主体控制逻辑};
\draw  (-4,2.5) rectangle (7,1.5);

\draw  (-1,0.5) rectangle (1,-3.5);
\draw  (2,0.5) rectangle (7,-3.5);
\draw  (2.5,0) rectangle (6.5,-2.5);
\node [right] at (2.5,-3) {输入模块(Lua解释器)};
\node [right] at (3,-0.5) {Lua输入文件};
\node [right] at (3.5,-1) {网格定义};
\node [right] at (3.5,-1.5) {材料定义};
\node [right] at (3.5,-2) {模式及参数};
\draw [latex new-latex new, arrow head=3mm]  (4,1.5)  -- (4,0);

\node at (0,-1.5) {内核};
\draw [latex new-latex new, arrow head=3mm]  (0,1.5)  -- (0,0.5);
\draw [latex new-latex new, arrow head=3mm]  (1,-1.5) -- (2.5,-1.5);

\draw  (-4,0.5) rectangle (-2,-3.5);
\node at (-3,-1.5) {输出};
\draw  (-9,0) rectangle (-5,-1);
\draw  (-5,-2) rectangle (-9,-3);
\node at (-7,-0.5) {输出文件(HDF5)};
\node at (-7,-2.5) {屏幕};
\draw [-latex new, arrow head=3mm] (-3,1.5) -- (-3,0.5);

\draw [latex new-, arrow head=3mm] (-5,-0.5) -- (-4,-0.5) ;
\draw [latex new-, arrow head=3mm] (-5,-2.5) -- (-4,-2.5) ;
\draw [latex new-, arrow head=3mm] (-2,-1.5) -- (-1,-1.5) ;

\end{tikzpicture}
\caption{\label{fig:program.structure.whole}\ProgramName 程序整体结构}
\end{figure}


\begin{figure}
\centering
\begin{tikzpicture}[scale=0.8, transform shape]
\draw  (-2.5,-4) rectangle (12,-5);
\node at (5,-4.5) { GPU};

\draw  (-2.5,-2.5) rectangle (12,-3.5);
\node at (5,-3) {显卡驱动};

\draw  (-2.5,-1) rectangle (12,-2);
\node at (5,-1.5) {CUDA};

\draw  (-2.5,2) rectangle (8,-0.5);
\node [right] at (-2,1.5) {Thrust};
\draw  (-2,1) rectangle (1,0);
\draw  (1.5,1) rectangle (4.5,0);
\draw  (5,1) rectangle (7.5,0);
\node [right] at (-1.5,0.5) {辅助函数};
\node at (3,0.75) { \small Global函数};
\node at (3,0.25) { \small 启动参数配置};
\node at (6.25,0.5) {显存管理};

\draw  (-2.5,6.5) rectangle (5,2.5);
\node [right] at (-2,6) {CUSP};
\draw (-2,4) rectangle (4.5,3);
\node at (1,3.5) {稀疏矩阵格式: DIA, ELL等};
\draw (-2,5.5) rectangle (4,4.5);
\node at (1,5) {迭代算法:CG, BiCGStab等};

\node at (10.25,1.25) {改进的};
\node at (10.25,0.75) {显存管理};
\draw [-latex new, arrow head=3mm] (7.5,0.5) -- (9,0.5);

\draw [very thick] (12,-0.5) -- (8.5,-0.5) 
     -- (8.5,2.5) -- (5.5,2.5) -- (5.5,7) 
     -- (-2.5,7) -- (-2.5,12.5) -- (12,12.5) 
     -- (12,-0.5);
\node [right] at (-2,12) {\ProgramName 程序内核};
\draw  (6,6.5) rectangle (11.5,4.5);
\node at (8.75,5.5) {能群间耦合处理};

\draw  (9,2) rectangle (11.5,0);
\draw  (6,4) rectangle (11.5,3);
\node at (8.75,3.5) {GPU端在线矩阵生成};

\draw  (-2,8.5) rectangle (11.5,7.5);
\node at (5,8) {本征值求解(源迭代)、固定源求解};
\draw  (-2,10) rectangle (11.5,9);
\node at (5,9.5) {双层网格加速};

\draw  (-2,11.5) rectangle (4.5,10.5);
\draw  (5,11.5) rectangle (11.5,10.5);
\node at (1,11) {临界计算};
\node at (8,11) {时空动力学计算};

\end{tikzpicture}
\caption{\label{fig:program.structure.core}\ProgramName 程序内核结构}
\end{figure}

GPU部分使用CUDA进行开发,版本为CUDA 5.0,
并使用了CUDA自带的Thrust通用C++模板库。
在线性代数库方面,现在业界尚没有较为完整、稳定、高效的稀疏矩阵库,
\ProgramName 程序使用了NVIDIA公司开发的开源稀疏矩阵库CUSP,
因为CUSP支持本文需要的DIA稀疏矩阵存储格式,并且实现较为高效。
CUSP在实现上也是一个C++模板库,大量依赖Thrust的实现,
并且使用了Thrust的显存管理模块,出于性能考虑本文替换了Thrust的显存管理模块。
%由于现在Thrust和CUSP并不支持多GPU并行,所以本程序也仅限使用单GPU进行加速。

在稀疏矩阵格式方面主要使用DIA格式(见\sectionref{sec:gpu.sparseformat.dia}),
并针对扩散方程离散的特点实现了扩散方程矩阵GPU端动态DIA格式生成功能,缩短了程序总计算时间。
为了对比其他格式,还添加了ELL、CSR、COO格式(使用CUSP)。

在输入文件格式方面,\ProgramName 使用了Lua语言作为输入文件格式。
Lua是一种快速、轻量级的嵌入式动态语言,设计目标即为容易嵌入到C语言中使用,
方便与C等语言交互,现已在PC游戏界广泛使用。
Lua当前版本5.2.2,基于MIT开源协议开源,
可从 \url{http://www.lua.org/download.html} 获得。
使用Lua语言作为输入文件不光可以显著减少解析输入文件的工作量,
而且还允许用户通过输入文件输入带有控制逻辑的Lua程序以对程序行为进行更为精细的控制,
\ProgramName 基于Lua提供了一种方面地输入动力学问题中材料、截面随时间变化的方式。

由于\ProgramName 程序是三维时空动力学程序,产生的数据包含5个维度(空间3维、能群1维、时间1维),
如果使用传统的纯文本格式输出,用户对数据进行后处理时则会十分不便,
所以\ProgramName 使用国际上较流行的一种针对科学计算的数据交换格式——HDF5作为直接输出格式。
HDF5是一种分层的、带有元数据的、支持多种数据格式的、针对科学计算的数据文件存储格式及相关技术。
它最早由美国国家超级计算应用中心(NCSA)提出,
现在由非营利的HDF Group管理和维护。
HDF5文件格式的读写库基于一种类似于BSD的协议开源\footnote{协议文本可以从
\url{http://www.hdfgroup.org/HDF5/doc/Copyright.html} 获取。},
当前版本为HDF5 1.8.10 patch1\footnote{可
从 \url{http://www.hdfgroup.org/downloads/index.html} 获取。}。

\begin{comment}

由于\ProgramName 程序使用人不能直接读取的HDF5格式进行输出,所以需要额外的工具进行数据后处理。
支持HDF5格式的第三方工具和编程语言比较丰富,本文则推荐Python语言作为后处理语言,
并提供一些Python程序作为示例。
Python语言是一种面向对象的动态语言,现在在学术界和工业界都十分流行,
不少GPU加速工具和科学可视化工具都使用Python编写,
所以本文也使用Python作为主要数据前处理、
后处理脚本语言。\footnote{Python可以从
\url{http://www.Python.org} 免费获取,本文使用 Python2.7.x 的语法。}

\end{comment}

\subsection{材料反应截面的输入}

对于扩散程序,输入数据中占主要部分的是各截面在不同空间网格上的取值,
\ProgramName 和传统确定论程序一样,采用两段式的输入方式,
即输入分为两部分:每个材料的各个截面和材料在空间网格上的分布,
前者一般用列表形式给出,后者一般用粗网格上的材料矩阵给出。
这种方式主要针对扩散计算中问题的常见特点:单个组件尺度内往往只有一种材料,
不同组件间的材料也往往有重复的。\footnote{但对于全堆均匀化后的截面参数,
即各个组件材料不同的情况,这种方式略显冗余。}

对于临界计算来说,以上这种两段式的输入方式已经能够满足实际计算需求,
但对于时空力学问题尚缺乏统一的材料描述方式。
对于时空动力学来说,材料定义的主体部分仍然和临界相同,
实际问题中往往大部分区域的材料、截面并不随时间变化,可以用两段式输入描述。
材料、截面随时间变化的部分基本可以分为两种模式:
\begin{enumerate}
\item
某个或某些材料的某些截面发生变化,即材料$M_i$的某截面$\Sigma_{M_i,j}$不再是定值,
而是随时间变化的函数$\Sigma_{M_i,j}(t)$。
在常见基准算例中,这种中情况常见于材料的部分成分发生改变时。
对于空间是一维、二维等低维形式的算例,由于计算中常取反应堆横截面,
高度方向被简化,控制棒移动的效果只能通过连续或阶跃改变固定网格处的材料反应截面来反映。

\item
材料在空间网格上的分布发生改变,如三维问题中,控制棒移动会改变控制棒路径上各网格的材料,
这时各网格上的截面从一种材料阶跃或连续地变为另一种材料的。
\end{enumerate}
实际计算中可能会出现以上两种模式的混合需求:
\begin{enumerate}
\setcounter{enumi}{2}

\item 只有某些固定区域上的某材料的截面发生变化。
这种情况在计算中一般通过人为地把改变的区域和不改变的区域指定成不同材料,
这样就可以隔离不同区域间的影响,使用上面模式1的输入方式。

\item
在三维问题中,控制棒从某网格某一侧进入并逐渐填满网格的过程是连续的,
但如果只使用上面第二种模式的方式则只能描述为阶跃变化(或是额外定义大量中间状态的材料),
带来额外的误差。为了提高精度,往往会对这些处于过度状态的网格(甚至包括它们附近的网格),
进行一定的均匀化或插值处理,得到过渡状态的截面。
而如何进行这方面的均匀化或插值则有多种方法,如按体积加权或按通量加权等,
不宜在程序中固化,\ProgramName 最好能提供某种接口来让用户控制
这些网格上的截面计算工作。

\end{enumerate}

为了给予用户精细地控制材料输入的方式,\ProgramName 选择了Lua这种动态语言作为材料输入的方式。
动态语言的特点是在运行时可以直接读取程序源代码并执行而不需要显式的编译过程,
这样就可以让用户把均匀化代码放在输入文件中,在运行时根据条件计算各网格上的材料。
如果不考虑效率问题,完全可以用这种动态语言的方式替换掉之前的二段式材料输入方式,
把材料在空间网格上的离散(及可能涉及的均匀化)过程暴露给用户,以增强程序的通用性。

但实际程序中很难使用这种方式,原因是程序运行速度的问题。
动态语言在一般经验中比编译型的语言慢两个数量级,
比较快的动态语言速度大约也比编译型语言慢一个数量级,
如果语言设计时就考虑到性能问题,而且动态语言的实现也经过充分优化,
并使用在线按需编译技术(一般称作JIT)后,其性能才能够勉强接近C语言。
现在这方面的动态语言只有少数的几个,如Lua的LuaJIT实现,
它是目前为止世界上最快的动态语言实现之一\footnote{参见:\url{http://luajit.org/luajit.html}};
还有最近刚出现的类Matlab语法的开源科学计算语言Julia(\url{http://julialang.org/}),
它使用了较为先进的LLVM的JIT在线编译器,性能也接近C语言的效率,
性能比较见Julia项目主页,但Julia项目现在并不成熟,而且不便于嵌入到\ProgramName 中。

而对于\ProgramName 程序这种细网有限差分程序,空间网格数量可达$10^7$的量级
\footnote{对于静态IAEA PWR三维基准问题(见第 \ref{sec:result.test.iaea} 节),
如果空间网格尺寸取1cm,则网格数量为$170\times170\times380=1.1\times10^7$。},
可能的方案有
\begin{enumerate}
\item
使用Lua的LuaJIT实现,性能接近C语言,但由于LuaJIT自身的限制只能适用于使用的内存不超过1G的情况。

\item
使用TinyCC(项目主页\url{http://bellard.org/tcc/})这种能嵌入到其他程序中的C编译器,
在运行时编译用户输入的材料处理函数以实现高速计算。
\end{enumerate}
但这两种方案都较为复杂,并有自己的不足。更为重要的是,
这两种方式只能在CPU端产生所有网格的截面数据,
在GPU计算前需要再复制到GPU显存上,大量数据的传输开销无法避免,
对于时空动力学问题,需要每个时间步都要产生截面数据并进行传输,时间上的开销太大。
根据前面的思路,也可以根据用户的输入在运行时编译为可直接在GPU上运行的PTX指令,
以避免数据传输问题,但这种方式的技术尚不成熟,而且对于开发人员和用户来说难度和工作量都要大的多,
目前基本没有实际应用的价值。

\begin{algorithm}
\ForEach(\tcc*[f]{遍历每种材料的截面信息}){输入文件中的材料截面信息}
{
  \If{截面信息通过时间相关函数生成}
  {
  调用该材料截面生成函数获得该时间步的截面定义\;
  }
  \Else
  {
  直接获取该材料的固定截面\;
}
}
预处理截面信息,并传输到GPU显存\;
\If{存在材料分布更新函数MatChangeFun}
{
  \If{上一时间步中,材料分布信息被更新过}
  {
    恢复初始的材料分布信息\;
  }
  在Lua环境内调用材料更新函数MatChangeFun,产生材料分布差分数据\;
  从Lua环境中取回材料分布差分数据\;
  根据材料分布差分数据更新材料分布信息\;
  更新GPU显存上的材料分布信息\;
}
在GPU端根据材料的截面信息和材料分布信息产生截面在空间网格上的分布\;
\setlabelname{每个时间步\ProgramName 对材料截面更新过程}
\caption{\label{alg:program.material.update}每个时间步\ProgramName 对材料截面更新过程}
\end{algorithm}

综合考虑以上各方面因素,\ProgramName 程序最后仍然采用了二段式的方式进行
数据输入,在时空动力学的材料输入上采用了动态语言的方式来实现功能上的可扩展性,
同时避免了大量的截面信息在CPU端产生的问题。主要方式为:
每个时间步开始时,由用户通过输入文件的Lua代码给出本时间步和初始时间步的材料在网格分布
信息的差分信息,即哪些网格上的材料发生了改变,并给出本时间步的新材料截面信息。
程序根据材料分布的差分信息和初始材料分布信息在CPU端产生新的材料分布矩阵,
新材料分布矩阵和新材料截面信息传送至GPU上,
GPU端程序根据信息直接在显存产生完整的材料分布数据。
在考虑到程序速度的情况下最大限度地照顾到前面提出的各种需求。
此外\ProgramName 还针对需要临时改变全部或某些材料截面的需求
添加了相应接口,简化输入文件修改的工作量。
每个时间步\ProgramName 对材料截面更新过程见\floatref{alg:program.material.update}。

需要说明的是,以上的材料更新过程和空间网格定义是关联的,
当需要在不同网格上进行计算时,程序会对不同网格上的材料信息分开处理,
允许用户在粗网和细网上分别定义不同的材料更新及均匀化方式。


\subsection{临界计算}
\label{sec:program.eigen}

临界计算是反应堆物理设计中常见的计算需求,同时也是动力学中计算通量初值所必需的,
所以\ProgramName 程序实现了临界计算功能。
临界计算部分,\ProgramName 主要使用:
\begin{enumerate}
\item CG-SG,使用CG方法从高能群到低能群逐群求解单群方程组。
\item BiCGStab-MG,使用BiCGStab方法直接求解所有能群联合方程组。
\end{enumerate}
两种方法。为了对比其他方法,程序还实现了如下求解方法:
\begin{enumerate}
\setcounter{enumi}{2}
\item Jacobi-SG,使用Jacobi迭代逐群求解。
\item Jacobi-MG,使用Jacobi对所有能群统一求解。
\item GMRES-MG,使用GMRES对所有能群统一求解。
\end{enumerate}
以上各迭代算法中,CG、BiCGStab、GMRES使用CUSP库的实现,
本文对其显存管理进行了改进。预条件算法使用对角线预条件算法,
见\sectionref{sec:gpu.krylov-precond}。

此外\ProgramName 为了加速细网问题的求解速度,
使用了MultiLevel方法改进迭代过程的初值,显著减少了迭代次数,
详见\sectionref{sec:equsolve.multimesh}。


\begin{algorithm}
读取输入文件的临界部分 \algoend
配置内核稀疏矩阵格式\;
初始化网格信息\;
\If{使用MultiLevel}
{
  初始化粗网网格信息 \algoend
  初始化粗网通量\;
  配置粗网求解算法及参数 \algoend
  求解粗网临界问题\;
  将粗网通量插值为细网通量\;
}
\Else
{
  初始化细网通量\;
}
配置求解算法及参数 \algoend
求解临界问题\;
通量归一化,输出结果\;
\setlabelname{\ProgramName 程序临界功能主流程伪代码}
\caption{\label{alg:program.eigen.main}\ProgramName 程序临界功能主流程伪代码}
\end{algorithm}


\begin{algorithm}
初始化CPU端和GPU端数据\;
根据材料定义和截面信息在GPU端直接产生的空间网格上的截面信息\;
根据网格信息配置DIA格式的迭代矩阵需要的空间\;
根据空间离散方式在GPU上填充各群迭代矩阵\;

根据当前通量计算第1群源项\;
计算对角线预条件算法的对角线矩阵\;
源迭代过程初始化\;
\Repeat{源迭代收敛}
{
  使用CG求解第1群通量 \algoend
  计算第2群源项\;
  使用CG求解第2群通量 \algoend
  更新第1群源项\;
  计算下一代的$K_\mathrm{eff}$ \algoend
  估计$K_\mathrm{eff}$和通量的误差\;
}
\setlabelname{\ProgramName 程序临界功能CG-SG算法伪代码}
\caption{\label{alg:program.eigen.cg-sg}\ProgramName 程序临界功能CG-SG算法伪代码}
\end{algorithm}


\begin{algorithm}
初始化CPU端和GPU端数据\;
根据材料定义和截面信息在GPU端直接产生的空间网格上的截面信息\;
根据网格信息配置DIA格式的迭代矩阵需要的空间\;
根据空间离散方式在GPU上填充迭代矩阵\;

根据当前通量计算源项\;
计算对角线预条件算法的对角线矩阵\;
源迭代过程初始化\;
\Repeat{源迭代收敛}
{
  使用BiCGStab求解各群通量 \algoend
  更新源项\;
  计算下一代的$K_\mathrm{eff}$ \algoend
  估计$K_\mathrm{eff}$和通量的误差\;
}
\setlabelname{\ProgramName 程序临界功能BiCGStab-MG算法伪代码}
\caption{\label{alg:program.eigen.bicgstab-mg}\ProgramName 程序临界功能BiCGStab-MG算法伪代码}
\end{algorithm}

临界部分的求解流程见\floatref{alg:program.eigen.main}。
以上各求解算法的实现可以分为两大类,逐群求解和联合求解,
这两类中的算法过程基本相近,所以只以CG-SG和BiCGStab-MG
为代表进行说明。
CG-SG算法的伪代码见\floatref{alg:program.eigen.cg-sg},
BiCGStab-MG算法的伪代码见\floatref{alg:program.eigen.bicgstab-mg}

\FloatBarrier

\subsection{时空动力学计算}

\subsubsection{时空动力学问题初值}
\label{sec:program.kinetics.keff-fix}

要计算时空动力学问题,首先要计算时变问题的初值,
实际中一般取刚好临界时的稳定状态为初始状态,
而实际算例中很少有恰好$k_\mathrm{eff}=1$的时候,
如果仅仅使用临界计算的通量作为初始通量分布并用\aeqref{equ:pro.diff.init.c}计算缓发中子源的话,
在第一个时间步中相当于材料截面有一个阶跃,使得总通量分布在第一个时间步会有一个明显的跳跃,
而且这也不符合\aeqref{equ:pro.diff.init.c}的计算条件。
所以一般计算中会对问题的参数进行修正(如程序NGFMN-K\cite{zhaowenbo}),使修正后的问题满足$k_\mathrm{eff}=1$,
本文对裂变项进行修正
\begin{align}
  \newcommand{\para}{\pb{\bm{x},t}}
  \nu\Sigma'_{f,g'}\para = \frac{1}{k_\mathrm{eff}}\nu\Sigma_{f,g'}\para
\end{align}
但实际计算中,$k_\mathrm{eff}$是由临界计算得到的,实际临界计算往往会有少量误差,
使得$k_\mathrm{eff}$的计算值和实际值之间有微小的差别,这与源迭代收敛条件有关。
但这个微小的误差会在后续的动力学过程中表现出来,即第一个时间步中总体通量会有一个小的阶跃,
而且在后续时间步中总体通量也会有小指数上升或小指数下降的趋势,影响时空动力学的计算和结果比较。
所以本文采用多次试算、修正的方式直至最终$k_\mathrm{eff}$足够接近于1,
算法伪代码见\floatref{alg:program.kinetics.keff-fix}。

\begin{algorithm}
初始化$c_L:=0.1, c_U:=10, c:=1, n:=0$\;
\While{$k_\mathrm{eff}^{(n)}$足够接近于$1$或$c_U-c_L$足够小}
{
  \lIf(\tcc*[f]{更新系数$c$的上下界})
  {$k>1$}{$\displaystyle c_U:=\frac{c+c_U}{2}$}
  \lElse(\tcc*[f]{每次减半是考虑到$k_\mathrm{eff}$的计算可能有误差})
  {$\displaystyle c_L:=\frac{c+c_L}{2}$}
  $\displaystyle c:=\frac{c}{\ k_\mathrm{eff}^{(n)}\ }$\;
  \lIf(\tcc*[f]{确保$c$不会发散})
  {$c>c_U$或$c<c_L$}
  {$\displaystyle c:=\frac{c_L+c_U}{2}$}
  使用$c$对裂变截面进行修正
  $\nu\Sigma'_{f,g',\bm{k}}\Big|_{t=0} := c\cdot\nu\Sigma_{f,g',\bm{k}}\Big|_{t=0}$\;
  重新计算临界问题,得到$k_\mathrm{eff}^{(n+1)}$ \algoend
  $n:=n+1$\;
}
\setlabelname{\ProgramName 程序时空动力学临界修正伪代码}
\caption{\label{alg:program.kinetics.keff-fix}
\ProgramName 程序时空动力学临界修正伪代码}
\end{algorithm}


\begin{comment}
\subsubsection{时空动力学与预条件算法}

由于复杂的预条件算法尚不能在GPU上并行化,只能在CPU上计算,
使得其启动(setup)时间往往较长,在静态临界计算中一般会导致总求解时间增加。
如果在每个时间步求解中分别使用这类复杂的预条件算法,则更加得不偿失。
但时空动力学计算中,材料变化一般相对较小,最后待求解的方程变化也并不大,
而预条件算法本身就是通过近似求解原方程来实现加速,
不要求与当前的方程完全符合,
所以可以在时空动力学启动计算初值时,
同时setup一个较复杂的预条件算法,得到初始状态的预条件矩阵,
在之后的每个时间步中总是使用这个预条件矩阵。

本文使用CUSP提供的SPAI预条件算法,
根据计算结果,上面这种方法对于网格数量较多的问题能够起到加速作用,
代价是时空动力学启动(init)时间增加。
\end{comment}


\subsubsection{时空动力学计算}

前面已经介绍,本文的时空动力学计算中,时间离散采用隐式向后差分格式,
每个时间步的通量只由上一步的通量及缓发中子源决定,
由于是隐式迭代,所以每个时间步要求解一个固定源扩散方程。
时空动力学计算流程见\floatref{alg:program.kinetics.loop}。

\begin{algorithm}
读取时空动力学计算参数并初始化\;
计算时空动力学初值
\tcc*[f]{见第 \ref{sec:program.eigen}
节及第 \ref{sec:program.kinetics.keff-fix} 节}
\;
%\If{使用SPAI预条件算法}
%{
%  读取SPAI设置并初始化SPAI\;
%}
\Repeat{完成所有时间步}
{
  初始化下一个时间步\;
  更新本时间步的材料、截面信息\;
  %\If{使用SPAI预条件算法}
  %{
  %  求解细网网格上的固定源问题(使用SPAI预条件矩阵)\;
  %}
  %\Else
  %{
    初始化对角线预条件算法,并求解细网网格上的固定源问题\;
  %}
  计算缓发中子源 \algoend
  计算当前时间步的最大通量及总功率\;
  将本时间步计算结果从GPU显存传送至内存\;
}

\setlabelname{\ProgramName 程序时空动力学主要过程伪代码}
\caption{\label{alg:program.kinetics.loop}
\ProgramName 程序时空动力学主要过程伪代码}
\end{algorithm}

同临界部分一样,固定源的多群扩散方程可以根据对能群处理的不同而有不同的求解方式,
但由于BiCGStab-MG方法只需要一层迭代循环,内迭代次数要比CG-SG方法少,
所以本文选择了BiCGStab-MG方法作为固定源方程求解算法,
伪代码见\floatref{alg:program.kinetics.bicgstab-mg}。

\begin{comment}

在实际计算中,

\begin{algorithm}
根据网格及截面信息在GPU显存上直接生成各群的迭代矩阵\;
初始化预条件算法的对角线矩阵\;
\Repeat{各群通量收敛}
{
  计算第1群的源项 \algoend
  使用CG算法求解第1群通量\;
  计算第2群的源项 \algoend
  使用CG算法求解第2群通量\;
  估算各群通量的误差\;
}
\setlabelname{\ProgramName 程序固定源CG-SG算法伪代码}
\caption{\label{alg:program.kinetics.cg-sg}
\ProgramName 程序固定源CG-SG算法伪代码}
\end{algorithm}

\end{comment}

\begin{algorithm}
根据网格及截面信息在GPU显存上直接生成迭代矩阵\;
初始化预条件算法的对角线矩阵\;
计算源项\;
使用BiCGStab算法直接求解各群通量\;
\setlabelname{\ProgramName 程序固定源BiCGStab-MG算法伪代码}
\caption{\label{alg:program.kinetics.bicgstab-mg}
\ProgramName 程序固定源BiCGStab-MG算法伪代码}
\end{algorithm}


\subsection{迭代收敛条件}

在迭代过程中,需要不断地对当前迭代的收敛程度进行估计,
本文使用本次迭代通量和上次迭代通量的绝对偏差的最大值进行对收敛程度进行估计,
为了排除通量幅值的影响,使用本次迭代的最大通量进行标准化,即
\begin{align}
e^{(p)}=\frac{\displaystyle \max_{\bm{k},g}\left|\phi_{\bm{k},g}^{(p)}-\phi_{\bm{k},g}^{(p-1)}\right|}
         {\displaystyle \max_{\bm{k},g}\phi_{\bm{k},g}^{(p)}}
\end{align}
其中$\bm{k}$为三维网格编号。


\subsection{MultiLevel方法}
\label{sec:equsolve.multimesh}

由于初值对于迭代算法的运行时间影响较大,
所以可以通过改善初值来实现总计算时间的缩减。
对于反应堆类问题,堆中的材料分布相对较简单,
尤其是在细网离散时往往会出现大片的网格材料相同的情况。
可以首先对问题进行粗网离散,可以用较低的开销进行求解,
获得一个较粗略的结果后,可以变换为一个较好的细网计算初值,
达到减少总计算时间的目的,
这种方法由文献\onlinecite{ginestar2001multilevel}最早提出,
称作MultiLevel方法。

本文采用如下方式计算临界计算中使用的通量初值:
产生原网格xyz方向网格数量均减半的空间网格划分,
即细网格上的8个网格对应到粗网上的1个网格,
使用如前所用的CG-SG等方法进行求解,该阶段用户可以通过输入文件自定义
每轮内迭代次数、外迭代收敛标准等控制变量,
在粗网上求解后把粗网上的通量插值为细网通量,
细网上的初始$K_\mathrm{eff}$取为粗网$K_\mathrm{eff}$%
\footnote{同一个扩散问题采用不同网格大小进行离散后得到的$K_\mathrm{eff}$并不完全一致,
略有差异。},
最后在细网上使用CG-SG等方法进行求解。

\subsection{显存管理}

CUSP的显存管理部分继承自Thrust的显存管理方式,
而Thrust的显存管理策略为需要时向CUDA申请,
不需要时向CUDA释放,CUDA本身的显存申请释放速度略慢。
虽然这种策略对于一般需求时能够应付,
但把CG、BiCGStab等算法放在源迭代过程中时会导致每次源迭代
开始及结束时都会进行显存的申请及释放,
会明显增加程序的运行时间。

为了消除这种无谓的时间开销,本文对Thrust的显存管理策略进行了修改,
增加缓存功能,即当上层算法对显存用完释放时,
不是直接调用CUDA释放,而是先放如一个可用显存池中。
这样当以后上层算法需要重新申请显存时,
可以直接从可用显存池中查找是否有适合的显存块,如果发现可用的,
则直接返回给上层算法,没有时才向CUDA进行申请。
通过这种方式,可以消除外迭代反复调用CUSP算法时出现的反复申请释放同样大小的显存的情况,
一般来讲从显存池中查找适合的显存的时间开销要小于向CUDA申请显存的开销,
从而达到了缩短程序运行时间的效果。
%\TODO 考虑添加结果比较

\begin{table}
\centering
\caption{不同显存管理策略下临界计算时间和显存占用峰值表}
\label{tab:program.cached_alloc}
\begin{tabular}{cccc}
\topline
计算条件 & 显存管理策略 & 计算时间/s & 显存占用峰值/MB\\
\midline
\multirow{2}{*}{5cm网格}
 & 无缓存 & 3.994 & 11.23\\
 & 有缓存 & 2.652 & 11.23\\
\multirow{2}{*}{2.5cm网格}
 & 无缓存 & 7.052 & 89.82\\
 & 有缓存 & 5.819 & 89.82\\
\multirow{2}{*}{2cm网格}
 & 无缓存 & 9.938 & 175.43\\
 & 有缓存 & 8.549 & 175.43\\
\multirow{2}{*}{1cm网格}
 & 无缓存 & 55.630 & 1403.42\\
 & 有缓存 & 53.430 & 1403.42\\

\multirow{2}{*}{2.5cm网格+MultiLevel}
 & 无缓存 & 4.790 & 93.00\\
 & 有缓存 & 3.338 & 101.05\\
\multirow{2}{*}{1cm网格+MultiLevel}
 & 无缓存 & 14.212 & 1453.17\\
 & 有缓存 & 12.512 & 1578.85\\
\bottomline
\end{tabular}
\end{table}

对了观察显存分配缓存的影响,这里使用静态IAEA PWR三维基准题进行计算,
不同情况下的计算时间和显存占用峰值见\floatref{tab:program.cached_alloc}。
可见,带缓存的显存管理策略以少量显存占用峰值增加为代价
(不使用MultiLevel时峰值显存增加很小)
减少了约1-2s的计算时间,对于小规模问题效果十分明显。
