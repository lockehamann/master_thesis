

\chapter{大型扩散方程GPU求解方法研究}

本课题主要使用中子扩散方程作为实际研究问题,
中子扩散方程在实际求解中大多使用有限差分方式进行空间离散,
得到的线性方程组为7对角实对称对角占优矩阵(三维、单一能群内)。
稳态和动力学扩散方程的求解主要涉及到大型7对角线矩阵的最大特征值计算
和大型线性方程组求解。
本章则主要研究如何在GPU上高效地求解这类问题。

\section{稀疏矩阵格式选择}

为了比较各种存储格式的速度,使用三维扩散临界计算程序来进行测试,
测试算例为三维IAEA基准题(见\sectionref{sec:result.test.iaea}),
各存储格式的计算时间见\floatref{tab:equsolve.spformat}及
\floatref{fig:equsolve.spformat},图中DM表示双层网格加速(见\sectionref{sec:equsolve.multimesh}),
SP表示单精度、DP表示双精度,每种情况计算5次,时间取最小值。

结果很清楚地显示:对于三维扩散方程,DIA和ELL格式明显优于CSR和COO格式,其中DIA性能最高。
这主要是因为DIA和ELL较为适合向量机型处理器的运算和内存访问方式,
而且DIA格式占用的空间最少(因为矩阵正好是7对角线矩阵)。
在CG和BiCGStab计算中,矩阵参与的部分是稀疏矩阵-向量乘法(以下简称为SpMV),
SpMV在GPU上的主要瓶颈是显存带宽\cite{bell2008spmv,baskaran2008optimizing}\footnote{现在的GPU运算能力太强,使得显存带宽成为瓶颈。},
所以DIA性能最好是预料之中。

\begin{sidewaystable}
\pdfrotate
\centering
\begin{minipage}{.8\linewidth}
\centering
\caption[不同稀疏矩阵格式求解三维临界扩散的时间表]
{\label{tab:equsolve.spformat}%
不同稀疏矩阵格式求解三维临界扩散的时间表(单位:s)%
\footnote{不同的存储格式对非零元有着不同的求和顺序,由于浮点误差存在,%
不同的格式要达到收敛标准所需要的迭代次数可能有差别,并导致总计算时间的改变。}
}
\begin{tabular}{ccccccccc}
\toprule
 \multirow {3}{*}{矩阵格式}  &
       \multicolumn{4}{c}{2.5cm $\times$ 2.5cm $\times$ 2.5cm}
       &\multicolumn{4}{c}{1.25cm $\times$ 1.25cm $\times$ 1.25cm} \\
 &\multicolumn{2}{c}{CG\footnote{内迭代每轮18次,下同。}}
 	   &\multicolumn{2}{c}{BiCGStab\footnote{内迭代每轮30次,下同。}}
       & \multicolumn{2}{c}{CG}& \multicolumn{2}{c}{BiCGStab}\\
 & SP& DP& SP& DP& SP& DP& SP& DP\\
\midrule
 DIA&  5.475&  7.956& 12.199& 19.391& 24.071& 43.415&  64.256& 116.922\\
 ELL&  6.255&  8.596& 14.711& 20.997& 29.141& 47.362&  81.261& 133.599\\
 CSR& 10.452& 13.042& 27.814& 34.960& 62.478& 82.929& 180.430& 238.462\\
 COO& 11.232& 13.432& 30.934& 36.629& 64.303& 82.384& 197.263& 248.618\\
 DIA DM\footnote{粗网内迭代每轮10次,下同。}
       &  3.838&  4.883&  6.349&  8.003&  9.266& 14.976&  14.165&  31.590\\
 ELL DM&  4.399&  5.351&  7.363&  8.626& 10.920& 16.224&  17.503&  35.322\\
 CSR DM&  5.975&  6.989& 11.513& 14.337& 19.734& 26.301&  38.438&  61.776\\
 COO DM&  6.474&  7.660& 11.840& 13.712& 20.733& 26.364&  42.400&  65.910\\
\bottomrule
\end{tabular}
\end{minipage}
\end{sidewaystable}

\begin{figure}
\centering
\begin{asy}
import graph;
size(13cm,15cm,IgnoreAspect);
real[] x=sequence(8);
real[] DIA={5.475, 7.956, 12.199, 19.391, 24.071, 43.415, 64.256, 116.922};
real[] ELL={6.255, 8.596, 14.711, 20.997, 29.141, 47.362, 81.261, 133.599};
real[] CSR={10.452, 13.042, 27.814, 34.960, 62.478, 82.929, 180.430, 238.462};
real[] COO={11.232, 13.432, 30.934, 36.629, 64.303, 82.384, 197.263, 248.618};
real[] DIAMM={3.838, 4.883, 6.349, 8.003, 9.266, 14.976, 14.165, 31.590};
real[] ELLMM={4.399, 5.351, 7.363, 8.626, 10.920, 16.224, 17.503, 35.322};
real[] CSRMM={5.975, 6.989, 11.513, 14.337, 19.734, 26.301, 38.438, 61.776};
real[] COOMM={6.474, 7.660, 11.840, 13.712, 20.733, 26.364, 42.400, 65.910};
scale(Linear,Log);
string[] month={
"2.5cm CG SP",
"2.5cm CG DP",
"2.5cm BiCGStab SP",
"2.5cm BiCGStab DP",
"1.25cm CG SP",
"1.25cm CG DP",
"1.25cm BiCGStab SP",
"1.25cm BiCGStab DP",
};
transform markersize = scale(1.5mm);
draw(graph(x,DIA),legend="DIA", marker(markersize*polygon(3)));
draw(graph(x,ELL),legend="ELL", marker(markersize*polygon(4)));
draw(graph(x,CSR),legend="CSR", marker(markersize*unitcircle));
draw(graph(x,COO),legend="COO", marker(markersize*cross(4))  );
draw(graph(x,DIAMM),legend="DIA DM", dashed, marker(markersize*polygon(3)));
draw(graph(x,ELLMM),legend="ELL DM", dashed, marker(markersize*polygon(4)));
draw(graph(x,CSRMM),legend="CSR DM", dashed, marker(markersize*unitcircle));
draw(graph(x,COOMM),legend="COO DM", dashed, marker(markersize*cross(4))  );
xaxis(BottomTop,LeftTicks(rotate(90)*Label(),new string(real x) {
return month[round(x)];}));
yaxis("$T/\mathrm{s}$",LeftRight,RightTicks);
add(legend(),point(NW),10SE);
\end{asy}
\caption{\label{fig:equsolve.spformat}不同稀疏矩阵格式求解三维临界扩散的时间}
\end{figure}


\section{迭代及预条件算法选择}

为了比较不同迭代求解方法和预处理器的效果,这里仍然选择用IAEA基准题进行测试,
网格大小分别取5cm、2.5cm、1cm进行测试。

选择的算法包括
\begin{enumerate}
\item Jacobi-SG,逐群使用Jacobi迭代进行求解。
\item Jacobi-MG,使用Jacobi对所有能群统一求解。
\item CG-SG,逐群使用CG迭代进行求解。
\item BiCGStab-MG,使用BiCGStab对所有能群统一求解。
\item GMRES-MG,使用GMRES对所有能群统一求解。
\end{enumerate}

由于在临界计算的源迭代中一般在每步内迭代中精确求解方程组,
往往是迭代一个较少的次数,可以达到大幅减少计算时间的目的。
这个内迭代次数和问题的规模和迭代算法、预条件算法都有关系,
如何根据问题规模选择最优的内迭代次数不在文本的讨论之内,
为了公平地比较各种算法,以下将分别寻找各种情况下最优的内迭代次数。

\subsection{Jacobi-SG}
\label{sec:equsolve.iter.jacobi-sg}

Jacobi-SG算法网格大小分别取5cm、2.5cm、2cm、1cm的计算结果
见\floatref{tab:equsolve.iter.jacobi-sg.5cm}、%
\floatref{tab:equsolve.iter.jacobi-sg.2.5cm}、%
\floatref{tab:equsolve.iter.jacobi-sg.2cm}和%
\floatref{tab:equsolve.iter.jacobi-sg.1cm},
从表中可见最优的内迭代次数分别为7、11、16、18。

\begin{datasheet}
\sectionref{sec:equsolve.iter.jacobi-sg}的数据表:
\floatref{tab:equsolve.iter.jacobi-sg.5cm}、
\floatref{tab:equsolve.iter.jacobi-sg.2.5cm}、
\floatref{tab:equsolve.iter.jacobi-sg.2cm}、
\floatref{tab:equsolve.iter.jacobi-sg.1cm}
。

\begin{table}
\centering
\caption{5cm 网格时 Jacobi-SG 不同内迭代次数的计算时间及总迭代次数}
\label{tab:equsolve.iter.jacobi-sg.5cm}
\begin{tabular}{cccc}
\toprule
内迭代次数 & 计算时间/s & 总内迭代次数 & 外迭代次数\\
\midrule
%1 & 2.855 & 4138 & 2069\\
2 & 1.965 & 4180 & 1045\\
3 & 1.701 & 4266 & 711\\
4 & 1.576 & 4368 & 546\\
5 & 1.513 & 4500 & 450\\
6 & 1.497 & 4644 & 387\\
7 & 1.420 & 4802 & 343\\
8 & 1.435 & 4976 & 311\\
9 & 1.482 & 5220 & 290\\
10 & 1.607 & 5500 & 275\\
20 & 2.012 & 8440 & 211\\
30 & 2.683 & 11580 & 193\\
40 & 3.463 & 14960 & 187\\
50 & 4.181 & 18400 & 184\\
\bottomrule
\end{tabular}
\end{table}

\begin{table}
\centering
\caption{2.5cm 网格时 Jacobi-SG 不同内迭代次数的计算时间及总迭代次数}
\label{tab:equsolve.iter.jacobi-sg.2.5cm}
\begin{tabular}{cccc}
\toprule
内迭代次数 & 计算时间/s & 总内迭代次数 & 外迭代次数\\
\midrule
2 & 12.870 & 15780 & 3945\\
3 & 11.466 & 15810 & 2635\\
4 & 10.889 & 15832 & 1979\\
5 & 9.906 & 15880 & 1588\\
6 & 9.547 & 15936 & 1328\\
7 & 9.516 & 15988 & 1142\\
8 & 9.391 & 16064 & 1004\\
9 & 9.376 & 16146 & 897\\
10 & 9.297 & 16240 & 812\\
11 & 9.048 & 16324 & 742\\
12 & 9.141 & 16440 & 685\\
13 & 9.329 & 16536 & 636\\
14 & 9.453 & 16660 & 595\\
15 & 9.048 & 16770 & 559\\
16 & 9.204 & 16896 & 528\\
17 & 9.141 & 17034 & 501\\
18 & 9.641 & 17172 & 477\\
19 & 9.594 & 17328 & 456\\
20 & 9.469 & 17480 & 437\\
30 & 10.078 & 19080 & 318\\
40 & 11.373 & 21840 & 273\\
50 & 12.776 & 24700 & 247\\
\bottomrule
\end{tabular}
\end{table}

\begin{table}
\centering
\caption{2cm 网格时 Jacobi-SG 不同内迭代次数的计算时间及总迭代次数}
\label{tab:equsolve.iter.jacobi-sg.2cm}
\begin{tabular}{cccc}
\toprule
内迭代次数 & 计算时间/s & 总内迭代次数 & 外迭代次数\\
\midrule
11 & 23.415 & 25256 & 1148\\
12 & 23.072 & 25320 & 1055\\
13 & 23.073 & 25428 & 978\\
14 & 23.493 & 25480 & 910\\
15 & 23.181 & 25590 & 853\\
16 & 22.948 & 25664 & 802\\
17 & 23.119 & 25772 & 758\\
18 & 23.104 & 25884 & 719\\
19 & 23.135 & 25992 & 684\\
20 & 23.150 & 26120 & 653\\
30 & 23.868 & 27420 & 457\\
40 & 25.756 & 29040 & 363\\
50 & 26.333 & 30800 & 308\\
\bottomrule
\end{tabular}
\end{table}


\begin{table}
\centering
\caption{1cm 网格时 Jacobi-SG 不同内迭代次数的计算时间及总迭代次数}
\label{tab:equsolve.iter.jacobi-sg.1cm}
\begin{tabular}{cccc}
\toprule
内迭代次数 & 计算时间/s & 总内迭代次数 & 外迭代次数\\
\midrule
14 & 518.420 & 96768 & 3456\\
16 & 511.619 & 96832 & 3026\\
18 & 509.309 & 96804 & 2689\\
20 & 512.507 & 96920 & 2423\\
30 & 511.556 & 97260 & 1621\\
40 & 510.714 & 97840 & 1223\\
50 & 510.136 & 98600 & 986\\
60 & 513.568 & 99480 & 829\\
70 & 521.774 & 100520 & 718\\
80 & 526.875 & 101600 & 635\\
90 & 528.966 & 102780 & 571\\
100 & 534.348 & 104000 & 520\\
\bottomrule
\end{tabular}
\end{table}

\end{datasheet}



\subsection{CG-SG}
\label{sec:equsolve.iter.cg-sg}

CG-SG算法网格大小分别取5cm、2.5cm、2cm、1cm的计算结果
见\floatref{tab:equsolve.iter.cg-sg.5cm}、%
\floatref{tab:equsolve.iter.cg-sg.2.5cm}、%
\floatref{tab:equsolve.iter.cg-sg.2cm}和%
\floatref{tab:equsolve.iter.cg-sg.1cm},
从表中可见最优的内迭代次数分别为4、7、10、17。


\begin{datasheet}
\sectionref{sec:equsolve.iter.cg-sg}的数据表:
\floatref{tab:equsolve.iter.cg-sg.5cm}、
\floatref{tab:equsolve.iter.cg-sg.2.5cm}、
\floatref{tab:equsolve.iter.cg-sg.2cm}、
\floatref{tab:equsolve.iter.cg-sg.1cm}
。

\begin{table}
\centering
\caption{5cm 网格时 CG-SG 不同内迭代次数的计算时间及总迭代次数}
\label{tab:equsolve.iter.cg-sg.5cm}
\begin{tabular}{cccc}
\toprule
内迭代次数 & 计算时间/s & 总内迭代次数 & 外迭代次数\\
\midrule
%1 & 4.540 & 4048 & 2024\\
2 & 1.576 & 1960 & 490\\
3 & 1.046 & 1464 & 244\\
4 & 0.920 & 1472 & 184\\
5 & 1.092 & 1840 & 184\\
6 & 1.186 & 2196 & 183\\
7 & 1.342 & 2562 & 183\\
8 & 1.467 & 2912 & 182\\
9 & 1.622 & 3276 & 182\\
10 & 1.685 & 3640 & 182\\
11 & 1.841 & 3988 & 182\\
12 & 1.966 & 4305 & 182\\
13 & 2.138 & 4595 & 182\\
14 & 2.246 & 4869 & 182\\
15 & 2.450 & 5124 & 182\\
16 & 2.372 & 5362 & 182\\
17 & 2.512 & 5576 & 182\\
18 & 2.559 & 5746 & 182\\
19 & 2.684 & 5888 & 182\\
20 & 2.792 & 6009 & 182\\
\bottomrule
\end{tabular}
\end{table}

\begin{table}
\centering
\caption{2.5cm 网格时 CG-SG 不同内迭代次数的计算时间及总迭代次数}
\label{tab:equsolve.iter.cg-sg.2.5cm}
\begin{tabular}{cccc}
\toprule
内迭代次数 & 计算时间/s & 总内迭代次数 & 外迭代次数\\
\midrule
%1 & 31.309 & 16256 & 8128\\
2 & 9.999 & 7472 & 1868\\
3 & 5.694 & 4740 & 790\\
4 & 3.837 & 3504 & 438\\
5 & 3.120 & 2910 & 291\\
6 & 2.855 & 2856 & 238\\
7 & 2.761 & 2800 & 200\\
8 & 2.792 & 2960 & 185\\
9 & 3.042 & 3312 & 184\\
10 & 3.385 & 3680 & 184\\
11 & 3.572 & 4026 & 183\\
12 & 3.947 & 4392 & 183\\
13 & 4.181 & 4758 & 183\\
14 & 4.493 & 5096 & 182\\
15 & 4.727 & 5460 & 182\\
16 & 5.070 & 5824 & 182\\
17 & 5.460 & 6188 & 182\\
18 & 5.787 & 6552 & 182\\
19 & 5.912 & 6916 & 182\\
20 & 6.146 & 7280 & 182\\
\bottomrule
\end{tabular}
\end{table}

\begin{table}
\centering
\caption{2cm 网格时 CG-SG 不同内迭代次数的计算时间及总迭代次数}
\label{tab:equsolve.iter.cg-sg.2cm}
\begin{tabular}{cccc}
\toprule
内迭代次数 & 计算时间/s & 总内迭代次数 & 外迭代次数\\
\midrule
%1 & 139.698 & 49450 & 24725\\
2 & 23.384 & 10940 & 2735\\
3 & 13.260 & 7410 & 1235\\
4 & 9.048 & 5528 & 691\\
5 & 6.927 & 4390 & 439\\
6 & 6.068 & 3864 & 322\\
7 & 5.757 & 3808 & 272\\
8 & 5.367 & 3696 & 231\\
9 & 5.726 & 3924 & 218\\
10 & 5.476 & 3700 & 185\\
11 & 5.897 & 4048 & 184\\
12 & 6.396 & 4416 & 184\\
13 & 6.739 & 4784 & 184\\
14 & 7.067 & 5124 & 183\\
15 & 7.395 & 5490 & 183\\
16 & 7.831 & 5856 & 183\\
17 & 8.611 & 6222 & 183\\
18 & 8.939 & 6588 & 183\\
19 & 8.939 & 6916 & 182\\
20 & 9.376 & 7280 & 182\\
\bottomrule
\end{tabular}
\end{table}


\begin{table}
\centering
\caption{1cm 网格时 CG-SG 不同内迭代次数的计算时间及总迭代次数}
\label{tab:equsolve.iter.cg-sg.1cm}
\begin{tabular}{cccc}
\toprule
内迭代次数 & 计算时间/s & 总内迭代次数 & 外迭代次数\\
\midrule
2 & 445.958 & 43336 & 10834\\
3 & 252.955 & 28458 & 4743\\
4 & 186.670 & 22616 & 2827\\
5 & 141.945 & 18000 & 1800\\
6 & 113.818 & 14928 & 1244\\
7 & 93.303 & 12558 & 897\\
8 & 78.983 & 10784 & 674\\
9 & 69.108 & 9594 & 533\\
10 & 60.591 & 8480 & 424\\
11 & 61.776 & 8580 & 390\\
12 & 53.102 & 7536 & 314\\
13 & 56.269 & 8034 & 309\\
14 & 50.981 & 7336 & 262\\
15 & 51.839 & 7500 & 250\\
16 & 50.357 & 7200 & 225\\
17 & 48.875 & 7106 & 209\\
18 & 52.806 & 7740 & 215\\
19 & 49.093 & 7106 & 187\\
20 & 50.216 & 7400 & 185\\
\bottomrule
\end{tabular}
\end{table}

\end{datasheet}


\subsection{BiCGStab-MG}
\label{sec:equsolve.iter.bicgstab-mg}

BiCGStab-MG算法网格大小分别取5cm、2.5cm、2cm、1cm的计算结果
见\floatref{tab:equsolve.iter.bicgstab-mg.5cm}、%
\floatref{tab:equsolve.iter.bicgstab-mg.2.5cm}、%
\floatref{tab:equsolve.iter.bicgstab-mg.2cm}和%
\floatref{tab:equsolve.iter.bicgstab-mg.1cm},
从表中可见最优的内迭代次数分别为3、8、10、21。


\begin{datasheet}
\sectionref{sec:equsolve.iter.bicgstab-mg}的数据表:
\floatref{tab:equsolve.iter.bicgstab-mg.5cm}、
\floatref{tab:equsolve.iter.bicgstab-mg.2.5cm}、
\floatref{tab:equsolve.iter.bicgstab-mg.2cm}、
\floatref{tab:equsolve.iter.bicgstab-mg.1cm}
。

\begin{table}
\centering
\caption{5cm 网格时 BiCGStab-MG 不同内迭代次数的计算时间及总迭代次数}
\label{tab:equsolve.iter.bicgstab-mg.5cm}
\begin{tabular}{cccc}
\toprule
内迭代次数 & 计算时间/s & 总内迭代次数 & 外迭代次数\\
\midrule
%1 & \multicolumn{3}{c}{不收敛} \\ %\footnote{Fail:Nan: KeffErr, PhiErr, }
2 & \multicolumn{3}{c}{不收敛} \\ %\footnote{Fail:Nan: KeffErr, PhiErr, }
3 & 0.998 & 624 & 208\\
4 & 1.108 & 752 & 188\\
5 & 1.264 & 915 & 183\\
6 & 1.419 & 1086 & 181\\
7 & 1.607 & 1267 & 181\\
8 & 1.763 & 1448 & 181\\
9 & 1.919 & 1629 & 181\\
10 & 2.091 & 1820 & 182\\
11 & 2.340 & 2013 & 183\\
12 & 2.434 & 2160 & 180\\
13 & 2.589 & 2366 & 182\\
14 & 2.777 & 2548 & 182\\
15 & 2.980 & 2730 & 182\\
16 & 3.214 & 2912 & 182\\
17 & 3.292 & 3094 & 182\\
18 & 3.448 & 3276 & 182\\
19 & 3.572 & 3458 & 182\\
20 & 3.869 & 3640 & 182\\
\bottomrule
\end{tabular}
\end{table}

\begin{table}
\centering
\caption{2.5cm 网格时 BiCGStab-MG 不同内迭代次数的计算时间及总迭代次数}
\label{tab:equsolve.iter.bicgstab-mg.2.5cm}
\begin{tabular}{cccc}
\toprule
内迭代次数 & 计算时间/s & 总内迭代次数 & 外迭代次数\\
\midrule
%1 & 3.900 & 900 & 900\\
2-5 & \multicolumn{3}{c}{不收敛} \\ %\footnote{Fail:Nan: KeffErr, PhiErr, }
6(超时) & >600 & >241926 & >40321 \\ %\footnote{Fail:Solve Time exceeds 600.000}
7 & 4.181 & 1638 & 234\\
8 & 4.056 & 1600 & 200\\
9 & 4.134 & 1647 & 183\\
10 & 4.446 & 1800 & 180\\
11 & 4.898 & 1991 & 181\\
12 & 5.132 & 2124 & 177\\
13 & 5.445 & 2275 & 175\\
14 & 5.834 & 2450 & 175\\
15 & 6.194 & 2625 & 175\\
16 & 6.521 & 2800 & 175\\
17 & 6.942 & 2992 & 176\\
18 & 7.332 & 3186 & 177\\
19 & 7.894 & 3401 & 179\\
20 & 8.222 & 3600 & 180\\
\bottomrule
\end{tabular}
\end{table}

\begin{table}
\centering
\caption{2cm 网格时 BiCGStab-MG 不同内迭代次数的计算时间及总迭代次数}
\label{tab:equsolve.iter.bicgstab-mg.2cm}
\begin{tabular}{cccc}
\toprule
内迭代次数 & 计算时间/s & 总内迭代次数 & 外迭代次数\\
\midrule
%1 & 12.152 & 1766 & 1766\\
2-7 & \multicolumn{3}{c}{不收敛} \\ %\footnote{Fail:Nan: KeffErr, PhiErr, }
8(超时) & >600 & >142232 & >17779 \\ %\footnote{Fail:Solve Time exceeds 600.000}
9 & 10.250 & 2412 & 268\\
10 & 8.954 & 2120 & 212\\
11 & 9.360 & 2233 & 203\\
12 & 9.313 & 2232 & 186\\
13 & 9.641 & 2301 & 177\\
14 & 10.452 & 2534 & 181\\
15 & 10.857 & 2655 & 177\\
16 & 11.388 & 2784 & 174\\
17 & 11.840 & 2941 & 173\\
18 & 12.527 & 3114 & 173\\
19 & 13.292 & 3287 & 173\\
20 & 13.900 & 3460 & 173\\
\bottomrule
\end{tabular}
\end{table}


\begin{table}
\centering
\caption{1cm 网格时 BiCGStab-MG 不同内迭代次数的计算时间及总迭代次数}
\label{tab:equsolve.iter.bicgstab-mg.1cm}
\begin{tabular}{cccc}
\toprule
内迭代次数 & 计算时间/s & 总内迭代次数 & 外迭代次数\\
\midrule
%1 & 600.929 & 16239 & 16239 \\ %Fail:Solve Time exceeds 600.000
2-17 & \multicolumn{3}{c}{不收敛} \\ %Fail:Nan: KeffErr, PhiErr,
18(超时) & >600 & >25650 & >1425 \\ %Fail:Solve Time exceeds 600.000
19 & 258.804 & 11134 & 586\\
20 & 154.456 & 6580 & 329\\
21 & 100.339 & 4305 & 205\\
22 & 105.627 & 4554 & 207\\
23 & 106.673 & 4554 & 198\\
24 & 107.936 & 4656 & 194\\
25 & 109.965 & 4725 & 189\\
26 & 115.472 & 4992 & 192\\
27 & 112.991 & 4887 & 181\\
28 & 120.354 & 5180 & 185\\
29 & 124.020 & 5336 & 184\\
30 & 129.511 & 5580 & 186\\
\bottomrule
\end{tabular}
\end{table}

\end{datasheet}




\section{双层网格加速}
\label{sec:equsolve.multimesh}
