
\chapter{结论与展望}

\section{结论}

本文主要研究了GPU加速在反应堆物理确定论计算中的应用。
首先介绍了目前GPU通用计算和科学计算的发展情况,
并以CUDA为例介绍了GPU编程的主要思路和编程模型。
此外还介绍了现代和Krylov子空间类线性方程求解算法
和它们在GPU上实现的工作。
对于稀疏矩阵在GPU上的存储格式也进行了介绍。

在上面的基础上,本文以三维扩散临界/时空动力学算法为例
进行了确定论方法的GPU并行化研究,
编写了使用GPU进行加速的\ProgramName 程序,
该程序对于静态IAEA PWR三维基准题、动态TWIGL二维基准题、
无热工反馈的动态LMW三维基准题进行了验证,
结果和同类程序符合得很好。
并对于三维IAEA基准题1cm网格下相对Citation程序
获得了最高272倍的加速效果,验证了GPU加速对于反应堆
物理确定论计算的重要意义。
对于时空动力学程序,\ProgramName 程序实现了一定条件下的
实时模拟,向前推进了目前反应堆物理实时仿真的最好结果。

此外本文还探讨了系数矩阵格式、迭代算法对扩散程序求解速度的影响,
指出DIA和ELL较为适合扩散程序的GPU并行化。
在迭代算法上,逐群求解的CG算法最适合三维扩散临界计算的GPU并行化。
对于动力学计算,BiCGStab方法可以进一步减少迭代次数,
较为适合动力学问题的GPU并行求解。

\section{展望}

目前GPU编程技术中CUDA使用的最为广泛,技术上也最为成熟,
在NVidia公司的大力推动下,CUDA的更新也十分迅速。
但CUDA编程相比传统CPU程序的编写要复杂的多,
不光需要了解GPU运算单元、存储体系的结构,
还需要自行管理显存和内存间的数据交换,
这都使得GPU并行程序的编写较为困难,
刚涉足于此的科技工作者上手难度较大。
此外CUDA技术只能在NVidia公司生产的显卡上使用,
对于AMD公司的独立显卡或是Intel公司的集成显卡都不支持。

可喜的是,近些年来除CUDA以外的其他技术也在蓬勃发展,
如可以跨硬件平台的OpenCL,C++ AMP等,
还有更容易使用的OpenACC等技术先后涌现。
虽然到目前为止,这些新技术的基础设施如编译器等还不太成熟,
生成的程序运行效率上可能还达不到CUDA的程序,
但随着相关技术的不断成熟,用户群的不断扩大,
这些新的技术将让GPU并行程序的开发工作量显著降低。

对于反应堆物理专业的科研工作者来说,
虽然目前支持OpenACC技术的编译器还很少,效果可能还不尽人意,
但随着计算机科学界在这方面研究的不断深入,编译器优化水平的不断提高,
OpenACC这样的指导性GPU并行化方式会更适合。
因为OpenACC抽象程度较高,可以脱离具体GPU参数进行开发,
可以显著降低相关程序的维护工作量,更适合非计算机专业的科学计算工作者。
