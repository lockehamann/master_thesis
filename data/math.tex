

\chapter{公式推导}

\section{扩散时空动力学偏微分方程}

多群扩散时空动力学方程为
\begin{align}
  \newcommand{\para}{\pb{\bm{x},t}}
  \left\{
  \begin{aligned}
    \frac{1}{v_g}\frac{\partial \phi_g\para}{\partial t}
    &=\nabla\cdot D_g\para \nabla\phi_g\para 
      -\Sigma_{t,g}\para \phi_g\para
      +\sum_{i=1}^I \chi_{i,g}\para \lambda_i C_i\para \\
          & \hspace{1cm}
      +\sum_{g'=1}^G\pb[B]{\chi_g\para \pb{1-\beta}\nu\Sigma_{f,g'}\para
                            +\Sigma_{g'\rightarrow g}\para}\phi_{g'}\para \\
    \frac{\partial C_i\para}{\partial t}
     &=\beta_i \sum_{g'=1}^G \nu\Sigma_{f,g'}\para \phi_{g'}\para
        -\lambda_i C_i\para \qquad i=1,2,\cdots,I
  \end{aligned}
  \right.
  \titlelabel{equ:pro.diff.equ}{多群扩散时空动力学方程}
\end{align}

如果初始条件为稳态
\begin{align}
  \newcommand{\para}{\pb{\bm{x},t}}
  \left\{
  \begin{aligned}
    \frac{\partial \phi_g\para}{\partial t}\Big|_{t=0} &=0 \\
    \frac{\partial C_i\para}{\partial t}\Big|_{t=0} &=0
  \end{aligned}
  \right.
  \label{equ:pro.diff.init.equ}
\end{align}

联立\aeqref{equ:pro.diff.equ}及\aeqref{equ:pro.diff.init.equ},
消去$C_i\pb{\bm{x},0}$可得
\begin{align}
  \newcommand{\para}{\pb{\bm{x},0}}
  \begin{aligned}
  &\nabla\cdot D_g\para \nabla\phi_g\para 
   -\Sigma_{t,g}\para \phi_g\para \\
  & \hspace{3cm}
   +\sum_{g'=1}^G\pb[B]{\chi_g\para \nu\Sigma_{f,g'}\para
                        +\Sigma_{g'\rightarrow g}\para}\phi_{g'}\para =0
  \end{aligned}
\end{align}

此为$k_\mathrm{eff}=1$时的稳态扩散方程,
由于问题的已知条件一般仅有$k_\mathrm{eff}\approx 1$,
所以一般求解普通的临界扩散方程作为通量初值。
\begin{align}
  \newcommand{\para}{\pb{\bm{x},0}}
  \begin{aligned}
  &\nabla\cdot D_g\para \nabla\phi_g\para 
   -\Sigma_{t,g}\para \phi_g\para \\
  & \hspace{3cm}
   +\sum_{g'=1}^G\pb[B]{\frac{1}{k_\mathrm{eff}}\chi_g\para \nu\Sigma_{f,g'}\para
                        +\Sigma_{g'\rightarrow g}\para}\phi_{g'}\para =0
  \end{aligned}
  \titlelabel{equ:pro.diff.init.diff.equ1}{扩散时空动力学问题的初始通量方程}
\end{align}

解出\aeqref{equ:pro.diff.init.diff.equ1}后
可由\aeqref{equ:pro.diff.init.equ}得到初始条件的$C_i\pb{\bm{x},0}$
\begin{align}
  \newcommand{\para}{\pb{\bm{x},0}}
  C_i\para = \frac{\beta_i}{\lambda_i}\sum_{g'=1}^G \nu\Sigma_{f,g'}\para\phi_{g'}\para
  \label{equ:pro.diff.init.c}
\end{align}

边界条件使用外推边界条件,即
\begin{align}
  \bm{n}\cdot\nabla\phi_g\pb{\bm{x},t} = -\frac{\phi_g\pb{\bm{x},t}}{\delta_g\pb{\bm{x},t}}
  \qquad \bm{x} \in \partial \mathcal{D}
  \titlelabel{equ:pro.diff.boundary.equ}{扩散方程外推边界条件}
\end{align}
其中$\partial \mathcal{D}$是待求解问题区域的边界面,
$\bm{n}$是边界面上的点$\bm{x}$在边界面上的法向量,
方向指向区域外。
这里不考虑中子从边界面离开又从另一处边界面进入的情况,即问题区域是一个凸空间。

外推长度取$\delta_g\pb{\bm{x},t}=2D_g\pb{\bm{x},t}$。

\section{几何空间网格划分}

\subsection{含外边界网格}
本文只考虑三维$xyz$坐标系,问题几何为立方体,
划分为$(K_x-2)\times(K_y-2)\times(K_z-2)$的结构网格,
每个方向的两个边界各增加一个边界网格,
则全部网格集合为$\mathcal{D}_{\bm{k}}=\big\{(k_x,k_y,k_z)\big|k_w = 0,1,\cdots,K_w-1 ; w=x,y,z\big\}$。

实际空间网格集记为$\underline{\mathcal{D}_{\bm{k}}}=\big\{(k_x,k_y,k_z)\big|k_w = 1,\cdots,K_w-2 ; w=x,y,z\big\}$。

边界网格集合记为$\partial \mathcal{D}_{\bm{k}}=\mathcal{D}_{\bm{k}} - \underline{\mathcal{D}_{\bm{k}}}$。

只在某一个方向上处于边界位置的网格集合记为
\begin{align}
  \begin{aligned}
  \underline{\partial \mathcal{D}_{\bm{k}}}
  &=\big\{(k_x,k_y,k_z)\big|k_w=0,K_w-1;k_v = 0,1,\cdots,K_v-1 ; v\in\{x,y,z\}-\{w\} ; w=x,y,z\big\}\\
  &=\big\{(k_x,k_y,k_z)\big|k_x=0,K_x-1;k_w = 0,1,\cdots,K_w-1 ; w=y,z\big\}
  \bigcup\\
  &\hspace{1cm}
  \big\{(k_x,k_y,k_z)\big|k_y=0,K_y-1;k_w = 0,1,\cdots,K_w-1 ; w=x,z\big\}
  \bigcup\\
  &\hspace{1cm}
  \big\{(k_x,k_y,k_z)\big|k_z=0,K_z-1;k_w = 0,1,\cdots,K_w-1 ; w=x,y\big\}
  \end{aligned}
\end{align}

$\underline{\partial \mathcal{D}_{\bm{k}}}$是实际参与计算的网格。
在$\underline{\partial \mathcal{D}_{\bm{k}}}$上定义边界网格的离散外法向量$\bm{n}_{\bm{k}}$,
对于某边界网格$\bm{k}=(k_x,k_y,k_z)$,若$k_w=0 \quad (w\in\{x,y,z\})$,则$\bm{n}_{\bm{k}}=-\hat{\bm{w}}$;
若$k_w=K_w-1$,则$\bm{n}_{\bm{k}}=\hat{\bm{w}}$,
其中$\hat{\bm{x}}=(1,0,0) \quad \hat{\bm{y}}=(0,1,0) \quad \hat{\bm{z}}=(0,0,1)$。

处于棱、角点上的边界网格集合记为
$\partial^2 \mathcal{D}_{\bm{k}} = \partial \mathcal{D}_{\bm{k}} - \underline{\partial \mathcal{D}_{\bm{k}}}$,
这部分边界网格不参与计算。

设$\vb{\cdot}$表示集合的元素个数,则有
\footnote{其中$\sum_{ \substack{w<v \\ w,v=x,y,z} } K_wK_v = K_xK_y+K_xK_z+K_yK_z$}
\begin{align}
  \begin{aligned}
    \vb[b]{\mathcal{D}_{\bm{k}}} &= \prod_{w=x,y,z}K_w\\
    \vb[b]{\underline{\mathcal{D}_{\bm{k}}}} &= \prod_{w=x,y,z}(K_w-2)
      = \prod_{w=x,y,z}K_w 
       -2\sum_{ \substack{w<v \\ w,v=x,y,z} } K_wK_v
       +4\sum_{w=x,y,z}K_w
       -8\\
    \vb[b]{\partial \mathcal{D}_{\bm{k}}} 
      &= \vb[b]{\mathcal{D}_{\bm{k}}} - \vb[b]{\underline{\mathcal{D}_{\bm{k}}}}
      = 2\sum_{ \substack{w<v \\ w,v=x,y,z} } K_wK_v
        -4\sum_{w=x,y,z}K_w
        +8  \\
    \vb[b]{\underline{\partial \mathcal{D}_{\bm{k}}}}
      &= 2\sum_{ \substack{w<v \\ w,v=x,y,z} } K_wK_v
        -8\sum_{w=x,y,z}K_w
        +24\\
    \vb[b]{\partial^2 \mathcal{D}_{\bm{k}}}
      &= \vb[b]{\partial \mathcal{D}_{\bm{k}}} - \vb[b]{\underline{\partial \mathcal{D}_{\bm{k}}}}
      =4\sum_{w=x,y,z}K_w - 16
  \end{aligned}
\end{align}


\subsection{不含外边界网格}

本文只考虑三维$xyz$坐标系,问题几何为立方体,
划分为$K_x\times K_y \times K_z$的结构网格,
则网格集合为$\mathcal{D}_{\bm{k}}=\big\{(k_x,k_y,k_z)\big|k_w = 0,1,\cdots,K_w-1 ; w=x,y,z\big\}$。
处于边界位置的网格集合记为
\begin{align}
  \begin{aligned}
  \underline{\partial \mathcal{D}_{\bm{k}}}
  &=\big\{(k_x,k_y,k_z)\big|k_x=0,K_x-1;k_w = 0,1,\cdots,K_w-1 ; w=y,z\big\}
  \bigcup\\
  &\hspace{1cm}
  \big\{(k_x,k_y,k_z)\big|k_y=0,K_y-1;k_w = 0,1,\cdots,K_w-1 ; w=x,z\big\}
  \bigcup\\
  &\hspace{1cm}
  \big\{(k_x,k_y,k_z)\big|k_z=0,K_z-1;k_w = 0,1,\cdots,K_w-1 ; w=x,y\big\}
  \end{aligned}
\end{align}

\TODO 在$\underline{\partial \mathcal{D}_{\bm{k}}}$上定义边界网格的离散外法向量$\bm{n}_{\bm{k}}$,
对于某边界网格$\bm{k}=(k_x,k_y,k_z)$,若$k_w=0 \quad (w\in\{x,y,z\})$,则$\bm{n}_{\bm{k}}=-\hat{\bm{w}}$;
若$k_w=K_w-1$,则$\bm{n}_{\bm{k}}=\hat{\bm{w}}$,
其中$\hat{\bm{x}}=(1,0,0) \quad \hat{\bm{y}}=(0,1,0) \quad \hat{\bm{z}}=(0,0,1)$。

设$\vb{\cdot}$表示集合的元素个数,则有
\footnote{其中$\sum_{ \substack{w<v \\ w,v=x,y,z} } K_wK_v = K_xK_y+K_xK_z+K_yK_z$}
\begin{align}
  \begin{aligned}
    \vb[b]{\mathcal{D}_{\bm{k}}} &= \prod_{w=x,y,z}K_w\\
    \vb[b]{\partial \mathcal{D}_{\bm{k}}} 
      &= \prod_{w=x,y,z}K_w - \prod_{w=x,y,z}\pb{K_w-2} \\
      &= 2\sum_{ \substack{w<v \\ w,v=x,y,z} } K_wK_v
        -4\sum_{w=x,y,z}K_w
        +8
  \end{aligned}
\end{align}




\section{直接解法}

\subsection{时间离散}


将\aeqref{equ:pro.diff.equ}对时间$t$进行离散。

\subsubsection{全隐式向后差分}

采用隐式向后差分格式
\begin{align}
  \newcommand{\para}[1][n]{\pb{\bm{x}}^{(#1)}}
  \left\{
  \begin{aligned}
    \frac{1}{v_g}\frac{\phi_g\para - \phi_g\para[n-1]}{\Delta t}
    &=\nabla\cdot D_g\para \nabla\phi_g\para 
      -\Sigma_{t,g}\para \phi_g\para \\
    & \hspace{1cm}
      +\sum_{g'=1}^G\pb[B]{\chi_g\para \pb{1-\beta}\nu\Sigma_{f,g'}\para
                           +\Sigma_{g'\rightarrow g}\para}\phi_{g'}\para \\
    &\hspace{1cm}
      +\sum_{i=1}^I \chi_{i,g}\para \lambda_i C_i\para \\
    \frac{C_i\para - C_i\para[n-1]}{\Delta t}
     &=\beta_i \sum_{g'=1}^G \nu\Sigma_{f,g'}\para \phi_{g'}\para
        -\lambda_i C_i\para
  \end{aligned}
  \right.
  \label{equ:pro.diff.dt.equ0}
\end{align}

解出$C_i\pb{\bm{x}}^{(n)}$可得
\begin{align}
  \newcommand{\para}[1][n]{\pb{\bm{x}}^{(#1)}}
  C_i\para = \frac{1}{1+\lambda_i\Delta t}
    \pb[B]{C_i\para[n-1]
    + \beta_i \Delta t \sum_{g'=1}^G \nu\Sigma_{f,g'}\para \phi_{g'}\para}
  \label{equ:pro.diff.dt.c}
\end{align}

代回\aeqref{equ:pro.diff.dt.equ0}得
\begin{align}
  \newcommand{\para}[1][n]{\pb{\bm{x}}^{(#1)}}
  \begin{aligned}
    &\quad \frac{1}{v_g}\frac{\phi_g\para - \phi_g\para[n-1]}{\Delta t} \\
    &=\nabla\cdot D_g\para \nabla\phi_g\para 
      -\Sigma_{t,g}\para \phi_g\para \\
    & \hspace{1cm}
      +\sum_{g'=1}^G\pb[B]{\chi_g\para \pb{1-\beta}\nu\Sigma_{f,g'}\para
                           +\Sigma_{g'\rightarrow g}\para}\phi_{g'}\para \\
    &\hspace{1cm}
      +\sum_{i=1}^I \frac{\chi_{i,g}\para \lambda_i}{1+\lambda_i\Delta t}
          \pb[B]{C_i\para[n-1] 
      + \beta_i \Delta t \sum_{g'=1}^G \nu\Sigma_{f,g'}\para \phi_{g'}\para}
  \end{aligned}
\end{align}

取$\chi_{i,g}=\chi_g$得
\begin{align}
  \newcommand{\para}[1][n]{\pb{\bm{x}}^{(#1)}}
  \begin{aligned}
    &\quad \frac{1}{v_g}\frac{\phi_g\para - \phi_g\para[n-1]}{\Delta t} \\
    &=\nabla\cdot D_g\para \nabla\phi_g\para 
      -\Sigma_{t,g}\para \phi_g\para 
      +\sum_{i=1}^I \frac{\chi_g\para \lambda_i}{1+\lambda_i\Delta t} C_i\para[n-1]\\
    & \hspace{1cm}
      +\sum_{g'=1}^G\pb[Bg]{\chi_g\para
        \pb[bg]{1-\beta 
          + \sum_{i=1}^I \frac{\lambda_i \beta_i \Delta t }{1+\lambda_i\Delta t}}
      \nu\Sigma_{f,g'}\para \\
    &\hspace{8cm}
         +\Sigma_{g'\rightarrow g}\para}\phi_{g'}\para
  \end{aligned}
\end{align}

记
\begin{align}
  \newcommand{\para}[1][n]{(\bm{x})^{(#1)}}
  S_C\para = \sum_{i=1}^I \frac{\lambda_i}{1+\lambda_i\Delta t} C_i\para[n-1]
  \titlelabel{equ:pro.diff.dt.sc}{离散扩散时空动力学中$S_C(\bm{x})^{(n)}$定义式} \\
  %
  B = 1-\beta + \sum_{i=1}^I \frac{\lambda_i \beta_i \Delta t }{1+\lambda_i\Delta t}
  \titlelabel{equ:pro.diff.dt.B}{离散扩散时空动力学中$B(\bm{x})^{(n)}$定义式}
\end{align}

则有
\begin{align}
  \newcommand{\para}[1][n]{\pb{\bm{x}}^{(#1)}}
  \begin{aligned}
    &\quad \frac{1}{v_g}\frac{\phi_g\para - \phi_g\para[n-1]}{\Delta t} \\
    &=\nabla\cdot D_g\para \nabla\phi_g\para 
      -\Sigma_{t,g}\para \phi_g\para + \chi_g\para S_C\para\\
    & \hspace{1cm}
      +\sum_{g'=1}^G\pb[B]{\chi_g\para
        B \nu\Sigma_{f,g'}\para
         +\Sigma_{g'\rightarrow g}\para}\phi_{g'}\para
  \end{aligned}
  \titlelabel{equ:pro.diff.dt.equ1}{时间$t$隐式向后差分离散后的扩散时空动力学通量$\phi$方程}
\end{align}


\subsubsection{裂变源显式}

采用隐式向后差分格式,但裂变源使用显式向前差分格式。
\begin{align}
  \newcommand{\para}[1][n]{\pb{\bm{x}}^{(#1)}}
  \left\{
  \begin{aligned}
    \frac{1}{v_g}\frac{\phi_g\para - \phi_g\para[n-1]}{\Delta t}
    &=\nabla\cdot D_g\para \nabla\phi_g\para 
      -\Sigma_{t,g}\para \phi_g\para \\
    & \hspace{1cm}
      +\sum_{g'=1}^G\pb[B]{\chi_g\para \pb{1-\beta}
                                \nu\Sigma_{f,g'}\para[n-1] \phi_{g'}\para[n-1]\\
    &\hspace{3cm}
                           +\Sigma_{g'\rightarrow g}\para \phi_{g'}\para} \\
    &\hspace{1cm}
      +\sum_{i=1}^I \chi_{i,g}\para \lambda_i C_i\para \\
    \frac{C_i\para - C_i\para[n-1]}{\Delta t}
     &=\beta_i \sum_{g'=1}^G \nu\Sigma_{f,g'}\para[n-1] \phi_{g'}\para[n-1]
        -\lambda_i C_i\para
  \end{aligned}
  \right.
  \label{equ:pro.diff.dt(f).equ0}
\end{align}

解出$C_i\pb{\bm{x}}^{(n)}$可得
\begin{align}
  \newcommand{\para}[1][n]{\pb{\bm{x}}^{(#1)}}
  C_i\para = \frac{1}{1+\lambda_i\Delta t}
    \pb[B]{C_i\para[n-1]
    + \beta_i \Delta t \sum_{g'=1}^G \nu\Sigma_{f,g'}\para[n-1] \phi_{g'}\para[n-1] }
  \label{equ:pro.diff.dt(f).c}
\end{align}


代回\aeqref{equ:pro.diff.dt(f).equ0},取$\chi_{i,g}=\chi_g$得
\begin{align}
  \newcommand{\para}[1][n]{\pb{\bm{x}}^{(#1)}}
  \begin{aligned}
    &\quad \frac{1}{v_g}\frac{\phi_g\para - \phi_g\para[n-1]}{\Delta t} \\
    &=\nabla\cdot D_g\para \nabla\phi_g\para 
      -\Sigma_{t,g}\para \phi_g\para 
      +\sum_{i=1}^I \frac{\chi_g\para \lambda_i}{1+\lambda_i\Delta t} C_i\para[n-1]\\
    & \hspace{1cm}
      +\sum_{g'=1}^G\pb[Bg]{\chi_g\para
        \pb[bg]{1-\beta 
          + \sum_{i=1}^I \frac{\lambda_i \beta_i \Delta t }{1+\lambda_i\Delta t}}
      \nu\Sigma_{f,g'}\para[n-1] \phi_{g'}\para[n-1]\\
    &\hspace{8cm}
         +\Sigma_{g'\rightarrow g}\para \phi_{g'}\para}
  \end{aligned}
\end{align}

代入\aeqref{equ:pro.diff.dt.sc}及\aeqref{equ:pro.diff.dt.B},并令
\begin{align}
  \newcommand{\para}[1][n]{\pb{\bm{x}}^{(#1)}}
  S_F\para = \sum_{g'=1}^G B \nu\Sigma_{f,g'}\para[n-1] \phi_{g'}\para[n-1]
\end{align}

可得
\begin{align}
  \newcommand{\para}[1][n]{\pb{\bm{x}}^{(#1)}}
  \begin{aligned}
    \frac{1}{v_g}\frac{\phi_g\para - \phi_g\para[n-1]}{\Delta t}
    &=\nabla\cdot D_g\para \nabla\phi_g\para 
      -\Sigma_{t,g}\para \phi_g\para  \\
    & \hspace{1cm}
       + \chi_g\para \pb[b]{S_C\para + S_F\para}\\
    & \hspace{1cm}
      +\sum_{g'=1}^G \Sigma_{g'\rightarrow g}\para \phi_{g'}\para
  \end{aligned}
  \titlelabel{equ:pro.diff.dt(f).equ1}{时间$t$裂变源显式离散后的扩散时空动力学通量$\phi$方程}
\end{align}



\subsection{空间离散}


在$xyz$坐标系中有一般离散关系
\begin{align}
  \phi(\bm{x}) &\rightarrow \phi_{\bm{k}}\\
  \Sigma(\bm{x}) &\rightarrow \Sigma_{\bm{k}}\\
  C_i(\bm{x}) &\rightarrow C_{i,\bm{k}}
\end{align}

其中$\bm{k}=(k_x,k_y,k_z)$为$xyz$空间离散后的网格坐标,
为方便起见这里暂时省略能群$g$,时间步长$n$等下标上标,下同。

离散的主要问题是微分项$\nabla\cdot D(\bm{x})\nabla\phi(\bm{x})$
和边界条件\aeqref{equ:pro.diff.boundary.equ}的离散方式。

首先考虑微分项,设$\nabla\cdot D(\bm{x})\nabla\phi(\bm{x})$对应的离散项为
$\pb[b]{\nabla\cdot D\nabla\phi}_{\bm{k}}$,
在$xyz$坐标系中,可取
\begin{align}
  \begin{aligned}
  \pb[b]{\nabla\cdot D\nabla\phi}_{\bm{k}}
    &=\sum_{w=x,y,z} \Sb[bg]{
      \frac{2D_{\bm{k}}D_{\bm{k}+\hat{\bm{w}}}\pb{\phi_{\bm{k}+\hat{\bm{w}}} - \phi_{\bm{k}}}}
           {\Delta w_{\bm{k}}\pb{D_{\bm{k}}\Delta w_{\bm{k}+\hat{\bm{w}}}+D_{\bm{k}+\hat{\bm{w}}}\Delta w_{\bm{k}}}}
           \\
    &\hspace{4cm} -\frac{2D_{\bm{k}}D_{\bm{k}-\hat{\bm{w}}}\pb{\phi_{\bm{k}} - \phi_{\bm{k}-\hat{\bm{w}}}}}
           {\Delta w_{\bm{k}}\pb{D_{\bm{k}}\Delta w_{\bm{k}-\hat{\bm{w}}}+D_{\bm{k}-\hat{\bm{w}}}\Delta w_{\bm{k}}}}
     }
  \end{aligned}
  \label{equ:dnabla2.equ0}
\end{align}

下面考虑边界条件\aeqref{equ:pro.diff.boundary.equ}的离散方式,
为方便起见本文只考虑方形问题区域,即问题区域的边界$\partial \mathcal{D}$仅包括与$xyz$坐标轴垂直的平面。

则边界条件\aeqref{equ:pro.diff.boundary.equ}可离散为\TODO
\begin{align}
  \frac{\phi_{\bm{k}}-\phi_{\bm{k}-\bm{n}_{\bm{k}}}}{|\Delta_{\bm{k}}\cdot \bm{n}_{\bm{k}}|}
   = -\frac{\phi_{\bm{k}}}{\delta_{\bm{k}}}
  \qquad \bm{k} \in \underline{\partial \mathcal{D}_{\bm{k}}}
\end{align}

其中
\begin{align}
  \Delta_{\bm{k}} = (\Delta x, \Delta y, \Delta z)
\end{align}

\subsubsection{初始通量方程}

\aeqref{equ:pro.diff.init.diff.equ1}的离散形式为
\begin{align}
  \pb[b]{\nabla\cdot D_g^{(0)} \nabla\phi_g^{(0)}}_{\bm{k}}
   -\Sigma_{t,g,\bm{k}}^{(0)} \phi_{g,\bm{k}}^{(0)}
   +\sum_{g'=1}^G\pb[B]{\frac{1}{k_\mathrm{eff}^{(0)}}\chi_{g,\bm{k}}^{(0)} \nu\Sigma_{f,g',\bm{k}}^{(0)}
                        +\Sigma_{g'\rightarrow g,\bm{k}}^{(0)}}\phi_{g',\bm{k}}^{(0)} =0
  \quad \bm{k} \in \underline{\mathcal{D}_{\bm{k}}}
\end{align}

\aeqref{equ:pro.diff.init.c}的离散形式为
\begin{align}
  C_{i,\bm{k}}^{(0)} = \frac{\beta_i}{\lambda_i}
    \sum_{g'=1}^G \nu\Sigma_{f,g',\bm{k}}^{(0)}\phi_{g',\bm{k}}^{(0)}
  \qquad \bm{k} \in \underline{\mathcal{D}_{\bm{k}}}
\end{align}


\subsubsection{全隐式向后差分}


则\aeqref{equ:pro.diff.dt.equ1}的离散形式为
\begin{align}
  \begin{aligned}
    \frac{1}{v_g}\frac{\phi_{g,\bm{k}}^{(n)} - \phi_{g,\bm{k}}^{(n-1)}}{\Delta t} 
    &=\pb[b]{\nabla\cdot D_{g}^{(n)} \nabla\phi_{g}^{(n)}}_{\bm{k}}
      -\Sigma_{t,g,\bm{k}}^{(n)} \phi_{g,\bm{k}}^{(n)} + \chi_{g,\bm{k}}^{(n)} S_{C,\bm{k}}^{(n)}\\
    & \hspace{1cm}
      +\sum_{g'=1}^G\pb[B]{\chi_{g,\bm{k}}^{(n)}
        B \nu\Sigma_{f,g',\bm{k}}^{(n)}
         +\Sigma_{g'\rightarrow g,\bm{k}}^{(n)}}\phi_{g',\bm{k}}^{(n)}
  \end{aligned}
  \qquad \bm{k} \in \underline{\mathcal{D}_{\bm{k}}}
  \label{equ:pro.diff.dt.dx.equ1}
\end{align}

其中
\begin{align}
  S_{C,\bm{k}}^{(n)} &= \sum_{i=1}^I \frac{\lambda_i}{1+\lambda_i\Delta t} C_{i,\bm{k}}^{(n-1)}
  \qquad \bm{k} \in \underline{\mathcal{D}_{\bm{k}}}
  \titlelabel{equ:pro.diff.dt.dx.sc}{离散扩散时空动力学中$S_{C,\bm{k}}^{(n)}$定义式}
\end{align}

\aeqref{equ:pro.diff.dt.c}的离散形式为
\begin{align}
  C_{i,\bm{k}}^{(n)} = \frac{1}{1+\lambda_i\Delta t}
    \pb[B]{C_{i,\bm{k}}^{(n-1)}
    + \beta_i \Delta t \sum_{g'=1}^G \nu\Sigma_{f,g',\bm{k}}^{(n)} \phi_{g',\bm{k}}^{(n)}}
  \qquad \bm{k} \in \underline{\mathcal{D}_{\bm{k}}}
\end{align}




\subsubsection{裂变源显式差分}


则\ateqref{equ:pro.diff.dt(f).equ1}的离散形式为
\begin{align}
  \begin{aligned}
    \frac{1}{v_g}\frac{\phi_{g,\bm{k}}^{(n)} - \phi_{g,\bm{k}}^{(n-1)}}{\Delta t} 
    &=\pb[b]{\nabla\cdot D_{g}^{(n)} \nabla\phi_{g}^{(n)}}_{\bm{k}}
      -\Sigma_{t,g,\bm{k}}^{(n)} \phi_{g,\bm{k}}^{(n)} 
      + \chi_{g,\bm{k}}^{(n)} \pb[b]{S_{C,\bm{k}}^{(n)} + S_{F,\bm{k}}^{(n)}}\\
    & \hspace{1cm}
      +\sum_{g'=1}^G \Sigma_{g'\rightarrow g,\bm{k}}^{(n)}\phi_{g',\bm{k}}^{(n)}
  \end{aligned}
  \qquad \bm{k} \in \underline{\mathcal{D}_{\bm{k}}}
  \label{equ:pro.diff.dt.dx.equ1}
\end{align}

其中$S_{C,\bm{k}}^{(n)}$定义同\aeqref{equ:pro.diff.dt.dx.sc},且
\begin{align}
  S_{F,\bm{k}}^{(n)} = \sum_{g'=1}^G B \nu\Sigma_{f,g',\bm{k}}^{(n-1)} \phi_{g',\bm{k}}^{(n-1)}
  \qquad \bm{k} \in \underline{\mathcal{D}_{\bm{k}}}
  \titlelabel{equ:pro.diff.dt.dx.sf}{离散扩散时空动力学中$S_{F,\bm{k}}^{(n)}$定义式}
\end{align}

\ateqref{equ:pro.diff.dt(f).c}的离散形式为
\begin{align}
  C_{i,\bm{k}}^{(n)} = \frac{1}{1+\lambda_i\Delta t}
    \pb[B]{C_{i,\bm{k}}^{(n-1)}
    + \beta_i \Delta t \sum_{g'=1}^G \nu\Sigma_{f,g',\bm{k}}^{(n-1)} \phi_{g',\bm{k}}^{(n-1)}}
  \qquad \bm{k} \in \underline{\mathcal{D}_{\bm{k}}}
\end{align}



\section{准静态}

考虑到反应堆物理计算中,一般总体通量幅度变化较快,
而相对功率分布即形状函数变化较慢,所以可以将中子通量分布分解为两部分
\begin{align}
  \phi\pb{\bm{x},E,t}=n(t) \psi\pb{\bm{x},E,t}
\end{align}
其中$n(t)$是幅度因子,$\psi\pb{\bm{x},E,t}$是形状因子。

将上式代入\ateqref{equ:pro.diff.equ},
并在方程两边作用算符$\int_D \diff{\bm{x}} \int \diff{E} \ w(\bm{r},E) $,
其中
\begin{align}
  \newcommand{\para}{\pb{\bm{x},t}}
  \begin{split}
    &\int_D \diff{\bm{x}} \sum_{g=1}^G
        w_g(\bm{x}) \frac{1}{v_g}\frac{\partial \phi_g\para}{\partial t} \\
    & \hspace{2cm}
      =\frac{\partial n(t)}{\partial t}\int_D \diff{\bm{x}} \sum_{g=1}^G \frac{w_g(\bm{x})}{v_g} \psi_g\para
      + n(t)\int_D \diff{\bm{x}} \sum_{g=1}^G \frac{w_g(\bm{x})}{v_g} \frac{\partial \psi_g\para}{\partial t}
  \end{split}
\end{align}
最后有
\begin{align}
  \newcommand{\para}{\pb{\bm{x},t}}
  \left\{
  \begin{aligned}
    \begin{split}
    &\frac{\partial n(t)}{n(t)\partial t}\int_D \diff{\bm{x}} \sum_{g=1}^G \frac{w_g(\bm{x})}{v_g} \psi_g\para
      + \int_D \diff{\bm{x}} \sum_{g=1}^G \frac{w_g(\bm{x})}{v_g} \frac{\partial \psi_g\para}{\partial t} \\
     & \hspace{1.5cm}
     =\int_D \diff{\bm{x}} \sum_{g=1}^G w_g(\bm{x})\pb[bg]{
       \sum_{i=1}^I \chi_{i,g}\para \lambda_i C_i\para \\
     &\hspace{2cm}
       +\nabla\cdot D_g\para \nabla\psi_g\para 
        -\Sigma_{t,g}\para \psi_g\para\\
     & \hspace{3cm}
       +\sum_{g'=1}^G\pb[B]{\chi_g\para \pb{1-\beta}\nu\Sigma_{f,g'}\para
                   +\Sigma_{g'\rightarrow g}\para}\psi_{g'}\para
       }
   \end{split}\\
   &\int_D \diff{\bm{x}} \sum_{g=1}^G w_g(\bm{x})\frac{\partial C_i\para}{\partial t}\\
   & \hspace{2cm}
    =\int_D \diff{\bm{x}} \sum_{g=1}^G w_g(\bm{x}) \pb[bg]{
      \beta_i \sum_{g'=1}^G \nu\Sigma_{f,g'}\para n(t)\psi_{g'}\para
       -\lambda_i C_i\para}
  \end{aligned}
  \right.
\end{align}
\TODO

\section{能群耦合\TODO}

本文不考虑向上散射,向下散射只散射到邻近的能群,则以上离散后的扩散方程可以写成如下形式
\begin{align}
  \begin{pmatrix}
  A_{11} & D_{12} & \cdots & D_{1G}\\
  D_{21} & A_{22} & &\\
   & \ddots & \ddots &\\
   & & D_{G-1,G} & A_{GG}
  \end{pmatrix}
  \begin{pmatrix}
  \phi_1 \\ \phi_2 \\ \vdots \\ \phi_G
  \end{pmatrix}
  =
  \begin{pmatrix}
  S_1 \\ S_2 \\ \vdots \\ S_G
  \end{pmatrix}
\end{align}

其中 $A_{gg}$是7对角对称阵,$D_{g_1g_2}$ 是对角阵。

两群情况则简化为
\begin{align}
  \begin{pmatrix}
  A_{11} & D_{12} \\
  D_{21} & A_{22}
  \end{pmatrix}
  \begin{pmatrix}
  \phi_1 \\ \phi_2
  \end{pmatrix}
  =
  \begin{pmatrix}
  S_1 \\ S_2
  \end{pmatrix}
\end{align}

\TODO

\section{多重网格法\TODO}

\onlinecite{trottenberg2001multigrid}
\onlinecite{briggs2000multigrid}
