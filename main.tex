%%% Local Variables:
%%% mode: latex
%%% TeX-master: t
%%% End:

\documentclass[master]{thuthesis}
% \documentclass[%
%   bachelor|master|doctor|postdoctor, % mandatory option
%   winfonts|nofonts|adobefonts, % mandatory only for bachelor and Linuxer
%   secret,
%   openany|openright,
%   arialtoc,arialtitle]{thuthesis}
% 当使用 XeLaTeX 编译时,本科生、Linux 用户需要加上 nofonts 选项;
% 当使用 PDFLaTeX 编译时,adobefonts 选项等效于 winfonts 选项(缺省选项)。

% 所有其它可能用到的包都统一放到这里了,可以根据自己的实际添加或者删除。
\usepackage[
addfootnotetoref
]{thutils}

\usepackage[xetex,hyperref]{xcolor}

\usepackage{tikz}
\usetikzlibrary{arrows.new}
\usetikzlibrary{decorations.pathreplacing}
\usetikzlibrary{datavisualization}


\usepackage[linesnumbered,boxed,algochapter,vlined]{algorithm2e}
\renewcommand\algorithmautorefname{算法}
\renewcommand{\algorithmcfname}{算法}
\makeatletter
\def\setlabelname#1{\def\@currentlabelname{#1}}
\makeatother

\usepackage{url}
\def\UrlBreaks{\do\/\do\.\do\-\do\#}
%\def\UrlBreaks{\do\A\do\B\do\C\do\D\do\E\do\F\do\G\do\H\do\I\do\J%
%\do\K\do\L\do\M\do\N\do\O\do\P\do\Q\do\R\do\S\do\T\do\U\do\V%
%\do\W\do\X\do\Y\do\Z\do\[\do\\\do\]\do\^\do\_\do\`\do\a\do\b%
%\do\c\do\d\do\e\do\f\do\g\do\h\do\i\do\j\do\k\do\l\do\m\do\n%
%\do\o\do\p\do\q\do\r\do\s\do\t\do\u\do\v\do\w\do\x\do\y\do\z%
%\do\.\do\@\do\\\do\/\do\!\do\_\do\|\do\;\do\>\do\]\do\)\do\,%
%\do\?\do\'\do+\do\=\do\#}

\usepackage[figuresright]{rotating}
\def\pdfrotate{\special{pdf: put @thispage <</Rotate 90>>}}

\usepackage{multirow}

\usepackage{answers}
\newenvironment{datasheetenv}[1]{}{}
\Newassociation{datasheet}{datasheetenv}{datasheetfile}

\usepackage{placeins}

\usepackage{multicol}

\usepackage{textcomp} %千分号

\newcommand{\TODO}{ \textcolor{blue}{TODO} }
\colorlet{BLUE}{blue}

%--------------------------------------------------------------------------
% 自定义函数

%bracket系列
\newcommand{\bracket}[4]
{\ensuremath{%
\ifthenelse{\equal{#1}{n}}{#3 #2 #4}{}%
\ifthenelse{\equal{#1}{b}}{\bigl#3 #2 \bigr#4}{}%
\ifthenelse{\equal{#1}{B}}{\Bigl#3 #2 \Bigr#4}{}%
\ifthenelse{\equal{#1}{bg}}{\biggl#3 #2 \biggr#4}{}%
\ifthenelse{\equal{#1}{Bg}}{\Biggl#3 #2 \Biggr#4}{}%
}}

\newcommand{\pbracket}[2]{\bracket{#1}{#2}{(}{)}}
\newcommand{\Sbracket}[2]{\bracket{#1}{#2}{[}{]}}
\newcommand{\Bbracket}[2]{\bracket{#1}{#2}{\lbrace}{\rbrace}}
\newcommand{\vbracket}[2]{\bracket{#1}{#2}{|}{|}}

\newcommand{\getsize}[2]
{%
\ifthenelse{\equal{#1}{n}}{#2}{}%
\ifthenelse{\equal{#1}{b}}{\big#2}{}%
\ifthenelse{\equal{#1}{B}}{\Big#2}{}%
\ifthenelse{\equal{#1}{bg}}{\bigg#2}{}%
\ifthenelse{\equal{#1}{Bg}}{\Bigg#2}{}%
}

\newcommand{\pb}[2][n]{\pbracket{#1}{#2}}
\newcommand{\Sb}[2][n]{\Sbracket{#1}{#2}}
\newcommand{\Bb}[2][n]{\Bbracket{#1}{#2}}
\newcommand{\vb}[2][n]{\vbracket{#1}{#2}}

\newcommand{\diff}[1]{\mathrm{d}#1}

\usepackage{datetime}
\newdateformat{mydate}{\THEYEAR-\twodigit{\THEMONTH}-\twodigit{\THEDAY}}
\newtimeformat{mytime}{\twodigit{\THEHOUR}:\twodigit{\THEMINUTE}}
\settimeformat{mytime}

\usepackage[draft=true,allpages=true]{draftmark}
\draftmarksetup{angle=0,grayness=0.5,xcoord=50,ycoord=-137,fontsize=12pt,
mark={DRAFT \mydate\today\hspace{5pt} \currenttime}}

%只显示三层目录
\setcounter{tocdepth}{2}

\begin{document}

% 定义所有的eps文件在 figures 子目录下
\graphicspath{{figures/}}


%%% 封面部分
\frontmatter

%%% Local Variables:
%%% mode: latex
%%% TeX-master: t
%%% End:

\ctitle{反应堆物理数值计算GPU加速方法研究}

\cdegree{工学硕士}


\cdepartment[工物]{工程物理系}
\cmajor{核能与核技术}
\cauthor{孙嘉龙} 
\csupervisor{余纲林助研}
% 日期自动生成,如果你要自己写就改这个cdate
%\cdate{\CJKdigits{\the\year}年\CJKnumber{\the\month}月}

\etitle{\TODO} 
% 这块比较复杂,需要分情况讨论:
% 1. 学术型硕士
%    \edegree:必须为Master of Arts或Master of Science(注意大小写)
%              “哲学、文学、历史学、法学、教育学、艺术学门类,公共管理学科
%               填写Master of Arts,其它填写Master of Science”
%    \emajor:“获得一级学科授权的学科填写一级学科名称,其它填写二级学科名称”
% 2. 专业型硕士
%    \edegree:“填写专业学位英文名称全称”
%    \emajor:“工程硕士填写工程领域,其它专业学位不填写此项”
% 3. 学术型博士
%    \edegree:Doctor of Philosophy(注意大小写)
%    \emajor:“获得一级学科授权的学科填写一级学科名称,其它填写二级学科名称”
% 4. 专业型博士
%    \edegree:“填写专业学位英文名称全称”
%    \emajor:不填写此项
\edegree{Master of Science} 
\emajor{Nuclear Sicence and Technology} 
\eauthor{Sun Jialong} 
\esupervisor{Lecturer YU Ganglin} 
% \edate{December, 2005}

% 定义中英文摘要和关键字
\begin{cabstract}
\TODO
\end{cabstract}

\ckeywords{\TeX, \LaTeX, CJK, 模板, 论文}

\begin{eabstract} 
\TODO
\end{eabstract}

\ekeywords{\TeX, \LaTeX, CJK, template, thesis}

%\makecover

% 目录
\tableofcontents

% 符号对照表
%\begin{denotation}

\item[HDF] Hierarchical Data Format

\end{denotation}


\Opensolutionfile{datasheetfile}
%%% 正文部分
\mainmatter

\def\ProgramName{RDGS}
\def\ProgramFullName{Real Diffusion GPU Solver}

\chapter{引言}
\section{研究背景}
\section{国内外研究现状}
\section{研究内容和论文组织结构}




\chapter{GPU科学计算简介}

\section{CUDA简介}

\subsection{CUDA总体设计}

CUDA翻译为统一计算设备架构,本身定位做一种包含CPU和GPU的编程模型,
不过实际上一般只用作GPU编程和GPU、CPU通讯编程。

CUDA把设备资源分为主机端和GPU端两部分,
主机端包含CPU、内存等正常C/C++程序可以访问到的资源,
GPU端包含多个SM(Streaming Multiprocessor)
或SMX(Next Generation Streaming Multiprocessor)
\footnote{从 NVIDIA显卡的 Kepler 架构开始,SM的规格大幅改变,改称为SMX。
为方便起见以下不再提SM,提到SMX时也包含SM。}
、显存等资源。

其中SMX代表GPU核心内的一个相对独立的向量处理单元,
类似传统的向量机中央处理器,这些SMX位于GPU的核心芯片内。
显存位于显卡PCB上,并被所有的SMX共享。

显存和主机内存是独立的,有各自的地址空间,
CPU端的代码不能直接读写显存,
GPU端的代码也不能直接读写内存,
需要程序员手工在显存和内存之间做数据传输。
CUDA允许在主机端申请所谓的\emph{页锁定主机内存}(Page-Locked Host Memory),
并允许GPU端直接访问,
由显卡驱动负责在内存和显卡之间进行自动数据传输。
此外,对于通用计算专用的Tesla显卡,
CUDA可以开启Unified Virtual Address Space功能,
即对内存和显存统一编址访问,可以省去一些编程上的繁琐操作。
\cite{cudadoc-cprogrammingguide}

\subsubsection{Global函数}

GPU上执行的代码需要放在专门的Kernel函数中,
这些函数在CUDA使用\_\_global\_\_进行标识,所以又称global函数。
Global函数只能由CPU端的代码通过特殊方式调用。
在CUDA 5.0之前global函数间不能相互调用,
global函数只能调用一种有\_\_device\_\_标识的函数(以下称为device函数)。
CUDA 5.0引入了Dynamic Parallelism功能,
允许在global函数内调用global函数,并定义了对应的语义,
该功能依赖计算能力\footnote{NVIDIA对其发布的GPU核心的功能进行划分的标准,当前Kepler架构的计算能力为3.0-3.5。}%
为3.5的GPU核心(如GK110,对应的显卡有GTX Titian、Tesla K20等),
详见文献\onlinecite{cudadoc-dynamicparallelism}。

Global函数是GPU上运行的程序的最基本单元,虽然global函数可以调用device函数,
但被调用的device函数的各种运行时配置都是依赖于直接或间接调用它的global函数。

Global函数实际运行时可以被一组线程同步执行,类似传统的向量机,
同步执行的线程数量在调用global函数的时候进行设置。

CUDA将运行一个global函数的线程分Grid、Block两个层级进行组织:
Grid代表所有参与的线程,一个Grid包含一个或多个Block,每个Block在Grid内都有自己的编号,
CUDA提供的Block编号可以是1维、2维或3维整数;
一个Block又包含一个或多个Thread,每个Thread在Block内也有自己的编号,
CUDA提供的Thread编号同样是1维、2维或3维整数,
这里的Thread就是一个实际的硬件线程,有自己的寄存器组等资源。
一个实际的线程组设置见\floatref{fig:gpu.cuda.blocks}。

\begin{figure}
\centering
\begin{tikzpicture}[scale=0.75, transform shape]
\def\TextBox#1#2#3{
\draw  (#1,#2) rectangle (#1+2,#2+1);
\node at (#1+1, #2+0.5) {#3};
}

\foreach \x in {0,...,2}
  \foreach \y in {0,...,1}
  {
    \TextBox{\x*2.5+1}{-\y*1.5+3}{Block(\x,\y)}
  }
\draw  (0.5,5) rectangle (8.5,1);
\node at (1.5,4.5) {\large Grid};

\foreach \x in {0,...,3}
  \foreach \y in {0,...,2}
  {
    \TextBox{\x*2.5}{-\y*1.5-1.5}{\small Thread(\x,\y)}
  }
\draw  (-0.5,0.5)  rectangle (10,-5);
\node at (1,0) {\large Block(2,1)};

\draw [dashed]  (6,2.5) edge (-0.5,0.5);
\draw [dashed]  (8,2.5) edge (10,0.5);
\draw [dashed]  (6,1.5) edge (-0.5,-5);
\draw [dashed]  (8,1.5) edge (10,-5);
\end{tikzpicture}
\caption[线程组层次结构示意图]{\label{fig:gpu.cuda.blocks}线程组层次结构示意图
\cite{cudadoc-cprogrammingguide}}
\end{figure}

同一个global函数调用时所涉及的线程均使用global函数的参数作为输入,
CUDA提供threadIdx、blockIdx等变量在代码中区分各个线程。
线程以Block为单元分配给GPU上的各个SMX独立执行,
Block之间没有任何直接的同步方式,
程序员不需要知道也不应该猜测各Block是如何在各个SMX执行的。
需要说明的是:强行使用显存作为Block间的同步可能会导致各SMX死锁。
实际Block间同步的最主要方式就是等待该global函数执行完,
此时所有Block状态都是确定的,即已经执行完。

这样GPU或驱动就可以根据实际GPU核心上的SMX数量来具体配置各Block是如何在SMX上执行的,
使得当SM数量不超过Block数量时global函数有了一定的并行扩展性,
见\floatref{fig:gpu.cuda.scalability}。
由于每个Block是运行在一个实际SMX上的,
所以SMX的寄存器、共享存储空间等资源会对Block的大小(包含的Thread数量)有一定的限制,
而CUDA对Grid大小(包含的Block数量)的限制则很小。
NVIDIA这样做是为了通过强制程序员对计算任务进行分割的方式获得了一定的程序并行扩展性,
同时也简化了同一系列不同规格GPU的设计,即通过增减SMX的数量来控制GPU计算能力。

\begin{figure}
\centering
\begin{tikzpicture}[scale=0.6, transform shape]
\def\TextBox#1#2#3{
\draw  (#1,#2) rectangle (#1+2,#2+1);
\node at (#1+1, #2+0.5) {#3};
}

\foreach \x in {0,...,3}
  \foreach \y in {0,...,1}
  {
    \TextBox{\x*2.5+1}{-\y*1.5+3}{Block(\x,\y)}
  }
\draw  (0.5,5) rectangle (11,1);
\node at (3,4.5) {\large Kernel函数的Grid};

\draw [dashed] (-3,-4) -- (16.5,-4);

\draw  (-1,-3.5) rectangle (4.5,-1);
\node at (1.5,-1.5) {\large 2个SMX的GPU};
\draw [-latex new, arrow head=3mm] (7,1) -- (9.5,-1);
\foreach \x in {0,...,1}
{
  \TextBox{\x*2.5-0.5}{-3}{SMX \x}
}
\foreach \t in {0,...,3}
{
  \foreach \x in {0,...,1}
  {
    \TextBox{\x*2.5-0.5}{-\t*2-6}{Block(\t,\x)}
  }
  \draw  (-1,-\t*2-4.75) rectangle (4.5,-\t*2-6.25);
}

\draw  (5.5,-3.5) rectangle (16,-1);
\node at (8,-1.5) {\large 4个SMX的GPU};
\draw [-latex new, arrow head=3mm] (3.5,1) -- (2,-1);
\foreach \x in {0,...,3}
{
  \TextBox{\x*2.5+6}{-3}{SMX \x}
}
\foreach \t in {0,...,1}
{
  \foreach \x in {0,...,3}
  {
    \TextBox{\x*2.5+6}{-\t*2-6}{Block(\x,\t)}
  }
  \draw  (5.5,-\t*2-4.75) rectangle (16,-\t*2-6.25);
}

\draw [dashed,-latex new, arrow head=3mm] (-1.5,-4.5) -- (-1.5,-13);
\foreach \t in {0,...,3}
{
  \node at  (-2.5,-\t*2-5.5) {\Large t=\t};
}

\draw (5,-0.5) -- (5,-12.5);
\end{tikzpicture}
\caption[CUDA程序的扩展性示意图]{\label{fig:gpu.cuda.scalability}CUDA程序的扩展性示意图
\cite{cudadoc-cprogrammingguide}}
\end{figure}

总的来说,GPU核心相当于一组不能单独编程的、以显存作为共享存储器的、带有自动负载平衡的小型向量机组。


\subsection{CUDA的GPU设备模型}

\begin{figure}
\centering
\begin{tikzpicture}[scale=0.75, transform shape]
%15*SMX
\def\SMX#1#2{
\draw  (#1+0, #2) rectangle (#1+1, #2+1);
\node at (#1+0.5, #2+0.5) {\small SMX};
}
\def\len{1.5}
\foreach \x in {0,...,8}
{ \SMX{\x*\len}{0} }
\foreach \x in {0,...,7}
{ \SMX{\x*\len+0.75}{-3} }

%L2 Cache
\draw  (0,-0.5) rectangle (13,-1.5);
\node at (6.5,-1) {L2缓存};


\def\Memory#1#2{
\draw  (#1,#2) rectangle (#1+2,#2+1);
\node at (#1+1,#2+0.5) {\small 显存控制器};
}
\foreach \i in {0,1,2}
{ \Memory{-2.5}{-\i*1.5} }
\foreach \i in {0,1,2}
{ \Memory{13.5}{-\i*1.5} }

\draw  (-2.5,2) rectangle (15.5,1.5);
\node at (6.5,1.75) {线程调度器(GigaThread Engine)};

\draw  (-2.5,3) rectangle (15.5,2.5);
\node at (6.5,2.75) {PCI Express 3.0接口};

\draw  (-3,4) rectangle (16,-3.5);
\node at (-1.5,3.5) {\Large GK110};
\end{tikzpicture}
\caption[GK110结构示意图]{\label{fig:gpu.cuda.gk110}GK110结构示意图
\cite{cudadoc-KeplerGK110ArchitectureWhitepaper}}
\end{figure}

GPU是由多个SMX组成的,例如Kepler架构GK110核心
(见\floatref{fig:gpu.cuda.gk110})就包含15个SMX,
其中每个SMX(见\floatref{fig:gpu.cuda.smx})
含有192个CUDA Core单元(支持单精度、整数计算,图示中Core)、
64个双精度浮点计算单元(图示中DP Unit)、
32个特殊函数计算单元(图示中SFU)、
32个读取/存储单元(图示中LD/ST),
4个Wrap调度器,65536个32bit寄存器。
SMX以32个Thread为一组(称为wrap)来调度Block中的线程,
每个SMX有4个wrap调度器和8个指令分派单元,这使得SMX可以同时执行4个wrap,
每个周期中每个wrap最多可以有两条不相关的指令被同时分派。
\cite{cudadoc-KeplerGK110ArchitectureWhitepaper}

\begin{figure}
\centering
%\includegraphics[scale=0.6]{smx.png}
\begin{tikzpicture}[scale=0.7, transform shape]
\def\TextBox#1#2#3#4
{
  \draw (#1,#2) rectangle (#1+#3,#2+1);
  \node at (#1+#3/2, #2+0.5) {#4};
}

\def\CoreLine#1#2
{
\foreach \i in {0,...,2}
{ \TextBox{#1+\i*1.5+0}{#2+0}{1}{Core} }
\TextBox{#1+4.5}{#2+0}{2}{DP Unit}

\foreach \i in {0,...,2}
{ \TextBox{#1+\i*1.5+7}{#2+0}{1}{Core} }
\TextBox{#1+11.5}{#2+0}{2}{DP Unit}

\TextBox{#1+14}{#2+0}{1.5}{LD/ST}
\TextBox{#1+16}{#2+0}{1}{SFU}
}

\CoreLine{0}{0}

\CoreLine{0}{-4}

\draw [decorate, decoration={brace, amplitude=10pt}] (-0.5,-4.25) -- (-0.5,1.25);
\node [left]at (-1,-1.5) {\large 32组};
\draw [loosely dashed] (0.5,-0.5) -- (0.5,-2.5);
\draw [loosely dashed] (8,-0.5) -- (8,-2.5);
\draw [loosely dashed] (16.5,-0.5) -- (16.5,-2.5);

\draw  (-0.5,2.5) rectangle (17.5,1.5);
\node at (8,2) {\large 寄存器文件(65536个 32bit寄存器)};

\foreach \x in {0,...,3}
{
  \def\xlen{4.5}
  \draw  (\x*\xlen-0.25,4) rectangle (\x*\xlen-0.25+4,3);
  \node  at (\x*\xlen+1.75,3.5) {\large Wrap调度器}; 
}
\draw  (-0.5,4.5) rectangle (17.5,5.5);
\node at (8,5) {\large 指令缓存};

\draw  (17.5,-4.5) rectangle (-0.5,-5.5);
\draw  (17.5,-6) rectangle (-0.5,-7);
\draw  (17.5,-7.5) rectangle (-0.5,-8.5);
\node at (8,-5) {\large 内部互联网络};
\node at (8,-6.5) {\large 64KB 共享存储区 / L1 缓存};
\node at (8,-8) {\large 48KB 只读数据缓存};

\foreach \x in {0,...,7}
{
  \foreach \y in {0,1}
  {
    \draw  (\x*2.25-0.5,-\y*1.5-9) rectangle (\x*2.25+1.5,-\y*1.5-10);
    \node at (\x*2.25+0.5,-\y*1.5-9.5) {纹理单元};
  }
}

\draw  (-3,-12) rectangle (18,6.5);
\node [right] at (-2.5,6) {\Large SMX};
\end{tikzpicture}
\caption[SMX结构示意图]{\label{fig:gpu.cuda.smx}SMX结构示意图
\cite{cudadoc-KeplerGK110ArchitectureWhitepaper}}
\end{figure}

SMX的存储器结构较为复杂,通用存储器的结构见\floatref{fig:gpu.cuda.memory},
其中寄存器部分和传统程序一样,并不直接对程序员可见,由CUDA编译器进行分配,
L1缓存、L2缓存对程序员也不是直接可见的,会在访问显存时自动进行调度。

\begin{figure}
\centering
\begin{tikzpicture}[scale=0.8, transform shape]

\draw  (0,3) ellipse (1 and 0.5);
\node at (0,3) {Thread};

\draw  (-1.5,0.5) rectangle (1.5,1.5);
\node at (0,1) {L1缓存};

\draw  (-5,0.5) rectangle (-2,1.5);
\node at (-3.5,1) {共享存储区};

\draw  (2,0.5) rectangle (5,1.5);
\node at (3.5,1) {只读数据缓存};

\draw [latex new-latex new, arrow head=3mm] (0,2.5) -- (0,1.5);
\draw [latex new-latex new, arrow head=3mm] (-0.5,2.5) -- (-3.5,1.5);
\draw [latex new-, arrow head=3mm] (0.5,2.5) -- (3.5,1.5);

\draw  (-5,-1.5) rectangle (5,-0.5);
\node at (0,-1) {\large L2缓存};
\draw [latex new-latex new, arrow head=3mm](0,0.5) -- (0,-0.5);
\draw [latex new-, arrow head=3mm](3.5,0.5) -- (3.5,-0.5);

\draw  (-5,-3.5) rectangle (5,-2.5);
\node at (0,-3) {\large 显存};
\draw [latex new-latex new, arrow head=3mm](0,-1.5) -- (0,-2.5);

\draw [decorate, decoration={brace, amplitude=7pt}] (-5.5,0) -- (-5.5,4);
\node [left] at (-5.75,2) {\large SMX};

\draw [decorate, decoration={brace, amplitude=7pt}] (-7,-2) -- (-7,4);
\node [left] at (-7.5,1) {\large GK110};

\draw (-5,2.5) rectangle (-2,3.5);
\node at (-3.5,3) {寄存器};
\draw [latex new-latex new, arrow head=3mm] (-1,3) -- (-2,3);
\end{tikzpicture}
\caption[Kepler存储器结构示意图]{\label{fig:gpu.cuda.memory}Kepler存储器结构示意图
\cite{cudadoc-KeplerGK110ArchitectureWhitepaper}}
\end{figure}

Global函数和device函数可以通过指针直接访问共享存储区、显存。
共享存储区位于SMX内,和L1缓存共用64KB空间,
在global函数启动时可以对共享存储区/L1缓存的分配进行设置,
分配的方式有:16KB/48KB、32KB/32KB(限Kepler架构)、48KB/16KB三种。
\cite{cudadoc-KeplerGK110ArchitectureWhitepaper}

在SMX上运行的Block中的所有线程共享SMX全部的65536个32位寄存器,
一旦分配完成,每个Thread线程就只能使用自己所有的寄存器,其他的寄存器则不可见。
每个Thread需要使用的寄存器数量由CUDA编译器控制。

共享存储区则对同一个Block内所有的线程可见,是同一个Block内的Thread协作、通讯的主要方式,
每个Block使用的共享存储区的大小在global函数启动时进行配置。

只读数据缓存可以用来加速在global函数运行过程中保持不便的数据的读取,
程序员可以通过带有const  \_\_restrict\_\_标识的指针进行访问,
也可以通过纹理模式访问。在Fermi及之前的架构中,该部分缓存只能通过纹理方式使用。
\cite{cudadoc-KeplerGK110ArchitectureWhitepaper}



\section{其他GPU编程技术简介}

\label{sec:gpu.other}
\subsection{OpenCL}

在以CUDA为代表的各种硬件相关的GPU、众核编程技术发展之后,
苹果公司提出异构平台计算的开放标准OpenCL,
标准文本见文献\onlinecite{opencl}。

在GPU通用计算领域起步较晚的AMD公司(原ATI公司被AMD收购)
放弃设计一个CUDA这样专为自己生产的GPU编程的编程技术,
直接采用OpenCL作为其GPU的主要开发技术。

OpenCL在很多设备概念、编程模型上照搬较为成熟的CUDA的设计,
但和CUDA相比仍有较大差异。
CUDA程序的编译过程主要发生在主程序开发过程中,CUDA C/C++编译器首先处理CUDA程序的源代码,
把代码分为GPU端和CPU端两部分,CPU端的代码交给GCC/MSVC等传统编译器进行编译,
GPU端的代码则由自己处理,生成相应的GPU PTX代码、主机端调用代码等部分,
最后和程序的其他部分链接成一个主程序。
这其中的PTX代码类似“GPU上的汇编代码”,实际上更接近于Java编译后的字节码,
PTX代码在运行时由主机端驱动程序编译成实际GPU的硬件指令交给GPU运行。
通过增加PTX这一层抽象,NVIDIA使得程序能够在GPU ISA(Instruction Set Architecture)
设计发生变化时仍然可以直接运行。

OpenCL代码语法和CUDA一样\footnote{CUDA后期加入了C++支持。}以
C语言为蓝本\footnote{现在也有为OpenCL增加C++支持的提议,%
见文献\onlinecite{gaster2013opencl}。},
但并不和普通的C代码混在一起,
而是直接以文本形式保存于程序或其他数据文件中,
在运行时由硬件驱动进行编译生成实际硬件代码运行。
所以说OpenCL程序的编译过程主要发生在运行时,
这种方式称为JIT,CUDA未来也计划引入JIT\cite{MarkGTC2013}。


现在支持OpenCL的主要硬件/厂商有:\cite{opencl-conformant-products}
\begin{enumerate}[1)]
\item Intel x86 CPU(及内嵌的Intel HD Graphics集成显卡)
\item AMD CPU/APU
\item AMD显卡
\item NVIDIA显卡
\item Intel MIC(Many Integrated Core) Architecture
\item ARM
\end{enumerate}
相对来说,AMD显卡和MIC上的OpenCL支持较好,因为这是它们上的主要编程方式,
NVIDIA对OpenCL的支持略差,某些情况下可能会有性能明显下降的情况。


\subsection{C++ AMP}

C++ AMP全称C++ Accelerated Massive Parallelism,是微软提出的一个开放的C++ GPU编程标准,
并给出了Windows平台上基于DirectX 11的实现。
C++ AMP的标准文本见文献\onlinecite{cppamp}。
C++ AMP和CUDA类似,都是在已有的语言上增加了新的部分来对GPU进行编程。
\cite{cppamp-overview,AdeGTC2013}

由于C++ AMP是一个开放标准,所以也可能会有进一步的发展。
2012年Intel公司展示了一个把C++ AMP代码编译到OpenCL的实现,使得C++ AMP跨系统跨平台成为可能。
\cite{cppamp-opencl}
不过暂时还没有该技术转化为产品的消息。
在Intel之后,LLVM\footnote{LLVM(Low Level Virtual Machine),
是一套开源的编译器基础构建,Mac系统的编译器如Clang大多基于LLVM构建。}社区中也出现了一个把C++ AMP代码编译到OpenCL的开源实现\cite{llvm-amp-opencl-prototype}。


\subsection{OpenACC与OpenMP}

由于之前的GPU编程技术需要程序员管理的细节过多,
在2011年11月CAPS、CRAY、NVIDIA和PGI等公司联合提出了并行计算标准OpenACC。
\cite{reyes2012comparative}
OpenACC标准文本可以从\url{http://www.openacc-standard.org/Downloads}下载。
OpenACC与OpenMP类似,是一种基于用户制导语句的半自动并行化标准。
OpenMP主要针对于共享存储器的CPU多核环境,
而OpenACC则主要面向CPU+GPU异构环境。

当前支持OpenACC的编译器主要为CAPS、CRAY、PGI等公司提供的商业编译器。

OpenACC已计划被纳入OpenMP 4.0标准。\cite{beyer2011openmp}



\section{GPU线性方程组求解算法简介}

数值计算中线性方程组的求解方法基本可以分为两大类:直接解法和迭代解法。
直接解法的计算量固定且容易估计,没有迭代收敛性问题,但算法本身并行度太低,
很难适合GPU这样的类向量机型处理器,所以以下主要介绍迭代解法。

%\TODO 增加说明

\subsection{传统迭代算法简介}
考虑如下形式的线性方程组
\begin{align}
  \bm{A}\bm{x}=\bm{b}
\end{align}
其中
\begin{align}
  \bm{A}=\begin{pmatrix}
  a_{11} & a_{12} & \cdots & a_{1n}\\
  a_{21} & a_{22} & \cdots & a_{2n}\\
  \vdots & \vdots & \ddots & \vdots\\
  a_{n1} & a_{n2} & \cdots & a_{nn}\\
  \end{pmatrix}
\end{align}
并记
\begin{align}
  \bm{D}=\begin{pmatrix}
      a_{11} &  &  & \\
       & a_{22} &  & \\
       &  & \ddots & \\
       &  &  & a_{nn}\\
      \end{pmatrix}
  \hspace{5pt}
  \bm{L}=\begin{pmatrix}
    0 &  &  & \\
    a_{21} & 0 &  & \\
    \vdots & \ddots & 0 & \\
    a_{n1} & \cdots & a_{n,n-1} & 0\\
    \end{pmatrix}
  \hspace{5pt}
  \bm{U}=\begin{pmatrix}
      0 & a_{12} & \cdots & a_{1n}\\
       & 0 & \ddots & \vdots\\
       &  & 0 & a_{n-1,n}\\
       &  &  & 0\\
      \end{pmatrix}
\end{align}

\subsubsection{Jacobi迭代}
Jacobi迭代的形式为
\begin{align}
  \bm{x}^{(k+1)}=\bm{D}^{-1}\pb[b]{\bm{b}-\pb{\bm{L}+\bm{U}}\bm{x}^{(k)}}
\end{align}
收敛条件为
\begin{align}
  \rho\pb[b]{\bm{D}^{-1}\pb{\bm{L}+\bm{U}}}<1
\end{align}
其中$\rho(\bm{M})$表示$\bm{M}$的谱半径。\cite{golub2012matrix}

\subsubsection{Gauss-Seidel迭代}
Gauss-Seidel迭代的形式为
\begin{align}
  \bm{x}^{(k+1)}=\pb{\bm{L}+\bm{D}}^{-1}\pb[b]{\bm{b}-\bm{U}\bm{x}^{(k)}}
\end{align}
当$\bm{A}$对称正定时,Gauss-Seidel迭代收敛。\cite{golub2012matrix}

Gauss-Seidel迭代在传统CPU求解线性方程组中使用广泛,
但其主要步骤$\pb{\bm{L}+\bm{D}}^{-1}$项的计算是一个串行过程, %(\TODO 解释)
较难在GPU这种向量机上实现,所以在GPU求解线性方程组中使用不多。

\subsubsection{逐次超松弛迭代}
逐次超松弛迭代是Jacobi迭代和Gauss-Seidel迭代的推广,
其迭代的形式为\cite{golub2012matrix}
\begin{align}
  \bm{x}^{(k+1)}=\pb{\bm{L}+\omega\bm{D}}^{-1}
                  \pb[b]{\omega\bm{b}-\pb{\omega\bm{U}+(\omega-1)\bm{D}}\bm{x}^{(k)}}
\end{align}

该迭代方法在GPU上面临和Gauss-Seidel迭代同样的问题。

\subsection{Krylov子空间类算法简介}
\def\algoend{;\hspace{0.4cm}}
\subsubsection{CG(Conjugate gradient)共轭梯度法}
CG方法可以用于求解$\bm{A}$对称正定时的线性方程,
其求解过程见\floatref{alg:gpu.cg}%
\footnote{由于本节的伪代码每行长度较短,
为节省版面故将多行放在一行内,用;号分隔。}。\cite{golub2012matrix}

\begin{algorithm}
\KwIn{$\bm{A},\bm{x}_0,\bm{b}$}
\KwOut{$\bm{x}_e$}
$\bm{r}_0 := \bm{b}-\bm{A}\bm{x}_0$ \algoend
$\bm{p}_0 := \bm{r}_0$ \algoend
$k:=0$\;

\While{ True }{
$\displaystyle \alpha_k:=\frac{\bm{r}_k^T\bm{r}_k}{\bm{p}_k^T\bm{A}\bm{p}_k}$\algoend
$\bm{x}_{k+1}:=\bm{x}_k+\alpha_k\bm{p}_k$ \algoend
$\bm{r}_{k+1}:=\bm{r}_k-\alpha_k\bm{A}\bm{p}_k$\;
\If{$\vb{\bm{r}_{k+1}}$ 足够小}{
  return $\bm{x}_e := \bm{x}_{k+1}$\;
}
$\displaystyle \beta_k := \frac{\bm{r}_{k+1}^T\bm{r}_{k+1}}{\bm{r}_k^T\bm{r}_k}$ \algoend
$\bm{p}_{k+1}:=\bm{r}_{k+1}+\beta_k\bm{p}_k$ \algoend
$k:=k+1$\;
}
\setlabelname{CG方法}
\caption{CG方法\label{alg:gpu.cg}}
\end{algorithm}

带预条件的CG方法见\floatref{alg:gpu.pcg}。\cite{golub2012matrix}
由于只需在\floatref{alg:gpu.pcg}中令$\bm{M}=\bm{I}$即可得到\floatref{alg:gpu.cg},
所以以下只给出带预条件的迭代算法。

\begin{algorithm}
\KwIn{$\bm{A},\bm{x}_0,\bm{b},\bm{M}$}
\KwOut{$\bm{x}_e$}
$\bm{r}_0 := \bm{b}-\bm{A}\bm{x}_0$ \algoend
$\bm{z}_0 := \bm{M}^{-1}\bm{r}_0$ \algoend
$\bm{p}_0 := \bm{r}_0$ \algoend
$k:=0$\;
\While{ True }{
$\displaystyle \alpha_k:=\frac{\bm{r}_k^T\bm{z}_k}{\bm{p}_k^T\bm{A}\bm{p}_k}$ \algoend
$\bm{x}_{k+1}:=\bm{x}_k+\alpha_k\bm{p}_k$ \algoend
$\bm{r}_{k+1}:=\bm{r}_k-\alpha_k\bm{A}\bm{p}_k$ \;
\If{$\vb{\bm{r}_{k+1}}$ 足够小}{
  return $\bm{x}_e := \bm{x}_{k+1}$\;
}
$\bm{z}_{k+1}:=\bm{M}^{-1}\bm{r}_{k+1}$ \algoend
$\displaystyle \beta_k := \frac{\bm{z}_{k+1}^T\bm{r}_{k+1}}{\bm{z}_k^T\bm{r}_k}$ \algoend
$\bm{p}_{k+1}:=\bm{z}_{k+1}+\beta_k\bm{p}_k$ \algoend
$k:=k+1$\;
}
\setlabelname{带预条件的CG方法}
\caption{\label{alg:gpu.pcg}带预条件的CG方法}
\end{algorithm}

\subsubsection{BiCGStab}
对于非对称正定的方程组,可以把CG方法扩展为BiCG(BiConjugate Gradient)方法。
\cite{press2007numericalrecipes}
但由于BiCG方法的数值稳定性较差,可能会遇到不收敛的情况,
所以文献\onlinecite{van1992bicgstab}提出了BiCGStab方法,
见\floatref{alg:gpu.pbicgstab}。

\begin{algorithm}
\KwIn{$\bm{A},\bm{x}_0,\bm{b},\bm{K}$}
\KwOut{$\bm{x}_e$}

$\bm{r}_0 := \bm{b}-\bm{A}\bm{x}_0$ \algoend
任取$\bm{r}_0^*$使得$\bm{r}_0^*\cdot\bm{r}_0\neq 0$ \algoend
$\rho_0:=\alpha:=\omega_0:=1$ \;
$\bm{v}_0:=\bm{p}_0 := \bm{0}$ \algoend
$k:=1$\;
\While{ True }{
$\rho_k:=\bm{r}_0^*\cdot\bm{r}_{k-1}$ \algoend
$\displaystyle \beta:=\frac{\rho_k}{\rho_{k-1}}\frac{\alpha}{\omega_{k-1}}$ \algoend
$\bm{p}_k:=\bm{r}_{k-1}+\beta\pb{\bm{p}_{k-1}-\omega_{k-1}\bm{v}_{k-1}}$ \;
$\bm{y}:=\bm{K}^{-1}\bm{p}_k$ \algoend
$\bm{v}_k:=\bm{Ay}$ \algoend
$\displaystyle \alpha:=\frac{\rho_k}{\bm{r}_0^*\cdot\bm{v}_k}$ \algoend
$\bm{s}:=\bm{r}_{k-1}-\alpha\bm{v}_k$ \algoend
$\bm{z}:=\bm{K}^{-1}\bm{s}$ \;
$\bm{t}:=\bm{Az}$ \algoend
$\displaystyle \omega_k:=\frac{(\bm{K}^{-1}\bm{t})\cdot(\bm{K}^{-1}\bm{s})}{(\bm{K}^{-1}\bm{t})\cdot(\bm{K}^{-1}\bm{t})}$ \algoend
$\bm{x}_{k}:=\bm{x}_{k-1}+\alpha\bm{y}+\omega_k\bm{z}$ \;
\If{$\bm{x}_{k}$收敛}{
  return $\bm{x}_e := \bm{x}_{k}$\;
}
$\bm{r}_k:=\bm{s}-\omega_k\bm{t}$ \algoend
$k:=k+1$\;
}
\setlabelname{带预条件的BiCGStab方法}
\caption{\label{alg:gpu.pbicgstab}带预条件的BiCGStab方法}
\end{algorithm}

\subsubsection{GMRES(Generalized Minimal RESidual method)广义最小残量法}
GMRES是一种可以用于求解非对阵矩阵的Krylov子空间方法,
由Saad和Schultz在1986年提出。\cite{saad1986gmres}

实际计算中GMRES需要再启动等技巧,这里只给出一个简单的GMRES算法,
见\floatref{alg:gpu.gmres}。\cite{golub2012matrix}

\begin{algorithm}
\KwIn{$\bm{A},\bm{x}_0,\bm{b}\text{,并有}\bm{A}\bm{x}_0\approx\bm{b}$}
\KwOut{$\bm{x}_e$}

$\bm{r}_0 := \bm{b}-\bm{A}\bm{x}_0$ \algoend
$h_{10}:=\| \bm{r}_0\|_2$ \algoend
$k:=0$\;
\While{ $h_{k+1,k}>0$ }{
$\displaystyle \bm{q}_{k+1}:=\frac{\bm{r}_k}{h_{k+1,k}}$ \algoend
$k:=k+1$ \algoend
$\bm{r}_k:=\bm{A}\bm{q}_k$\;
\For{$i:=1 \  \mathrm{to} \   k$}
{
$h_{ik}:=\bm{q}_i^T\cdot\bm{r}_k$ \algoend
$\bm{r}_k:=\bm{r}_k-h_{ik}\bm{q}_i$\;
}
$h_{k+1,k}:=\| \bm{r}_k\|_2$ \;
$\bm{x}=\bm{x}_0+\bm{Q}_k\bm{y}_k$
其中$\bm{y}_k$使$\left\| h_{10}e_1-\overline{\bm{H}_m}\bm{y}_k \right\|_2$最小\;
}
$\bm{x}_e:=\bm{x}_k$\;
\setlabelname{GMRES方法}
\caption{\label{alg:gpu.gmres}GMRES方法}
\end{algorithm}


\subsection{预条件算法}
\label{sec:gpu.krylov-precond}
当矩阵$\bm{A}$的条件数较大时,以上的各种迭代方法的收敛速度往往并不理想,
为了改善在大条件数情况下的收敛速度,Krylov方法往往和预条件处理共同使用。
预条件处理的思路是把原始问题同解变换为一个条件数较小的问题。
预条件的方法一般是找到一个矩阵$\bm{P}$使得$\bm{P^{-1}A}$的条件数较小,
且方程组$\bm{Px}=\bm{b}$较容易求解。\cite{saad2003iterative}

以下列举一些常见的预条件算法
\begin{enumerate}
\item Jacobi,又称对角线预处理

取\cite{saad2003iterative}
\begin{align}
  \bm{P}=\begin{pmatrix}
  \displaystyle \frac{1}{A_{11}} & & &\\
  & \displaystyle \frac{1}{A_{22}} & &\\
  & & \ddots &\\
  & & & \displaystyle \frac{1}{A_{nn}}\\
  \end{pmatrix}
\end{align}

\item SPAI(SParse Approximate Inverse)\cite{saad2003iterative}

计算使$\|\bm{AT}-\bm{I}\|_F$最小的$\bm{T}$,
取$\bm{P}=\bm{T}^{-1}$。
$\bm{T}$的求解有多种算法,
如文献\onlinecite{grote1997parallel,bridson2001multiresolution}。

\item IC(Incomplete Cholesky factorization)

参见文献\onlinecite{golub2012matrix}第10.3.2节。

\item ILU(Incomplete LU Factorizations)

参见文献\onlinecite{saad2003iterative}第10.3节。

\end{enumerate}
上面介绍的预条件算法除了Jacobi具有天然的并行度外,
其他算法的Setup过程的串行度较高,较难在GPU上实现。


\begin{comment}

\subsubsection{多重网格}

\begin{table}
\centering
\caption{不同求解算法求解2维Poisson问题需要的计算量\cite{trottenberg2000multigrid}}
\begin{minipage}{.7\linewidth}
\centering
\begin{tabular}{lc}
\topline
求解方法 & 需要的计算量\footnote{记N为未知元数量。}\\
\midline
高斯消去 & $\mathcal{O}(N^2)$\\
Jacobi迭代 & $\mathcal{O}(N^2\log\epsilon)$\\
Gauss-Seidel迭代 & $\mathcal{O}(N^2\log\epsilon)$\\
超松弛迭代 & $\mathcal{O}(N^{3/2}\log\epsilon)$\\
CG & $\mathcal{O}(N^{3/2}\log\epsilon)$\\
IC CG & $\mathcal{O}(N^{5/4}\log\epsilon)$\\
ADI & $\mathcal{O}(N\log N)$\\
完全多重网格法 & $\mathcal{O}(N)$\\
\bottomline
\end{tabular}
\end{minipage}
\end{table}

\end{comment}

\subsection{常见稀疏矩阵存储格式介绍}
\label{sec:gpu.sparseformat}

稀疏矩阵的特点是矩阵中大量元素为0,非零元素比例较小,
分布有时具有一定的规律,如何快速高效地读写稀疏矩阵就是一个可以深入研究的方向。
目前这方面的基础研究已经较为完善,对应的稀疏矩阵表示、存储方法一般称为系数矩阵存储格式,
下面介绍一些常见的系数矩阵存储格式。

\subsubsection{坐标格式(COO)}
COO格式直接存储每个非零元素的位置和值,
如矩阵
\begin{align}
\bm{A}=\begin{pmatrix}
1 & 7 & 0\\
0 & 2 & 0\\
5 & 0 & 3
\end{pmatrix}
\label{equ:gpu.sparseformat.example.matrix3}
\end{align}
的COO格式表示为\footnote{下标从0开始计数,下同。}
\begin{align*}
\mathrm{rows}=\begin{pmatrix}
0 & 0 & 1 & 2 & 2  
\end{pmatrix}
\\
\mathrm{cols}=\begin{pmatrix}
0 & 1 & 1 & 0 & 2
\end{pmatrix}
\\
\mathrm{values}=\begin{pmatrix}
1 & 7 & 2 & 5 & 3
\end{pmatrix}
\end{align*}
其中每个非零元素分别存储:所在行(rows)、所在列(cols)、元素值(values),
三个数组中的位置要对应,而且一般会按照行列进行排序,
例如第一行第二列的元素7,对应rows、cols、values中下标为1的位置。


\subsubsection{行压缩格式(CSC)}
COO格式如果按行列位置对非零元素进行排序,则在行数组(rows)或列数组(cols)之一中
会连续出现很多相同的值,这是由于对应的非零元素属于相同的行或列。
为了减少这种冗余,CSC格式把每行的非零元连续存放,使用ptr数组记录每行元素的起始下标,
相对于COO节省了存储行号的空间。
如矩阵\aeqref{equ:gpu.sparseformat.example.matrix3}
的CSC格式表示为
\begin{align*}
\mathrm{ptr}&=\begin{pmatrix}
0 & 2 & 3  
\end{pmatrix}
\\
\mathrm{cols}&=\begin{pmatrix}
0 & 1 & 1 & 0 & 2
\end{pmatrix}
\\
\mathrm{values}&=\begin{pmatrix}
1 & 7 & 2 & 5 & 3
\end{pmatrix}
\end{align*}
ptr第一个元素为0,表示cols、values数组第一行的元素从下标0开始,
ptr第二个元素为2,表示第一行只有2个元素,第二行的元素从下表2开始。

如果把行列的存储方式调换,则可得到另一种格式,即列压缩格式(CSC)。

\subsubsection{对角线格式(DIA)}
\label{sec:gpu.sparseformat.dia}

对角线格式是专门用于存储对角线稀疏矩阵的一种格式,
其思路是只存储有非零元素的对角线。
如矩阵\cite{bell2008spmv}
\begin{align}
\bm{A}=\begin{pmatrix}
1 & 7 & 0 & 0\\
0 & 2 & 8 & 0\\
5 & 0 & 3 & 9\\
0 & 6 & 0 & 4
\end{pmatrix}
\label{equ:gpu.sparseformat.example.matrix4}
\end{align}
的DIA格式表示为\footnote{其中$\_$表示DIA格式不使用的任意值,下同。}
\begin{align*}
\mathrm{data}=\begin{pmatrix}
\_ & 1 & 7\\
\_ & 2 & 8\\
5 & 3 & 9\\
6 & 4 & \_
\end{pmatrix}
\quad
\mathrm{offsets}=\begin{pmatrix}
-2 & 0 & 1\\
\end{pmatrix}
\end{align*}
该矩阵有三个条对角线(5,6)、(1,2,3,4)、(7,8,9),对角线元素在data中存储(文本采用按列存放),
需要注意的是主对角线以上的对角线和以下的对角线在data矩阵中对其的方式不同。
这三条对角线的位置信息存储在offsets数组中,
-2表示主对角线下方第二条,0表示主对角线,1表示主对角线上方第一条。
data矩阵的列和offsets的元素要一一对应,
而且一般会按照对角线位置排序后进行存储。

\subsubsection{ELLPACK格式(ELL)}
ELL是为向量机设计的一种稀疏矩阵存储格式,\cite{grimes1979itpack}
用于存储$M\times N$的矩阵且每行最多只有$k$个元素的情况。
如矩阵\aeqref{equ:gpu.sparseformat.example.matrix4}
的ELL格式表示为\cite{bell2008spmv}
\begin{align*}
\mathrm{data}=\begin{pmatrix}
1 & 7 & \_\\
2 & 8 & \_\\
5 & 3 & 9 \\
6 & 4 & \_
\end{pmatrix}
\quad
\mathrm{indices}=\begin{pmatrix}
0 & 1 & \_\\
1 & 2 & \_\\
0 & 2 & 3\\
1 & 4 & \_
\end{pmatrix}
\end{align*}
ELL格式为每行保留k个位置,在上面的例子中每行最多有3个元素,
data中的一行对应原矩阵中的一行,所以data的列数为3。
data中每行元素的列号存储在indices矩阵中,
例如,第2行第3列的元素8,在data矩阵中对应第2行第2列的位置,
该元素的列号3存储在indices矩阵中第2行第2列的位置上。
同一行的元素顺序可以随意,一般会按列号进行排序。


\subsubsection{Hybrid格式(HYB)}
由于ELL格式只适合每行元素个数相差不多的情况,
所以又进一步出现了Hybrid格式,即同时使用ELL格式和COO格式,
ELL格式存储系数矩阵中较为规整的部分(每行元素数量接近),
COO格式则存储剩下的比较不规则的非零元。\cite{bell2008spmv}
如矩阵\aeqref{equ:gpu.sparseformat.example.matrix4}
可拆分为两个矩阵
\begin{align*}
\bm{A}=\begin{pmatrix}
1 & 7 & 0 & 0\\
0 & 2 & 8 & 0\\
5 & 0 & 3 & 0\\
0 & 6 & 0 & 4
\end{pmatrix}
+
\begin{pmatrix}
0 &  &  & \\
 & 0 &  & \\
 &  & 0 & 9\\
 &  &  & 0
\end{pmatrix}
\end{align*}
分别使用ELL和COO格式进行存储,
ELL格式部分为
\begin{align*}
\mathrm{data}=\begin{pmatrix}
1 & 7\\
2 & 8\\
5 & 3\\
6 & 4
\end{pmatrix}
\quad
\mathrm{indices}=\begin{pmatrix}
0 & 1\\
1 & 2\\
0 & 2\\
1 & 4
\end{pmatrix}
\end{align*}
COO格式部分为
\begin{align*}
\mathrm{rows}=\begin{pmatrix}
2
\end{pmatrix}
\\
\mathrm{cols}=\begin{pmatrix}
3
\end{pmatrix}
\\
\mathrm{values}=\begin{pmatrix}
9
\end{pmatrix}
\end{align*}

HYB格式中,ELL部分每行存储多少个元素不是固定的,
可以根据需要进行变化。






\chapter{\ProgramName 程序理论推导及开发}

\ProgramName (\ProgramFullName)是本文开发的基于GPU加速的、三维2群扩散细网有限差分、稳态/时空动力学求解程序。


\section{中子扩散时空动力学理论推导}

\subsection{稳态公式推导}

多群扩散时空动力学方程为
\begin{align}
  \newcommand{\para}{\pb{\bm{x},t}}
  \left\{
  \begin{aligned}
    \frac{1}{v_g}\frac{\partial \phi_g\para}{\partial t}
    &=\nabla\cdot D_g\para \nabla\phi_g\para 
      -\Sigma_{t,g}\para \phi_g\para
      +\sum_{i=1}^I \chi_{i,g}\para \lambda_i C_i\para \\
          & \hspace{1cm}
      +\sum_{g'=1}^G\pb[B]{\chi_g\para \pb{1-\beta}\nu\Sigma_{f,g'}\para
                            +\Sigma_{g'\rightarrow g}\para}\phi_{g'}\para \\
    \frac{\partial C_i\para}{\partial t}
     &=\beta_i \sum_{g'=1}^G \nu\Sigma_{f,g'}\para \phi_{g'}\para
        -\lambda_i C_i\para \qquad i=1,2,\cdots,I
  \end{aligned}
  \right.
  \titlelabel{equ:pro.diff.equ}{多群扩散时空动力学方程}
\end{align}

如果初始条件为稳态则有
\begin{align}
  \newcommand{\para}{\pb{\bm{x},t}}
  \left\{
  \begin{aligned}
    \frac{\partial \phi_g\para}{\partial t}\Big|_{t=0} &=0 \\
    \frac{\partial C_i\para}{\partial t}\Big|_{t=0} &=0
  \end{aligned}
  \right.
  \label{equ:pro.diff.init.equ}
\end{align}

联立\aeqref{equ:pro.diff.equ}及\aeqref{equ:pro.diff.init.equ},
消去$C_i\pb{\bm{x},0}$可得
\begin{align}
  \newcommand{\para}{\pb{\bm{x},0}}
  \begin{aligned}
  &\nabla\cdot D_g\para \nabla\phi_g\para 
   -\Sigma_{t,g}\para \phi_g\para \\
  & \hspace{3cm}
   +\sum_{g'=1}^G\pb[B]{\chi_g\para \nu\Sigma_{f,g'}\para
                        +\Sigma_{g'\rightarrow g}\para}\phi_{g'}\para =0
  \end{aligned}
\end{align}

此为$k_\mathrm{eff}=1$时的稳态扩散方程,
由于问题的已知条件一般仅有$k_\mathrm{eff}\approx 1$,
所以先求解普通的临界扩散方程。
\begin{align}
  \newcommand{\para}{\pb{\bm{x},0}}
  \begin{aligned}
  &\nabla\cdot D_g\para \nabla\phi_g\para 
   -\Sigma_{t,g}\para \phi_g\para \\
  & \hspace{3cm}
   +\sum_{g'=1}^G\pb[B]{\frac{1}{k_\mathrm{eff}}\chi_g\para \nu\Sigma_{f,g'}\para
                        +\Sigma_{g'\rightarrow g}\para}\phi_{g'}\para =0
  \end{aligned}
  \titlelabel{equ:pro.diff.init.diff.equ1}{扩散时空动力学问题的初始通量方程}
\end{align}
然后对裂变截面进行修正
\begin{align}
  \newcommand{\para}{\pb{\bm{x},t}}
  \nu\Sigma'_{f,g'}\para = \frac{1}{k_\mathrm{eff}}\nu\Sigma_{f,g'}\para
\end{align}
修正后的问题的$k_\mathrm{eff}=1$,可以适用\aeqref{equ:pro.diff.init.equ},
求得到初始条件的$C_i\pb{\bm{x},0}$
\begin{align}
  \newcommand{\para}{\pb{\bm{x},0}}
  C_i\para = \frac{\beta_i}{\lambda_i}\sum_{g'=1}^G \nu\Sigma_{f,g'}\para\phi_{g'}\para
  \label{equ:pro.diff.init.c}
\end{align}

边界条件为反射边界条件时
\begin{align}
  \bm{n}\cdot\nabla\phi_g\pb{\bm{x},t} = 0
  \qquad \bm{x} \in \partial \mathcal{D}
  \titlelabel{equ:pro.diff.boundary.equ}{扩散方程反射边界条件}
\end{align}
其中$\partial \mathcal{D}$是待求解问题区域的边界面,
$\bm{n}$是边界面上的点$\bm{x}$在边界面上的法向量,
方向指向区域外。

边界条件为0边界条件时
\begin{align}
  \phi_g\pb{\bm{x},t} = 0
  \qquad \bm{x} \in \partial \mathcal{D}
\end{align}

\subsubsection{空间离散}

程序使用三维直角坐标网格,空间划分为$K_x\times K_y \times K_z$个结构网格,
则网格集合为
\begin{align}
  \mathcal{D}_{\bm{k}}=\big\{(k_x,k_y,k_z)\big|k_w = 0,1,\cdots,K_w-1 ; w=x,y,z\big\}
\end{align}

在$xyz$坐标系中有一般离散关系
\begin{align}
  \phi(\bm{x}) &\rightarrow \phi_{\bm{k}} &
  \Sigma(\bm{x}) &\rightarrow \Sigma_{\bm{k}} &
  C_i(\bm{x}) &\rightarrow C_{i,\bm{k}}
\end{align}
其中$\bm{k}=(k_x,k_y,k_z)$为$xyz$空间离散后的网格坐标,
为方便起见这里暂时省略能群$g$,时间步长$n$等下标上标,下同。

离散的主要问题是微分项$\nabla\cdot D(\bm{x})\nabla\phi(\bm{x})$的离散方式,
设$\nabla\cdot D(\bm{x})\nabla\phi(\bm{x})$对应的离散项记为
$\pb[b]{\nabla\cdot D\nabla\phi}_{\bm{k}}$。
本文的网格通量值取在网格中心位置,所以一个网格内$D$相同,
相邻的网格如$D_{\bm{k}}$与$D_{\bm{k}\pm\hat{\bm{x}}}$、
$D_{\bm{k}\pm\hat{\bm{y}}}$、$D_{\bm{k}\pm\hat{\bm{z}}}$
\footnote{记$\hat{\bm{x}}=(1,0,0)^T$,$\hat{\bm{y}}=(0,1,0)^T$,$\hat{\bm{z}}=(0,0,1)^T$。}
(这三项以下简记为$D_{\bm{k}\pm\hat{\bm{w}}}$其中$w=x,y,z$)
未必相同,根据文献\onlinecite{xiezhongsheng}第二章的推导,
本文将相邻网格交界面上的$D$取为
\begin{align}
D_{\bm{k}\pm\frac{1}{2}\hat{\bm{w}}}
  =\frac{2D_{\bm{k}}D_{\bm{k}\pm\hat{\bm{w}}}}{D_{\bm{k}}+D_{\bm{k}\pm\hat{\bm{w}}}}
\end{align}
则
\begin{align*}
\pb[b]{\nabla\cdot D\nabla\phi}_{\bm{k}}
 &= \pb[bg]{ D_{\bm{k}\pm\frac{1}{2}\hat{\bm{x}}}\frac{\phi_{\bm{k}+\hat{\bm{x}}} - \phi_{\bm{k}}}{\Delta x^2}
    -D_{\bm{k}\pm\frac{1}{2}\hat{\bm{x}}}\frac{\phi_{\bm{k}+\hat{\bm{x}}} - \phi_{\bm{k}}}{\Delta x^2}
    }\\
 &\hspace{1cm}+
    \pb[bg]{ D_{\bm{k}\pm\frac{1}{2}\hat{\bm{y}}}\frac{\phi_{\bm{k}+\hat{\bm{y}}} - \phi_{\bm{k}}}{\Delta y^2}
       -D_{\bm{k}\pm\frac{1}{2}\hat{\bm{y}}}\frac{\phi_{\bm{k}+\hat{\bm{y}}} - \phi_{\bm{k}}}{\Delta y^2}
       }\\
 &\hspace{1cm}+
    \pb[bg]{ D_{\bm{k}\pm\frac{1}{2}\hat{\bm{z}}}\frac{\phi_{\bm{k}+\hat{\bm{z}}} - \phi_{\bm{k}}}{\Delta z^2}
       -D_{\bm{k}\pm\frac{1}{2}\hat{\bm{z}}}\frac{\phi_{\bm{k}+\hat{\bm{z}}} - \phi_{\bm{k}}}{\Delta z^2}
       }\\
 &= \sum_{w=x,y,z} \pb[bg]{ D_{\bm{k}\pm\frac{1}{2}\hat{\bm{w}}}\frac{\phi_{\bm{k}+\hat{\bm{w}}} - \phi_{\bm{k}}}{\Delta w^2}
   -D_{\bm{k}\pm\frac{1}{2}\hat{\bm{w}}}\frac{\phi_{\bm{k}+\hat{\bm{w}}} - \phi_{\bm{k}}}{\Delta w^2}
   }
\end{align*}
即
\begin{align}
  \pb[b]{\nabla\cdot D\nabla\phi}_{\bm{k}}
    &=\sum_{w=x,y,z} \Sb[bg]{
           \frac{2D_{\bm{k}}D_{\bm{k}+\hat{\bm{w}}}\pb{\phi_{\bm{k}+\hat{\bm{w}}} - \phi_{\bm{k}}}}
                {\Delta w^2\pb{D_{\bm{k}}+D_{\bm{k}+\hat{\bm{w}}}}}
          -\frac{2D_{\bm{k}}D_{\bm{k}-\hat{\bm{w}}}\pb{\phi_{\bm{k}} - \phi_{\bm{k}-\hat{\bm{w}}}}}
                {\Delta w^2\pb{D_{\bm{k}}+D_{\bm{k}-\hat{\bm{w}}}}}
          }
  \label{equ:dnabla2.equ0}
\end{align}
其中$\Delta x,\Delta y,\Delta z$是$x,y,z$方向上的网格尺寸。

则\aeqref{equ:pro.diff.init.diff.equ1}的离散形式为
\begin{align}
  \pb[b]{\nabla\cdot D_g^{(0)} \nabla\phi_g^{(0)}}_{\bm{k}}
   -\Sigma_{t,g,\bm{k}}^{(0)} \phi_{g,\bm{k}}^{(0)}
   +\sum_{g'=1}^G\pb[B]{\frac{1}{k_\mathrm{eff}^{(0)}}\chi_{g,\bm{k}}^{(0)} \nu\Sigma_{f,g',\bm{k}}^{(0)}
                        +\Sigma_{g'\rightarrow g,\bm{k}}^{(0)}}\phi_{g',\bm{k}}^{(0)} =0
  \quad \bm{k} \in \mathcal{D}_{\bm{k}}
\end{align}

\aeqref{equ:pro.diff.init.c}的离散形式为
\begin{align}
  C_{i,\bm{k}}^{(0)} = \frac{\beta_i}{\lambda_i}
    \sum_{g'=1}^G \nu\Sigma_{f,g',\bm{k}}^{(0)}\phi_{g',\bm{k}}^{(0)}
  \qquad \bm{k} \in \mathcal{D}_{\bm{k}}
\end{align}

\subsection{动力学公式推导}

将\aeqref{equ:pro.diff.equ}对时间$t$采用全隐式向后差分格式进行离散得
\begin{align}
  \newcommand{\para}[1][n]{\pb{\bm{x}}^{(#1)}}
  \left\{
  \begin{aligned}
    \frac{1}{v_g}\frac{\phi_g\para - \phi_g\para[n-1]}{\Delta t}
    &=\nabla\cdot D_g\para \nabla\phi_g\para 
      -\Sigma_{t,g}\para \phi_g\para \\
    & \hspace{1cm}
      +\sum_{g'=1}^G\pb[B]{\chi_g\para \pb{1-\beta}\nu\Sigma_{f,g'}\para
                           +\Sigma_{g'\rightarrow g}\para}\phi_{g'}\para \\
    &\hspace{1cm}
      +\sum_{i=1}^I \chi_{i,g}\para \lambda_i C_i\para \\
    \frac{C_i\para - C_i\para[n-1]}{\Delta t}
     &=\beta_i \sum_{g'=1}^G \nu\Sigma_{f,g'}\para \phi_{g'}\para
        -\lambda_i C_i\para
  \end{aligned}
  \right.
  \label{equ:pro.diff.dt.equ0}
\end{align}

解出$C_i\pb{\bm{x}}^{(n)}$可得
\begin{align}
  \newcommand{\para}[1][n]{\pb{\bm{x}}^{(#1)}}
  C_i\para = \frac{1}{1+\lambda_i\Delta t}
    \pb[B]{C_i\para[n-1]
    + \beta_i \Delta t \sum_{g'=1}^G \nu\Sigma_{f,g'}\para \phi_{g'}\para}
  \label{equ:pro.diff.dt.c}
\end{align}

代回\aeqref{equ:pro.diff.dt.equ0}得
\begin{align}
  \newcommand{\para}[1][n]{\pb{\bm{x}}^{(#1)}}
  \begin{aligned}
    &\quad \frac{1}{v_g}\frac{\phi_g\para - \phi_g\para[n-1]}{\Delta t} \\
    &=\nabla\cdot D_g\para \nabla\phi_g\para 
      -\Sigma_{t,g}\para \phi_g\para \\
    & \hspace{1cm}
      +\sum_{g'=1}^G\pb[B]{\chi_g\para \pb{1-\beta}\nu\Sigma_{f,g'}\para
                           +\Sigma_{g'\rightarrow g}\para}\phi_{g'}\para \\
    &\hspace{1cm}
      +\sum_{i=1}^I \frac{\chi_{i,g}\para \lambda_i}{1+\lambda_i\Delta t}
          \pb[B]{C_i\para[n-1] 
      + \beta_i \Delta t \sum_{g'=1}^G \nu\Sigma_{f,g'}\para \phi_{g'}\para}
  \end{aligned}
\end{align}

取$\chi_{i,g}=\chi_g$得
\begin{align}
  \newcommand{\para}[1][n]{\pb{\bm{x}}^{(#1)}}
  \begin{aligned}
    &\quad \frac{1}{v_g}\frac{\phi_g\para - \phi_g\para[n-1]}{\Delta t} \\
    &=\nabla\cdot D_g\para \nabla\phi_g\para 
      -\Sigma_{t,g}\para \phi_g\para 
      +\sum_{i=1}^I \frac{\chi_g\para \lambda_i}{1+\lambda_i\Delta t} C_i\para[n-1]\\
    & \hspace{1cm}
      +\sum_{g'=1}^G\pb[Bg]{\chi_g\para
        \pb[bg]{1-\beta 
          + \sum_{i=1}^I \frac{\lambda_i \beta_i \Delta t }{1+\lambda_i\Delta t}}
      \nu\Sigma_{f,g'}\para \\
    &\hspace{8cm}
         +\Sigma_{g'\rightarrow g}\para}\phi_{g'}\para
  \end{aligned}
\end{align}

记
\begin{align}
  \newcommand{\para}[1][n]{(\bm{x})^{(#1)}}
  S_C\para = \sum_{i=1}^I \frac{\lambda_i}{1+\lambda_i\Delta t} C_i\para[n-1]
  \titlelabel{equ:pro.diff.dt.sc}{离散扩散时空动力学中$S_C(\bm{x})^{(n)}$定义式} \\
  %
  B = 1-\beta + \sum_{i=1}^I \frac{\lambda_i \beta_i \Delta t }{1+\lambda_i\Delta t}
  \titlelabel{equ:pro.diff.dt.B}{离散扩散时空动力学中$B(\bm{x})^{(n)}$定义式}
\end{align}

则有
\begin{align}
  \newcommand{\para}[1][n]{\pb{\bm{x}}^{(#1)}}
  \begin{aligned}
    &\quad \frac{1}{v_g}\frac{\phi_g\para - \phi_g\para[n-1]}{\Delta t} \\
    &=\nabla\cdot D_g\para \nabla\phi_g\para 
      -\Sigma_{t,g}\para \phi_g\para + \chi_g\para S_C\para\\
    & \hspace{1cm}
      +\sum_{g'=1}^G\pb[B]{\chi_g\para
        B \nu\Sigma_{f,g'}\para
         +\Sigma_{g'\rightarrow g}\para}\phi_{g'}\para
  \end{aligned}
  \titlelabel{equ:pro.diff.dt.equ1}{时间$t$隐式向后差分离散后的扩散时空动力学通量$\phi$方程}
\end{align}
此为固定源扩散问题,可通过解线性方程进行求解。

\subsubsection{空间离散}
同临界问题部分,\aeqref{equ:pro.diff.dt.equ1}的离散形式为
\begin{align}
  \begin{aligned}
    \frac{1}{v_g}\frac{\phi_{g,\bm{k}}^{(n)} - \phi_{g,\bm{k}}^{(n-1)}}{\Delta t} 
    &=\pb[b]{\nabla\cdot D_{g}^{(n)} \nabla\phi_{g}^{(n)}}_{\bm{k}}
      -\Sigma_{t,g,\bm{k}}^{(n)} \phi_{g,\bm{k}}^{(n)} + \chi_{g,\bm{k}}^{(n)} S_{C,\bm{k}}^{(n)}\\
    & \hspace{1cm}
      +\sum_{g'=1}^G\pb[B]{\chi_{g,\bm{k}}^{(n)}
        B \nu\Sigma_{f,g',\bm{k}}^{(n)}
         +\Sigma_{g'\rightarrow g,\bm{k}}^{(n)}}\phi_{g',\bm{k}}^{(n)}
  \end{aligned}
  \qquad \bm{k} \in \mathcal{D}_{\bm{k}}
  \label{equ:pro.diff.dt.dx.equ1}
\end{align}

其中
\begin{align}
  S_{C,\bm{k}}^{(n)} &= \sum_{i=1}^I \frac{\lambda_i}{1+\lambda_i\Delta t} C_{i,\bm{k}}^{(n-1)}
  \qquad \bm{k} \in \mathcal{D}_{\bm{k}}
  \titlelabel{equ:pro.diff.dt.dx.sc}{离散扩散时空动力学中$S_{C,\bm{k}}^{(n)}$定义式}
\end{align}

\aeqref{equ:pro.diff.dt.c}的离散形式为
\begin{align}
  C_{i,\bm{k}}^{(n)} = \frac{1}{1+\lambda_i\Delta t}
    \pb[B]{C_{i,\bm{k}}^{(n-1)}
    + \beta_i \Delta t \sum_{g'=1}^G \nu\Sigma_{f,g',\bm{k}}^{(n)} \phi_{g',\bm{k}}^{(n)}}
  \qquad \bm{k} \in \mathcal{D}_{\bm{k}}
\end{align}



\subsection{能群耦合}

本文不考虑向上散射,向下散射只散射到邻近的能群,
则以上离散后的固定源方程可以写成如下形式
\begin{align}
  \begin{pmatrix}
  A_{11} & D_{12} & \cdots & D_{1G}\\
  D_{21} & A_{22} & &\\
   & \ddots & \ddots &\\
   & & D_{G-1,G} & A_{GG}
  \end{pmatrix}
  \begin{pmatrix}
  \phi_1 \\ \phi_2 \\ \vdots \\ \phi_G
  \end{pmatrix}
  =
  \begin{pmatrix}
  S_1 \\ S_2 \\ \vdots \\ S_G
  \end{pmatrix}
\end{align}
其中 $A_{gg}$是7对角对称阵,$D_{g_1g_2}$ 是对角阵。

两群情况则简化为
\begin{align}
  \begin{pmatrix}
  A_{11} & D_{12} \\
  D_{21} & A_{22}
  \end{pmatrix}
  \begin{pmatrix}
  \phi_1 \\ \phi_2
  \end{pmatrix}
  =
  \begin{pmatrix}
  S_1 \\ S_2
  \end{pmatrix}
  \label{equ:program.group.g2equ.fixs}
\end{align}

对于\aeqref{equ:program.group.g2equ.fixs},
可直接使用支持非对称矩阵的迭代算法进行求解,
如Jacobi迭代、BiCGStab、GMRES等。

不过在实际计算中,如果能够更充分的利用$A_{ii}$、
$D_{ij}$的实对称矩阵性质往往可以加速计算过程。
这样就有逐群迭代求解过程
\begin{align}
  \left\{
  \begin{aligned}
    \phi_1^{(k+1)}&=A_{11}^{-1}\pb[B]{S_1-D_{12}\phi_2^{(k)}}\\
    \phi_2^{(k+1)}&=A_{22}^{-1}\pb[B]{S_2-D_{21}\phi_1^{(k+1)}}
  \end{aligned}
  \right.
\end{align}

对于临界问题,则可写成如下形式的广义特征值问题
\begin{align}
  \begin{pmatrix}
  A_{11} &  \\
   & A_{22}
  \end{pmatrix}
  \begin{pmatrix}
  \phi_1 \\ \phi_2
  \end{pmatrix}
  =\frac{1}{k_\mathrm{eff}}
    \begin{pmatrix}
    F_{11} & F_{12} \\
    S_{21} & 0
    \end{pmatrix}
  \begin{pmatrix}
    \phi_1 \\ \phi_2
  \end{pmatrix}
\end{align}
其中 $A_{gg}$是7对角对称阵,$F_{g_1g_2}$、$S_{g_1g_2}$ 是对角阵。

如果使用源迭代进行求解,则每轮外迭代的分群迭代形式为
\begin{align}
  \left\{
  \begin{aligned}
    \mathcal{S}^{(k+1)}&=\frac{1}{k_\mathrm{eff}^{(k)}}\pb[B]{F_{11}\phi_1^{(k)}+F_{12}\phi_2^{(k)}}\\
    \phi_1^{(k+1)}&=A_{11}^{-1}\mathcal{S}^{(k+1)}\\
    \phi_2^{(k+1)}&=A_{22}^{-1}S_{21}\phi_1^{(k+1)}\\
    k_\mathrm{eff}^{(k+1)}&=k_\mathrm{eff}^{(k)}\frac{\left\|\mathcal{S}^{(k+1)}\right\|}{\left\|\mathcal{S}^{(k)}\right\|}
  \end{aligned}
  \right.
  \label{equ:program.group.g2equ.keff}
\end{align}

\aeqref{equ:program.group.g2equ.fixs}和\aeqref{equ:program.group.g2equ.keff}
中的$A_{ii}^{-1}$项可以使用支持实对称正定的迭代算法进行求解,
如Jacobi、CG等。


\section{\ProgramName 程序开发}

\subsection{整体介绍}
\ProgramName 程序整体结构见\floatref{fig:program.structure.whole},
其中内核部分是程序的主要部分,结构见\floatref{fig:program.structure.core}。

\begin{figure}[h]
\centering
\begin{tikzpicture}[scale=0.8, transform shape]

\node at (1.5,2) {主体控制逻辑};
\draw  (-4,2.5) rectangle (7,1.5);

\draw  (-1,0.5) rectangle (1,-3.5);
\draw  (2,0.5) rectangle (7,-3.5);
\draw  (2.5,0) rectangle (6.5,-2.5);
\node [right] at (2.5,-3) {输入模块(Lua解释器)};
\node [right] at (3,-0.5) {Lua输入文件};
\node [right] at (3.5,-1) {网格定义};
\node [right] at (3.5,-1.5) {材料定义};
\node [right] at (3.5,-2) {模式及参数};
\draw [latex new-latex new, arrow head=3mm]  (4,1.5)  -- (4,0);

\node at (0,-1.5) {内核};
\draw [latex new-latex new, arrow head=3mm]  (0,1.5)  -- (0,0.5);
\draw [latex new-latex new, arrow head=3mm]  (1,-1.5) -- (2.5,-1.5);

\draw  (-4,0.5) rectangle (-2,-3.5);
\node at (-3,-1.5) {输出};
\draw  (-9,0) rectangle (-5,-1);
\draw  (-5,-2) rectangle (-9,-3);
\node at (-7,-0.5) {输出文件(HDF5)};
\node at (-7,-2.5) {屏幕};
\draw [-latex new, arrow head=3mm] (-3,1.5) -- (-3,0.5);

\draw [latex new-, arrow head=3mm] (-5,-0.5) -- (-4,-0.5) ;
\draw [latex new-, arrow head=3mm] (-5,-2.5) -- (-4,-2.5) ;
\draw [latex new-, arrow head=3mm] (-2,-1.5) -- (-1,-1.5) ;

\end{tikzpicture}
\caption{\label{fig:program.structure.whole}\ProgramName 程序整体结构}
\end{figure}


\begin{figure}
\centering
\begin{tikzpicture}[scale=0.8, transform shape]
\draw  (-2.5,-4) rectangle (12,-5);
\node at (5,-4.5) { GPU};

\draw  (-2.5,-2.5) rectangle (12,-3.5);
\node at (5,-3) {显卡驱动};

\draw  (-2.5,-1) rectangle (12,-2);
\node at (5,-1.5) {CUDA};

\draw  (-2.5,2) rectangle (8,-0.5);
\node [right] at (-2,1.5) {Thrust};
\draw  (-2,1) rectangle (1,0);
\draw  (1.5,1) rectangle (4.5,0);
\draw  (5,1) rectangle (7.5,0);
\node [right] at (-1.5,0.5) {辅助函数};
\node at (3,0.75) { \small Global函数};
\node at (3,0.25) { \small 启动参数配置};
\node at (6.25,0.5) {显存管理};

\draw  (-2.5,6.5) rectangle (5,2.5);
\node [right] at (-2,6) {CUSP};
\draw (-2,4) rectangle (4.5,3);
\node at (1,3.5) {稀疏矩阵格式: DIA, ELL等};
\draw (-2,5.5) rectangle (4,4.5);
\node at (1,5) {迭代算法:CG, BiCGStab等};

\node at (10.25,1.25) {改进的};
\node at (10.25,0.75) {显存管理};
\draw [-latex new, arrow head=3mm] (7.5,0.5) -- (9,0.5);

\draw [very thick] (12,-0.5) -- (8.5,-0.5) 
     -- (8.5,2.5) -- (5.5,2.5) -- (5.5,7) 
     -- (-2.5,7) -- (-2.5,12.5) -- (12,12.5) 
     -- (12,-0.5);
\node [right] at (-2,12) {\ProgramName 程序内核};
\draw  (6,6.5) rectangle (11.5,4.5);
\node at (8.75,5.5) {能群间耦合处理};

\draw  (9,2) rectangle (11.5,0);
\draw  (6,4) rectangle (11.5,3);
\node at (8.75,3.5) {GPU端在线矩阵生成};

\draw  (-2,8.5) rectangle (11.5,7.5);
\node at (5,8) {本征值求解(源迭代)、固定源求解};
\draw  (-2,10) rectangle (11.5,9);
\node at (5,9.5) {双层网格加速};

\draw  (-2,11.5) rectangle (4.5,10.5);
\draw  (5,11.5) rectangle (11.5,10.5);
\node at (1,11) {临界计算};
\node at (8,11) {时空动力学计算};

\end{tikzpicture}
\caption{\label{fig:program.structure.core}\ProgramName 程序内核结构}
\end{figure}

GPU部分使用CUDA进行开发,版本为CUDA 5.0,
并使用了CUDA自带的Thrust通用C++模板库。
在线性代数库方面,现在业界尚没有较为完整、稳定、高效的稀疏矩阵库,
\ProgramName 程序使用了NVIDIA公司开发的开源稀疏矩阵库CUSP,
因为CUSP支持本文需要的DIA稀疏矩阵存储格式,并且实现较为高效。
CUSP在实现上也是一个C++模板库,大量依赖Thrust的实现,
并且使用了Thrust的显存管理模块,出于性能考虑本文替换了Thrust的显存管理模块。
%由于现在Thrust和CUSP并不支持多GPU并行,所以本程序也仅限使用单GPU进行加速。

在稀疏矩阵格式方面主要使用DIA格式(见\sectionref{sec:gpu.sparseformat.dia}),
并针对扩散方程离散的特点实现了扩散方程矩阵GPU端动态DIA格式生成功能,缩短了程序总计算时间。
为了对比其他格式,还添加了ELL、CSR、COO格式(使用CUSP)。

在输入文件格式方面,\ProgramName 使用了Lua语言作为输入文件格式。
Lua是一种快速、轻量级的嵌入式动态语言,设计目标即为容易嵌入到C语言中使用,
方便与C等语言交互,现已在PC游戏界广泛使用。
Lua当前版本5.2.2,基于MIT开源协议开源,
可从 \url{http://www.lua.org/download.html} 获得。
使用Lua语言作为输入文件不光可以显著减少解析输入文件的工作量,
而且还允许用户通过输入文件输入带有控制逻辑的Lua程序以对程序行为进行更为精细的控制,
\ProgramName 基于Lua提供了一种方面地输入动力学问题中材料、截面随时间变化的方式。

由于\ProgramName 程序是三维时空动力学程序,产生的数据包含5个维度(空间3维、能群1维、时间1维),
如果使用传统的纯文本格式输出,用户对数据进行后处理时则会十分不便,
所以\ProgramName 使用国际上较流行的一种针对科学计算的数据交换格式——HDF5作为直接输出格式。
HDF5是一种分层的、带有元数据的、支持多种数据格式的、针对科学计算的数据文件存储格式及相关技术。
它最早由美国国家超级计算应用中心(NCSA)提出,
现在由非营利的HDF Group管理和维护。
HDF5文件格式的读写库基于一种类似于BSD的协议开源\footnote{协议文本可以从
\url{http://www.hdfgroup.org/HDF5/doc/Copyright.html} 获取。},
当前版本为HDF5 1.8.10 patch1\footnote{可
从 \url{http://www.hdfgroup.org/downloads/index.html} 获取。}。

\begin{comment}

由于\ProgramName 程序使用人不能直接读取的HDF5格式进行输出,所以需要额外的工具进行数据后处理。
支持HDF5格式的第三方工具和编程语言比较丰富,本文则推荐Python语言作为后处理语言,
并提供一些Python程序作为示例。
Python语言是一种面向对象的动态语言,现在在学术界和工业界都十分流行,
不少GPU加速工具和科学可视化工具都使用Python编写,
所以本文也使用Python作为主要数据前处理、
后处理脚本语言。\footnote{Python可以从
\url{http://www.Python.org} 免费获取,本文使用 Python2.7.x 的语法。}

\end{comment}

\subsection{材料反应截面的输入}

对于扩散程序,输入数据中占主要部分的是各截面在不同空间网格上的取值,
\ProgramName 和传统确定论程序一样,采用两段式的输入方式,
即输入分为两部分:每个材料的各个截面和材料在空间网格上的分布,
前者一般用列表形式给出,后者一般用粗网格上的材料矩阵给出。
这种方式主要针对扩散计算中问题的常见特点:单个组件尺度内往往只有一种材料,
不同组件间的材料也往往有重复的。\footnote{但对于全堆均匀化后的截面参数,
即各个组件材料不同的情况,这种方式略显冗余。}

对于临界计算来说,以上这种两段式的输入方式已经能够满足实际计算需求,
但对于时空力学问题尚缺乏统一的材料描述方式。
对于时空动力学来说,材料定义的主体部分仍然和临界相同,
实际问题中往往大部分区域的材料、截面并不随时间变化,可以用两段式输入描述。
材料、截面随时间变化的部分基本可以分为两种模式:
\begin{enumerate}
\item
某个或某些材料的某些截面发生变化,即材料$M_i$的某截面$\Sigma_{M_i,j}$不再是定值,
而是随时间变化的函数$\Sigma_{M_i,j}(t)$。
在常见基准算例中,这种中情况常见于材料的部分成分发生改变时。
对于空间是一维、二维等低维形式的算例,由于计算中常取反应堆横截面,
高度方向被简化,控制棒移动的效果只能通过连续或阶跃改变固定网格处的材料反应截面来反映。

\item
材料在空间网格上的分布发生改变,如三维问题中,控制棒移动会改变控制棒路径上各网格的材料,
这时各网格上的截面从一种材料阶跃或连续地变为另一种材料的。
\end{enumerate}
实际计算中可能会出现以上两种模式的混合需求:
\begin{enumerate}
\setcounter{enumi}{2}

\item 只有某些固定区域上的某材料的截面发生变化。
这种情况在计算中一般通过人为地把改变的区域和不改变的区域指定成不同材料,
这样就可以隔离不同区域间的影响,使用上面模式1的输入方式。

\item
在三维问题中,控制棒从某网格某一侧进入并逐渐填满网格的过程是连续的,
但如果只使用上面第二种模式的方式则只能描述为阶跃变化(或是额外定义大量中间状态的材料),
带来额外的误差。为了提高精度,往往会对这些处于过度状态的网格(甚至包括它们附近的网格),
进行一定的均匀化或插值处理,得到过渡状态的截面。
而如何进行这方面的均匀化或插值则有多种方法,如按体积加权或按通量加权等,
不宜在程序中固化,\ProgramName 最好能提供某种接口来让用户控制
这些网格上的截面计算工作。

\end{enumerate}

为了给予用户精细地控制材料输入的方式,\ProgramName 选择了Lua这种动态语言作为材料输入的方式。
动态语言的特点是在运行时可以直接读取程序源代码并执行而不需要显式的编译过程,
这样就可以让用户把均匀化代码放在输入文件中,在运行时根据条件计算各网格上的材料。
如果不考虑效率问题,完全可以用这种动态语言的方式替换掉之前的二段式材料输入方式,
把材料在空间网格上的离散(及可能涉及的均匀化)过程暴露给用户,以增强程序的通用性。

但实际程序中很难使用这种方式,原因是程序运行速度的问题。
动态语言在一般经验中比编译型的语言慢两个数量级,
比较快的动态语言速度大约也比编译型语言慢一个数量级,
如果语言设计时就考虑到性能问题,而且动态语言的实现也经过充分优化,
并使用在线按需编译技术(一般称作JIT)后,其性能才能够勉强接近C语言。
现在这方面的动态语言只有少数的几个,如Lua的LuaJIT实现,
它是目前为止世界上最快的动态语言实现之一\footnote{参见:\url{http://luajit.org/luajit.html}};
还有最近刚出现的类Matlab语法的开源科学计算语言Julia(\url{http://julialang.org/}),
它使用了较为先进的LLVM的JIT在线编译器,性能也接近C语言的效率,
性能比较见Julia项目主页,但Julia项目现在并不成熟,而且不便于嵌入到\ProgramName 中。

而对于\ProgramName 程序这种细网有限差分程序,空间网格数量可达$10^7$的量级
\footnote{对于静态IAEA PWR三维基准问题(见第 \ref{sec:result.test.iaea} 节),
如果空间网格尺寸取1cm,则网格数量为$170\times170\times380=1.1\times10^7$。},
可能的方案有
\begin{enumerate}
\item
使用Lua的LuaJIT实现,性能接近C语言,但由于LuaJIT自身的限制只能适用于使用的内存不超过1G的情况。

\item
使用TinyCC(项目主页\url{http://bellard.org/tcc/})这种能嵌入到其他程序中的C编译器,
在运行时编译用户输入的材料处理函数以实现高速计算。
\end{enumerate}
但这两种方案都较为复杂,并有自己的不足。更为重要的是,
这两种方式只能在CPU端产生所有网格的截面数据,
在GPU计算前需要再复制到GPU显存上,大量数据的传输开销无法避免,
对于时空动力学问题,需要每个时间步都要产生截面数据并进行传输,时间上的开销太大。
根据前面的思路,也可以根据用户的输入在运行时编译为可直接在GPU上运行的PTX指令,
以避免数据传输问题,但这种方式的技术尚不成熟,而且对于开发人员和用户来说难度和工作量都要大的多,
目前基本没有实际应用的价值。

\begin{algorithm}
\ForEach(\tcc*[f]{遍历每种材料的截面信息}){输入文件中的材料截面信息}
{
  \If{截面信息通过时间相关函数生成}
  {
  调用该材料截面生成函数获得该时间步的截面定义\;
  }
  \Else
  {
  直接获取该材料的固定截面\;
}
}
预处理截面信息,并传输到GPU显存\;
\If{存在材料分布更新函数MatChangeFun}
{
  \If{上一时间步中,材料分布信息被更新过}
  {
    恢复初始的材料分布信息\;
  }
  在Lua环境内调用材料更新函数MatChangeFun,产生材料分布差分数据\;
  从Lua环境中取回材料分布差分数据\;
  根据材料分布差分数据更新材料分布信息\;
  更新GPU显存上的材料分布信息\;
}
在GPU端根据材料的截面信息和材料分布信息产生截面在空间网格上的分布\;
\setlabelname{每个时间步\ProgramName 对材料截面更新过程}
\caption{\label{alg:program.material.update}每个时间步\ProgramName 对材料截面更新过程}
\end{algorithm}

综合考虑以上各方面因素,\ProgramName 程序最后仍然采用了二段式的方式进行
数据输入,在时空动力学的材料输入上采用了动态语言的方式来实现功能上的可扩展性,
同时避免了大量的截面信息在CPU端产生的问题。主要方式为:
每个时间步开始时,由用户通过输入文件的Lua代码给出本时间步和初始时间步的材料在网格分布
信息的差分信息,即哪些网格上的材料发生了改变,并给出本时间步的新材料截面信息。
程序根据材料分布的差分信息和初始材料分布信息在CPU端产生新的材料分布矩阵,
新材料分布矩阵和新材料截面信息传送至GPU上,
GPU端程序根据信息直接在显存产生完整的材料分布数据。
在考虑到程序速度的情况下最大限度地照顾到前面提出的各种需求。
此外\ProgramName 还针对需要临时改变全部或某些材料截面的需求
添加了相应接口,简化输入文件修改的工作量。
每个时间步\ProgramName 对材料截面更新过程见\floatref{alg:program.material.update}。

需要说明的是,以上的材料更新过程和空间网格定义是关联的,
当需要在不同网格上进行计算时,程序会对不同网格上的材料信息分开处理,
允许用户在粗网和细网上分别定义不同的材料更新及均匀化方式。


\subsection{临界计算}
\label{sec:program.eigen}

临界计算是反应堆物理设计中常见的计算需求,同时也是动力学中计算通量初值所必需的,
所以\ProgramName 程序实现了临界计算功能。
临界计算部分,\ProgramName 主要使用:
\begin{enumerate}
\item CG-SG,使用CG方法从高能群到低能群逐群求解单群方程组。
\item BiCGStab-MG,使用BiCGStab方法直接求解所有能群联合方程组。
\end{enumerate}
两种方法。为了对比其他方法,程序还实现了如下求解方法:
\begin{enumerate}
\setcounter{enumi}{2}
\item Jacobi-SG,使用Jacobi迭代逐群求解。
\item Jacobi-MG,使用Jacobi对所有能群统一求解。
\item GMRES-MG,使用GMRES对所有能群统一求解。
\end{enumerate}
以上各迭代算法中,CG、BiCGStab、GMRES使用CUSP库的实现,
本文对其显存管理进行了改进。预条件算法使用对角线预条件算法,
见\sectionref{sec:gpu.krylov-precond}。

此外\ProgramName 为了加速细网问题的求解速度,
使用了MultiLevel方法改进迭代过程的初值,显著减少了迭代次数,
详见\sectionref{sec:equsolve.multimesh}。


\begin{algorithm}
读取输入文件的临界部分 \algoend
配置内核稀疏矩阵格式\;
初始化网格信息\;
\If{使用MultiLevel}
{
  初始化粗网网格信息 \algoend
  初始化粗网通量\;
  配置粗网求解算法及参数 \algoend
  求解粗网临界问题\;
  将粗网通量插值为细网通量\;
}
\Else
{
  初始化细网通量\;
}
配置求解算法及参数 \algoend
求解临界问题\;
通量归一化,输出结果\;
\setlabelname{\ProgramName 程序临界功能主流程伪代码}
\caption{\label{alg:program.eigen.main}\ProgramName 程序临界功能主流程伪代码}
\end{algorithm}


\begin{algorithm}
初始化CPU端和GPU端数据\;
根据材料定义和截面信息在GPU端直接产生的空间网格上的截面信息\;
根据网格信息配置DIA格式的迭代矩阵需要的空间\;
根据空间离散方式在GPU上填充各群迭代矩阵\;

根据当前通量计算第1群源项\;
计算对角线预条件算法的对角线矩阵\;
源迭代过程初始化\;
\Repeat{源迭代收敛}
{
  使用CG求解第1群通量 \algoend
  计算第2群源项\;
  使用CG求解第2群通量 \algoend
  更新第1群源项\;
  计算下一代的$K_\mathrm{eff}$ \algoend
  估计$K_\mathrm{eff}$和通量的误差\;
}
\setlabelname{\ProgramName 程序临界功能CG-SG算法伪代码}
\caption{\label{alg:program.eigen.cg-sg}\ProgramName 程序临界功能CG-SG算法伪代码}
\end{algorithm}


\begin{algorithm}
初始化CPU端和GPU端数据\;
根据材料定义和截面信息在GPU端直接产生的空间网格上的截面信息\;
根据网格信息配置DIA格式的迭代矩阵需要的空间\;
根据空间离散方式在GPU上填充迭代矩阵\;

根据当前通量计算源项\;
计算对角线预条件算法的对角线矩阵\;
源迭代过程初始化\;
\Repeat{源迭代收敛}
{
  使用BiCGStab求解各群通量 \algoend
  更新源项\;
  计算下一代的$K_\mathrm{eff}$ \algoend
  估计$K_\mathrm{eff}$和通量的误差\;
}
\setlabelname{\ProgramName 程序临界功能BiCGStab-MG算法伪代码}
\caption{\label{alg:program.eigen.bicgstab-mg}\ProgramName 程序临界功能BiCGStab-MG算法伪代码}
\end{algorithm}

临界部分的求解流程见\floatref{alg:program.eigen.main}。
以上各求解算法的实现可以分为两大类,逐群求解和联合求解,
这两类中的算法过程基本相近,所以只以CG-SG和BiCGStab-MG
为代表进行说明。
CG-SG算法的伪代码见\floatref{alg:program.eigen.cg-sg},
BiCGStab-MG算法的伪代码见\floatref{alg:program.eigen.bicgstab-mg}

\FloatBarrier

\subsection{时空动力学计算}

\subsubsection{时空动力学问题初值}
\label{sec:program.kinetics.keff-fix}

要计算时空动力学问题,首先要计算时变问题的初值,
实际中一般取刚好临界时的稳定状态为初始状态,
而实际算例中很少有恰好$k_\mathrm{eff}=1$的时候,
如果仅仅使用临界计算的通量作为初始通量分布并用\aeqref{equ:pro.diff.init.c}计算缓发中子源的话,
在第一个时间步中相当于材料截面有一个阶跃,使得总通量分布在第一个时间步会有一个明显的跳跃,
而且这也不符合\aeqref{equ:pro.diff.init.c}的计算条件。
所以一般计算中会对问题的参数进行修正(如程序NGFMN-K\cite{zhaowenbo}),使修正后的问题满足$k_\mathrm{eff}=1$,
本文对裂变项进行修正
\begin{align}
  \newcommand{\para}{\pb{\bm{x},t}}
  \nu\Sigma'_{f,g'}\para = \frac{1}{k_\mathrm{eff}}\nu\Sigma_{f,g'}\para
\end{align}
但实际计算中,$k_\mathrm{eff}$是由临界计算得到的,实际临界计算往往会有少量误差,
使得$k_\mathrm{eff}$的计算值和实际值之间有微小的差别,这与源迭代收敛条件有关。
但这个微小的误差会在后续的动力学过程中表现出来,即第一个时间步中总体通量会有一个小的阶跃,
而且在后续时间步中总体通量也会有小指数上升或小指数下降的趋势,影响时空动力学的计算和结果比较。
所以本文采用多次试算、修正的方式直至最终$k_\mathrm{eff}$足够接近于1,
算法伪代码见\floatref{alg:program.kinetics.keff-fix}。

\begin{algorithm}
初始化$c_L:=0.1, c_U:=10, c:=1, n:=0$\;
\While{$k_\mathrm{eff}^{(n)}$足够接近于$1$或$c_U-c_L$足够小}
{
  \lIf(\tcc*[f]{更新系数$c$的上下界})
  {$k>1$}{$\displaystyle c_U:=\frac{c+c_U}{2}$}
  \lElse(\tcc*[f]{每次减半是考虑到$k_\mathrm{eff}$的计算可能有误差})
  {$\displaystyle c_L:=\frac{c+c_L}{2}$}
  $\displaystyle c:=\frac{c}{\ k_\mathrm{eff}^{(n)}\ }$\;
  \lIf(\tcc*[f]{确保$c$不会发散})
  {$c>c_U$或$c<c_L$}
  {$\displaystyle c:=\frac{c_L+c_U}{2}$}
  使用$c$对裂变截面进行修正
  $\nu\Sigma'_{f,g',\bm{k}}\Big|_{t=0} := c\cdot\nu\Sigma_{f,g',\bm{k}}\Big|_{t=0}$\;
  重新计算临界问题,得到$k_\mathrm{eff}^{(n+1)}$ \algoend
  $n:=n+1$\;
}
\setlabelname{\ProgramName 程序时空动力学临界修正伪代码}
\caption{\label{alg:program.kinetics.keff-fix}
\ProgramName 程序时空动力学临界修正伪代码}
\end{algorithm}


\begin{comment}
\subsubsection{时空动力学与预条件算法}

由于复杂的预条件算法尚不能在GPU上并行化,只能在CPU上计算,
使得其启动(setup)时间往往较长,在静态临界计算中一般会导致总求解时间增加。
如果在每个时间步求解中分别使用这类复杂的预条件算法,则更加得不偿失。
但时空动力学计算中,材料变化一般相对较小,最后待求解的方程变化也并不大,
而预条件算法本身就是通过近似求解原方程来实现加速,
不要求与当前的方程完全符合,
所以可以在时空动力学启动计算初值时,
同时setup一个较复杂的预条件算法,得到初始状态的预条件矩阵,
在之后的每个时间步中总是使用这个预条件矩阵。

本文使用CUSP提供的SPAI预条件算法,
根据计算结果,上面这种方法对于网格数量较多的问题能够起到加速作用,
代价是时空动力学启动(init)时间增加。
\end{comment}


\subsubsection{时空动力学计算}

前面已经介绍,本文的时空动力学计算中,时间离散采用隐式向后差分格式,
每个时间步的通量只由上一步的通量及缓发中子源决定,
由于是隐式迭代,所以每个时间步要求解一个固定源扩散方程。
时空动力学计算流程见\floatref{alg:program.kinetics.loop}。

\begin{algorithm}
读取时空动力学计算参数并初始化\;
计算时空动力学初值
\tcc*[f]{见第 \ref{sec:program.eigen}
节及第 \ref{sec:program.kinetics.keff-fix} 节}
\;
%\If{使用SPAI预条件算法}
%{
%  读取SPAI设置并初始化SPAI\;
%}
\Repeat{完成所有时间步}
{
  初始化下一个时间步\;
  更新本时间步的材料、截面信息\;
  %\If{使用SPAI预条件算法}
  %{
  %  求解细网网格上的固定源问题(使用SPAI预条件矩阵)\;
  %}
  %\Else
  %{
    初始化对角线预条件算法,并求解细网网格上的固定源问题\;
  %}
  计算缓发中子源 \algoend
  计算当前时间步的最大通量及总功率\;
  将本时间步计算结果从GPU显存传送至内存\;
}

\setlabelname{\ProgramName 程序时空动力学主要过程伪代码}
\caption{\label{alg:program.kinetics.loop}
\ProgramName 程序时空动力学主要过程伪代码}
\end{algorithm}

同临界部分一样,固定源的多群扩散方程可以根据对能群处理的不同而有不同的求解方式,
但由于BiCGStab-MG方法只需要一层迭代循环,内迭代次数要比CG-SG方法少,
所以本文选择了BiCGStab-MG方法作为固定源方程求解算法,
伪代码见\floatref{alg:program.kinetics.bicgstab-mg}。

\begin{comment}

在实际计算中,

\begin{algorithm}
根据网格及截面信息在GPU显存上直接生成各群的迭代矩阵\;
初始化预条件算法的对角线矩阵\;
\Repeat{各群通量收敛}
{
  计算第1群的源项 \algoend
  使用CG算法求解第1群通量\;
  计算第2群的源项 \algoend
  使用CG算法求解第2群通量\;
  估算各群通量的误差\;
}
\setlabelname{\ProgramName 程序固定源CG-SG算法伪代码}
\caption{\label{alg:program.kinetics.cg-sg}
\ProgramName 程序固定源CG-SG算法伪代码}
\end{algorithm}

\end{comment}

\begin{algorithm}
根据网格及截面信息在GPU显存上直接生成迭代矩阵\;
初始化预条件算法的对角线矩阵\;
计算源项\;
使用BiCGStab算法直接求解各群通量\;
\setlabelname{\ProgramName 程序固定源BiCGStab-MG算法伪代码}
\caption{\label{alg:program.kinetics.bicgstab-mg}
\ProgramName 程序固定源BiCGStab-MG算法伪代码}
\end{algorithm}


\subsection{迭代收敛条件}

在迭代过程中,需要不断地对当前迭代的收敛程度进行估计,
本文使用本次迭代通量和上次迭代通量的绝对偏差的最大值进行对收敛程度进行估计,
为了排除通量幅值的影响,使用本次迭代的最大通量进行标准化,即
\begin{align}
e^{(p)}=\frac{\displaystyle \max_{\bm{k},g}\left|\phi_{\bm{k},g}^{(p)}-\phi_{\bm{k},g}^{(p-1)}\right|}
         {\displaystyle \max_{\bm{k},g}\phi_{\bm{k},g}^{(p)}}
\end{align}
其中$\bm{k}$为三维网格编号。


\subsection{MultiLevel方法}
\label{sec:equsolve.multimesh}

由于初值对于迭代算法的运行时间影响较大,
所以可以通过改善初值来实现总计算时间的缩减。
对于反应堆类问题,堆中的材料分布相对较简单,
尤其是在细网离散时往往会出现大片的网格材料相同的情况。
可以首先对问题进行粗网离散,可以用较低的开销进行求解,
获得一个较粗略的结果后,可以变换为一个较好的细网计算初值,
达到减少总计算时间的目的,
这种方法由文献\onlinecite{ginestar2001multilevel}最早提出,
称作MultiLevel方法。

本文采用如下方式计算临界计算中使用的通量初值:
产生原网格xyz方向网格数量均减半的空间网格划分,
即细网格上的8个网格对应到粗网上的1个网格,
使用如前所用的CG-SG等方法进行求解,该阶段用户可以通过输入文件自定义
每轮内迭代次数、外迭代收敛标准等控制变量,
在粗网上求解后把粗网上的通量插值为细网通量,
细网上的初始$K_\mathrm{eff}$取为粗网$K_\mathrm{eff}$%
\footnote{同一个扩散问题采用不同网格大小进行离散后得到的$K_\mathrm{eff}$并不完全一致,
略有差异。},
最后在细网上使用CG-SG等方法进行求解。

\subsection{显存管理}

CUSP的显存管理部分继承自Thrust的显存管理方式,
而Thrust的显存管理策略为需要时向CUDA申请,
不需要时向CUDA释放,CUDA本身的显存申请释放速度略慢。
虽然这种策略对于一般需求时能够应付,
但把CG、BiCGStab等算法放在源迭代过程中时会导致每次源迭代
开始及结束时都会进行显存的申请及释放,
会明显增加程序的运行时间。

为了消除这种无谓的时间开销,本文对Thrust的显存管理策略进行了修改,
增加缓存功能,即当上层算法对显存用完释放时,
不是直接调用CUDA释放,而是先放如一个可用显存池中。
这样当以后上层算法需要重新申请显存时,
可以直接从可用显存池中查找是否有适合的显存块,如果发现可用的,
则直接返回给上层算法,没有时才向CUDA进行申请。
通过这种方式,可以消除外迭代反复调用CUSP算法时出现的反复申请释放同样大小的显存的情况,
一般来讲从显存池中查找适合的显存的时间开销要小于向CUDA申请显存的开销,
从而达到了缩短程序运行时间的效果。
%\TODO 考虑添加结果比较

\begin{table}
\centering
\caption{不同显存管理策略下临界计算时间和显存占用峰值表}
\label{tab:program.cached_alloc}
\begin{tabular}{cccc}
\topline
计算条件 & 显存管理策略 & 计算时间/s & 显存占用峰值/MB\\
\midline
\multirow{2}{*}{5cm网格}
 & 无缓存 & 3.994 & 11.23\\
 & 有缓存 & 2.652 & 11.23\\
\multirow{2}{*}{2.5cm网格}
 & 无缓存 & 7.052 & 89.82\\
 & 有缓存 & 5.819 & 89.82\\
\multirow{2}{*}{2cm网格}
 & 无缓存 & 9.938 & 175.43\\
 & 有缓存 & 8.549 & 175.43\\
\multirow{2}{*}{1cm网格}
 & 无缓存 & 55.630 & 1403.42\\
 & 有缓存 & 53.430 & 1403.42\\

\multirow{2}{*}{2.5cm网格+MultiLevel}
 & 无缓存 & 4.790 & 93.00\\
 & 有缓存 & 3.338 & 101.05\\
\multirow{2}{*}{1cm网格+MultiLevel}
 & 无缓存 & 14.212 & 1453.17\\
 & 有缓存 & 12.512 & 1578.85\\
\bottomline
\end{tabular}
\end{table}

对了观察显存分配缓存的影响,这里使用静态IAEA PWR三维基准题进行计算,
不同情况下的计算时间和显存占用峰值见\floatref{tab:program.cached_alloc}。
可见,带缓存的显存管理策略以少量显存占用峰值增加为代价
(不使用MultiLevel时峰值显存增加很小)
减少了约1-2s的计算时间,对于小规模问题效果十分明显。




\chapter{数值结果验证}
\section{基准算例}
\subsection{静态三维IAEA PWR基准问题}
\label{sec:result.test.iaea}
\subsection{动态二维TWIGL基准问题}
\subsection{动态三维LMW基准问题}
\section{数值结果及验证}
\section{参数影响对比}




\chapter{大型扩散方程GPU求解方法研究}

本课题主要使用中子扩散方程作为实际研究问题,
中子扩散方程在实际求解中大多使用有限差分方式进行空间离散,
得到的线性方程组为7对角实对称对角占优矩阵(三维、单一能群内)。
稳态和动力学扩散方程的求解主要涉及到大型7对角线矩阵的最大特征值计算
和大型线性方程组求解。
本章则主要研究如何在GPU上高效地求解这类问题。

\section{稀疏矩阵格式选择}

为了比较各种存储格式的速度,使用三维扩散临界计算程序来进行测试,
测试算例为三维IAEA基准题(见\sectionref{sec:result.test.iaea}),
各存储格式的计算时间见\floatref{tab:equsolve.spformat}及
\floatref{fig:equsolve.spformat},图中DM表示双层网格加速(见\sectionref{sec:equsolve.multimesh}),
SP表示单精度、DP表示双精度,每种情况计算5次,时间取最小值。

结果很清楚地显示:对于三维扩散方程,DIA和ELL格式明显优于CSR和COO格式,其中DIA性能最高。
这主要是因为DIA和ELL较为适合向量机型处理器的运算和内存访问方式,
而且DIA格式占用的空间最少(因为矩阵正好是7对角线矩阵)。
在CG和BiCGStab计算中,矩阵参与的部分是稀疏矩阵-向量乘法(以下简称为SpMV),
SpMV在GPU上的主要瓶颈是显存带宽\cite{bell2008spmv,baskaran2008optimizing}\footnote{现在的GPU运算能力太强,使得显存带宽成为瓶颈。},
所以DIA性能最好是预料之中。

\begin{sidewaystable}
\pdfrotate
\centering
\begin{minipage}{.8\linewidth}
\centering
\caption[不同稀疏矩阵格式求解三维临界扩散的时间表]
{\label{tab:equsolve.spformat}%
不同稀疏矩阵格式求解三维临界扩散的时间表(单位:s)%
\footnote{不同的存储格式对非零元有着不同的求和顺序,由于浮点误差存在,%
不同的格式要达到收敛标准所需要的迭代次数可能有差别,并导致总计算时间的改变。}
}
\begin{tabular}{ccccccccc}
\toprule
 \multirow {3}{*}{矩阵格式}  &
       \multicolumn{4}{c}{2.5cm $\times$ 2.5cm $\times$ 2.5cm}
       &\multicolumn{4}{c}{1.25cm $\times$ 1.25cm $\times$ 1.25cm} \\
 &\multicolumn{2}{c}{CG\footnote{内迭代每轮18次,下同。}}
 	   &\multicolumn{2}{c}{BiCGStab\footnote{内迭代每轮30次,下同。}}
       & \multicolumn{2}{c}{CG}& \multicolumn{2}{c}{BiCGStab}\\
 & SP& DP& SP& DP& SP& DP& SP& DP\\
\midrule
 DIA&  5.475&  7.956& 12.199& 19.391& 24.071& 43.415&  64.256& 116.922\\
 ELL&  6.255&  8.596& 14.711& 20.997& 29.141& 47.362&  81.261& 133.599\\
 CSR& 10.452& 13.042& 27.814& 34.960& 62.478& 82.929& 180.430& 238.462\\
 COO& 11.232& 13.432& 30.934& 36.629& 64.303& 82.384& 197.263& 248.618\\
 DIA DM\footnote{粗网内迭代每轮10次,下同。}
       &  3.838&  4.883&  6.349&  8.003&  9.266& 14.976&  14.165&  31.590\\
 ELL DM&  4.399&  5.351&  7.363&  8.626& 10.920& 16.224&  17.503&  35.322\\
 CSR DM&  5.975&  6.989& 11.513& 14.337& 19.734& 26.301&  38.438&  61.776\\
 COO DM&  6.474&  7.660& 11.840& 13.712& 20.733& 26.364&  42.400&  65.910\\
\bottomrule
\end{tabular}
\end{minipage}
\end{sidewaystable}

\begin{figure}
\centering
\begin{asy}
import graph;
size(13cm,15cm,IgnoreAspect);
real[] x=sequence(8);
real[] DIA={5.475, 7.956, 12.199, 19.391, 24.071, 43.415, 64.256, 116.922};
real[] ELL={6.255, 8.596, 14.711, 20.997, 29.141, 47.362, 81.261, 133.599};
real[] CSR={10.452, 13.042, 27.814, 34.960, 62.478, 82.929, 180.430, 238.462};
real[] COO={11.232, 13.432, 30.934, 36.629, 64.303, 82.384, 197.263, 248.618};
real[] DIAMM={3.838, 4.883, 6.349, 8.003, 9.266, 14.976, 14.165, 31.590};
real[] ELLMM={4.399, 5.351, 7.363, 8.626, 10.920, 16.224, 17.503, 35.322};
real[] CSRMM={5.975, 6.989, 11.513, 14.337, 19.734, 26.301, 38.438, 61.776};
real[] COOMM={6.474, 7.660, 11.840, 13.712, 20.733, 26.364, 42.400, 65.910};
scale(Linear,Log);
string[] month={
"2.5cm CG SP",
"2.5cm CG DP",
"2.5cm BiCGStab SP",
"2.5cm BiCGStab DP",
"1.25cm CG SP",
"1.25cm CG DP",
"1.25cm BiCGStab SP",
"1.25cm BiCGStab DP",
};
transform markersize = scale(1.5mm);
draw(graph(x,DIA),legend="DIA", marker(markersize*polygon(3)));
draw(graph(x,ELL),legend="ELL", marker(markersize*polygon(4)));
draw(graph(x,CSR),legend="CSR", marker(markersize*unitcircle));
draw(graph(x,COO),legend="COO", marker(markersize*cross(4))  );
draw(graph(x,DIAMM),legend="DIA DM", dashed, marker(markersize*polygon(3)));
draw(graph(x,ELLMM),legend="ELL DM", dashed, marker(markersize*polygon(4)));
draw(graph(x,CSRMM),legend="CSR DM", dashed, marker(markersize*unitcircle));
draw(graph(x,COOMM),legend="COO DM", dashed, marker(markersize*cross(4))  );
xaxis(BottomTop,LeftTicks(rotate(90)*Label(),new string(real x) {
return month[round(x)];}));
yaxis("$T/\mathrm{s}$",LeftRight,RightTicks);
add(legend(),point(NW),10SE);
\end{asy}
\caption{\label{fig:equsolve.spformat}不同稀疏矩阵格式求解三维临界扩散的时间}
\end{figure}


\section{迭代及预条件算法选择}

为了比较不同迭代求解方法和预处理器的效果,这里仍然选择用IAEA基准题进行测试,
网格大小分别取5cm、2.5cm、1cm进行测试。

选择的算法包括
\begin{enumerate}
\item Jacobi-SG,逐群使用Jacobi迭代进行求解。
\item Jacobi-MG,使用Jacobi对所有能群统一求解。
\item CG-SG,逐群使用CG迭代进行求解。
\item BiCGStab-MG,使用BiCGStab对所有能群统一求解。
\item GMRES-MG,使用GMRES对所有能群统一求解。
\end{enumerate}

由于在临界计算的源迭代中一般在每步内迭代中精确求解方程组,
往往是迭代一个较少的次数,可以达到大幅减少计算时间的目的。
这个内迭代次数和问题的规模和迭代算法、预条件算法都有关系,
如何根据问题规模选择最优的内迭代次数不在文本的讨论之内,
为了公平地比较各种算法,以下将分别寻找各种情况下最优的内迭代次数。

\subsection{Jacobi-SG}
\label{sec:equsolve.iter.jacobi-sg}

Jacobi-SG算法网格大小分别取5cm、2.5cm、2cm、1cm的计算结果
见\floatref{tab:equsolve.iter.jacobi-sg.5cm}、%
\floatref{tab:equsolve.iter.jacobi-sg.2.5cm}、%
\floatref{tab:equsolve.iter.jacobi-sg.2cm}和%
\floatref{tab:equsolve.iter.jacobi-sg.1cm},
从表中可见最优的内迭代次数分别为7、11、16、18。

\begin{datasheet}
\sectionref{sec:equsolve.iter.jacobi-sg}的数据表:
\floatref{tab:equsolve.iter.jacobi-sg.5cm}、
\floatref{tab:equsolve.iter.jacobi-sg.2.5cm}、
\floatref{tab:equsolve.iter.jacobi-sg.2cm}、
\floatref{tab:equsolve.iter.jacobi-sg.1cm}
。

\begin{table}
\centering
\caption{5cm 网格时 Jacobi-SG 不同内迭代次数的计算时间及总迭代次数}
\label{tab:equsolve.iter.jacobi-sg.5cm}
\begin{tabular}{cccc}
\toprule
内迭代次数 & 计算时间/s & 总内迭代次数 & 外迭代次数\\
\midrule
%1 & 2.855 & 4138 & 2069\\
2 & 1.965 & 4180 & 1045\\
3 & 1.701 & 4266 & 711\\
4 & 1.576 & 4368 & 546\\
5 & 1.513 & 4500 & 450\\
6 & 1.497 & 4644 & 387\\
7 & 1.420 & 4802 & 343\\
8 & 1.435 & 4976 & 311\\
9 & 1.482 & 5220 & 290\\
10 & 1.607 & 5500 & 275\\
20 & 2.012 & 8440 & 211\\
30 & 2.683 & 11580 & 193\\
40 & 3.463 & 14960 & 187\\
50 & 4.181 & 18400 & 184\\
\bottomrule
\end{tabular}
\end{table}

\begin{table}
\centering
\caption{2.5cm 网格时 Jacobi-SG 不同内迭代次数的计算时间及总迭代次数}
\label{tab:equsolve.iter.jacobi-sg.2.5cm}
\begin{tabular}{cccc}
\toprule
内迭代次数 & 计算时间/s & 总内迭代次数 & 外迭代次数\\
\midrule
2 & 12.870 & 15780 & 3945\\
3 & 11.466 & 15810 & 2635\\
4 & 10.889 & 15832 & 1979\\
5 & 9.906 & 15880 & 1588\\
6 & 9.547 & 15936 & 1328\\
7 & 9.516 & 15988 & 1142\\
8 & 9.391 & 16064 & 1004\\
9 & 9.376 & 16146 & 897\\
10 & 9.297 & 16240 & 812\\
11 & 9.048 & 16324 & 742\\
12 & 9.141 & 16440 & 685\\
13 & 9.329 & 16536 & 636\\
14 & 9.453 & 16660 & 595\\
15 & 9.048 & 16770 & 559\\
16 & 9.204 & 16896 & 528\\
17 & 9.141 & 17034 & 501\\
18 & 9.641 & 17172 & 477\\
19 & 9.594 & 17328 & 456\\
20 & 9.469 & 17480 & 437\\
30 & 10.078 & 19080 & 318\\
40 & 11.373 & 21840 & 273\\
50 & 12.776 & 24700 & 247\\
\bottomrule
\end{tabular}
\end{table}

\begin{table}
\centering
\caption{2cm 网格时 Jacobi-SG 不同内迭代次数的计算时间及总迭代次数}
\label{tab:equsolve.iter.jacobi-sg.2cm}
\begin{tabular}{cccc}
\toprule
内迭代次数 & 计算时间/s & 总内迭代次数 & 外迭代次数\\
\midrule
11 & 23.415 & 25256 & 1148\\
12 & 23.072 & 25320 & 1055\\
13 & 23.073 & 25428 & 978\\
14 & 23.493 & 25480 & 910\\
15 & 23.181 & 25590 & 853\\
16 & 22.948 & 25664 & 802\\
17 & 23.119 & 25772 & 758\\
18 & 23.104 & 25884 & 719\\
19 & 23.135 & 25992 & 684\\
20 & 23.150 & 26120 & 653\\
30 & 23.868 & 27420 & 457\\
40 & 25.756 & 29040 & 363\\
50 & 26.333 & 30800 & 308\\
\bottomrule
\end{tabular}
\end{table}


\begin{table}
\centering
\caption{1cm 网格时 Jacobi-SG 不同内迭代次数的计算时间及总迭代次数}
\label{tab:equsolve.iter.jacobi-sg.1cm}
\begin{tabular}{cccc}
\toprule
内迭代次数 & 计算时间/s & 总内迭代次数 & 外迭代次数\\
\midrule
14 & 518.420 & 96768 & 3456\\
16 & 511.619 & 96832 & 3026\\
18 & 509.309 & 96804 & 2689\\
20 & 512.507 & 96920 & 2423\\
30 & 511.556 & 97260 & 1621\\
40 & 510.714 & 97840 & 1223\\
50 & 510.136 & 98600 & 986\\
60 & 513.568 & 99480 & 829\\
70 & 521.774 & 100520 & 718\\
80 & 526.875 & 101600 & 635\\
90 & 528.966 & 102780 & 571\\
100 & 534.348 & 104000 & 520\\
\bottomrule
\end{tabular}
\end{table}

\end{datasheet}



\subsection{CG-SG}
\label{sec:equsolve.iter.cg-sg}

CG-SG算法网格大小分别取5cm、2.5cm、2cm、1cm的计算结果
见\floatref{tab:equsolve.iter.cg-sg.5cm}、%
\floatref{tab:equsolve.iter.cg-sg.2.5cm}、%
\floatref{tab:equsolve.iter.cg-sg.2cm}和%
\floatref{tab:equsolve.iter.cg-sg.1cm},
从表中可见最优的内迭代次数分别为4、7、10、17。


\begin{datasheet}
\sectionref{sec:equsolve.iter.cg-sg}的数据表:
\floatref{tab:equsolve.iter.cg-sg.5cm}、
\floatref{tab:equsolve.iter.cg-sg.2.5cm}、
\floatref{tab:equsolve.iter.cg-sg.2cm}、
\floatref{tab:equsolve.iter.cg-sg.1cm}
。

\begin{table}
\centering
\caption{5cm 网格时 CG-SG 不同内迭代次数的计算时间及总迭代次数}
\label{tab:equsolve.iter.cg-sg.5cm}
\begin{tabular}{cccc}
\toprule
内迭代次数 & 计算时间/s & 总内迭代次数 & 外迭代次数\\
\midrule
%1 & 4.540 & 4048 & 2024\\
2 & 1.576 & 1960 & 490\\
3 & 1.046 & 1464 & 244\\
4 & 0.920 & 1472 & 184\\
5 & 1.092 & 1840 & 184\\
6 & 1.186 & 2196 & 183\\
7 & 1.342 & 2562 & 183\\
8 & 1.467 & 2912 & 182\\
9 & 1.622 & 3276 & 182\\
10 & 1.685 & 3640 & 182\\
11 & 1.841 & 3988 & 182\\
12 & 1.966 & 4305 & 182\\
13 & 2.138 & 4595 & 182\\
14 & 2.246 & 4869 & 182\\
15 & 2.450 & 5124 & 182\\
16 & 2.372 & 5362 & 182\\
17 & 2.512 & 5576 & 182\\
18 & 2.559 & 5746 & 182\\
19 & 2.684 & 5888 & 182\\
20 & 2.792 & 6009 & 182\\
\bottomrule
\end{tabular}
\end{table}

\begin{table}
\centering
\caption{2.5cm 网格时 CG-SG 不同内迭代次数的计算时间及总迭代次数}
\label{tab:equsolve.iter.cg-sg.2.5cm}
\begin{tabular}{cccc}
\toprule
内迭代次数 & 计算时间/s & 总内迭代次数 & 外迭代次数\\
\midrule
%1 & 31.309 & 16256 & 8128\\
2 & 9.999 & 7472 & 1868\\
3 & 5.694 & 4740 & 790\\
4 & 3.837 & 3504 & 438\\
5 & 3.120 & 2910 & 291\\
6 & 2.855 & 2856 & 238\\
7 & 2.761 & 2800 & 200\\
8 & 2.792 & 2960 & 185\\
9 & 3.042 & 3312 & 184\\
10 & 3.385 & 3680 & 184\\
11 & 3.572 & 4026 & 183\\
12 & 3.947 & 4392 & 183\\
13 & 4.181 & 4758 & 183\\
14 & 4.493 & 5096 & 182\\
15 & 4.727 & 5460 & 182\\
16 & 5.070 & 5824 & 182\\
17 & 5.460 & 6188 & 182\\
18 & 5.787 & 6552 & 182\\
19 & 5.912 & 6916 & 182\\
20 & 6.146 & 7280 & 182\\
\bottomrule
\end{tabular}
\end{table}

\begin{table}
\centering
\caption{2cm 网格时 CG-SG 不同内迭代次数的计算时间及总迭代次数}
\label{tab:equsolve.iter.cg-sg.2cm}
\begin{tabular}{cccc}
\toprule
内迭代次数 & 计算时间/s & 总内迭代次数 & 外迭代次数\\
\midrule
%1 & 139.698 & 49450 & 24725\\
2 & 23.384 & 10940 & 2735\\
3 & 13.260 & 7410 & 1235\\
4 & 9.048 & 5528 & 691\\
5 & 6.927 & 4390 & 439\\
6 & 6.068 & 3864 & 322\\
7 & 5.757 & 3808 & 272\\
8 & 5.367 & 3696 & 231\\
9 & 5.726 & 3924 & 218\\
10 & 5.476 & 3700 & 185\\
11 & 5.897 & 4048 & 184\\
12 & 6.396 & 4416 & 184\\
13 & 6.739 & 4784 & 184\\
14 & 7.067 & 5124 & 183\\
15 & 7.395 & 5490 & 183\\
16 & 7.831 & 5856 & 183\\
17 & 8.611 & 6222 & 183\\
18 & 8.939 & 6588 & 183\\
19 & 8.939 & 6916 & 182\\
20 & 9.376 & 7280 & 182\\
\bottomrule
\end{tabular}
\end{table}


\begin{table}
\centering
\caption{1cm 网格时 CG-SG 不同内迭代次数的计算时间及总迭代次数}
\label{tab:equsolve.iter.cg-sg.1cm}
\begin{tabular}{cccc}
\toprule
内迭代次数 & 计算时间/s & 总内迭代次数 & 外迭代次数\\
\midrule
2 & 445.958 & 43336 & 10834\\
3 & 252.955 & 28458 & 4743\\
4 & 186.670 & 22616 & 2827\\
5 & 141.945 & 18000 & 1800\\
6 & 113.818 & 14928 & 1244\\
7 & 93.303 & 12558 & 897\\
8 & 78.983 & 10784 & 674\\
9 & 69.108 & 9594 & 533\\
10 & 60.591 & 8480 & 424\\
11 & 61.776 & 8580 & 390\\
12 & 53.102 & 7536 & 314\\
13 & 56.269 & 8034 & 309\\
14 & 50.981 & 7336 & 262\\
15 & 51.839 & 7500 & 250\\
16 & 50.357 & 7200 & 225\\
17 & 48.875 & 7106 & 209\\
18 & 52.806 & 7740 & 215\\
19 & 49.093 & 7106 & 187\\
20 & 50.216 & 7400 & 185\\
\bottomrule
\end{tabular}
\end{table}

\end{datasheet}


\subsection{BiCGStab-MG}
\label{sec:equsolve.iter.bicgstab-mg}

BiCGStab-MG算法网格大小分别取5cm、2.5cm、2cm、1cm的计算结果
见\floatref{tab:equsolve.iter.bicgstab-mg.5cm}、%
\floatref{tab:equsolve.iter.bicgstab-mg.2.5cm}、%
\floatref{tab:equsolve.iter.bicgstab-mg.2cm}和%
\floatref{tab:equsolve.iter.bicgstab-mg.1cm},
从表中可见最优的内迭代次数分别为3、8、10、21。


\begin{datasheet}
\sectionref{sec:equsolve.iter.bicgstab-mg}的数据表:
\floatref{tab:equsolve.iter.bicgstab-mg.5cm}、
\floatref{tab:equsolve.iter.bicgstab-mg.2.5cm}、
\floatref{tab:equsolve.iter.bicgstab-mg.2cm}、
\floatref{tab:equsolve.iter.bicgstab-mg.1cm}
。

\begin{table}
\centering
\caption{5cm 网格时 BiCGStab-MG 不同内迭代次数的计算时间及总迭代次数}
\label{tab:equsolve.iter.bicgstab-mg.5cm}
\begin{tabular}{cccc}
\toprule
内迭代次数 & 计算时间/s & 总内迭代次数 & 外迭代次数\\
\midrule
%1 & \multicolumn{3}{c}{不收敛} \\ %\footnote{Fail:Nan: KeffErr, PhiErr, }
2 & \multicolumn{3}{c}{不收敛} \\ %\footnote{Fail:Nan: KeffErr, PhiErr, }
3 & 0.998 & 624 & 208\\
4 & 1.108 & 752 & 188\\
5 & 1.264 & 915 & 183\\
6 & 1.419 & 1086 & 181\\
7 & 1.607 & 1267 & 181\\
8 & 1.763 & 1448 & 181\\
9 & 1.919 & 1629 & 181\\
10 & 2.091 & 1820 & 182\\
11 & 2.340 & 2013 & 183\\
12 & 2.434 & 2160 & 180\\
13 & 2.589 & 2366 & 182\\
14 & 2.777 & 2548 & 182\\
15 & 2.980 & 2730 & 182\\
16 & 3.214 & 2912 & 182\\
17 & 3.292 & 3094 & 182\\
18 & 3.448 & 3276 & 182\\
19 & 3.572 & 3458 & 182\\
20 & 3.869 & 3640 & 182\\
\bottomrule
\end{tabular}
\end{table}

\begin{table}
\centering
\caption{2.5cm 网格时 BiCGStab-MG 不同内迭代次数的计算时间及总迭代次数}
\label{tab:equsolve.iter.bicgstab-mg.2.5cm}
\begin{tabular}{cccc}
\toprule
内迭代次数 & 计算时间/s & 总内迭代次数 & 外迭代次数\\
\midrule
%1 & 3.900 & 900 & 900\\
2-5 & \multicolumn{3}{c}{不收敛} \\ %\footnote{Fail:Nan: KeffErr, PhiErr, }
6(超时) & >600 & >241926 & >40321 \\ %\footnote{Fail:Solve Time exceeds 600.000}
7 & 4.181 & 1638 & 234\\
8 & 4.056 & 1600 & 200\\
9 & 4.134 & 1647 & 183\\
10 & 4.446 & 1800 & 180\\
11 & 4.898 & 1991 & 181\\
12 & 5.132 & 2124 & 177\\
13 & 5.445 & 2275 & 175\\
14 & 5.834 & 2450 & 175\\
15 & 6.194 & 2625 & 175\\
16 & 6.521 & 2800 & 175\\
17 & 6.942 & 2992 & 176\\
18 & 7.332 & 3186 & 177\\
19 & 7.894 & 3401 & 179\\
20 & 8.222 & 3600 & 180\\
\bottomrule
\end{tabular}
\end{table}

\begin{table}
\centering
\caption{2cm 网格时 BiCGStab-MG 不同内迭代次数的计算时间及总迭代次数}
\label{tab:equsolve.iter.bicgstab-mg.2cm}
\begin{tabular}{cccc}
\toprule
内迭代次数 & 计算时间/s & 总内迭代次数 & 外迭代次数\\
\midrule
%1 & 12.152 & 1766 & 1766\\
2-7 & \multicolumn{3}{c}{不收敛} \\ %\footnote{Fail:Nan: KeffErr, PhiErr, }
8(超时) & >600 & >142232 & >17779 \\ %\footnote{Fail:Solve Time exceeds 600.000}
9 & 10.250 & 2412 & 268\\
10 & 8.954 & 2120 & 212\\
11 & 9.360 & 2233 & 203\\
12 & 9.313 & 2232 & 186\\
13 & 9.641 & 2301 & 177\\
14 & 10.452 & 2534 & 181\\
15 & 10.857 & 2655 & 177\\
16 & 11.388 & 2784 & 174\\
17 & 11.840 & 2941 & 173\\
18 & 12.527 & 3114 & 173\\
19 & 13.292 & 3287 & 173\\
20 & 13.900 & 3460 & 173\\
\bottomrule
\end{tabular}
\end{table}


\begin{table}
\centering
\caption{1cm 网格时 BiCGStab-MG 不同内迭代次数的计算时间及总迭代次数}
\label{tab:equsolve.iter.bicgstab-mg.1cm}
\begin{tabular}{cccc}
\toprule
内迭代次数 & 计算时间/s & 总内迭代次数 & 外迭代次数\\
\midrule
%1 & 600.929 & 16239 & 16239 \\ %Fail:Solve Time exceeds 600.000
2-17 & \multicolumn{3}{c}{不收敛} \\ %Fail:Nan: KeffErr, PhiErr,
18(超时) & >600 & >25650 & >1425 \\ %Fail:Solve Time exceeds 600.000
19 & 258.804 & 11134 & 586\\
20 & 154.456 & 6580 & 329\\
21 & 100.339 & 4305 & 205\\
22 & 105.627 & 4554 & 207\\
23 & 106.673 & 4554 & 198\\
24 & 107.936 & 4656 & 194\\
25 & 109.965 & 4725 & 189\\
26 & 115.472 & 4992 & 192\\
27 & 112.991 & 4887 & 181\\
28 & 120.354 & 5180 & 185\\
29 & 124.020 & 5336 & 184\\
30 & 129.511 & 5580 & 186\\
\bottomrule
\end{tabular}
\end{table}

\end{datasheet}




\section{双层网格加速}
\label{sec:equsolve.multimesh}




\chapter{结论与展望}

\Closesolutionfile{datasheetfile}

%%% 其它部分
\backmatter



% 参考文献
\bibliographystyle{thubib}
\bibliography{ref/refs}


% 致谢
%

\begin{ack}
  衷心感谢导师余纲林老师和核能所所长王侃教授对本人的指导。
  余纲林老师在学术上和生活中的悉心指导让我终生受益,
  王侃教授的严格要求使我受益匪浅,
  在此我衷心地感谢他们。
  如果没有余纲林老师和王侃教授,我肯定取得不了这样的成绩。
  
  此外还要感谢工物系核能所的张鹏、李林森、李泽光、佘顶、
  徐琪、余健开等众位师兄和同学对我的帮助,
  在本课题的研究过程中,他们都或多或少的参与了讨论,
  给了我不少启发和帮助,在这里我也向他们表示感谢。
  
  最后,还要感谢 \thuthesis,它让我的论文写作过程轻松了很多。

\end{ack}


% 附录
\begin{appendix}

\chapter{原始数据表}

\input{datasheetfile}

\iffalse

程序性能结果应使用几何平均数。\cite{fleming1986not}

\chapter{课题程序使用说明}
\section{运行环境}
\section{输入文件格式}
\section{输出文件格式}
\subsection{HDF5简介}



\TODO 使用简介

\subsection{Python简介}
\subsection{后处理脚本示例}
\section{编译说明}

\fi

%%%% Local Variables: 
%%% mode: latex
%%% TeX-master: "../main"
%%% End: 

\chapter{TEMP}

程序性能结果应使用几何平均数。\cite{fleming1986not}

\end{appendix}

% 个人简历
%\begin{resume}

  \resumeitem{个人简历}

  1988 年 5 月 11 日出生于黑龙江省省齐齐哈尔市。
  
  2007 年 9 月考入清华大学工程物理系工程物理专业,2011年 7 月本科毕业并获得工学学士学位。
  
  2011 年 9 月免试进入清华大学工程物理系攻读硕士学位至今。

  \resumeitem{发表的学术论文} % 发表的和录用的合在一起

  \begin{enumerate}[{[}1{]}]
  \item 
  \end{enumerate}

\end{resume}

\end{document}
