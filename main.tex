%%% Local Variables:
%%% mode: latex
%%% TeX-master: t
%%% End:

\documentclass[master]{thuthesis}
% \documentclass[%
%   bachelor|master|doctor|postdoctor, % mandatory option
%   winfonts|nofonts|adobefonts, % mandatory only for bachelor and Linuxer
%   secret,
%   openany|openright,
%   arialtoc,arialtitle]{thuthesis}
% 当使用 XeLaTeX 编译时,本科生、Linux 用户需要加上 nofonts 选项;
% 当使用 PDFLaTeX 编译时,adobefonts 选项等效于 winfonts 选项(缺省选项)。

% 所有其它可能用到的包都统一放到这里了,可以根据自己的实际添加或者删除。
\usepackage[
addfootnotetoref
]{thutils}

\usepackage[xetex,hyperref]{xcolor}

\usepackage{tikz}
\usetikzlibrary{arrows.new}
\usetikzlibrary{decorations.pathreplacing}
\usetikzlibrary{datavisualization}


\usepackage[linesnumbered,boxed,algochapter,vlined]{algorithm2e}
\renewcommand\algorithmautorefname{算法}
\renewcommand{\algorithmcfname}{算法}
\makeatletter
\def\setlabelname#1{\def\@currentlabelname{#1}}
\makeatother

\usepackage{url}
\def\UrlBreaks{\do\/\do\.\do\-\do\#}
%\def\UrlBreaks{\do\A\do\B\do\C\do\D\do\E\do\F\do\G\do\H\do\I\do\J%
%\do\K\do\L\do\M\do\N\do\O\do\P\do\Q\do\R\do\S\do\T\do\U\do\V%
%\do\W\do\X\do\Y\do\Z\do\[\do\\\do\]\do\^\do\_\do\`\do\a\do\b%
%\do\c\do\d\do\e\do\f\do\g\do\h\do\i\do\j\do\k\do\l\do\m\do\n%
%\do\o\do\p\do\q\do\r\do\s\do\t\do\u\do\v\do\w\do\x\do\y\do\z%
%\do\.\do\@\do\\\do\/\do\!\do\_\do\|\do\;\do\>\do\]\do\)\do\,%
%\do\?\do\'\do+\do\=\do\#}

\usepackage[figuresright]{rotating}
\def\pdfrotate{\special{pdf: put @thispage <</Rotate 90>>}}

\usepackage{multirow}

\usepackage{answers}
\newenvironment{datasheetenv}[1]{}{}
\Newassociation{datasheet}{datasheetenv}{datasheetfile}

\usepackage{placeins}

\usepackage{multicol}

\usepackage{textcomp} %千分号

\newcommand{\TODO}{ \textcolor{blue}{TODO} }
\colorlet{BLUE}{blue}

%--------------------------------------------------------------------------
% 自定义函数

%bracket系列
\newcommand{\bracket}[4]
{\ensuremath{%
\ifthenelse{\equal{#1}{n}}{#3 #2 #4}{}%
\ifthenelse{\equal{#1}{b}}{\bigl#3 #2 \bigr#4}{}%
\ifthenelse{\equal{#1}{B}}{\Bigl#3 #2 \Bigr#4}{}%
\ifthenelse{\equal{#1}{bg}}{\biggl#3 #2 \biggr#4}{}%
\ifthenelse{\equal{#1}{Bg}}{\Biggl#3 #2 \Biggr#4}{}%
}}

\newcommand{\pbracket}[2]{\bracket{#1}{#2}{(}{)}}
\newcommand{\Sbracket}[2]{\bracket{#1}{#2}{[}{]}}
\newcommand{\Bbracket}[2]{\bracket{#1}{#2}{\lbrace}{\rbrace}}
\newcommand{\vbracket}[2]{\bracket{#1}{#2}{|}{|}}

\newcommand{\getsize}[2]
{%
\ifthenelse{\equal{#1}{n}}{#2}{}%
\ifthenelse{\equal{#1}{b}}{\big#2}{}%
\ifthenelse{\equal{#1}{B}}{\Big#2}{}%
\ifthenelse{\equal{#1}{bg}}{\bigg#2}{}%
\ifthenelse{\equal{#1}{Bg}}{\Bigg#2}{}%
}

\newcommand{\pb}[2][n]{\pbracket{#1}{#2}}
\newcommand{\Sb}[2][n]{\Sbracket{#1}{#2}}
\newcommand{\Bb}[2][n]{\Bbracket{#1}{#2}}
\newcommand{\vb}[2][n]{\vbracket{#1}{#2}}

\newcommand{\diff}[1]{\mathrm{d}#1}

\usepackage{datetime}
\newdateformat{mydate}{\THEYEAR-\twodigit{\THEMONTH}-\twodigit{\THEDAY}}
\newtimeformat{mytime}{\twodigit{\THEHOUR}:\twodigit{\THEMINUTE}}
\settimeformat{mytime}

\usepackage[draft=true,allpages=true]{draftmark}
\draftmarksetup{angle=0,grayness=0.5,xcoord=50,ycoord=-137,fontsize=12pt,
mark={DRAFT \mydate\today\hspace{5pt} \currenttime}}

%只显示三层目录
\setcounter{tocdepth}{2}

\begin{document}

% 定义所有的eps文件在 figures 子目录下
\graphicspath{{figures/}}


%%% 封面部分
\frontmatter

%%% Local Variables:
%%% mode: latex
%%% TeX-master: t
%%% End:
\secretlevel{绝密} \secretyear{2100}

\ctitle{清华大学学位论文 \LaTeX\ 模板\\使用示例文档}
% 根据自己的情况选,不用这样复杂
\makeatletter
\ifthu@bachelor\relax\else
  \ifthu@doctor
    \cdegree{工学博士}
  \else
    \ifthu@master
      \cdegree{工学硕士}
    \fi
  \fi
\fi
\makeatother


\cdepartment[计算机]{计算机科学与技术系}
\cmajor{计算机科学与技术}
\cauthor{薛瑞尼} 
\csupervisor{郑纬民教授}
% 如果没有副指导老师或者联合指导老师,把下面两行相应的删除即可。
\cassosupervisor{陈文光教授}
\ccosupervisor{某某某教授}
% 日期自动生成,如果你要自己写就改这个cdate
%\cdate{\CJKdigits{\the\year}年\CJKnumber{\the\month}月}

% 博士后部分
% \cfirstdiscipline{计算机科学与技术}
% \cseconddiscipline{系统结构}
% \postdoctordate{2009年7月——2011年7月}

\etitle{An Introduction to \LaTeX{} Thesis Template of Tsinghua University} 
% 这块比较复杂,需要分情况讨论:
% 1. 学术型硕士
%    \edegree:必须为Master of Arts或Master of Science(注意大小写)
%              “哲学、文学、历史学、法学、教育学、艺术学门类,公共管理学科
%               填写Master of Arts,其它填写Master of Science”
%    \emajor:“获得一级学科授权的学科填写一级学科名称,其它填写二级学科名称”
% 2. 专业型硕士
%    \edegree:“填写专业学位英文名称全称”
%    \emajor:“工程硕士填写工程领域,其它专业学位不填写此项”
% 3. 学术型博士
%    \edegree:Doctor of Philosophy(注意大小写)
%    \emajor:“获得一级学科授权的学科填写一级学科名称,其它填写二级学科名称”
% 4. 专业型博士
%    \edegree:“填写专业学位英文名称全称”
%    \emajor:不填写此项
\edegree{Doctor of Engineering} 
\emajor{Computer Science and Technology} 
\eauthor{Xue Ruini} 
\esupervisor{Professor Zheng Weimin} 
\eassosupervisor{Chen Wenguang} 
% 这个日期也会自动生成,你要改么?
% \edate{December, 2005}

% 定义中英文摘要和关键字
\begin{cabstract}
  论文的摘要是对论文研究内容和成果的高度概括。摘要应对论文所研究的问题及其研究目
  的进行描述,对研究方法和过程进行简单介绍,对研究成果和所得结论进行概括。摘要应
  具有独立性和自明性,其内容应包含与论文全文同等量的主要信息。使读者即使不阅读全
  文,通过摘要就能了解论文的总体内容和主要成果。

  论文摘要的书写应力求精确、简明。切忌写成对论文书写内容进行提要的形式,尤其要避
  免“第 1 章……;第 2 章……;……”这种或类似的陈述方式。

  本文介绍清华大学论文模板 \thuthesis{} 的使用方法。本模板符合学校的本科、硕士、
  博士论文格式要求。

  本文的创新点主要有:
  \begin{itemize}
    \item 用例子来解释模板的使用方法;
    \item 用废话来填充无关紧要的部分;
    \item 一边学习摸索一边编写新代码。
  \end{itemize}

  关键词是为了文献标引工作、用以表示全文主要内容信息的单词或术语。关键词不超过 5
  个,每个关键词中间用分号分隔。(模板作者注:关键词分隔符不用考虑,模板会自动处
  理。英文关键词同理。)
\end{cabstract}

\ckeywords{\TeX, \LaTeX, CJK, 模板, 论文}

\begin{eabstract} 
   An abstract of a dissertation is a summary and extraction of research work
   and contributions. Included in an abstract should be description of research
   topic and research objective, brief introduction to methodology and research
   process, and summarization of conclusion and contributions of the
   research. An abstract should be characterized by independence and clarity and
   carry identical information with the dissertation. It should be such that the
   general idea and major contributions of the dissertation are conveyed without
   reading the dissertation. 

   An abstract should be concise and to the point. It is a misunderstanding to
   make an abstract an outline of the dissertation and words ``the first
   chapter'', ``the second chapter'' and the like should be avoided in the
   abstract.

   Key words are terms used in a dissertation for indexing, reflecting core
   information of the dissertation. An abstract may contain a maximum of 5 key
   words, with semi-colons used in between to separate one another.
\end{eabstract}

\ekeywords{\TeX, \LaTeX, CJK, template, thesis}

%\makecover

% 目录
\tableofcontents

% 符号对照表
%\begin{denotation}

\item[HDF] Hierarchical Data Format
\item[GPU] Graphics Processing Unit
\item[CPU] Central processing unit
\item[CG] Conjugate Gradient
\item[\ProgramName] \ProgramFullName

\item[PWR] Pressurized Water Reactor
\item[API] Application Programming Interface
\item[CUDA] Compute Unified Device Architecture
\item[MPI] Message Passing Interface
\item[SIMD] Single instruction, multiple data
\item[MIMD] multiple instruction, multiple data
\item[ADI] Alternating direction implicit method
\item[SM] Streaming Multiprocessor
\item[SMX] Next Generation Streaming Multiprocessor
\item[PTX] Parallel Thread Execution
\item[ISA] Instruction Set Architecture
\item[JIT] Just-in-time Compiler
\item[GMRES] Generalized Minimal RESidual method

\end{denotation}


\Opensolutionfile{datasheetfile}
%%% 正文部分
\mainmatter

\def\ProgramName{RDGS}
\def\ProgramFullName{Real Diffusion GPU Solver}


\chapter{引言}
\section{研究背景与意义}

自2011年4月日本福岛地震及其引发福岛第一核电站受损以来,
核能的安全性又再次为世界各方所关注,
全世界及整个核能行业对核能安全性的要求进一步提高,
处在核能行业内的反应堆物理方向也面临着
不断提高的计算精度和计算速度的巨大需求。

一直以来,反应堆物理计算中最为主要的中子输运方程的求解方法
分为两条线:对输运方程进行简化并离散为线性方程组相关问题的确定论方法
和直接使用蒙特卡罗方法对中子输运方程及实际中子反应过程进行联合求解的蒙特卡罗方法。
后者主要依赖于蒙特卡罗方法对高维偏微分方程的直接求解能力,
但同样也面临着蒙特卡罗方法计算时间较长、误差具有随机性、
提高计算精度的代价巨大等等问题。
这方面的代表程序是有着悠久历史的MCNP程序\cite{forster2004mcnp},
至今已发展到MCNP5,不久后MCNP6也将发布;
国内近几年也有不少蒙特卡罗程序正在开发,
比较有代表性的是清华大学工程物理系核能所开发的
针对反应堆物理计算的蒙特卡罗程序RMC\cite{li2010development}。
由于蒙特卡罗方法需要的计算成本相当高,
但却有相当高的天然粒子并行特性,适合多机多节点并行计算,
所以直至近些年计算机技术充分发展后才具有全堆计算能力,
目前蒙特卡罗程序全堆精细计算仍常在大型机或超级计算机上进行,
如文献\onlinecite{she2013development}。

确定论方法方面有三个主要分支:
不考虑角度的扩散近似方法,$P_N$、$S_N$等角度离散方法,
能够处理任意几何的特征线方法。
这些方法在空间上大多使用有规则网格或有限元不规则网格的空间离散方式。
也有自适应结构网格方面的工作,如文献\onlinecite{wang2009three},
研究了三维扩散方程的自适应结构网格方法。
虽然特征线方法在理论上可以支持任意几何,
但不少这方面的工作由于工作量或加速方面的原因而对几何模型进行限制。
在能量方面,确定论方法一般只做分群近似,
非几何、角度方面的因素则大多交给群截面构造方面处理。
即使做了上述各种近似,确定论方法面对全堆问题时仍然面临计算量、
存储量太大的问题,\cite{azmy1997multiprocessing}
所以在此基础上发展了各种节块方法来减少空间网格数量。
目前工程中全堆计算大多采用组件-堆芯的两步法,
即先用细群输运对全反射边界组件进行求解,得到组件的均匀化少群截面,再由三维堆芯程序求解。
两步法的问题在于求解时使用的组件的边界条件不准确,由此带来误差。
确定论方法面对堆芯级问题的细网直接求解仍然未见较为完整的解决方案。
而随着对于反应堆安全的关注越来越多,要求越来越高,
对于细网直接求解的需求也在不断增加。

一直以来,大规模线性方程的求解和矩阵特征值的计算都依赖于大型机或超级计算机
等多节点分布式计算系统才能在有效的时间内进行求解,
但随着个人计算机技术的不断发展,传统用于显示加速的图形加速卡(现一般称为显卡)
的计算能力和IO(Input/Output)能力的不断增强,
使其已经能够满足某些科学计算或通用计算的需求。
由此发展出了GPU并行化路线,GPU是指显卡的核心部分——图形处理单元。
2013年,民用显卡Nvidia Geforce GTX Titan的峰值单精度浮点计算能力已经高达4.7TFlop/s,
远高于现在CPU单芯片的计算能力。
而2000年6月的TOP500 Super Computer \cite{meuer2001top500}
排行世界第一的超级计算机的峰值浮点性能也只有3.2TFlop/s
(见 \url{http://www.top500.org/site/48748})。
如何利用这样强大的计算能力的问题已经早就摆在了各个行业的面前。

\section{国内外研究现状}

\subsection{反应堆物理GPU并行化研究}
\subsubsection{确定论方法的GPU并行化}

确定论的GPU并行化最早见于2010年,
文献\onlinecite{prayudhatama2010gpu}首先对
一维扩散程序的GPU并行化进行了探索。%相对于文中提供的CPU对比程序获得了最高70倍的加速。
此后文献\onlinecite{kodamastudy}和\onlinecite{xuqi_gpu_old}(国内)分别对
三维扩散的GPU并行化进行了研究,
相对于实际CPU扩散程序取得了3-5倍的加速效果。

2011年,国内龚春叶等人\cite{gong2011gpu,gong2012particle,gongyechun}
和文献\onlinecite{jamelot2011high}
对三维$S_N$方法的GPU并行化进行了研究,
前者对于三维$S_N$方法相对于文中提供的对比程序有2-8倍左右的加速。
2012年Oak Ridge National Laboratory研究了GPU超级计算机上的
三维$S_N$方法大规模并行化,实现了35TFlop/s的计算能力,
同时指出$S_N$方法的并行可扩展性较差。\cite{baker2012high}

此外文献\onlinecite{jamelot2011high,kirschenmann2011parallel,kodamastudy}
等对三维$\mathrm{S}P_N$方法的并行化进行了研究,
相对于他们的CPU串行实现取得了最高36的加速。

2013年,文献\onlinecite{talamo2013numerical}和\onlinecite{zhangzhizhu}(国内)实现了
三维特征线方法的GPU加速,后者最高取得了将近100倍的加速效果。

\subsubsection{蒙特卡罗方法的GPU并行化}

2009年,文献\onlinecite{nelson2009monte}较早地研究了
蒙特卡罗方法的GPU并行化,并取得了11-20倍的加速效果。
此后国内的龚春叶实现了MCNP的GPU并行化\cite{gongyechun},
实现了10倍左右的加速。
近些年来蒙卡方法的GPU并行化仍在不断发展中,
相关的文献较多,如\onlinecite{ding2011evaluation,willert2012hybrid,
liu_monte_2012,marcus2012mcmini,xuqians}等,
结果大多加速比有限或对模型有简化处理。
蒙特卡罗算法的主要问题是,程序行为是伪随机的,
很难对其进行静态预测,此外还有:不规则的内存访问,
控制流动态不规则改变,行为和并行度显著依赖于输入等特点。
这些都给蒙特卡罗方法的GPU并行化带来了较大困难,
导致直接进行GPU移植的效果很差。\cite{Martin2013}
蒙特卡罗方法的GPU并行化想要取得较好的成果可能需要对
传统计算方式做较大的修改。\cite{lichenglong}

\subsection{数值加速方法}

确定论方法将中子输运方程离散为线性方程组后,
就需要对其进行求解,对于三维问题,方程组的带求解变量数很大,
为了在合理的时间内进行求解则势必进行加速。

扩散方法离散后的方程为主对角占优线性方程的求解或本征值计算,
可使用数值计算领域的研究成果。输运问题如$S_N$方法或特征线方法
虽然同样离散为线性方程组,但由于其物理特性有中子飞行方向的概念,
对于固定方向往往可以按某种空间网格顺序依次解出,
即系数矩阵是分块三角的。
一般求解过程中会利用这种特性,而不是直接套用通用方法求解,
但遗憾的是这种类似于高斯消去法中回代操作的过程串行度很高,
即各元素间的计算次序依赖很多,不利于GPU并行。
对于输运问题,文献\onlinecite{alcouffe1977diffusion}提出了使用扩散方法加速
输运计算的扩散综合加速DSA(Diffusion Synthetic Acceleration)方法。

除此以外,反应堆物理领域中还大量使用了各种节块类方法,
其主要思想是通过提高每个空间网格对通量分布的描述能力来实现同等精度下
大量减少空间网格数目,但节块法不能直接产生细网网格上的通量分布,
需要通过通量重构等方法产生,这都引入了一定程度的误差。

对于扩散方程,早在1989年文献\onlinecite{suetomi1989conjugate}
就尝试对二维圆柱几何应用共轭梯度法CG(Conjugate Gradient),
并研究了不同预条件算法对收敛次数的影响。
文献\onlinecite{suetomi1991conjugate}和\onlinecite{gupta_krylov_2004}
分别研究了CG等Krylov类方法对于扩散计算中k本征值问题的求解,
并对源迭代过程进行了改进。
1998年,文献\onlinecite{so1998mapping}对于二维扩散问题研究了
CG方法在SIMD、MIMD等向量机、并行机上的性能。

除了迭代算法外,文献\onlinecite{ginestar2001multilevel}提出了
Multilevel方法,即通过在粗网上进行预先求解来给细网求解过程提供一个更为精确的通量分布初值,
利用粗网格求解代价较小的特点来加速细网计算。

\subsection{动力学}

扩散动力学实时,三维13440个网格,0.25s实时。\cite{宋英明2010}

三维节块准静态,1936个节块,最高55倍运行。\cite{丁小川2011}

三维准静态,1452个网格,对于0.1s达到实时。\cite{齐克林1996}


\subsection{temp}
对于稠密线性代数计算GPU单芯性能可达CPU单芯的7倍,
稀疏线性代数性能更是高达12倍。\footnote{测试环境:CPU组为Intel Sandy Bridge E5-2687W 8核 3.1GHz,
计算程序MKL 10.3.6;GPU组为 K20X,软件环境cuBLAS 5.0,不记数据传输时间。}
\cite{WillGTC2013}

GPU上稀疏矩阵的预条件算法和直接求解算法尚在不断研究、发展中。\cite{KyleGTC2013}

目前GPU上对角线方程求解研究的较为充分的仍然是三对角线方程组\cite{LiWenChangGTC2013}

多GPU\cite{LeviGTC2013}、
MPI+CUDA\cite{Jiri2013}、GPU集群\cite{StefanGTC2013}等
多GPU多节点混合并行技术也在不断发展中。

\section{研究内容和论文组织结构}





\chapter{GPU科学计算简介}

\section{CUDA简介}

\subsection{CUDA总体设计}

CUDA翻译为统一计算设备架构,本身定位做一种包含CPU和GPU的编程模型,
不过实际上一般只用作GPU编程和GPU、CPU通讯编程。

CUDA把设备资源分为主机端和GPU端两部分,
主机端包含CPU、内存等正常C/C++程序可以访问到的资源,
GPU端包含多个SM(Streaming Multiprocessor)
或SMX(Next Generation Streaming Multiprocessor)
\footnote{从 NVIDIA显卡的 Kepler 架构开始,SM的规格大幅改变,改称为SMX。
为方便起见以下不再提SM,提到SMX时也包含SM。}
、显存等资源。

其中SMX代表GPU核心内的一个相对独立的向量处理单元,
类似传统的向量机中央处理器,这些SMX位于GPU的核心芯片内。
显存位于显卡PCB上,并被所有的SMX共享。

显存和主机内存是独立的,有各自的地址空间,
CPU端的代码不能直接读写显存,
GPU端的代码也不能直接读写内存,
需要程序员手工在显存和内存之间做数据传输。
CUDA允许在主机端申请所谓的\emph{页锁定主机内存}(Page-Locked Host Memory),
并允许GPU端直接访问,
由显卡驱动负责在内存和显卡之间进行自动数据传输。
此外,对于通用计算专用的Tesla显卡,
CUDA可以开启Unified Virtual Address Space功能,
即对内存和显存统一编址访问,可以省去一些编程上的繁琐操作。
\cite{cudadoc-cprogrammingguide}

\subsubsection{Global函数}

GPU上执行的代码需要放在专门的Kernel函数中,
这些函数在CUDA使用\_\_global\_\_进行标识,所以又称global函数。
Global函数只能由CPU端的代码通过特殊方式调用。
在CUDA 5.0之前global函数间不能相互调用,
global函数只能调用一种有\_\_device\_\_标识的函数(以下称为device函数)。
CUDA 5.0引入了Dynamic Parallelism功能,
允许在global函数内调用global函数,并定义了对应的语义,
该功能依赖计算能力\footnote{NVIDIA对其发布的GPU核心的功能进行划分的标准,当前Kepler架构的计算能力为3.0-3.5。}%
为3.5的GPU核心(如GK110,对应的显卡有GTX Titian、Tesla K20等),
详见文献\onlinecite{cudadoc-dynamicparallelism}。

Global函数是GPU上运行的程序的最基本单元,虽然global函数可以调用device函数,
但被调用的device函数的各种运行时配置都是依赖于直接或间接调用它的global函数。

Global函数实际运行时可以被一组线程同步执行,类似传统的向量机,
同步执行的线程数量在调用global函数的时候进行设置。

CUDA将运行一个global函数的线程分Grid、Block两个层级进行组织:
Grid代表所有参与的线程,一个Grid包含一个或多个Block,每个Block在Grid内都有自己的编号,
CUDA提供的Block编号可以是1维、2维或3维整数;
一个Block又包含一个或多个Thread,每个Thread在Block内也有自己的编号,
CUDA提供的Thread编号同样是1维、2维或3维整数,
这里的Thread就是一个实际的硬件线程,有自己的寄存器组等资源。
一个实际的线程组设置见\floatref{fig:gpu.cuda.blocks}。

\begin{figure}
\centering
\begin{tikzpicture}[scale=0.75, transform shape]
\def\TextBox#1#2#3{
\draw  (#1,#2) rectangle (#1+2,#2+1);
\node at (#1+1, #2+0.5) {#3};
}

\foreach \x in {0,...,2}
  \foreach \y in {0,...,1}
  {
    \TextBox{\x*2.5+1}{-\y*1.5+3}{Block(\x,\y)}
  }
\draw  (0.5,5) rectangle (8.5,1);
\node at (1.5,4.5) {\large Grid};

\foreach \x in {0,...,3}
  \foreach \y in {0,...,2}
  {
    \TextBox{\x*2.5}{-\y*1.5-1.5}{\small Thread(\x,\y)}
  }
\draw  (-0.5,0.5)  rectangle (10,-5);
\node at (1,0) {\large Block(2,1)};

\draw [dashed]  (6,2.5) edge (-0.5,0.5);
\draw [dashed]  (8,2.5) edge (10,0.5);
\draw [dashed]  (6,1.5) edge (-0.5,-5);
\draw [dashed]  (8,1.5) edge (10,-5);
\end{tikzpicture}
\caption[线程组层次结构示意图]{\label{fig:gpu.cuda.blocks}线程组层次结构示意图
\cite{cudadoc-cprogrammingguide}}
\end{figure}

同一个global函数调用时所涉及的线程均使用global函数的参数作为输入,
CUDA提供threadIdx、blockIdx等变量在代码中区分各个线程。
线程以Block为单元分配给GPU上的各个SMX独立执行,
Block之间没有任何直接的同步方式,
程序员不需要知道也不应该猜测各Block是如何在各个SMX执行的。
需要说明的是:强行使用显存作为Block间的同步可能会导致各SMX死锁。
实际Block间同步的最主要方式就是等待该global函数执行完,
此时所有Block状态都是确定的,即已经执行完。

这样GPU或驱动就可以根据实际GPU核心上的SMX数量来具体配置各Block是如何在SMX上执行的,
使得当SM数量不超过Block数量时global函数有了一定的并行扩展性,
见\floatref{fig:gpu.cuda.scalability}。
由于每个Block是运行在一个实际SMX上的,
所以SMX的寄存器、共享存储空间等资源会对Block的大小(包含的Thread数量)有一定的限制,
而CUDA对Grid大小(包含的Block数量)的限制则很小。
NVIDIA这样做是为了通过强制程序员对计算任务进行分割的方式获得了一定的程序并行扩展性,
同时也简化了同一系列不同规格GPU的设计,即通过增减SMX的数量来控制GPU计算能力。

\begin{figure}
\centering
\begin{tikzpicture}[scale=0.6, transform shape]
\def\TextBox#1#2#3{
\draw  (#1,#2) rectangle (#1+2,#2+1);
\node at (#1+1, #2+0.5) {#3};
}

\foreach \x in {0,...,3}
  \foreach \y in {0,...,1}
  {
    \TextBox{\x*2.5+1}{-\y*1.5+3}{Block(\x,\y)}
  }
\draw  (0.5,5) rectangle (11,1);
\node at (3,4.5) {\large Kernel函数的Grid};

\draw [dashed] (-3,-4) -- (16.5,-4);

\draw  (-1,-3.5) rectangle (4.5,-1);
\node at (1.5,-1.5) {\large 2个SMX的GPU};
\draw [-latex new, arrow head=3mm] (7,1) -- (9.5,-1);
\foreach \x in {0,...,1}
{
  \TextBox{\x*2.5-0.5}{-3}{SMX \x}
}
\foreach \t in {0,...,3}
{
  \foreach \x in {0,...,1}
  {
    \TextBox{\x*2.5-0.5}{-\t*2-6}{Block(\t,\x)}
  }
  \draw  (-1,-\t*2-4.75) rectangle (4.5,-\t*2-6.25);
}

\draw  (5.5,-3.5) rectangle (16,-1);
\node at (8,-1.5) {\large 4个SMX的GPU};
\draw [-latex new, arrow head=3mm] (3.5,1) -- (2,-1);
\foreach \x in {0,...,3}
{
  \TextBox{\x*2.5+6}{-3}{SMX \x}
}
\foreach \t in {0,...,1}
{
  \foreach \x in {0,...,3}
  {
    \TextBox{\x*2.5+6}{-\t*2-6}{Block(\x,\t)}
  }
  \draw  (5.5,-\t*2-4.75) rectangle (16,-\t*2-6.25);
}

\draw [dashed,-latex new, arrow head=3mm] (-1.5,-4.5) -- (-1.5,-13);
\foreach \t in {0,...,3}
{
  \node at  (-2.5,-\t*2-5.5) {\Large t=\t};
}

\draw (5,-0.5) -- (5,-12.5);
\end{tikzpicture}
\caption[CUDA程序的扩展性示意图]{\label{fig:gpu.cuda.scalability}CUDA程序的扩展性示意图
\cite{cudadoc-cprogrammingguide}}
\end{figure}

总的来说,GPU核心相当于一组不能单独编程的、以显存作为共享存储器的、带有自动负载平衡的小型向量机组。


\subsection{CUDA的GPU设备模型}

\begin{figure}
\centering
\begin{tikzpicture}[scale=0.75, transform shape]
%15*SMX
\def\SMX#1#2{
\draw  (#1+0, #2) rectangle (#1+1, #2+1);
\node at (#1+0.5, #2+0.5) {\small SMX};
}
\def\len{1.5}
\foreach \x in {0,...,8}
{ \SMX{\x*\len}{0} }
\foreach \x in {0,...,7}
{ \SMX{\x*\len+0.75}{-3} }

%L2 Cache
\draw  (0,-0.5) rectangle (13,-1.5);
\node at (6.5,-1) {L2缓存};


\def\Memory#1#2{
\draw  (#1,#2) rectangle (#1+2,#2+1);
\node at (#1+1,#2+0.5) {\small 显存控制器};
}
\foreach \i in {0,1,2}
{ \Memory{-2.5}{-\i*1.5} }
\foreach \i in {0,1,2}
{ \Memory{13.5}{-\i*1.5} }

\draw  (-2.5,2) rectangle (15.5,1.5);
\node at (6.5,1.75) {线程调度器(GigaThread Engine)};

\draw  (-2.5,3) rectangle (15.5,2.5);
\node at (6.5,2.75) {PCI Express 3.0接口};

\draw  (-3,4) rectangle (16,-3.5);
\node at (-1.5,3.5) {\Large GK110};
\end{tikzpicture}
\caption[GK110结构示意图]{\label{fig:gpu.cuda.gk110}GK110结构示意图
\cite{cudadoc-KeplerGK110ArchitectureWhitepaper}}
\end{figure}

GPU是由多个SMX组成的,例如Kepler架构GK110核心
(见\floatref{fig:gpu.cuda.gk110})就包含15个SMX,
其中每个SMX(见\floatref{fig:gpu.cuda.smx})
含有192个CUDA Core单元(支持单精度、整数计算,图示中Core)、
64个双精度浮点计算单元(图示中DP Unit)、
32个特殊函数计算单元(图示中SFU)、
32个读取/存储单元(图示中LD/ST),
4个Wrap调度器,65536个32bit寄存器。
SMX以32个Thread为一组(称为wrap)来调度Block中的线程,
每个SMX有4个wrap调度器和8个指令分派单元,这使得SMX可以同时执行4个wrap,
每个周期中每个wrap最多可以有两条不相关的指令被同时分派。
\cite{cudadoc-KeplerGK110ArchitectureWhitepaper}

\begin{figure}
\centering
%\includegraphics[scale=0.6]{smx.png}
\begin{tikzpicture}[scale=0.7, transform shape]
\def\TextBox#1#2#3#4
{
  \draw (#1,#2) rectangle (#1+#3,#2+1);
  \node at (#1+#3/2, #2+0.5) {#4};
}

\def\CoreLine#1#2
{
\foreach \i in {0,...,2}
{ \TextBox{#1+\i*1.5+0}{#2+0}{1}{Core} }
\TextBox{#1+4.5}{#2+0}{2}{DP Unit}

\foreach \i in {0,...,2}
{ \TextBox{#1+\i*1.5+7}{#2+0}{1}{Core} }
\TextBox{#1+11.5}{#2+0}{2}{DP Unit}

\TextBox{#1+14}{#2+0}{1.5}{LD/ST}
\TextBox{#1+16}{#2+0}{1}{SFU}
}

\CoreLine{0}{0}

\CoreLine{0}{-4}

\draw [decorate, decoration={brace, amplitude=10pt}] (-0.5,-4.25) -- (-0.5,1.25);
\node [left]at (-1,-1.5) {\large 32组};
\draw [loosely dashed] (0.5,-0.5) -- (0.5,-2.5);
\draw [loosely dashed] (8,-0.5) -- (8,-2.5);
\draw [loosely dashed] (16.5,-0.5) -- (16.5,-2.5);

\draw  (-0.5,2.5) rectangle (17.5,1.5);
\node at (8,2) {\large 寄存器文件(65536个 32bit寄存器)};

\foreach \x in {0,...,3}
{
  \def\xlen{4.5}
  \draw  (\x*\xlen-0.25,4) rectangle (\x*\xlen-0.25+4,3);
  \node  at (\x*\xlen+1.75,3.5) {\large Wrap调度器}; 
}
\draw  (-0.5,4.5) rectangle (17.5,5.5);
\node at (8,5) {\large 指令缓存};

\draw  (17.5,-4.5) rectangle (-0.5,-5.5);
\draw  (17.5,-6) rectangle (-0.5,-7);
\draw  (17.5,-7.5) rectangle (-0.5,-8.5);
\node at (8,-5) {\large 内部互联网络};
\node at (8,-6.5) {\large 64KB 共享存储区 / L1 缓存};
\node at (8,-8) {\large 48KB 只读数据缓存};

\foreach \x in {0,...,7}
{
  \foreach \y in {0,1}
  {
    \draw  (\x*2.25-0.5,-\y*1.5-9) rectangle (\x*2.25+1.5,-\y*1.5-10);
    \node at (\x*2.25+0.5,-\y*1.5-9.5) {纹理单元};
  }
}

\draw  (-3,-12) rectangle (18,6.5);
\node [right] at (-2.5,6) {\Large SMX};
\end{tikzpicture}
\caption[SMX结构示意图]{\label{fig:gpu.cuda.smx}SMX结构示意图
\cite{cudadoc-KeplerGK110ArchitectureWhitepaper}}
\end{figure}

SMX的存储器结构较为复杂,通用存储器的结构见\floatref{fig:gpu.cuda.memory},
其中寄存器部分和传统程序一样,并不直接对程序员可见,由CUDA编译器进行分配,
L1缓存、L2缓存对程序员也不是直接可见的,会在访问显存时自动进行调度。

\begin{figure}
\centering
\begin{tikzpicture}[scale=0.8, transform shape]

\draw  (0,3) ellipse (1 and 0.5);
\node at (0,3) {Thread};

\draw  (-1.5,0.5) rectangle (1.5,1.5);
\node at (0,1) {L1缓存};

\draw  (-5,0.5) rectangle (-2,1.5);
\node at (-3.5,1) {共享存储区};

\draw  (2,0.5) rectangle (5,1.5);
\node at (3.5,1) {只读数据缓存};

\draw [latex new-latex new, arrow head=3mm] (0,2.5) -- (0,1.5);
\draw [latex new-latex new, arrow head=3mm] (-0.5,2.5) -- (-3.5,1.5);
\draw [latex new-, arrow head=3mm] (0.5,2.5) -- (3.5,1.5);

\draw  (-5,-1.5) rectangle (5,-0.5);
\node at (0,-1) {\large L2缓存};
\draw [latex new-latex new, arrow head=3mm](0,0.5) -- (0,-0.5);
\draw [latex new-, arrow head=3mm](3.5,0.5) -- (3.5,-0.5);

\draw  (-5,-3.5) rectangle (5,-2.5);
\node at (0,-3) {\large 显存};
\draw [latex new-latex new, arrow head=3mm](0,-1.5) -- (0,-2.5);

\draw [decorate, decoration={brace, amplitude=7pt}] (-5.5,0) -- (-5.5,4);
\node [left] at (-5.75,2) {\large SMX};

\draw [decorate, decoration={brace, amplitude=7pt}] (-7,-2) -- (-7,4);
\node [left] at (-7.5,1) {\large GK110};

\draw (-5,2.5) rectangle (-2,3.5);
\node at (-3.5,3) {寄存器};
\draw [latex new-latex new, arrow head=3mm] (-1,3) -- (-2,3);
\end{tikzpicture}
\caption[Kepler存储器结构示意图]{\label{fig:gpu.cuda.memory}Kepler存储器结构示意图
\cite{cudadoc-KeplerGK110ArchitectureWhitepaper}}
\end{figure}

Global函数和device函数可以通过指针直接访问共享存储区、显存。
共享存储区位于SMX内,和L1缓存共用64KB空间,
在global函数启动时可以对共享存储区/L1缓存的分配进行设置,
分配的方式有:16KB/48KB、32KB/32KB(限Kepler架构)、48KB/16KB三种。
\cite{cudadoc-KeplerGK110ArchitectureWhitepaper}

在SMX上运行的Block中的所有线程共享SMX全部的65536个32位寄存器,
一旦分配完成,每个Thread线程就只能使用自己所有的寄存器,其他的寄存器则不可见。
每个Thread需要使用的寄存器数量由CUDA编译器控制。

共享存储区则对同一个Block内所有的线程可见,是同一个Block内的Thread协作、通讯的主要方式,
每个Block使用的共享存储区的大小在global函数启动时进行配置。

只读数据缓存可以用来加速在global函数运行过程中保持不便的数据的读取,
程序员可以通过带有const  \_\_restrict\_\_标识的指针进行访问,
也可以通过纹理模式访问。在Fermi及之前的架构中,该部分缓存只能通过纹理方式使用。
\cite{cudadoc-KeplerGK110ArchitectureWhitepaper}



\section{其他GPU编程技术简介}

\label{sec:gpu.other}
\subsection{OpenCL}

在以CUDA为代表的各种硬件相关的GPU、众核编程技术发展之后,
苹果公司提出异构平台计算的开放标准OpenCL,
标准文本见文献\onlinecite{opencl}。

在GPU通用计算领域起步较晚的AMD公司(原ATI公司被AMD收购)
放弃设计一个CUDA这样专为自己生产的GPU编程的编程技术,
直接采用OpenCL作为其GPU的主要开发技术。

OpenCL在很多设备概念、编程模型上照搬较为成熟的CUDA的设计,
但和CUDA相比仍有较大差异。
CUDA程序的编译过程主要发生在主程序开发过程中,CUDA C/C++编译器首先处理CUDA程序的源代码,
把代码分为GPU端和CPU端两部分,CPU端的代码交给GCC/MSVC等传统编译器进行编译,
GPU端的代码则由自己处理,生成相应的GPU PTX代码、主机端调用代码等部分,
最后和程序的其他部分链接成一个主程序。
这其中的PTX代码类似“GPU上的汇编代码”,实际上更接近于Java编译后的字节码,
PTX代码在运行时由主机端驱动程序编译成实际GPU的硬件指令交给GPU运行。
通过增加PTX这一层抽象,NVIDIA使得程序能够在GPU ISA(Instruction Set Architecture)
设计发生变化时仍然可以直接运行。

OpenCL代码语法和CUDA一样\footnote{CUDA后期加入了C++支持。}以
C语言为蓝本\footnote{现在也有为OpenCL增加C++支持的提议,%
见文献\onlinecite{gaster2013opencl}。},
但并不和普通的C代码混在一起,
而是直接以文本形式保存于程序或其他数据文件中,
在运行时由硬件驱动进行编译生成实际硬件代码运行。
所以说OpenCL程序的编译过程主要发生在运行时,
这种方式称为JIT,CUDA未来也计划引入JIT\cite{MarkGTC2013}。


现在支持OpenCL的主要硬件/厂商有:\cite{opencl-conformant-products}
\begin{enumerate}[1)]
\item Intel x86 CPU(及内嵌的Intel HD Graphics集成显卡)
\item AMD CPU/APU
\item AMD显卡
\item NVIDIA显卡
\item Intel MIC(Many Integrated Core) Architecture
\item ARM
\end{enumerate}
相对来说,AMD显卡和MIC上的OpenCL支持较好,因为这是它们上的主要编程方式,
NVIDIA对OpenCL的支持略差,某些情况下可能会有性能明显下降的情况。


\subsection{C++ AMP}

C++ AMP全称C++ Accelerated Massive Parallelism,是微软提出的一个开放的C++ GPU编程标准,
并给出了Windows平台上基于DirectX 11的实现。
C++ AMP的标准文本见文献\onlinecite{cppamp}。
C++ AMP和CUDA类似,都是在已有的语言上增加了新的部分来对GPU进行编程。
\cite{cppamp-overview,AdeGTC2013}

由于C++ AMP是一个开放标准,所以也可能会有进一步的发展。
2012年Intel公司展示了一个把C++ AMP代码编译到OpenCL的实现,使得C++ AMP跨系统跨平台成为可能。
\cite{cppamp-opencl}
不过暂时还没有该技术转化为产品的消息。
在Intel之后,LLVM\footnote{LLVM(Low Level Virtual Machine),
是一套开源的编译器基础构建,Mac系统的编译器如Clang大多基于LLVM构建。}社区中也出现了一个把C++ AMP代码编译到OpenCL的开源实现\cite{llvm-amp-opencl-prototype}。


\subsection{OpenACC与OpenMP}

由于之前的GPU编程技术需要程序员管理的细节过多,
在2011年11月CAPS、CRAY、NVIDIA和PGI等公司联合提出了并行计算标准OpenACC。
\cite{reyes2012comparative}
OpenACC标准文本可以从\url{http://www.openacc-standard.org/Downloads}下载。
OpenACC与OpenMP类似,是一种基于用户制导语句的半自动并行化标准。
OpenMP主要针对于共享存储器的CPU多核环境,
而OpenACC则主要面向CPU+GPU异构环境。

当前支持OpenACC的编译器主要为CAPS、CRAY、PGI等公司提供的商业编译器。

OpenACC已计划被纳入OpenMP 4.0标准。\cite{beyer2011openmp}



\section{GPU线性方程组求解算法简介}

数值计算中线性方程组的求解方法基本可以分为两大类:直接解法和迭代解法。
直接解法的计算量固定且容易估计,没有迭代收敛性问题,但算法本身并行度太低,
很难适合GPU这样的类向量机型处理器,所以以下主要介绍迭代解法。

%\TODO 增加说明

\subsection{传统迭代算法简介}
考虑如下形式的线性方程组
\begin{align}
  \bm{A}\bm{x}=\bm{b}
\end{align}
其中
\begin{align}
  \bm{A}=\begin{pmatrix}
  a_{11} & a_{12} & \cdots & a_{1n}\\
  a_{21} & a_{22} & \cdots & a_{2n}\\
  \vdots & \vdots & \ddots & \vdots\\
  a_{n1} & a_{n2} & \cdots & a_{nn}\\
  \end{pmatrix}
\end{align}
并记
\begin{align}
  \bm{D}=\begin{pmatrix}
      a_{11} &  &  & \\
       & a_{22} &  & \\
       &  & \ddots & \\
       &  &  & a_{nn}\\
      \end{pmatrix}
  \hspace{5pt}
  \bm{L}=\begin{pmatrix}
    0 &  &  & \\
    a_{21} & 0 &  & \\
    \vdots & \ddots & 0 & \\
    a_{n1} & \cdots & a_{n,n-1} & 0\\
    \end{pmatrix}
  \hspace{5pt}
  \bm{U}=\begin{pmatrix}
      0 & a_{12} & \cdots & a_{1n}\\
       & 0 & \ddots & \vdots\\
       &  & 0 & a_{n-1,n}\\
       &  &  & 0\\
      \end{pmatrix}
\end{align}

\subsubsection{Jacobi迭代}
Jacobi迭代的形式为
\begin{align}
  \bm{x}^{(k+1)}=\bm{D}^{-1}\pb[b]{\bm{b}-\pb{\bm{L}+\bm{U}}\bm{x}^{(k)}}
\end{align}
收敛条件为
\begin{align}
  \rho\pb[b]{\bm{D}^{-1}\pb{\bm{L}+\bm{U}}}<1
\end{align}
其中$\rho(\bm{M})$表示$\bm{M}$的谱半径。\cite{golub2012matrix}

\subsubsection{Gauss-Seidel迭代}
Gauss-Seidel迭代的形式为
\begin{align}
  \bm{x}^{(k+1)}=\pb{\bm{L}+\bm{D}}^{-1}\pb[b]{\bm{b}-\bm{U}\bm{x}^{(k)}}
\end{align}
当$\bm{A}$对称正定时,Gauss-Seidel迭代收敛。\cite{golub2012matrix}

Gauss-Seidel迭代在传统CPU求解线性方程组中使用广泛,
但其主要步骤$\pb{\bm{L}+\bm{D}}^{-1}$项的计算是一个串行过程, %(\TODO 解释)
较难在GPU这种向量机上实现,所以在GPU求解线性方程组中使用不多。

\subsubsection{逐次超松弛迭代}
逐次超松弛迭代是Jacobi迭代和Gauss-Seidel迭代的推广,
其迭代的形式为\cite{golub2012matrix}
\begin{align}
  \bm{x}^{(k+1)}=\pb{\bm{L}+\omega\bm{D}}^{-1}
                  \pb[b]{\omega\bm{b}-\pb{\omega\bm{U}+(\omega-1)\bm{D}}\bm{x}^{(k)}}
\end{align}

该迭代方法在GPU上面临和Gauss-Seidel迭代同样的问题。

\subsection{Krylov子空间类算法简介}
\def\algoend{;\hspace{0.4cm}}
\subsubsection{CG(Conjugate gradient)共轭梯度法}
CG方法可以用于求解$\bm{A}$对称正定时的线性方程,
其求解过程见\floatref{alg:gpu.cg}%
\footnote{由于本节的伪代码每行长度较短,
为节省版面故将多行放在一行内,用;号分隔。}。\cite{golub2012matrix}

\begin{algorithm}
\KwIn{$\bm{A},\bm{x}_0,\bm{b}$}
\KwOut{$\bm{x}_e$}
$\bm{r}_0 := \bm{b}-\bm{A}\bm{x}_0$ \algoend
$\bm{p}_0 := \bm{r}_0$ \algoend
$k:=0$\;

\While{ True }{
$\displaystyle \alpha_k:=\frac{\bm{r}_k^T\bm{r}_k}{\bm{p}_k^T\bm{A}\bm{p}_k}$\algoend
$\bm{x}_{k+1}:=\bm{x}_k+\alpha_k\bm{p}_k$ \algoend
$\bm{r}_{k+1}:=\bm{r}_k-\alpha_k\bm{A}\bm{p}_k$\;
\If{$\vb{\bm{r}_{k+1}}$ 足够小}{
  return $\bm{x}_e := \bm{x}_{k+1}$\;
}
$\displaystyle \beta_k := \frac{\bm{r}_{k+1}^T\bm{r}_{k+1}}{\bm{r}_k^T\bm{r}_k}$ \algoend
$\bm{p}_{k+1}:=\bm{r}_{k+1}+\beta_k\bm{p}_k$ \algoend
$k:=k+1$\;
}
\setlabelname{CG方法}
\caption{CG方法\label{alg:gpu.cg}}
\end{algorithm}

带预条件的CG方法见\floatref{alg:gpu.pcg}。\cite{golub2012matrix}
由于只需在\floatref{alg:gpu.pcg}中令$\bm{M}=\bm{I}$即可得到\floatref{alg:gpu.cg},
所以以下只给出带预条件的迭代算法。

\begin{algorithm}
\KwIn{$\bm{A},\bm{x}_0,\bm{b},\bm{M}$}
\KwOut{$\bm{x}_e$}
$\bm{r}_0 := \bm{b}-\bm{A}\bm{x}_0$ \algoend
$\bm{z}_0 := \bm{M}^{-1}\bm{r}_0$ \algoend
$\bm{p}_0 := \bm{r}_0$ \algoend
$k:=0$\;
\While{ True }{
$\displaystyle \alpha_k:=\frac{\bm{r}_k^T\bm{z}_k}{\bm{p}_k^T\bm{A}\bm{p}_k}$ \algoend
$\bm{x}_{k+1}:=\bm{x}_k+\alpha_k\bm{p}_k$ \algoend
$\bm{r}_{k+1}:=\bm{r}_k-\alpha_k\bm{A}\bm{p}_k$ \;
\If{$\vb{\bm{r}_{k+1}}$ 足够小}{
  return $\bm{x}_e := \bm{x}_{k+1}$\;
}
$\bm{z}_{k+1}:=\bm{M}^{-1}\bm{r}_{k+1}$ \algoend
$\displaystyle \beta_k := \frac{\bm{z}_{k+1}^T\bm{r}_{k+1}}{\bm{z}_k^T\bm{r}_k}$ \algoend
$\bm{p}_{k+1}:=\bm{z}_{k+1}+\beta_k\bm{p}_k$ \algoend
$k:=k+1$\;
}
\setlabelname{带预条件的CG方法}
\caption{\label{alg:gpu.pcg}带预条件的CG方法}
\end{algorithm}

\subsubsection{BiCGStab}
对于非对称正定的方程组,可以把CG方法扩展为BiCG(BiConjugate Gradient)方法。
\cite{press2007numericalrecipes}
但由于BiCG方法的数值稳定性较差,可能会遇到不收敛的情况,
所以文献\onlinecite{van1992bicgstab}提出了BiCGStab方法,
见\floatref{alg:gpu.pbicgstab}。

\begin{algorithm}
\KwIn{$\bm{A},\bm{x}_0,\bm{b},\bm{K}$}
\KwOut{$\bm{x}_e$}

$\bm{r}_0 := \bm{b}-\bm{A}\bm{x}_0$ \algoend
任取$\bm{r}_0^*$使得$\bm{r}_0^*\cdot\bm{r}_0\neq 0$ \algoend
$\rho_0:=\alpha:=\omega_0:=1$ \;
$\bm{v}_0:=\bm{p}_0 := \bm{0}$ \algoend
$k:=1$\;
\While{ True }{
$\rho_k:=\bm{r}_0^*\cdot\bm{r}_{k-1}$ \algoend
$\displaystyle \beta:=\frac{\rho_k}{\rho_{k-1}}\frac{\alpha}{\omega_{k-1}}$ \algoend
$\bm{p}_k:=\bm{r}_{k-1}+\beta\pb{\bm{p}_{k-1}-\omega_{k-1}\bm{v}_{k-1}}$ \;
$\bm{y}:=\bm{K}^{-1}\bm{p}_k$ \algoend
$\bm{v}_k:=\bm{Ay}$ \algoend
$\displaystyle \alpha:=\frac{\rho_k}{\bm{r}_0^*\cdot\bm{v}_k}$ \algoend
$\bm{s}:=\bm{r}_{k-1}-\alpha\bm{v}_k$ \algoend
$\bm{z}:=\bm{K}^{-1}\bm{s}$ \;
$\bm{t}:=\bm{Az}$ \algoend
$\displaystyle \omega_k:=\frac{(\bm{K}^{-1}\bm{t})\cdot(\bm{K}^{-1}\bm{s})}{(\bm{K}^{-1}\bm{t})\cdot(\bm{K}^{-1}\bm{t})}$ \algoend
$\bm{x}_{k}:=\bm{x}_{k-1}+\alpha\bm{y}+\omega_k\bm{z}$ \;
\If{$\bm{x}_{k}$收敛}{
  return $\bm{x}_e := \bm{x}_{k}$\;
}
$\bm{r}_k:=\bm{s}-\omega_k\bm{t}$ \algoend
$k:=k+1$\;
}
\setlabelname{带预条件的BiCGStab方法}
\caption{\label{alg:gpu.pbicgstab}带预条件的BiCGStab方法}
\end{algorithm}

\subsubsection{GMRES(Generalized Minimal RESidual method)广义最小残量法}
GMRES是一种可以用于求解非对阵矩阵的Krylov子空间方法,
由Saad和Schultz在1986年提出。\cite{saad1986gmres}

实际计算中GMRES需要再启动等技巧,这里只给出一个简单的GMRES算法,
见\floatref{alg:gpu.gmres}。\cite{golub2012matrix}

\begin{algorithm}
\KwIn{$\bm{A},\bm{x}_0,\bm{b}\text{,并有}\bm{A}\bm{x}_0\approx\bm{b}$}
\KwOut{$\bm{x}_e$}

$\bm{r}_0 := \bm{b}-\bm{A}\bm{x}_0$ \algoend
$h_{10}:=\| \bm{r}_0\|_2$ \algoend
$k:=0$\;
\While{ $h_{k+1,k}>0$ }{
$\displaystyle \bm{q}_{k+1}:=\frac{\bm{r}_k}{h_{k+1,k}}$ \algoend
$k:=k+1$ \algoend
$\bm{r}_k:=\bm{A}\bm{q}_k$\;
\For{$i:=1 \  \mathrm{to} \   k$}
{
$h_{ik}:=\bm{q}_i^T\cdot\bm{r}_k$ \algoend
$\bm{r}_k:=\bm{r}_k-h_{ik}\bm{q}_i$\;
}
$h_{k+1,k}:=\| \bm{r}_k\|_2$ \;
$\bm{x}=\bm{x}_0+\bm{Q}_k\bm{y}_k$
其中$\bm{y}_k$使$\left\| h_{10}e_1-\overline{\bm{H}_m}\bm{y}_k \right\|_2$最小\;
}
$\bm{x}_e:=\bm{x}_k$\;
\setlabelname{GMRES方法}
\caption{\label{alg:gpu.gmres}GMRES方法}
\end{algorithm}


\subsection{预条件算法}
\label{sec:gpu.krylov-precond}
当矩阵$\bm{A}$的条件数较大时,以上的各种迭代方法的收敛速度往往并不理想,
为了改善在大条件数情况下的收敛速度,Krylov方法往往和预条件处理共同使用。
预条件处理的思路是把原始问题同解变换为一个条件数较小的问题。
预条件的方法一般是找到一个矩阵$\bm{P}$使得$\bm{P^{-1}A}$的条件数较小,
且方程组$\bm{Px}=\bm{b}$较容易求解。\cite{saad2003iterative}

以下列举一些常见的预条件算法
\begin{enumerate}
\item Jacobi,又称对角线预处理

取\cite{saad2003iterative}
\begin{align}
  \bm{P}=\begin{pmatrix}
  \displaystyle \frac{1}{A_{11}} & & &\\
  & \displaystyle \frac{1}{A_{22}} & &\\
  & & \ddots &\\
  & & & \displaystyle \frac{1}{A_{nn}}\\
  \end{pmatrix}
\end{align}

\item SPAI(SParse Approximate Inverse)\cite{saad2003iterative}

计算使$\|\bm{AT}-\bm{I}\|_F$最小的$\bm{T}$,
取$\bm{P}=\bm{T}^{-1}$。
$\bm{T}$的求解有多种算法,
如文献\onlinecite{grote1997parallel,bridson2001multiresolution}。

\item IC(Incomplete Cholesky factorization)

参见文献\onlinecite{golub2012matrix}第10.3.2节。

\item ILU(Incomplete LU Factorizations)

参见文献\onlinecite{saad2003iterative}第10.3节。

\end{enumerate}
上面介绍的预条件算法除了Jacobi具有天然的并行度外,
其他算法的Setup过程的串行度较高,较难在GPU上实现。


\begin{comment}

\subsubsection{多重网格}

\begin{table}
\centering
\caption{不同求解算法求解2维Poisson问题需要的计算量\cite{trottenberg2000multigrid}}
\begin{minipage}{.7\linewidth}
\centering
\begin{tabular}{lc}
\topline
求解方法 & 需要的计算量\footnote{记N为未知元数量。}\\
\midline
高斯消去 & $\mathcal{O}(N^2)$\\
Jacobi迭代 & $\mathcal{O}(N^2\log\epsilon)$\\
Gauss-Seidel迭代 & $\mathcal{O}(N^2\log\epsilon)$\\
超松弛迭代 & $\mathcal{O}(N^{3/2}\log\epsilon)$\\
CG & $\mathcal{O}(N^{3/2}\log\epsilon)$\\
IC CG & $\mathcal{O}(N^{5/4}\log\epsilon)$\\
ADI & $\mathcal{O}(N\log N)$\\
完全多重网格法 & $\mathcal{O}(N)$\\
\bottomline
\end{tabular}
\end{minipage}
\end{table}

\end{comment}

\subsection{常见稀疏矩阵存储格式介绍}
\label{sec:gpu.sparseformat}

稀疏矩阵的特点是矩阵中大量元素为0,非零元素比例较小,
分布有时具有一定的规律,如何快速高效地读写稀疏矩阵就是一个可以深入研究的方向。
目前这方面的基础研究已经较为完善,对应的稀疏矩阵表示、存储方法一般称为系数矩阵存储格式,
下面介绍一些常见的系数矩阵存储格式。

\subsubsection{坐标格式(COO)}
COO格式直接存储每个非零元素的位置和值,
如矩阵
\begin{align}
\bm{A}=\begin{pmatrix}
1 & 7 & 0\\
0 & 2 & 0\\
5 & 0 & 3
\end{pmatrix}
\label{equ:gpu.sparseformat.example.matrix3}
\end{align}
的COO格式表示为\footnote{下标从0开始计数,下同。}
\begin{align*}
\mathrm{rows}=\begin{pmatrix}
0 & 0 & 1 & 2 & 2  
\end{pmatrix}
\\
\mathrm{cols}=\begin{pmatrix}
0 & 1 & 1 & 0 & 2
\end{pmatrix}
\\
\mathrm{values}=\begin{pmatrix}
1 & 7 & 2 & 5 & 3
\end{pmatrix}
\end{align*}
其中每个非零元素分别存储:所在行(rows)、所在列(cols)、元素值(values),
三个数组中的位置要对应,而且一般会按照行列进行排序,
例如第一行第二列的元素7,对应rows、cols、values中下标为1的位置。


\subsubsection{行压缩格式(CSC)}
COO格式如果按行列位置对非零元素进行排序,则在行数组(rows)或列数组(cols)之一中
会连续出现很多相同的值,这是由于对应的非零元素属于相同的行或列。
为了减少这种冗余,CSC格式把每行的非零元连续存放,使用ptr数组记录每行元素的起始下标,
相对于COO节省了存储行号的空间。
如矩阵\aeqref{equ:gpu.sparseformat.example.matrix3}
的CSC格式表示为
\begin{align*}
\mathrm{ptr}&=\begin{pmatrix}
0 & 2 & 3  
\end{pmatrix}
\\
\mathrm{cols}&=\begin{pmatrix}
0 & 1 & 1 & 0 & 2
\end{pmatrix}
\\
\mathrm{values}&=\begin{pmatrix}
1 & 7 & 2 & 5 & 3
\end{pmatrix}
\end{align*}
ptr第一个元素为0,表示cols、values数组第一行的元素从下标0开始,
ptr第二个元素为2,表示第一行只有2个元素,第二行的元素从下表2开始。

如果把行列的存储方式调换,则可得到另一种格式,即列压缩格式(CSC)。

\subsubsection{对角线格式(DIA)}
\label{sec:gpu.sparseformat.dia}

对角线格式是专门用于存储对角线稀疏矩阵的一种格式,
其思路是只存储有非零元素的对角线。
如矩阵\cite{bell2008spmv}
\begin{align}
\bm{A}=\begin{pmatrix}
1 & 7 & 0 & 0\\
0 & 2 & 8 & 0\\
5 & 0 & 3 & 9\\
0 & 6 & 0 & 4
\end{pmatrix}
\label{equ:gpu.sparseformat.example.matrix4}
\end{align}
的DIA格式表示为\footnote{其中$\_$表示DIA格式不使用的任意值,下同。}
\begin{align*}
\mathrm{data}=\begin{pmatrix}
\_ & 1 & 7\\
\_ & 2 & 8\\
5 & 3 & 9\\
6 & 4 & \_
\end{pmatrix}
\quad
\mathrm{offsets}=\begin{pmatrix}
-2 & 0 & 1\\
\end{pmatrix}
\end{align*}
该矩阵有三个条对角线(5,6)、(1,2,3,4)、(7,8,9),对角线元素在data中存储(文本采用按列存放),
需要注意的是主对角线以上的对角线和以下的对角线在data矩阵中对其的方式不同。
这三条对角线的位置信息存储在offsets数组中,
-2表示主对角线下方第二条,0表示主对角线,1表示主对角线上方第一条。
data矩阵的列和offsets的元素要一一对应,
而且一般会按照对角线位置排序后进行存储。

\subsubsection{ELLPACK格式(ELL)}
ELL是为向量机设计的一种稀疏矩阵存储格式,\cite{grimes1979itpack}
用于存储$M\times N$的矩阵且每行最多只有$k$个元素的情况。
如矩阵\aeqref{equ:gpu.sparseformat.example.matrix4}
的ELL格式表示为\cite{bell2008spmv}
\begin{align*}
\mathrm{data}=\begin{pmatrix}
1 & 7 & \_\\
2 & 8 & \_\\
5 & 3 & 9 \\
6 & 4 & \_
\end{pmatrix}
\quad
\mathrm{indices}=\begin{pmatrix}
0 & 1 & \_\\
1 & 2 & \_\\
0 & 2 & 3\\
1 & 4 & \_
\end{pmatrix}
\end{align*}
ELL格式为每行保留k个位置,在上面的例子中每行最多有3个元素,
data中的一行对应原矩阵中的一行,所以data的列数为3。
data中每行元素的列号存储在indices矩阵中,
例如,第2行第3列的元素8,在data矩阵中对应第2行第2列的位置,
该元素的列号3存储在indices矩阵中第2行第2列的位置上。
同一行的元素顺序可以随意,一般会按列号进行排序。


\subsubsection{Hybrid格式(HYB)}
由于ELL格式只适合每行元素个数相差不多的情况,
所以又进一步出现了Hybrid格式,即同时使用ELL格式和COO格式,
ELL格式存储系数矩阵中较为规整的部分(每行元素数量接近),
COO格式则存储剩下的比较不规则的非零元。\cite{bell2008spmv}
如矩阵\aeqref{equ:gpu.sparseformat.example.matrix4}
可拆分为两个矩阵
\begin{align*}
\bm{A}=\begin{pmatrix}
1 & 7 & 0 & 0\\
0 & 2 & 8 & 0\\
5 & 0 & 3 & 0\\
0 & 6 & 0 & 4
\end{pmatrix}
+
\begin{pmatrix}
0 &  &  & \\
 & 0 &  & \\
 &  & 0 & 9\\
 &  &  & 0
\end{pmatrix}
\end{align*}
分别使用ELL和COO格式进行存储,
ELL格式部分为
\begin{align*}
\mathrm{data}=\begin{pmatrix}
1 & 7\\
2 & 8\\
5 & 3\\
6 & 4
\end{pmatrix}
\quad
\mathrm{indices}=\begin{pmatrix}
0 & 1\\
1 & 2\\
0 & 2\\
1 & 4
\end{pmatrix}
\end{align*}
COO格式部分为
\begin{align*}
\mathrm{rows}=\begin{pmatrix}
2
\end{pmatrix}
\\
\mathrm{cols}=\begin{pmatrix}
3
\end{pmatrix}
\\
\mathrm{values}=\begin{pmatrix}
9
\end{pmatrix}
\end{align*}

HYB格式中,ELL部分每行存储多少个元素不是固定的,
可以根据需要进行变化。






\chapter{\ProgramName 程序理论推导及开发}

\ProgramName (\ProgramFullName)是本文开发的基于GPU加速的、三维2群扩散细网有限差分、稳态/时空动力学求解程序。


\section{中子扩散时空动力学理论推导}

\subsection{稳态公式推导}

多群扩散时空动力学方程为
\begin{align}
  \newcommand{\para}{\pb{\bm{x},t}}
  \left\{
  \begin{aligned}
    \frac{1}{v_g}\frac{\partial \phi_g\para}{\partial t}
    &=\nabla\cdot D_g\para \nabla\phi_g\para 
      -\Sigma_{t,g}\para \phi_g\para
      +\sum_{i=1}^I \chi_{i,g}\para \lambda_i C_i\para \\
          & \hspace{1cm}
      +\sum_{g'=1}^G\pb[B]{\chi_g\para \pb{1-\beta}\nu\Sigma_{f,g'}\para
                            +\Sigma_{g'\rightarrow g}\para}\phi_{g'}\para \\
    \frac{\partial C_i\para}{\partial t}
     &=\beta_i \sum_{g'=1}^G \nu\Sigma_{f,g'}\para \phi_{g'}\para
        -\lambda_i C_i\para \qquad i=1,2,\cdots,I
  \end{aligned}
  \right.
  \titlelabel{equ:pro.diff.equ}{多群扩散时空动力学方程}
\end{align}

如果初始条件为稳态则有
\begin{align}
  \newcommand{\para}{\pb{\bm{x},t}}
  \left\{
  \begin{aligned}
    \frac{\partial \phi_g\para}{\partial t}\Big|_{t=0} &=0 \\
    \frac{\partial C_i\para}{\partial t}\Big|_{t=0} &=0
  \end{aligned}
  \right.
  \label{equ:pro.diff.init.equ}
\end{align}

联立\aeqref{equ:pro.diff.equ}及\aeqref{equ:pro.diff.init.equ},
消去$C_i\pb{\bm{x},0}$可得
\begin{align}
  \newcommand{\para}{\pb{\bm{x},0}}
  \begin{aligned}
  &\nabla\cdot D_g\para \nabla\phi_g\para 
   -\Sigma_{t,g}\para \phi_g\para \\
  & \hspace{3cm}
   +\sum_{g'=1}^G\pb[B]{\chi_g\para \nu\Sigma_{f,g'}\para
                        +\Sigma_{g'\rightarrow g}\para}\phi_{g'}\para =0
  \end{aligned}
\end{align}

此为$k_\mathrm{eff}=1$时的稳态扩散方程,
由于问题的已知条件一般仅有$k_\mathrm{eff}\approx 1$,
所以先求解普通的临界扩散方程。
\begin{align}
  \newcommand{\para}{\pb{\bm{x},0}}
  \begin{aligned}
  &\nabla\cdot D_g\para \nabla\phi_g\para 
   -\Sigma_{t,g}\para \phi_g\para \\
  & \hspace{3cm}
   +\sum_{g'=1}^G\pb[B]{\frac{1}{k_\mathrm{eff}}\chi_g\para \nu\Sigma_{f,g'}\para
                        +\Sigma_{g'\rightarrow g}\para}\phi_{g'}\para =0
  \end{aligned}
  \titlelabel{equ:pro.diff.init.diff.equ1}{扩散时空动力学问题的初始通量方程}
\end{align}
然后对裂变截面进行修正
\begin{align}
  \newcommand{\para}{\pb{\bm{x},t}}
  \nu\Sigma'_{f,g'}\para = \frac{1}{k_\mathrm{eff}}\nu\Sigma_{f,g'}\para
\end{align}
修正后的问题的$k_\mathrm{eff}=1$,可以适用\aeqref{equ:pro.diff.init.equ},
求得到初始条件的$C_i\pb{\bm{x},0}$
\begin{align}
  \newcommand{\para}{\pb{\bm{x},0}}
  C_i\para = \frac{\beta_i}{\lambda_i}\sum_{g'=1}^G \nu\Sigma_{f,g'}\para\phi_{g'}\para
  \label{equ:pro.diff.init.c}
\end{align}

边界条件为反射边界条件时
\begin{align}
  \bm{n}\cdot\nabla\phi_g\pb{\bm{x},t} = 0
  \qquad \bm{x} \in \partial \mathcal{D}
  \titlelabel{equ:pro.diff.boundary.equ}{扩散方程反射边界条件}
\end{align}
其中$\partial \mathcal{D}$是待求解问题区域的边界面,
$\bm{n}$是边界面上的点$\bm{x}$在边界面上的法向量,
方向指向区域外。

边界条件为0边界条件时
\begin{align}
  \phi_g\pb{\bm{x},t} = 0
  \qquad \bm{x} \in \partial \mathcal{D}
\end{align}

\subsubsection{空间离散}

程序使用三维直角坐标网格,空间划分为$K_x\times K_y \times K_z$个结构网格,
则网格集合为
\begin{align}
  \mathcal{D}_{\bm{k}}=\big\{(k_x,k_y,k_z)\big|k_w = 0,1,\cdots,K_w-1 ; w=x,y,z\big\}
\end{align}

在$xyz$坐标系中有一般离散关系
\begin{align}
  \phi(\bm{x}) &\rightarrow \phi_{\bm{k}} &
  \Sigma(\bm{x}) &\rightarrow \Sigma_{\bm{k}} &
  C_i(\bm{x}) &\rightarrow C_{i,\bm{k}}
\end{align}

其中$\bm{k}=(k_x,k_y,k_z)$为$xyz$空间离散后的网格坐标,
为方便起见这里暂时省略能群$g$,时间步长$n$等下标上标,下同。

离散的主要问题是微分项$\nabla\cdot D(\bm{x})\nabla\phi(\bm{x})$的离散方式,
设$\nabla\cdot D(\bm{x})\nabla\phi(\bm{x})$对应的离散项为
$\pb[b]{\nabla\cdot D\nabla\phi}_{\bm{k}}$,
在$xyz$坐标系中,可取
\begin{align}
  \begin{aligned}
  \pb[b]{\nabla\cdot D\nabla\phi}_{\bm{k}}
    &=\sum_{w=x,y,z} \Sb[bg]{
      \frac{2D_{\bm{k}}D_{\bm{k}+\hat{\bm{w}}}\pb{\phi_{\bm{k}+\hat{\bm{w}}} - \phi_{\bm{k}}}}
           {\Delta w_{\bm{k}}\pb{D_{\bm{k}}\Delta w_{\bm{k}+\hat{\bm{w}}}+D_{\bm{k}+\hat{\bm{w}}}\Delta w_{\bm{k}}}}
           \\
    &\hspace{4cm} -\frac{2D_{\bm{k}}D_{\bm{k}-\hat{\bm{w}}}\pb{\phi_{\bm{k}} - \phi_{\bm{k}-\hat{\bm{w}}}}}
           {\Delta w_{\bm{k}}\pb{D_{\bm{k}}\Delta w_{\bm{k}-\hat{\bm{w}}}+D_{\bm{k}-\hat{\bm{w}}}\Delta w_{\bm{k}}}}
     }
  \end{aligned}
  \label{equ:dnabla2.equ0}
\end{align}
其中$\Delta w_{\bm{k}}$是网格$\bm{k}$在$w$方向上的长度。

则\aeqref{equ:pro.diff.init.diff.equ1}的离散形式为
\begin{align}
  \pb[b]{\nabla\cdot D_g^{(0)} \nabla\phi_g^{(0)}}_{\bm{k}}
   -\Sigma_{t,g,\bm{k}}^{(0)} \phi_{g,\bm{k}}^{(0)}
   +\sum_{g'=1}^G\pb[B]{\frac{1}{k_\mathrm{eff}^{(0)}}\chi_{g,\bm{k}}^{(0)} \nu\Sigma_{f,g',\bm{k}}^{(0)}
                        +\Sigma_{g'\rightarrow g,\bm{k}}^{(0)}}\phi_{g',\bm{k}}^{(0)} =0
  \quad \bm{k} \in \mathcal{D}_{\bm{k}}
\end{align}

\aeqref{equ:pro.diff.init.c}的离散形式为
\begin{align}
  C_{i,\bm{k}}^{(0)} = \frac{\beta_i}{\lambda_i}
    \sum_{g'=1}^G \nu\Sigma_{f,g',\bm{k}}^{(0)}\phi_{g',\bm{k}}^{(0)}
  \qquad \bm{k} \in \mathcal{D}_{\bm{k}}
\end{align}

\subsection{动力学公式推导}

将\aeqref{equ:pro.diff.equ}对时间$t$采用全隐式向后差分格式进行离散得
\begin{align}
  \newcommand{\para}[1][n]{\pb{\bm{x}}^{(#1)}}
  \left\{
  \begin{aligned}
    \frac{1}{v_g}\frac{\phi_g\para - \phi_g\para[n-1]}{\Delta t}
    &=\nabla\cdot D_g\para \nabla\phi_g\para 
      -\Sigma_{t,g}\para \phi_g\para \\
    & \hspace{1cm}
      +\sum_{g'=1}^G\pb[B]{\chi_g\para \pb{1-\beta}\nu\Sigma_{f,g'}\para
                           +\Sigma_{g'\rightarrow g}\para}\phi_{g'}\para \\
    &\hspace{1cm}
      +\sum_{i=1}^I \chi_{i,g}\para \lambda_i C_i\para \\
    \frac{C_i\para - C_i\para[n-1]}{\Delta t}
     &=\beta_i \sum_{g'=1}^G \nu\Sigma_{f,g'}\para \phi_{g'}\para
        -\lambda_i C_i\para
  \end{aligned}
  \right.
  \label{equ:pro.diff.dt.equ0}
\end{align}

解出$C_i\pb{\bm{x}}^{(n)}$可得
\begin{align}
  \newcommand{\para}[1][n]{\pb{\bm{x}}^{(#1)}}
  C_i\para = \frac{1}{1+\lambda_i\Delta t}
    \pb[B]{C_i\para[n-1]
    + \beta_i \Delta t \sum_{g'=1}^G \nu\Sigma_{f,g'}\para \phi_{g'}\para}
  \label{equ:pro.diff.dt.c}
\end{align}

代回\aeqref{equ:pro.diff.dt.equ0}得
\begin{align}
  \newcommand{\para}[1][n]{\pb{\bm{x}}^{(#1)}}
  \begin{aligned}
    &\quad \frac{1}{v_g}\frac{\phi_g\para - \phi_g\para[n-1]}{\Delta t} \\
    &=\nabla\cdot D_g\para \nabla\phi_g\para 
      -\Sigma_{t,g}\para \phi_g\para \\
    & \hspace{1cm}
      +\sum_{g'=1}^G\pb[B]{\chi_g\para \pb{1-\beta}\nu\Sigma_{f,g'}\para
                           +\Sigma_{g'\rightarrow g}\para}\phi_{g'}\para \\
    &\hspace{1cm}
      +\sum_{i=1}^I \frac{\chi_{i,g}\para \lambda_i}{1+\lambda_i\Delta t}
          \pb[B]{C_i\para[n-1] 
      + \beta_i \Delta t \sum_{g'=1}^G \nu\Sigma_{f,g'}\para \phi_{g'}\para}
  \end{aligned}
\end{align}

取$\chi_{i,g}=\chi_g$得
\begin{align}
  \newcommand{\para}[1][n]{\pb{\bm{x}}^{(#1)}}
  \begin{aligned}
    &\quad \frac{1}{v_g}\frac{\phi_g\para - \phi_g\para[n-1]}{\Delta t} \\
    &=\nabla\cdot D_g\para \nabla\phi_g\para 
      -\Sigma_{t,g}\para \phi_g\para 
      +\sum_{i=1}^I \frac{\chi_g\para \lambda_i}{1+\lambda_i\Delta t} C_i\para[n-1]\\
    & \hspace{1cm}
      +\sum_{g'=1}^G\pb[Bg]{\chi_g\para
        \pb[bg]{1-\beta 
          + \sum_{i=1}^I \frac{\lambda_i \beta_i \Delta t }{1+\lambda_i\Delta t}}
      \nu\Sigma_{f,g'}\para \\
    &\hspace{8cm}
         +\Sigma_{g'\rightarrow g}\para}\phi_{g'}\para
  \end{aligned}
\end{align}

记
\begin{align}
  \newcommand{\para}[1][n]{(\bm{x})^{(#1)}}
  S_C\para = \sum_{i=1}^I \frac{\lambda_i}{1+\lambda_i\Delta t} C_i\para[n-1]
  \titlelabel{equ:pro.diff.dt.sc}{离散扩散时空动力学中$S_C(\bm{x})^{(n)}$定义式} \\
  %
  B = 1-\beta + \sum_{i=1}^I \frac{\lambda_i \beta_i \Delta t }{1+\lambda_i\Delta t}
  \titlelabel{equ:pro.diff.dt.B}{离散扩散时空动力学中$B(\bm{x})^{(n)}$定义式}
\end{align}

则有
\begin{align}
  \newcommand{\para}[1][n]{\pb{\bm{x}}^{(#1)}}
  \begin{aligned}
    &\quad \frac{1}{v_g}\frac{\phi_g\para - \phi_g\para[n-1]}{\Delta t} \\
    &=\nabla\cdot D_g\para \nabla\phi_g\para 
      -\Sigma_{t,g}\para \phi_g\para + \chi_g\para S_C\para\\
    & \hspace{1cm}
      +\sum_{g'=1}^G\pb[B]{\chi_g\para
        B \nu\Sigma_{f,g'}\para
         +\Sigma_{g'\rightarrow g}\para}\phi_{g'}\para
  \end{aligned}
  \titlelabel{equ:pro.diff.dt.equ1}{时间$t$隐式向后差分离散后的扩散时空动力学通量$\phi$方程}
\end{align}
此为固定源扩散问题,可通过解线性方程进行求解。

\subsubsection{空间离散}
同临界问题部分,\aeqref{equ:pro.diff.dt.equ1}的离散形式为
\begin{align}
  \begin{aligned}
    \frac{1}{v_g}\frac{\phi_{g,\bm{k}}^{(n)} - \phi_{g,\bm{k}}^{(n-1)}}{\Delta t} 
    &=\pb[b]{\nabla\cdot D_{g}^{(n)} \nabla\phi_{g}^{(n)}}_{\bm{k}}
      -\Sigma_{t,g,\bm{k}}^{(n)} \phi_{g,\bm{k}}^{(n)} + \chi_{g,\bm{k}}^{(n)} S_{C,\bm{k}}^{(n)}\\
    & \hspace{1cm}
      +\sum_{g'=1}^G\pb[B]{\chi_{g,\bm{k}}^{(n)}
        B \nu\Sigma_{f,g',\bm{k}}^{(n)}
         +\Sigma_{g'\rightarrow g,\bm{k}}^{(n)}}\phi_{g',\bm{k}}^{(n)}
  \end{aligned}
  \qquad \bm{k} \in \mathcal{D}_{\bm{k}}
  \label{equ:pro.diff.dt.dx.equ1}
\end{align}

其中
\begin{align}
  S_{C,\bm{k}}^{(n)} &= \sum_{i=1}^I \frac{\lambda_i}{1+\lambda_i\Delta t} C_{i,\bm{k}}^{(n-1)}
  \qquad \bm{k} \in \mathcal{D}_{\bm{k}}
  \titlelabel{equ:pro.diff.dt.dx.sc}{离散扩散时空动力学中$S_{C,\bm{k}}^{(n)}$定义式}
\end{align}

\aeqref{equ:pro.diff.dt.c}的离散形式为
\begin{align}
  C_{i,\bm{k}}^{(n)} = \frac{1}{1+\lambda_i\Delta t}
    \pb[B]{C_{i,\bm{k}}^{(n-1)}
    + \beta_i \Delta t \sum_{g'=1}^G \nu\Sigma_{f,g',\bm{k}}^{(n)} \phi_{g',\bm{k}}^{(n)}}
  \qquad \bm{k} \in \mathcal{D}_{\bm{k}}
\end{align}



\subsection{能群耦合}

本文不考虑向上散射,向下散射只散射到邻近的能群,
则以上离散后的固定源方程可以写成如下形式
\begin{align}
  \begin{pmatrix}
  A_{11} & D_{12} & \cdots & D_{1G}\\
  D_{21} & A_{22} & &\\
   & \ddots & \ddots &\\
   & & D_{G-1,G} & A_{GG}
  \end{pmatrix}
  \begin{pmatrix}
  \phi_1 \\ \phi_2 \\ \vdots \\ \phi_G
  \end{pmatrix}
  =
  \begin{pmatrix}
  S_1 \\ S_2 \\ \vdots \\ S_G
  \end{pmatrix}
\end{align}
其中 $A_{gg}$是7对角对称阵,$D_{g_1g_2}$ 是对角阵。

两群情况则简化为
\begin{align}
  \begin{pmatrix}
  A_{11} & D_{12} \\
  D_{21} & A_{22}
  \end{pmatrix}
  \begin{pmatrix}
  \phi_1 \\ \phi_2
  \end{pmatrix}
  =
  \begin{pmatrix}
  S_1 \\ S_2
  \end{pmatrix}
  \label{equ:program.group.g2equ.fixs}
\end{align}

对于\aeqref{equ:program.group.g2equ.fixs},
可直接使用支持非对称矩阵的迭代算法进行求解,
如Jacobi迭代、BiCGStab、GMRES等。

不过在实际计算中,如果能够更充分的利用$A_{ii}$、
$D_{ij}$的实对称矩阵性质往往可以加速计算过程。
这样就有逐群迭代求解过程
\begin{align}
  \left\{
  \begin{aligned}
    \phi_1^{(k+1)}&=A_{11}^{-1}\pb[B]{S_1-D_{12}\phi_2^{(k)}}\\
    \phi_2^{(k+1)}&=A_{22}^{-1}\pb[B]{S_2-D_{21}\phi_1^{(k+1)}}
  \end{aligned}
  \right.
\end{align}

对于临界问题,则可写成如下形式的广义特征值问题
\begin{align}
  \begin{pmatrix}
  A_{11} &  \\
   & A_{22}
  \end{pmatrix}
  \begin{pmatrix}
  \phi_1 \\ \phi_2
  \end{pmatrix}
  =\frac{1}{k_\mathrm{eff}}
    \begin{pmatrix}
    F_{11} & F_{12} \\
    S_{21} & 0
    \end{pmatrix}
  \begin{pmatrix}
    \phi_1 \\ \phi_2
  \end{pmatrix}
\end{align}
其中 $A_{gg}$是7对角对称阵,$F_{g_1g_2}$、$S_{g_1g_2}$ 是对角阵。

如果使用源迭代进行求解,则每轮外迭代的分群迭代形式为(略去源迭代$k_\mathrm{eff}$相关处理)
\begin{align}
  \left\{
  \begin{aligned}
    \phi_1^{(k+1)}&=A_{11}^{-1}\pb[B]{F_{11}\phi_1^{(k)}+F_{12}\phi_2^{(k)}}\\
    \phi_2^{(k+1)}&=A_{22}^{-1}S_{21}\phi_1^{(k+1)}
  \end{aligned}
  \right.
  \label{equ:program.group.g2equ.keff}
\end{align}

\aeqref{equ:program.group.g2equ.fixs}和\aeqref{equ:program.group.g2equ.keff}
中的$A_{ii}^{-1}$项可以使用支持实对称正定的迭代算法进行求解,
如Jacobi、CG等。


\section{\ProgramName 程序开发}

\subsection{整体介绍}
\ProgramName 程序整体结构见\floatref{fig:program.structure.whole},
其中内核部分是程序的主要部分,结构见\floatref{fig:program.structure.core}。

\begin{figure}[h]
\centering
\begin{tikzpicture}[scale=0.8, transform shape]

\node at (1.5,2) {主体控制逻辑};
\draw  (-4,2.5) rectangle (7,1.5);

\draw  (-1,0.5) rectangle (1,-3.5);
\draw  (2,0.5) rectangle (7,-3.5);
\draw  (2.5,0) rectangle (6.5,-2.5);
\node [right] at (2.5,-3) {输入模块(Lua解释器)};
\node [right] at (3,-0.5) {Lua输入文件};
\node [right] at (3.5,-1) {网格定义};
\node [right] at (3.5,-1.5) {材料定义};
\node [right] at (3.5,-2) {模式及参数};
\draw [latex new-latex new, arrow head=3mm]  (4,1.5)  -- (4,0);

\node at (0,-1.5) {内核};
\draw [latex new-latex new, arrow head=3mm]  (0,1.5)  -- (0,0.5);
\draw [latex new-latex new, arrow head=3mm]  (1,-1.5) -- (2.5,-1.5);

\draw  (-4,0.5) rectangle (-2,-3.5);
\node at (-3,-1.5) {输出};
\draw  (-9,0) rectangle (-5,-1);
\draw  (-5,-2) rectangle (-9,-3);
\node at (-7,-0.5) {输出文件(HDF5)};
\node at (-7,-2.5) {屏幕};
\draw [-latex new, arrow head=3mm] (-3,1.5) -- (-3,0.5);

\draw [latex new-, arrow head=3mm] (-5,-0.5) -- (-4,-0.5) ;
\draw [latex new-, arrow head=3mm] (-5,-2.5) -- (-4,-2.5) ;
\draw [latex new-, arrow head=3mm] (-2,-1.5) -- (-1,-1.5) ;

\end{tikzpicture}
\caption{\label{fig:program.structure.whole}\ProgramName 程序整体结构}
\end{figure}


\begin{figure}
\centering
\begin{tikzpicture}[scale=0.8, transform shape]
\draw  (-2.5,-4) rectangle (12,-5);
\node at (5,-4.5) { GPU};

\draw  (-2.5,-2.5) rectangle (12,-3.5);
\node at (5,-3) {显卡驱动};

\draw  (-2.5,-1) rectangle (12,-2);
\node at (5,-1.5) {CUDA};

\draw  (-2.5,2) rectangle (8,-0.5);
\node [right] at (-2,1.5) {Thrust};
\draw  (-2,1) rectangle (1,0);
\draw  (1.5,1) rectangle (4.5,0);
\draw  (5,1) rectangle (7.5,0);
\node [right] at (-1.5,0.5) {辅助函数};
\node at (3,0.75) { \small Global函数};
\node at (3,0.25) { \small 启动参数配置};
\node at (6.25,0.5) {显存管理};

\draw  (-2.5,6.5) rectangle (5,2.5);
\node [right] at (-2,6) {CUSP};
\draw (-2,4) rectangle (4.5,3);
\node at (1,3.5) {稀疏矩阵格式: DIA, ELL等};
\draw (-2,5.5) rectangle (4,4.5);
\node at (1,5) {迭代算法:CG, BiCGStab等};

\node at (10.25,1.25) {改进的};
\node at (10.25,0.75) {显存管理};
\draw [-latex new, arrow head=3mm] (7.5,0.5) -- (9,0.5);

\draw [very thick] (12,-0.5) -- (8.5,-0.5) 
     -- (8.5,2.5) -- (5.5,2.5) -- (5.5,7) 
     -- (-2.5,7) -- (-2.5,12.5) -- (12,12.5) 
     -- (12,-0.5);
\node [right] at (-2,12) {\ProgramName 程序内核};
\draw  (6,6.5) rectangle (11.5,4.5);
\node at (8.75,5.5) {能群间耦合处理};

\draw  (9,2) rectangle (11.5,0);
\draw  (6,4) rectangle (11.5,3);
\node at (8.75,3.5) {GPU端在线矩阵生成};

\draw  (-2,8.5) rectangle (11.5,7.5);
\node at (5,8) {本征值求解(源迭代)、固定源求解};
\draw  (-2,10) rectangle (11.5,9);
\node at (5,9.5) {双层网格加速};

\draw  (-2,11.5) rectangle (4.5,10.5);
\draw  (5,11.5) rectangle (11.5,10.5);
\node at (1,11) {临界计算};
\node at (8,11) {时空动力学计算};

\end{tikzpicture}
\caption{\label{fig:program.structure.core}\ProgramName 程序内核结构}
\end{figure}

GPU部分使用CUDA进行开发,版本为CUDA 5.0,
并使用了CUDA自带的Thrust通用C++模板库。
在线性代数库方面,现在业界尚没有较为完整、稳定、高效的稀疏矩阵库,
\ProgramName 程序使用了NVIDIA公司开发的开源稀疏矩阵库CUSP,
因为CUSP支持本文需要的DIA稀疏矩阵存储格式,并且实现较为高效。
CUSP在实现上也是一个C++模板库,大量依赖Thrust的实现,
并且使用了Thrust的显存管理模块,出于性能考虑本文替换了Thrust的显存管理模块。
%由于现在Thrust和CUSP并不支持多GPU并行,所以本程序也仅限使用单GPU进行加速。

在稀疏矩阵格式方面主要使用DIA格式(见\sectionref{sec:gpu.sparseformat.dia}),
并针对扩散方程离散的特点实现了扩散方程矩阵GPU端动态DIA格式生成功能,缩短了程序总计算时间。
为了对比其他格式,还添加了ELL、CSR、COO格式(使用CUSP)。

在输入文件格式方面,\ProgramName 使用了Lua语言作为输入文件格式。
Lua是一种快速、轻量级的嵌入式动态语言,设计目标即为容易嵌入到C语言中使用,
方便与C等语言交互,现已在PC游戏界广泛使用。
Lua当前版本5.2.2,基于MIT开源协议开源,
可从 \url{http://www.lua.org/download.html} 获得。
使用Lua语言作为输入文件不光可以显著减少解析输入文件的工作量,
而且还允许用户通过输入文件输入带有控制逻辑的Lua程序以对程序行为进行更为精细的控制,
\ProgramName 基于Lua提供了一种方面地输入动力学问题中材料、截面随时间变化的方式。

由于\ProgramName 程序是三维时空动力学程序,产生的数据包含5个维度(空间3维、能群1维、时间1维),
如果使用传统的纯文本格式输出,用户对数据进行后处理时则会十分不便,
所以\ProgramName 使用国际上较流行的一种针对科学计算的数据交换格式——HDF5作为直接输出格式。
HDF5是一种分层的、带有元数据的、支持多种数据格式的、针对科学计算的数据文件存储格式及相关技术。
它最早由美国国家超级计算应用中心(NCSA)提出,
现在由非营利的HDF Group管理和维护。
HDF5文件格式的读写库基于一种类似于BSD的协议开源\footnote{协议文本可以从
\url{http://www.hdfgroup.org/HDF5/doc/Copyright.html} 获取。},
当前版本为HDF5 1.8.10 patch1\footnote{可
从 \url{http://www.hdfgroup.org/downloads/index.html} 获取。}。

\begin{comment}

由于\ProgramName 程序使用人不能直接读取的HDF5格式进行输出,所以需要额外的工具进行数据后处理。
支持HDF5格式的第三方工具和编程语言比较丰富,本文则推荐Python语言作为后处理语言,
并提供一些Python程序作为示例。
Python语言是一种面向对象的动态语言,现在在学术界和工业界都十分流行,
不少GPU加速工具和科学可视化工具都使用Python编写,
所以本文也使用Python作为主要数据前处理、
后处理脚本语言。\footnote{Python可以从
\url{http://www.Python.org} 免费获取,本文使用 Python2.7.x 的语法。}

\end{comment}

\subsection{材料反应截面的输入}

对于扩散程序,输入数据中占主要部分的是各截面在不同空间网格上的取值,
\ProgramName 和传统确定论程序一样,采用两段式的输入方式,
即输入分为两部分:每个材料的各个截面和材料在空间网格上的分布,
前者一般用列表形式给出,后者一般用粗网格上的材料矩阵给出。
这种方式主要针对扩散计算中问题的常见特点:单个组件尺度内往往只有一种材料,
不同组件间的材料也往往有重复的。\footnote{但对于全堆均匀化后的截面参数,
即各个组件材料不同的情况,这种方式略显冗余。}

对于临界计算来说,以上这种两段式的输入方式已经能够满足实际计算需求,
但对于时空力学问题尚缺乏统一的材料描述方式。
对于时空动力学来说,材料定义的主体部分仍然和临界相同,
实际问题中往往大部分区域的材料、截面并不随时间变化,可以用两段式输入描述。
材料、截面随时间变化的部分基本可以分为两种模式:
\begin{enumerate}
\item
某个或某些材料的某些截面发生变化,即材料$M_i$的某截面$\Sigma_{M_i,j}$不再是定值,
而是随时间变化的函数$\Sigma_{M_i,j}(t)$。
在常见基准算例中,这种中情况常见于材料的部分成分发生改变时。
对于空间是一维、二维等低维形式的算例,由于计算中常取反应堆横截面,
高度方向被简化,控制棒移动的效果只能通过连续或阶跃改变固定网格处的材料反应截面来反映。

\item
材料在空间网格上的分布发生改变,如三维问题中,控制棒移动会改变控制棒路径上各网格的材料,
这时各网格上的截面从一种材料阶跃或连续地变为另一种材料的。
\end{enumerate}
实际计算中可能会出现以上两种模式的混合需求:
\begin{enumerate}
\setcounter{enumi}{2}

\item 只有某些固定区域上的某材料的截面发生变化。
这种情况在计算中一般通过人为地把改变的区域和不改变的区域指定成不同材料,
这样就可以隔离不同区域间的影响,使用上面模式1的输入方式。

\item
在三维问题中,控制棒从某网格某一侧进入并逐渐填满网格的过程是连续的,
但如果只使用上面第二种模式的方式则只能描述为阶跃变化(或是额外定义大量中间状态的材料),
带来额外的误差。为了提高精度,往往会对这些处于过度状态的网格(甚至包括它们附近的网格),
进行一定的均匀化或插值处理,得到过渡状态的截面。
而如何进行这方面的均匀化或插值则有多种方法,如按体积加权或按通量加权等,
不宜在程序中固化,\ProgramName 最好能提供某种接口来让用户控制
这些网格上的截面计算工作。

\end{enumerate}

为了给予用户精细地控制材料输入的方式,\ProgramName 选择了Lua这种动态语言作为材料输入的方式。
动态语言的特点是在运行时可以直接读取程序源代码并执行而不需要显式的编译过程,
这样就可以让用户把均匀化代码放在输入文件中,在运行时根据条件计算各网格上的材料。
如果不考虑效率问题,完全可以用这种动态语言的方式替换掉之前的二段式材料输入方式,
把材料在空间网格上的离散(及可能涉及的均匀化)过程暴露给用户,以增强程序的通用性。

但实际程序中很难使用这种方式,原因是程序运行速度的问题。
动态语言在一般经验中比编译型的语言慢两个数量级,
比较快的动态语言速度大约也比编译型语言慢一个数量级,
如果语言设计时就考虑到性能问题,而且动态语言的实现也经过充分优化,
并使用在线按需编译技术(一般称作JIT)后,其性能才能够勉强接近C语言。
现在这方面的动态语言只有少数的几个,如Lua的LuaJIT实现,
它是目前为止世界上最快的动态语言实现之一\footnote{参见:\url{http://luajit.org/luajit.html}};
还有最近刚出现的类Matlab语法的开源科学计算语言Julia(\url{http://julialang.org/}),
它使用了较为先进的LLVM的JIT在线编译器,性能也接近C语言的效率,
性能比较见Julia项目主页,但Julia项目现在并不成熟,而且不便于嵌入到\ProgramName 中。

而对于\ProgramName 程序这种细网有限差分程序,空间网格数量可达$10^7$的量级
\footnote{对于静态IAEA PWR三维基准问题(见第 \ref{sec:result.test.iaea} 节),
如果空间网格尺寸取1cm,则网格数量为$170\times170\times380=1.1\times10^7$。},
可能的方案有
\begin{enumerate}
\item
使用Lua的LuaJIT实现,性能接近C语言,但由于LuaJIT自身的限制只能适用于使用的内存不超过1G的情况。

\item
使用TinyCC(项目主页\url{http://bellard.org/tcc/})这种能嵌入到其他程序中的C编译器,
在运行时编译用户输入的材料处理函数以实现高速计算。
\end{enumerate}
但这两种方案都较为复杂,并有自己的不足。更为重要的是,
这两种方式只能在CPU端产生所有网格的截面数据,
在GPU计算前需要再复制到GPU显存上,大量数据的传输开销无法避免,
对于时空动力学问题,需要每个时间步都要产生截面数据并进行传输,时间上的开销太大。
根据前面的思路,也可以根据用户的输入在运行时编译为可直接在GPU上运行的PTX指令,
以避免数据传输问题,但这种方式的技术尚不成熟,而且对于开发人员和用户来说难度和工作量都要大的多,
目前基本没有实际应用的价值。

\begin{algorithm}
\ForEach(\tcc*[f]{遍历每种材料的截面信息}){输入文件中的材料截面信息}
{
  \If{截面信息通过时间相关函数生成}
  {
  调用该材料截面生成函数获得该时间步的截面定义\;
  }
  \Else
  {
  直接获取该材料的固定截面\;
}
}
预处理截面信息,并传输到GPU显存\;
\If{存在材料分布更新函数MatChangeFun}
{
  \If{上一时间步中,材料分布信息被更新过}
  {
    恢复初始的材料分布信息\;
  }
  在Lua环境内调用材料更新函数MatChangeFun,产生材料分布差分数据\;
  从Lua环境中取回材料分布差分数据\;
  根据材料分布差分数据更新材料分布信息\;
  更新GPU显存上的材料分布信息\;
}
在GPU端根据材料的截面信息和材料分布信息产生截面在空间网格上的分布\;
\setlabelname{每个时间步\ProgramName 对材料截面更新过程}
\caption{\label{alg:program.material.update}每个时间步\ProgramName 对材料截面更新过程}
\end{algorithm}

综合考虑以上各方面因素,\ProgramName 程序最后仍然采用了二段式的方式进行
数据输入,在时空动力学的材料输入上采用了动态语言的方式来实现功能上的可扩展性,
同时避免了大量的截面信息在CPU端产生的问题。主要方式为:
每个时间步开始时,由用户通过输入文件的Lua代码给出本时间步和初始时间步的材料在网格分布
信息的差分信息,即哪些网格上的材料发生了改变,并给出本时间步的新材料截面信息。
程序根据材料分布的差分信息和初始材料分布信息在CPU端产生新的材料分布矩阵,
新材料分布矩阵和新材料截面信息传送至GPU上,
GPU端程序根据信息直接在显存产生完整的材料分布数据。
在考虑到程序速度的情况下最大限度地照顾到前面提出的各种需求。
此外\ProgramName 还针对需要临时改变全部或某些材料截面的需求
添加了相应接口,简化输入文件修改的工作量。
每个时间步\ProgramName 对材料截面更新过程见\floatref{alg:program.material.update}。

需要说明的是,以上的材料更新过程和空间网格定义是关联的,
当需要在不同网格上进行计算时,程序会对不同网格上的材料信息分开处理,
允许用户在粗网和细网上分别定义不同的材料更新及均匀化方式。


\subsection{临界计算}
\label{sec:program.eigen}

临界计算是反应堆物理设计中常见的计算需求,同时也是动力学中计算通量初值所必需的,
所以\ProgramName 程序实现了临界计算功能。
临界计算部分,\ProgramName 主要使用:
\begin{enumerate}
\item CG-SG,使用CG方法从高能群到低能群逐群求解单群方程组。
\item BiCGStab-MG,使用BiCGStab方法直接求解所有能群联合方程组。
\end{enumerate}
两种方法。为了对比其他方法,程序还实现了如下求解方法:
\begin{enumerate}
\setcounter{enumi}{2}
\item Jacobi-SG,使用Jacobi迭代逐群求解。
\item Jacobi-MG,使用Jacobi对所有能群统一求解。
\item GMRES-MG,使用GMRES对所有能群统一求解。
\end{enumerate}
以上各迭代算法中,CG、BiCGStab、GMRES使用CUSP库的实现,
本文对其显存管理进行了改进。预条件算法使用对角线预条件算法,
见\sectionref{sec:gpu.krylov-precond}。

此外\ProgramName 为了加速细网问题的求解速度,
使用了粗网预求解方法改进迭代过程的初值,显著减少了迭代次数,
详见\sectionref{sec:equsolve.multimesh}。


\begin{algorithm}
读取输入文件的临界部分 \algoend
配置内核稀疏矩阵格式\;
初始化网格信息\;
\If{需要进行粗网预求解}
{
  初始化粗网网格信息 \algoend
  初始化粗网通量\;
  配置粗网求解算法及参数 \algoend
  求解粗网临界问题\;
  将粗网通量插值为细网通量\;
}
\Else
{
  初始化细网通量\;
}
配置求解算法及参数 \algoend
求解临界问题\;
通量归一化,输出结果\;
\setlabelname{\ProgramName 程序临界功能主流程伪代码}
\caption{\label{alg:program.eigen.main}\ProgramName 程序临界功能主流程伪代码}
\end{algorithm}


\begin{algorithm}
初始化CPU端和GPU端数据\;
根据材料定义和截面信息在GPU端直接产生的空间网格上的截面信息\;
根据网格信息配置DIA格式的迭代矩阵需要的空间\;
根据空间离散方式在GPU上填充各群迭代矩阵\;

根据当前通量计算第1群源项\;
计算对角线预条件算法的对角线矩阵\;
源迭代过程初始化\;
\Repeat{源迭代收敛}
{
  使用CG求解第1群通量 \algoend
  计算第2群源项\;
  使用CG求解第2群通量 \algoend
  更新第1群源项\;
  计算下一代的$K_\mathrm{eff}$ \algoend
  估计$K_\mathrm{eff}$和通量的误差\;
}
\setlabelname{\ProgramName 程序临界功能CG-SG算法伪代码}
\caption{\label{alg:program.eigen.cg-sg}\ProgramName 程序临界功能CG-SG算法伪代码}
\end{algorithm}


\begin{algorithm}
初始化CPU端和GPU端数据\;
根据材料定义和截面信息在GPU端直接产生的空间网格上的截面信息\;
根据网格信息配置DIA格式的迭代矩阵需要的空间\;
根据空间离散方式在GPU上填充迭代矩阵\;

根据当前通量计算源项\;
计算对角线预条件算法的对角线矩阵\;
源迭代过程初始化\;
\Repeat{源迭代收敛}
{
  使用BiCGStab求解各群通量 \algoend
  更新源项\;
  计算下一代的$K_\mathrm{eff}$ \algoend
  估计$K_\mathrm{eff}$和通量的误差\;
}
\setlabelname{\ProgramName 程序临界功能BiCGStab-MG算法伪代码}
\caption{\label{alg:program.eigen.bicgstab-mg}\ProgramName 程序临界功能BiCGStab-MG算法伪代码}
\end{algorithm}

临界部分的求解流程见\floatref{alg:program.eigen.main}。
以上各求解算法的实现可以分为两大类,逐群求解和联合求解,
这两类中的算法过程基本相近,所以只以CG-SG和BiCGStab-MG
为代表进行说明。
CG-SG算法的伪代码见\floatref{alg:program.eigen.cg-sg},
BiCGStab-MG算法的伪代码见\floatref{alg:program.eigen.bicgstab-mg}

\FloatBarrier

\subsection{时空动力学计算}

\subsubsection{时空动力学问题初值}
\label{sec:program.kinetics.keff-fix}

要计算时空动力学问题,首先要计算时变问题的初值,
实际中一般取刚好临界时的稳定状态为初始状态,
而实际算例中很少有恰好$k_\mathrm{eff}=1$的时候,
如果仅仅使用临界计算的通量作为初始通量分布并用\aeqref{equ:pro.diff.init.c}计算缓发中子源的话,
在第一个时间步中相当于材料截面有一个阶跃,使得总通量分布在第一个时间步会有一个明显的跳跃,
而且这也不符合\aeqref{equ:pro.diff.init.c}的计算条件。
所以一般计算中会对问题的参数进行修正(如程序NGFMN-K\cite{zhaowenbo}),使修正后的问题满足$k_\mathrm{eff}=1$,
本文对裂变项进行修正
\begin{align}
  \newcommand{\para}{\pb{\bm{x},t}}
  \nu\Sigma'_{f,g'}\para = \frac{1}{k_\mathrm{eff}}\nu\Sigma_{f,g'}\para
\end{align}
但实际计算中,$k_\mathrm{eff}$是由临界计算得到的,实际临界计算往往会有少量误差,
使得$k_\mathrm{eff}$的计算值和实际值之间有微小的差别,这与源迭代收敛条件有关。
但这个微小的误差会在后续的动力学过程中表现出来,即第一个时间步中总体通量会有一个小的阶跃,
而且在后续时间步中总体通量也会有小指数上升或小指数下降的趋势,影响时空动力学的计算和结果比较。
所以本文采用多次试算、修正的方式直至最终$k_\mathrm{eff}$足够接近于1,
算法伪代码见\floatref{alg:program.kinetics.keff-fix}。

\begin{algorithm}
初始化$c_L:=0.1, c_U:=10, c:=1, n:=0$\;
\While{$k_\mathrm{eff}^{(n)}$足够接近于$1$或$c_U-c_L$足够小}
{
  \lIf(\tcc*[f]{更新系数$c$的上下界})
  {$k>1$}{$\displaystyle c_U:=\frac{c+c_U}{2}$}
  \lElse(\tcc*[f]{每次减半是考虑到$k_\mathrm{eff}$的计算可能有误差})
  {$\displaystyle c_L:=\frac{c+c_L}{2}$}
  $\displaystyle c:=\frac{c}{\ k_\mathrm{eff}^{(n)}\ }$\;
  \If(\tcc*[f]{确保$c$不会发散})
  {$c>c_U$或$c<c_L$}
  {$\displaystyle c:=\frac{c_L+c_U}{2}$}
  使用$c$对裂变截面进行修正
  $\nu\Sigma'_{f,g',\bm{k}}\Big|_{t=0} := c\cdot\nu\Sigma_{f,g',\bm{k}}\Big|_{t=0}$\;
  重新计算临界问题,得到$k_\mathrm{eff}^{(n+1)}$\;
  $n:=n+1$\;
}
\setlabelname{\ProgramName 程序时空动力学临界修正伪代码}
\caption{\label{alg:program.kinetics.keff-fix}
\ProgramName 程序时空动力学临界修正伪代码}
\end{algorithm}

\subsubsection{时空动力学计算}

前面已经介绍,本文的时空动力学计算中,时间离散采用隐式向后差分格式,
每个时间步的通量只由上一步的通量及缓发中子源决定,
由于是隐式迭代,所以每个时间步要求解一个固定源扩散方程。
时空动力学计算流程见\floatref{alg:program.kinetics.loop}。

\begin{algorithm}
读取时空动力学计算参数并初始化\;
\If{需要进行粗网预求解}
{
  初始化粗网网格信息\;
}
计算时空动力学初值
\tcc*[f]{见第 \ref{sec:program.eigen}
节及第 \ref{sec:program.kinetics.keff-fix} 节}
\;
\Repeat{完成所有时间步}
{
  初始化下一个时间步\;
  更新本时间步的材料、截面信息\;
  \If{需要进行粗网预求解}
  {
    将上一时间步的细网通量及缓发中子源映射到粗网\;
    更新粗网网格材料、截面信息\;
    求解粗网上的固定源问题并计算缓发中子源\;
    将粗网通量及缓发中子源插值到细网网格\;
  }
  求解细网网格上的固定源问题并计算缓发中子源\;
  计算当前时间步的最大通量及总功率\;
  将本时间步计算结果从GPU显存传送至内存\;
}

\setlabelname{\ProgramName 程序时空动力学主要过程伪代码}
\caption{\label{alg:program.kinetics.loop}
\ProgramName 程序时空动力学主要过程伪代码}
\end{algorithm}

同临界部分一样,固定源的多群扩散方程可以根据对能群处理的不同而有不同的求解方式,
本文实现了CG-SG和BiCGStab-MG两种固定源方程求解算法,
分别对应逐群求解和联合求解,
伪代码分别见\floatref{alg:program.kinetics.cg-sg}及%
\floatref{alg:program.kinetics.bicgstab-mg}。

\begin{algorithm}
根据网格及截面信息在GPU显存上直接生成各群的迭代矩阵\;
初始化预条件算法的对角线矩阵\;
\Repeat{各群通量收敛}
{
  计算第1群的源项\;
  使用CG算法求解第1群通量\;
  计算第2群的源项\;
  使用CG算法求解第2群通量\;
  估算各群通量的误差\;
}
\setlabelname{\ProgramName 程序固定源CG-SG算法伪代码}
\caption{\label{alg:program.kinetics.cg-sg}
\ProgramName 程序固定源CG-SG算法伪代码}
\end{algorithm}


\begin{algorithm}
根据网格及截面信息在GPU显存上直接生成迭代矩阵\;
初始化预条件算法的对角线矩阵\;
计算源项\;
使用BiCGStab算法直接求解各群通量\;
\setlabelname{\ProgramName 程序固定源BiCGStab-MG算法伪代码}
\caption{\label{alg:program.kinetics.bicgstab-mg}
\ProgramName 程序固定源BiCGStab-MG算法伪代码}
\end{algorithm}


\subsubsection{显存管理}

CUSP的显存管理部分继承自Thrust的显存管理方式,
而Thrust的显存管理策略为需要时向CUDA申请,
不需要时向CUDA释放,CUDA本身的显存申请释放速度略慢。
虽然这种策略对于一般需求时能够应付,
但把CG、BiCGStab等算法放在源迭代过程中时会导致每次源迭代
开始及结束时都会进行显存的申请及释放,
会明显增加程序的运行时间。

为了消除这种无谓的时间开销,本文对Thrust的显存管理策略进行了修改,
增加缓存功能,即当上层算法对显存用完释放时,
不是直接调用CUDA释放,而是先放如一个可用显存池中。
这样当以后上层算法需要重新申请显存时,
可以直接从可用显存池中查找是否有适合的显存块,如果发现可用的,
则直接返回给上层算法,没有时才向CUDA进行申请。
通过这种方式,可以消除外迭代反复调用CUSP算法时出现的反复申请释放同样大小的显存的情况,
一般来讲从显存池中查找适合的显存的时间开销要小于向CUDA申请显存的开销,
从而达到了缩短程序运行时间的效果。
%\TODO 考虑添加结果比较

对了观察显存分配缓存的影响,这里使用静态IAEA PWR三维基准题进行计算,
不同情况下的计算时间和显存占用峰值见\floatref{tab:program.cached_alloc}。
可见,带缓存的显存管理策略以少量显存占用峰值增加为代价
(不使用粗网预求解时峰值显存增加很小)
减少了约1-2s的计算时间,对于小规模问题效果十分明显。

\begin{table}
\centering
\caption{不同显存管理策略下临界计算时间和显存占用峰值表}
\label{tab:program.cached_alloc}
\begin{tabular}{cccc}
\topline
计算条件 & 显存管理策略 & 计算时间/s & 显存占用峰值/MB\\
\midline
\multirow{2}{*}{5cm网格}
 & 无缓存 & 3.994 & 11.23\\
 & 有缓存 & 2.652 & 11.23\\
\multirow{2}{*}{2.5cm网格}
 & 无缓存 & 7.052 & 89.82\\
 & 有缓存 & 5.819 & 89.82\\
\multirow{2}{*}{2cm网格}
 & 无缓存 & 9.938 & 175.43\\
 & 有缓存 & 8.549 & 175.43\\
\multirow{2}{*}{1cm网格}
 & 无缓存 & 55.630 & 1403.42\\
 & 有缓存 & 53.430 & 1403.42\\

\multirow{2}{*}{2.5cm网格+粗网预求解}
 & 无缓存 & 4.790 & 93.00\\
 & 有缓存 & 3.338 & 101.05\\
\multirow{2}{*}{1cm网格+粗网预求解}
 & 无缓存 & 14.212 & 1453.17\\
 & 有缓存 & 12.512 & 1578.85\\
\bottomline
\end{tabular}
\end{table}

\subsection{迭代收敛条件}

在迭代过程中,需要不断地对当前迭代的收敛程度进行估计,
本文使用本次迭代通量和上次迭代通量的绝对偏差的最大值进行对收敛程度进行估计,
为了排除通量幅值的影响,使用本次迭代的最大通量进行标准化,即
\begin{align}
e^{(p)}=\frac{\displaystyle \max_{\bm{k},g}\left|\phi_{\bm{k},g}^{(p)}-\phi_{\bm{k},g}^{(p-1)}\right|}
         {\displaystyle \max_{\bm{k},g}\phi_{\bm{k},g}^{(p)}}
\end{align}
其中$\bm{k}$为三维网格编号。


\subsection{Multilevel方法}
\label{sec:equsolve.multimesh}

由于初值对于迭代算法的运行时间影响较大,
所以可以通过改善初值来实现总计算时间的缩减。
对于反应堆类问题,堆中的材料分布相对较简单,
尤其是在细网离散时往往会出现大片的网格材料相同的情况。
可以首先对问题进行粗网离散,可以用较低的开销进行求解,
获得一个较粗略的结果后,可以变换为一个较好的细网计算初值,
达到减少总计算时间的目的,
这种方法由文献\onlinecite{ginestar2001multilevel}最早提出,
称作Multilevel方法。

本课题采用如下方式计算通量初值:
产生原网格xyz方向网格数量均减半的空间网格划分,
使用如前所用的CG-SG方法进行求解,该阶段用户可以通过输入文件自定义
每轮内迭代次数、外迭代收敛标准等控制变量,
在粗网上求解后把粗网上的通量插值为细网通量,
细网上的初始$K_\mathrm{eff}$取为粗网$K_\mathrm{eff}$即可%
\footnote{同一个扩散问题采用不同网格大小进行离散后得到的$K_\mathrm{eff}$并不完全一致,
略有差异。}。

实际求解中,Multilevel方法对于稳态计算有显著的加速效果。



\chapter{数值结果验证}

\section{基准算例概述}

为了验证\ProgramName 程序的临界计算及时空动力学计算功能,
本章使用常用的三个不含热工反馈的时空动力学基准题对程序的结果进行验证。
临界计算的对比程序选用广泛使用的细网有限差分多群扩散程序——Citation。

\subsection{静态IAEA PWR三维基准问题}
\label{sec:result.test.iaea}

IAEA 三维压水堆基准题是三维两群扩散基准题\cite{center1977benchmark},
由177个燃料组件构成,堆芯取$1/4$,
大小为170cm $\times$ 170cm $\times$ 380cm,
堆芯的几何结构见\floatref{fig:result.test.iaea},
堆芯中心为对称边界条件,外侧边界入流为0,外侧也可取等效边界条件
\begin{align}
  \frac{\partial \phi_g}{\partial n}=-\frac{0.4692}{D_g}\phi_g
\end{align}

各材料界面见\floatref{tab:result.test.iaea.mat},
裂变谱取$\chi_1=1$,$\chi_2=0$。

\begin{table}
\centering
\caption{\label{tab:result.test.iaea.mat}静态IAEA PWR 三维基准题材料截面}
\begin{tabular}{cccccc}
\topline
材料 & 能群$g$ & $D_g/\mathrm{cm}$ & $\Sigma_{a,g}/\mathrm{cm}^{-1}$
    & $\nu\Sigma_{f,g}/n,\mathrm{cm}^{-1}$
    & $\Sigma_{s,1\rightarrow2}/\mathrm{cm}^{-1}$\\
\midline
\multirow{2}{*}{M1} 
  & 1 & 1.5 & 0.01 & 0 & \multirow{2}{*}{0.02} \\
  & 2 & 0.4 & 0.08 & 0.135 &\\
\multirow{2}{*}{M2} 
  & 1 & 1.5 & 0.01 & 0 & \multirow{2}{*}{0.02} \\
  & 2 & 0.4 & 0.085 & 0.135 &\\
\multirow{2}{*}{M3} 
  & 1 & 1.5 & 0.01 & 0 & \multirow{2}{*}{0.02} \\
  & 2 & 0.4 & 0.13 & 0.135 &\\
\multirow{2}{*}{M4} 
  & 1 & 2.0 & 0 & 0 & \multirow{2}{*}{0.04} \\
  & 2 & 0.3 & 0.01 & 0 &\\
\multirow{2}{*}{M5} 
  & 1 & 2.0 & 0 & 0 & \multirow{2}{*}{0.04} \\
  & 2 & 0.3 & 0.055 & 0 &\\
\bottomline
\end{tabular}
\end{table}

\begin{figure}
\centering
\begin{subfigure}{\textwidth}
\centering
\begin{tikzpicture}[scale=0.7, transform shape]
\def\lenscale{0.06}

\def\x#1{#1*\lenscale}
\def\zz#1{#1*\lenscale-8}
\def\z#1{#1*\lenscale}

\draw [very thick] (0,0) -- (\x{170},0) 
            -- (\x{170}, \zz{380}) -- (0, \zz{380});
\draw [loosely dashdotted] (0,0) -- (0, \zz{380});
\draw (0,\z{20}) -- (\x{150},\z{20}) -- (\x{150},\zz{360}) -- (0,\zz{360});
\draw (\x{10},\z{20}) -- (\x{10},\zz{380});
\draw [dashed] (\x{30},\zz{380}) -- (\x{30},\zz{280}) -- (\x{50},\zz{280}) -- (\x{50},\zz{380});
\draw (\x{70},\z{20}) -- (\x{70},\zz{380});
\draw (\x{90}, \z{20}) -- (\x{90},\zz{380});
\draw (\x{130},\z{20}) -- (\x{130},\zz{360});
\node at (\x{10},\z{80}) {\huge $\approx$};
\node at (\x{70},\z{80}) {\huge $\approx$};
\node at (\x{90},\z{80}) {\huge $\approx$};
\node at (\x{130},\z{80}) {\huge $\approx$};
\node at (\x{150},\z{80}) {\huge $\approx$};
\node at (\x{170},\z{80}) {\huge $\approx$};

\node [left] at (0,0) {\Large 0};
\node [left] at (0,\z{20}) {\Large 20};
\node [left] at (0,\zz{280}) {\Large 280};
\node [left] at (0,\zz{360}) {\Large 360};
\node [left] at (0,\zz{380}) {\Large 380};

\foreach \xp in {0, 10,30,50,70,90,130,150,170}
{
\node [above] at (\x{\xp},\zz{380}) {\Large \xp};
}

\node at (\x{160}, \zz{260}) {\Large M4};
\node at (\x{140}, \zz{260}) {\Large M1};
\node at (\x{110}, \zz{260}) {\Large M2};
\node at (\x{80}, \zz{260}) {\Large M3};
\node at (\x{40}, \zz{260}) {\Large M2};

\draw [-latex new, arrow head=3mm] (-1,\zz{260}) -- (\x{5}, \zz{260});
\node [left] at (-1, \zz{260}) {\Large M3};

%\draw [-latex new, arrow head=3mm] (-1,\zz{320}) -- (\x{40}, \zz{320});
%\node [left] at (-1, \zz{320}) {\Large Mp3};
\node at (\x{40}, \zz{320}) {\Large M3};

\node at (\x{80}, \zz{370}) {\Large M5};
\node at (\x{60}, \zz{370}) {\Large M4};
\node at (\x{40}, \zz{370}) {\Large M5};
\node at (\x{20}, \zz{370}) {\Large M4};

\draw [-latex new, arrow head=3mm] (-1,\zz{370}) -- (\x{5}, \zz{370});
\node [left] at (-1, \zz{370}) {\Large M5};

\node [above] at (-0.5,\zz{380}+0.5) {\Large cm};
\node [above] at (\x{170}+1,\zz{380}) {\Large cm};
\end{tikzpicture}
\caption{纵截面图}
\end{subfigure}
\begin{subfigure}{\textwidth}
\centering
\begin{tikzpicture}[scale=0.7, transform shape]
\def\lenscale{0.06}
\def\x#1{#1*\lenscale}

\draw [loosely dashdotted] (\x{170},0) -- (0,0) -- (0,\x{170});
%\draw [loosely dashdotted] (0,0) -- (\x{170},\x{170});

\draw (\x{10},0) -- (\x{10},\x{10}) -- (0,\x{10});

\draw (\x{70},0) -- (\x{70},\x{10}) -- (\x{90},\x{10}) -- (\x{90},0);
\draw (0,\x{70}) -- (\x{10},\x{70}) -- (\x{10},\x{90}) -- (0,\x{90});
\node [left] at (-1,\x{80}) {\Large M3};
\draw [-latex new, arrow head=3mm] (-1,\x{80}) -- (\x{5}, \x{80});
\node at (\x{80},\x{5}) {\Large M3};

\draw (\x{30},\x{30}) rectangle (\x{50},\x{50});
\node at (\x{40},\x{40}) {\Large M3};

\draw (0,\x{130}) -- (\x{30},\x{130}) -- (\x{30},\x{110})
           -- (\x{70},\x{110}) -- (\x{70},\x{70})
           -- (\x{110},\x{70}) -- (\x{110},\x{30})-- (\x{130},\x{30})
           -- (\x{130},0);
\node at (\x{40},\x{80}) {\Large M2};

\draw (\x{70},\x{70}) rectangle (\x{90},\x{90});
\node at (\x{80},\x{80}) {\Large M3};

\draw (0,\x{150}) -- (\x{50},\x{150}) -- (\x{50},\x{130})
           -- (\x{90},\x{130}) -- (\x{90},\x{110})
           -- (\x{110},\x{110})
           -- (\x{110},\x{90}) -- (\x{130},\x{90})-- (\x{130},\x{50})
           -- (\x{150},\x{50}) -- (\x{150},0);
\node at (\x{50},\x{120}) {\Large M1};

\draw [very thick] (0,\x{170}) -- (\x{70},\x{170}) -- (\x{70},\x{150})
           -- (\x{110},\x{150}) -- (\x{110},\x{130})
           -- (\x{130},\x{130})
           -- (\x{130},\x{110}) -- (\x{150},\x{110})-- (\x{150},\x{70})
           -- (\x{170},\x{70}) -- (\x{170},0);
\node at (\x{70},\x{140}) {\Large M4};

\foreach \xp in {0, 10,30,50,70,90,130,150,170}
{
\node [below] at (\x{\xp},0) {\Large \xp};
\node [left] at (0,\x{\xp}) {\Large \xp};
}

\node [below] at (\x{170}+1,0) {\Large cm};
\node [left] at (0, \x{170}+0.5) {\Large cm};

\draw [-latex new, arrow head=3mm](-1,\x{5}) -- (\x{5},\x{5});
\node [left] at (-1,\x{5}) {\Large M3};

\end{tikzpicture}
\caption{横截面图}
\end{subfigure}
\caption{\label{fig:result.test.iaea}静态IAEA PWR 三维基准题堆芯几何}
\end{figure}

\FloatBarrier
\subsection{动态TWIGL二维基准问题}
\label{sec:result.test.twigl}
TWIGL是二维两群扩散时空动力学基准题,
在1969年由Hageman和Yasinsky
于文献\onlinecite{hageman1969comparison}中提出\cite{gehin1992quasi},
堆芯取$1/4$,大小为80cm $\times$ 80cm,
堆芯的几何结构见\floatref{fig:result.test.twigl},
堆芯中心为对称边界条件,外侧边界入流为0。

各材料截面见\floatref{tab:result.test.twigl.mat},
裂变谱和缓发中子谱取$\chi_1=1$,$\chi_2=0$。
群速度为
\begin{align}
  \left\{
  \begin{aligned}
  v_1&=1.0\times10^7\mathrm{cm/s}\\
  v_2&=2.0\times10^5\mathrm{cm/s}
  \end{aligned}
  \right.
\end{align}
缓发中子为单组缓发中子
\begin{align}
  \left\{
  \begin{aligned}
  \beta&=0.0075\\
  \lambda&=0.08\mathrm{s}^{-1}
  \end{aligned}
  \right.
\end{align}
反应性引入方式
\begin{enumerate}
\item 阶跃引入:
\begin{align}
\Delta\Sigma_{a,2}&=-0.0035\mathrm{cm}^{-1} \quad t=0
\end{align}

\item 线性引入:
\begin{align}
\Sigma_{a,2}(t)&=\begin{cases}
    (1-0.11667t)\Sigma_{a,2}(0) & t\le 0.2\\
    0.97666\Sigma_{a,2}(0) & t > 0.2
  \end{cases}
\end{align}


\end{enumerate}

\begin{table}
\centering
\caption{\label{tab:result.test.twigl.mat}动态TWIGL二维基准题材料截面}
\begin{tabular}{cccccc}
\topline
材料 & 能群$g$ & $D_g/\mathrm{cm}$ & $\Sigma_{a,g}/\mathrm{cm}^{-1}$
    & $\nu\Sigma_{f,g}/n,\mathrm{cm}^{-1}$
    & $\Sigma_{s,1\rightarrow2}/\mathrm{cm}^{-1}$\\
\midline
\multirow{2}{*}{M1} 
  & 1 & 1.4 & 0.01 & 0.007 & \multirow{2}{*}{0.01} \\
  & 2 & 0.4 & 0.15 & 0.2 &\\
\multirow{2}{*}{M2} 
  & 1 & 1.4 & 0.01 & 0.007 & \multirow{2}{*}{0.01} \\
  & 2 & 0.4 & 0.15 & 0.2 &\\
\multirow{2}{*}{M3} 
  & 1 & 1.3 & 0.008 & 0.003 & \multirow{2}{*}{0.01} \\
  & 2 & 0.5 & 0.05 & 0.06 &\\
\bottomline
\end{tabular}
\end{table}

\begin{figure}
\centering
\begin{tikzpicture}[scale=0.7, transform shape]
\def\lenscale{0.075}
\def\x#1{#1*\lenscale}

\draw [loosely dashdotted] (\x{80},0) -- (0,0) -- (0,\x{80});

\draw (\x{24},0) -- (\x{24},\x{56});
\draw (0,\x{24}) -- (\x{56},\x{24});
\draw (\x{56},0) -- (\x{56},\x{56});
\draw (0,\x{56}) -- (\x{56},\x{56});
\draw (\x{80},0) -- (\x{80},\x{80});
\draw (0,\x{80}) -- (\x{80},\x{80});

\node at (\x{12},\x{12}) {\Large M3};
\node at (\x{40},\x{12}) {\Large M2};
\node at (\x{12},\x{40}) {\Large M2};
\node at (\x{40},\x{40}) {\Large M1};
\node at (\x{68},\x{68}) {\Large M1};

\foreach \xp in {0, 24, 56,80}
{
\node [below] at (\x{\xp},0) {\Large \xp};
\node [left] at (0,\x{\xp}) {\Large \xp};
}

\node [below] at (\x{80}+1,0) {\Large cm};
\node [left] at (0, \x{80}+0.5) {\Large cm};

\end{tikzpicture}
\caption{\label{fig:result.test.twigl}动态TWIGL二维基准题堆芯几何}
\end{figure}

\FloatBarrier
\subsection{动态LMW三维基准问题}

LMW(Langenbuch-Maurer-Werner)基准题\cite{langenbuch1977coarse,gehin1992quasi}是
三维两群扩散中子时空动力学基准题,
大小为$1/4$堆芯,几何尺寸110cm $\times$ 110cm $\times$ 200cm,
见\floatref{fig:result.test.lmw}。

\begin{figure}
\centering
\begin{subfigure}{.45\textwidth}
\hspace{-1.5cm}
\begin{tikzpicture}[scale=0.7, transform shape]
\def\lenscale{0.075}
\def\x#1{#1*\lenscale}

\draw [loosely dashdotted] (\x{110},0) -- (0,0) -- (0,\x{110});

\draw (\x{10},\x{0}) -- (\x{10},\x{10}) -- (\x{0},\x{10});
\draw (\x{50},\x{0}) -- (\x{50},\x{10}) -- (\x{70},\x{10}) -- (\x{70},\x{0});
\draw (\x{0},\x{50}) -- (\x{10},\x{50}) -- (\x{10},\x{70}) -- (\x{0},\x{70});
\node at (\x{30},\x{10}) {\Large M1};
\node at (\x{60},\x{5}) {\Large M2};
\node at (\x{5},\x{60}) {\Large M2};
\node at (\x{5},\x{5}) {\Large M2};

\draw (\x{30},\x{30}) rectangle (\x{50},\x{50});
\node at (\x{40},\x{40}) {\Large M2};

\draw (\x{70},\x{0}) -- (\x{70},\x{50}) -- (\x{50},\x{50})
      -- (\x{50},\x{70}) -- (\x{0},\x{70});
\node at (\x{60},\x{60}) {\Large M3};

\draw (\x{90},\x{0}) -- (\x{90},\x{70}) -- (\x{70},\x{70})
      -- (\x{70},\x{90}) -- (\x{0},\x{90});
\node at (\x{80},\x{80}) {\Large M4};

\draw (\x{110},\x{0}) -- (\x{110},\x{90}) -- (\x{90},\x{90})
      -- (\x{90},\x{110}) -- (\x{0},\x{110});

\foreach \xp in {0, 10,30,50,70,90,110}
{
\node [below] at (\x{\xp},0) {\Large \xp};
\node [left] at (0,\x{\xp}) {\Large \xp};
}

\node [below] at (\x{110}+1,0) {\Large cm};
\node [left] at (0, \x{110}+0.5) {\Large cm};

\draw [-latex new, arrow head=3mm] (-1.25,\x{50}) -- (\x{0},\x{60});
\draw [-latex new, arrow head=3mm] (-1.25,\x{50}) -- (\x{60},\x{10});
\node [left] at (-1.25,\x{50}) {\Large 第一组};

\draw [-latex new, arrow head=3mm] (-1.25,\x{20}) -- (\x{5},\x{10});
\draw [-latex new, arrow head=3mm] (-1.25,\x{20}) -- (\x{30},\x{40});
\node [left] at (-1.25,\x{20}) {\Large 第二组};

\end{tikzpicture}
\caption{横截面图}
\end{subfigure}
\\[1cm]
\begin{subfigure}{.45\textwidth}
\begin{tikzpicture}[scale=0.7, transform shape]
\def\lenscale{0.06}

\def\x#1{#1*\lenscale}
\def\zz#1{#1*\lenscale}
\def\z#1{#1*\lenscale}

\draw [very thick] (0,0) -- (\x{110},0) 
            -- (\x{110}, \zz{200}) -- (0, \zz{200});
\draw [loosely dashdotted] (0,0) -- (0, \zz{200});

\draw (0,\z{20}) -- (\x{90},\z{20}) -- (\x{90},\x{180}) -- (0,\zz{180});
\draw (\x{70},\x{20}) -- (\x{70},\x{180});

\node [left] at (0,0) {\Large 0};
\node [left] at (0,\z{20}) {\Large 20};
\node [left] at (0,\z{60}) {\Large 60};
\node [left] at (0,\z{100}) {\Large 100};
\node [left] at (0,\zz{180}) {\Large 180};
\node [left] at (0,\zz{200}) {\Large 200};

\node at (\x{30},\x{40}) {\Large M1};
\node at (\x{80},\x{100}) {\Large M3};
\node at (\x{100},\x{100}) {\Large M4};
\node at (\x{20},\x{190}) {\Large M4};
\node at (\x{40},\x{190}) {\Large M2};
\node at (\x{60},\x{190}) {\Large M2};
\draw [-latex new, arrow head=3mm] (-1,\x{190}) -- (\x{5},\x{190});
\node [left] at (-1,\x{190}) {\Large M2};

\draw [dashed] (\x{0},\x{100}) -- (\x{10},\x{100}) -- (\x{10},\x{200});
\draw (\x{10},\x{180}) -- (\x{10},\x{200});
\draw (\x{50},\x{200}) -- (\x{50},\x{100}) -- (\x{70},\x{100}) -- (\x{70},\x{200});
\draw [dashed] (\x{30},\x{180}) -- (\x{30},\x{200});

\node at (\x{60},\x{140}) {\Large M2};
\draw [-latex new, arrow head=3mm] (-1,\x{140}) -- (\x{5},\x{140});
\node [left] at (-1,\x{140}) {\Large M2};

\draw [-latex new, arrow head=3mm] (-1,\x{70}) -- (\x{5},\x{100});
\draw [-latex new, arrow head=3mm] (-1,\x{70}) -- (\x{60},\x{100});
\node [left] at (-1,\x{70}) {\Large 第一组};
\draw [-latex new, arrow head=3mm] (-1,\x{160}) -- (\x{5},\x{180});
\draw [-latex new, arrow head=3mm] (-1,\x{160}) -- (\x{40},\x{180});
\node [left] at (-1,\x{160}) {\Large 第二组};

\foreach \xp in {0,10,30,50,70,90,110}
{
\node [above] at (\x{\xp},\zz{200}) {\Large \xp};
}

\node at (-0.75,\zz{200}+0.75) {\Large cm};
\end{tikzpicture}
\caption{纵截面图(初始时控制棒位置)}
\end{subfigure}
\begin{subfigure}{.45\textwidth}
\begin{tikzpicture}[scale=0.7, transform shape]
\def\lenscale{0.06}

\def\x#1{#1*\lenscale}
\def\zz#1{#1*\lenscale}
\def\z#1{#1*\lenscale}

\draw [very thick] (0,0) -- (\x{110},0) 
            -- (\x{110}, \zz{200}) -- (0, \zz{200});
\draw [loosely dashdotted] (0,0) -- (0, \zz{200});

\draw (0,\z{20}) -- (\x{90},\z{20}) -- (\x{90},\x{180}) -- (0,\zz{180});
\draw (\x{70},\x{20}) -- (\x{70},\x{180});

\node [left] at (0,0) {\Large 0};
\node [left] at (0,\z{20}) {\Large 20};
\node [left] at (0,\z{60}) {\Large 60};
\node [left] at (0,\z{100}) {\Large 100};
\node [left] at (0,\zz{180}) {\Large 180};
\node [left] at (0,\zz{200}) {\Large 200};

\node at (\x{30},\x{40}) {\Large M1};
\node at (\x{80},\x{100}) {\Large M3};
\node at (\x{100},\x{100}) {\Large M4};
\node at (\x{20},\x{190}) {\Large M4};
\node at (\x{40},\x{190}) {\Large M2};
\node at (\x{60},\x{190}) {\Large M2};
\draw [-latex new, arrow head=3mm] (-1,\x{190}) -- (\x{5},\x{190});
\node [left] at (-1,\x{190}) {\Large M2};

\draw (\x{0},\x{60}) -- (\x{10},\x{60}) -- (\x{10},\x{200});
\draw [dashed] (\x{30},\x{200}) -- (\x{30},\x{60}) -- (\x{50},\x{60}) -- (\x{50},\x{200});
\draw (\x{50},\x{180}) -- (\x{50},\x{200});
\draw (\x{70},\x{180}) -- (\x{70},\x{200});

\draw [-latex new, arrow head=3mm] (-1,\x{30}) -- (\x{5},\x{60});
\draw [-latex new, arrow head=3mm] (-1,\x{30}) -- (\x{40},\x{60});
\node [left] at (-1,\x{30}) {\Large 第二组};
\draw [-latex new, arrow head=3mm] (-1,\x{160}) -- (\x{5},\x{180});
\draw [-latex new, arrow head=3mm] (-1,\x{160}) -- (\x{60},\x{180});
\node [left] at (-1,\x{160}) {\Large 第一组};

\node at (\x{40},\x{120}) {\Large M2};
\draw [-latex new, arrow head=3mm] (-1,\x{120}) -- (\x{5},\x{120});
\node [left] at (-1,\x{120}) {\Large M2};

\foreach \xp in {0,10,30,50,70,90,110}
{
\node [above] at (\x{\xp},\zz{200}) {\Large \xp};
}

\node at (-0.75,\zz{200}+0.75) {\Large cm};
\end{tikzpicture}
\caption{纵截面图(最终控制棒位置)}
\end{subfigure}
\caption{\label{fig:result.test.lmw}动态LMW三维基准题堆芯几何}
\end{figure}

\begin{table}
\centering
\caption{\label{tab:result.test.lmw.mat}动态LMW三维基准题材料截面}
\begin{tabular}{cccccc}
\topline
材料 & 能群$g$ & $D_g/\mathrm{cm}$ & $\Sigma_{a,g}/\mathrm{cm}^{-1}$
    & $\nu\Sigma_{f,g}/n,\mathrm{cm}^{-1}$
    & $\Sigma_{s,1\rightarrow2}/\mathrm{cm}^{-1}$\\
\midline
\multirow{2}{*}{M1} 
  & 1 & 1.423913 & 0.01040206 & 0.006477691 & \multirow{2}{*}{0.0175555} \\
  & 2 & 0.356306 & 0.08766217 & 0.1127328 &\\
\multirow{2}{*}{M2} 
  & 1 & 1.423913 & 0.01095206 & 0.006477691 & \multirow{2}{*}{0.0175555} \\
  & 2 & 0.356306 & 0.09146217 & 0.1127328 &\\
\multirow{2}{*}{M3} 
  & 1 & 1.425611 & 0.01099263 & 0.007503284 & \multirow{2}{*}{0.01717768} \\
  & 2 & 0.350574 & 0.09925634 & 0.1378004 &\\
\multirow{2}{*}{M4} 
  & 1 & 1.634227 & 0.002660573 & 0 & \multirow{2}{*}{0.02759693} \\
  & 2 & 0.264002 & 0.04936351 & 0 &\\
\bottomline
\end{tabular}
\end{table}


算例包含两组控制棒,开始时第一组控制棒以3cm/s的速度拔出,
26.67s时停止。第二组控制棒从7.5s开始以同样的速度插入,47.5s时停止,
控制棒位置随时间变化见\floatref{fig:result.test.lmw.rob}。
\begin{figure}
\centering
\includegraphics[scale=0.9]{result-test-lmw-rob}
\caption{\label{fig:result.test.lmw.rob}LMW算例控制棒位置随时间变化图}
\end{figure}

裂变谱和缓发中子谱取$\chi_1=1$,$\chi_2=0$。
群速度为
\begin{align}
  \left\{
  \begin{aligned}
  v_1&=1.25\times10^7\mathrm{cm/s}\\
  v_2&=2.5\times10^5\mathrm{cm/s}
  \end{aligned}
  \right.
\end{align}
缓发中子有6组,见\floatref{tab:result.test.lmw.mat.c}。

\begin{table}
\centering
\caption{\label{tab:result.test.lmw.mat.c}动态LMW三维基准题缓发中子数据}
\begin{tabular}{ccc}
\topline
缓发中子组 & 缓发中子份额$\beta_i$ & 衰变常数$\lambda_i/\mathrm{s}^{-1}$\\
\midline
1 & 0.000247 & 0.0127\\
2 & 0.0013845 & 0.0317\\
3 & 0.001222 & 0.1150\\
4 & 0.0026455 & 0.3110\\
5 & 0.000832 & 1.400\\
6 & 0.000169 & 3.870\\
\bottomline
\end{tabular}
\end{table}


\section{数值结果验证}

在比较\ProgramName 的计算结果和参考结果之前,
先介绍本文使用的通量结果偏差的计算方式。

对于临界问题,程序计算得到的通量分布结果的绝对值无意义,
有意义的是各网格上通量的相对大小关系,
所以一般都需要对结果进行归一化,
再比较归一化后的通量结果。
那么选择的归一化方法就会影响到计算结果的对比,
常见的归一化方式是总功率归一化和最大通量归一化,
但这两种方法都有各自的问题:
由于堆芯中通量分布在不同的网格上可相差数个量级,
如果按照堆芯总功率进行归一化,那么堆芯中部功率较高的部分对归一化系数影响很大,
高通量区域的误差会通过归一化系数转移到到低通量区域上;
同理使用最大通量归一化时,通量最高处的误差会影响到其他所有点的误差结果。
在常见的组件级通量统计中,这种问题并不明显,
但在细网结果的相对误差比较中,这种影响就被放大了。

为了减小高通量网格对归一化系数的影响,本文使用所有网格上的相对幅值的均值作为
归一化系数,即待验证的通量分布$\phi$和参考通量分布$\psi$间的相对归一化系数取为
\begin{align}
  \eta = \frac{\ \displaystyle\sum_{\bm{k},g}\frac{\phi_{\bm{k},g}}{\psi_{\bm{k},g}}\ }
              {\displaystyle\sum_{\bm{k},g}1}
\end{align}
其中$\bm{k}$为空间网格编号,
则归一化后的各网格的最大相对误差为
\begin{align}
  e_\mathrm{max} = \max_{\bm{k},g}
      \left|
        \frac{\phi_{\bm{k},g}}{\eta\cdot\psi_{\bm{k},g}} - 1
      \right|
\end{align}
本文以后提到的最大相对误差均是指上式所定义的最大相对误差,
而且为了更好地反映有实际意义的堆芯内的通量的相对误差,
一般把堆芯活性区以外的水反射层处通量极低的网格滤掉。

本文中如无特殊声明,计算环境均为
\begin{itemize}
\item CPU: 2$\times$ Intel Xeon CPU X5670 2.93GHz (双路服务器)
\item 内存:24GB
\item GPU:NVIDIA GeForce GTX TITAN,6GB显存
\item 操作系统:Windows 7 SP1 64位
\end{itemize}

\subsection{静态IAEA PWR三维基准题}

使用Citation的计算结果作为参考解,
通量收敛精度$\epsilon_\phi$取$5\times10^{-7}$,
对于5cm、2.5cm、2cm、1cm的$k_\mathrm{eff}$计算结果
见\floatref{tab:result.iaea.citation}。
\ProgramName 的通量收敛精度取$\epsilon'_\phi=10^{-6}$,
需要注意的是\ProgramName 和Citation的收敛精度计算方式不同,
计算结果见\floatref{tab:result.iaea.self},
其中$\max\epsilon_{\phi_{A,g}}$表示各组件通量的最大相对误差。

可见在该精度下\ProgramName 计算的$k_\mathrm{eff}$ 偏差小于$1\times10^{-5}$,
细网通量最大相对误差小于$10^{-3}$,
对于Multilevel方法通量最大相对误差小于$9\times10^{-4}$。
1cm网格$1/8$堆芯的组件通量相对偏差结果见\floatref{fig:result.iaea.aphitable.cg}及
\floatref{fig:result.iaea.aphitable.cgml},
其中$\phi_R$是\ProgramName 的计算结果,$\phi_C$是Citation的计算结果,
$\epsilon(\phi)$是两者的相对偏差。


\begin{table}
\centering
\caption{静态IAEA PWR三维基准题Citation计算结果}
\label{tab:result.iaea.citation}
\begin{tabular}{ccr}
\topline
网格尺寸/cm & $k_\mathrm{eff}$ & 计算时间/s \\
\midline
5.0 & 1.0288101 &    6.36 \\
2.5 & 1.0290707 &  233.70 \\
2.0 & 1.0291338 &  856.08 \\
1.0 & 1.0292318 & 38005.0 \\
\bottomline
\end{tabular}
\end{table}

\begin{table}
\centering
\caption{静态IAEA PWR三维基准题\ProgramName 计算结果及误差}
\label{tab:result.iaea.self}
\subcaptionbox{CG-SG算法\label{tab:result.iaea.self.cg}}{
\begin{tabular}{crcccc}
\topline
网格尺寸/cm 
        & 计算时间/s & $k_\mathrm{eff}$ 
        & $\epsilon_{k_\mathrm{eff}}$
        & $\max\epsilon_{\phi_{\bm{k},g}}$
        & $\max\epsilon_{\phi_{A,g}}$\\
\midline
5.0 &    2.637 & 1.0288103 & $1.9\times10^{-7}$ & $5.7\times10^{-4}$ & $1.6\times10^{-5}$\\
2.5 &    8.268 & 1.0290725 & $1.8\times10^{-6}$ & $5.8\times10^{-4}$ & $8.3\times10^{-6}$\\
2.0 &   15.444 & 1.0291343 & $4.9\times10^{-7}$ & $5.9\times10^{-4}$ & $7.5\times10^{-6}$\\
1.0 &  164.877 & 1.0292398 & $8.0\times10^{-6}$ & $1.0\times10^{-3}$ & $6.0\times10^{-5}$\\
\bottomline
\end{tabular}
}
\\[1cm]
\subcaptionbox{CG-SG Multilevel算法\label{tab:result.iaea.self.cgml}}{
\begin{tabular}{crcccc}
\topline
网格尺寸/cm 
        & 计算时间/s & $k_\mathrm{eff}$ 
        & $\epsilon_{k_\mathrm{eff}}$
        & $\max\epsilon_{\phi_{\bm{k},g}}$
        & $\max\epsilon_{\phi_{A,g}}$\\
\midline
2.5 &    6.006 & 1.0290727 & $2.0\times10^{-6}$ & $6.3\times10^{-4}$ & $3.5\times10^{-5}$\\
1.0 &   78.749 & 1.0292401 & $8.2\times10^{-6}$ & $8.8\times10^{-4}$ & $4.2\times10^{-5}$\\
\bottomline
\end{tabular}
}
\end{table}


\begin{figure}
\centering
\subcaptionbox{快群}{
\begin{tikzpicture}
\def\di#1#2#3#4#5{
\def\boxx{1.6}
\def\boxy{1.3}
\draw (#1*\boxx, #2*\boxy) -- +(\boxx,0) -- +(\boxx,\boxy) -- +(0,\boxy) -- +(0,0);
\node [right] at (#1*\boxx,#2*\boxy+1) {#3};
\node [right]  at (#1*\boxx,#2*\boxy+0.65) {#4};
\node [right]  at (#1*\boxx,#2*\boxy+0.25) {#5};
}
\di{0}{5}{$\phi_R$}{$\phi_C$}{$\epsilon(\phi)$}
\di{0}{0}{0.70262}{0.70262}{1.3E-05}
\di{1}{0}{0.90547}{0.90548}{1.0E-05}
\di{1}{1}{0.97516}{0.97519}{2.9E-05}
\di{2}{0}{0.99228}{0.99227}{0.2E-05}
\di{2}{1}{0.99999}{1.00000}{1.3E-05}
\di{2}{2}{0.97502}{0.97504}{2.7E-05}
\di{3}{0}{0.84486}{0.84487}{1.0E-05}
\di{3}{1}{0.90290}{0.90295}{4.8E-05}
\di{3}{2}{0.91784}{0.91789}{5.3E-05}
\di{3}{3}{0.82549}{0.82553}{4.0E-05}
\di{4}{0}{0.59191}{0.59190}{0.7E-05}
\di{4}{1}{0.76170}{0.76171}{1.1E-05}
\di{4}{2}{0.82799}{0.82801}{2.7E-05}
\di{4}{3}{0.69022}{0.69022}{0.1E-05}
\di{4}{4}{0.45968}{0.45968}{0.5E-05}
\di{5}{0}{0.68223}{0.68221}{2.2E-05}
\di{5}{1}{0.74415}{0.74414}{0.6E-05}
\di{5}{2}{0.76470}{0.76471}{0.9E-05}
\di{5}{3}{0.64706}{0.64705}{2.1E-05}
\di{5}{4}{0.47278}{0.47278}{0.0E-05}
\di{5}{5}{0.31979}{0.31978}{2.4E-05}
\di{6}{0}{0.67966}{0.67965}{2.3E-05}
\di{6}{1}{0.68848}{0.68846}{2.9E-05}
\di{6}{2}{0.66881}{0.66879}{3.4E-05}
\di{6}{3}{0.51772}{0.51771}{1.6E-05}
\di{6}{4}{0.33060}{0.33059}{2.1E-05}
\di{7}{0}{0.47720}{0.47720}{1.2E-05}
\di{7}{1}{0.46513}{0.46512}{0.9E-05}
\di{7}{2}{0.38826}{0.38825}{2.6E-05}
\end{tikzpicture}
}

\subcaptionbox{热群}{
\begin{tikzpicture}
\def\di#1#2#3#4#5{
\def\boxx{1.6}
\def\boxy{1.3}
\draw (#1*\boxx, #2*\boxy) -- +(\boxx,0) -- +(\boxx,\boxy) -- +(0,\boxy) -- +(0,0);
\node [right] at (#1*\boxx,#2*\boxy+1) {#3};
\node [right]  at (#1*\boxx,#2*\boxy+0.65) {#4};
\node [right]  at (#1*\boxx,#2*\boxy+0.25) {#5};
}
\di{0}{5}{$\phi_R$}{$\phi_C$}{$\epsilon(\phi)$}
\di{0}{0}{0.12722}{0.12723}{4.1E-05}
\di{1}{0}{0.22095}{0.22095}{2.7E-05}
\di{1}{1}{0.24050}{0.24051}{2.4E-05}
\di{2}{0}{0.24513}{0.24513}{2.0E-05}
\di{2}{1}{0.24621}{0.24621}{2.5E-05}
\di{2}{2}{0.23376}{0.23377}{6.0E-05}
\di{3}{0}{0.20640}{0.20640}{1.6E-05}
\di{3}{1}{0.22285}{0.22286}{1.5E-05}
\di{3}{2}{0.22612}{0.22612}{1.8E-05}
\di{3}{3}{0.20386}{0.20386}{1.7E-05}
\di{4}{0}{0.10719}{0.10719}{1.8E-05}
\di{4}{1}{0.18626}{0.18626}{0.4E-05}
\di{4}{2}{0.20503}{0.20503}{0.1E-05}
\di{4}{3}{0.16896}{0.16896}{0.1E-05}
\di{4}{4}{0.08353}{0.08354}{1.0E-05}
\di{5}{0}{0.16699}{0.16699}{0.5E-05}
\di{5}{1}{0.18443}{0.18442}{1.5E-05}
\di{5}{2}{0.19002}{0.19002}{0.8E-05}
\di{5}{3}{0.16091}{0.16090}{1.6E-05}
\di{5}{4}{0.12166}{0.12166}{0.9E-05}
\di{5}{5}{0.10275}{0.10275}{1.1E-05}
\di{6}{0}{0.16904}{0.16903}{2.7E-05}
\di{6}{1}{0.17154}{0.17154}{0.8E-05}
\di{6}{2}{0.17471}{0.17470}{1.5E-05}
\di{6}{3}{0.15023}{0.15023}{2.1E-05}
\di{6}{4}{0.10541}{0.10541}{1.0E-05}
\di{7}{0}{0.13792}{0.13791}{2.7E-05}
\di{7}{1}{0.13411}{0.13410}{2.3E-05}
\di{7}{2}{0.12457}{0.12457}{2.4E-05}
\end{tikzpicture}
}
\caption{静态IAEA三维基准题\ProgramName 程序 CG-SG 1cm网格$1/8$堆芯组件通量相对偏差}
\label{fig:result.iaea.aphitable.cg}
\end{figure}


\begin{figure}
\centering
\subcaptionbox{快群}{
\begin{tikzpicture}
\def\di#1#2#3#4#5{
\def\boxx{1.6}
\def\boxy{1.3}
\draw (#1*\boxx, #2*\boxy) -- +(\boxx,0) -- +(\boxx,\boxy) -- +(0,\boxy) -- +(0,0);
\node [right] at (#1*\boxx,#2*\boxy+1) {#3};
\node [right]  at (#1*\boxx,#2*\boxy+0.65) {#4};
\node [right]  at (#1*\boxx,#2*\boxy+0.25) {#5};
}
\di{0}{5}{$\phi_R$}{$\phi_C$}{$\epsilon(\phi)$}
\di{0}{0}{0.70263}{0.70262}{0.6E-05}
\di{1}{0}{0.90548}{0.90548}{0.9E-05}
\di{1}{1}{0.97518}{0.97519}{0.3E-05}
\di{2}{0}{0.99229}{0.99227}{2.0E-05}
\di{2}{1}{1.00001}{1.00000}{0.8E-05}
\di{2}{2}{0.97504}{0.97504}{0.6E-05}
\di{3}{0}{0.84487}{0.84487}{0.5E-05}
\di{3}{1}{0.90291}{0.90295}{3.7E-05}
\di{3}{2}{0.91785}{0.91789}{4.2E-05}
\di{3}{3}{0.82550}{0.82553}{3.4E-05}
\di{4}{0}{0.59191}{0.59190}{1.0E-05}
\di{4}{1}{0.76171}{0.76171}{0.6E-05}
\di{4}{2}{0.82799}{0.82801}{2.4E-05}
\di{4}{3}{0.69022}{0.69022}{0.2E-05}
\di{4}{4}{0.45968}{0.45968}{0.5E-05}
\di{5}{0}{0.68222}{0.68221}{1.5E-05}
\di{5}{1}{0.74414}{0.74414}{0.0E-05}
\di{5}{2}{0.76470}{0.76471}{1.4E-05}
\di{5}{3}{0.64706}{0.64705}{1.3E-05}
\di{5}{4}{0.47278}{0.47278}{1.0E-05}
\di{5}{5}{0.31978}{0.31978}{1.4E-05}
\di{6}{0}{0.67965}{0.67965}{0.9E-05}
\di{6}{1}{0.68847}{0.68846}{1.4E-05}
\di{6}{2}{0.66880}{0.66879}{2.3E-05}
\di{6}{3}{0.51772}{0.51771}{0.3E-05}
\di{6}{4}{0.33059}{0.33059}{0.6E-05}
\di{7}{0}{0.47719}{0.47720}{0.7E-05}
\di{7}{1}{0.46512}{0.46512}{0.6E-05}
\di{7}{2}{0.38825}{0.38825}{1.0E-05}
\end{tikzpicture}
}

\subcaptionbox{热群}{
\begin{tikzpicture}
\def\di#1#2#3#4#5{
\def\boxx{1.6}
\def\boxy{1.3}
\draw (#1*\boxx, #2*\boxy) -- +(\boxx,0) -- +(\boxx,\boxy) -- +(0,\boxy) -- +(0,0);
\node [right] at (#1*\boxx,#2*\boxy+1) {#3};
\node [right]  at (#1*\boxx,#2*\boxy+0.65) {#4};
\node [right]  at (#1*\boxx,#2*\boxy+0.25) {#5};
}
\di{0}{5}{$\phi_R$}{$\phi_C$}{$\epsilon(\phi)$}
\di{0}{0}{0.12722}{0.12723}{1.4E-05}
\di{1}{0}{0.22095}{0.22095}{0.7E-05}
\di{1}{1}{0.24051}{0.24051}{0.2E-05}
\di{2}{0}{0.24513}{0.24513}{0.3E-05}
\di{2}{1}{0.24621}{0.24621}{0.3E-05}
\di{2}{2}{0.23377}{0.23377}{3.5E-05}
\di{3}{0}{0.20640}{0.20640}{0.2E-05}
\di{3}{1}{0.22286}{0.22286}{0.1E-05}
\di{3}{2}{0.22612}{0.22612}{0.4E-05}
\di{3}{3}{0.20386}{0.20386}{1.1E-05}
\di{4}{0}{0.10719}{0.10719}{0.9E-05}
\di{4}{1}{0.18626}{0.18626}{0.2E-05}
\di{4}{2}{0.20503}{0.20503}{0.4E-05}
\di{4}{3}{0.16896}{0.16896}{0.1E-05}
\di{4}{4}{0.08353}{0.08354}{1.0E-05}
\di{5}{0}{0.16699}{0.16699}{0.2E-05}
\di{5}{1}{0.18442}{0.18442}{0.9E-05}
\di{5}{2}{0.19002}{0.19002}{0.4E-05}
\di{5}{3}{0.16091}{0.16090}{1.0E-05}
\di{5}{4}{0.12166}{0.12166}{0.0E-05}
\di{5}{5}{0.10275}{0.10275}{0.1E-05}
\di{6}{0}{0.16904}{0.16903}{1.0E-05}
\di{6}{1}{0.17153}{0.17154}{0.5E-05}
\di{6}{2}{0.17471}{0.17470}{0.3E-05}
\di{6}{3}{0.15023}{0.15023}{0.9E-05}
\di{6}{4}{0.10541}{0.10541}{0.0E-05}
\di{7}{0}{0.13792}{0.13791}{0.8E-05}
\di{7}{1}{0.13410}{0.13410}{0.8E-05}
\di{7}{2}{0.12457}{0.12457}{0.6E-05}
\end{tikzpicture}
}
\caption{静态IAEA三维基准题\ProgramName 程序 CG-SG Multilevel 1cm网格$1/8$堆芯组件通量相对偏差}
\label{fig:result.iaea.aphitable.cgml}
\end{figure}

\subsection{动态TWIGL二维基准问题}

动态TWIGL二维基准题的计算结果及对比见\floatref{tab:result.twigl.power-compare},
可见\ProgramName 和各种节块程序的结果符合的较好。

\begin{table}
\centering
\begin{minipage}{\textwidth}
\centering
\caption{动态TWIGL二维基准问题计算结果(堆芯相对功率)\label{tab:result.twigl.power-compare}}
\subcaptionbox{阶跃反应性基准题\label{tab:result.twigl.power-compare.1}}
{
\begin{tabular}{cccccc}
\topline
系统时刻/s & QUANDRY\footnote{使用解析节块法,时间步长0.01s,
             数据来源\onlinecite{smith1979analytic,zhaowenbo}。}
         & NEM\footnote{使用节块展开法,网格8cm$\times$8cm,时间步长0.005s,
             数据来源\onlinecite{bandini1990three,zhaowenbo}。}
         & NGFMN-K\footnote{使用节块格林函数法,网格8cm$\times$8cm,计算方法DIRK(5,4)-E,
             计算精度1e-5,数据来源\onlinecite{zhaowenbo}。}
         & SPANDEX\footnote{使用变时间步节块展开法,网格4cm$\times$4cm,时间步长0.0001s,
             数据来源\onlinecite{aviles1993development,sutton1996diffusion}。}
         & \ProgramName \footnote{本文工作,细网有限差分法,
             网格划分1cm$\times$1cm,时间步长$0.01$s。}
         \\
\midline
0.1 & 2.064 & 2.060 & 2.061 & 2.062 & 2.060 \\
0.2 & 2.076 & 2.078 & 2.079 & 2.079 & 2.079 \\
0.3 & 2.095 & 2.095 & 2.096 & 2.096 & 2.096 \\
0.4 & 2.112 & 2.113 & 2.114 & 2.114 & 2.114 \\
0.5 & 2.130 & 2.131 & 2.131 & 2.131 & 2.131 \\
\bottomline
\end{tabular}
}
\\[1cm]
\subcaptionbox{线性反应性基准题\label{tab:result.twigl.power-compare.2}}
{
\begin{tabular}{cccccc}
\topline
系统时刻/s & QUANDRY\footnote{时间步长0.005s,
             数据来源\onlinecite{smith1979analytic,zhaowenbo}。}
         & NEM\footnote{网格8cm$\times$8cm,时间步长0.005s,
             数据来源\onlinecite{bandini1990three,zhaowenbo}。}
         & NGFMN-K\footnote{网格8cm$\times$8cm,计算方法DIRK(5,4)-E,
             计算精度1e-5,数据来源\onlinecite{zhaowenbo}。}
         & SPANDEX\footnote{网格4cm$\times$4cm,时间步长0.0001s,
             数据来源\onlinecite{aviles1993development,sutton1996diffusion}。}
         & \ProgramName \footnote{网格划分1cm$\times$1cm,时间步长$0.005$s。}
         \\
\midline
0.1 & 1.305 & 1.309 & 1.309 & 1.309 & 1.309 \\
0.2 & 1.954 & 1.962 & 1.960 & 1.960 & 1.962 \\
0.3 & 2.074 & 2.075 & 2.075 & 2.075 & 2.076 \\
0.4 & 2.092 & 2.092 & 2.092 & 2.092 & 2.093 \\
0.5 & 2.109 & 2.110 & 2.110 & 2.110 & 2.111 \\
\bottomline
\end{tabular}
}
\end{minipage}
\end{table}


\subsection{动态LMW三维基准问题}

动态LMW三维基准题的计算结果及对比见\floatref{tab:result.lmw.power-compare}。

\begin{sidewaystable}
\pdfrotate
\centering
\begin{minipage}{0.8\textwidth}
\centering
\caption{动态LMW三维基准问题计算结果(堆芯相对功率)}
\label{tab:result.lmw.power-compare}
\begin{tabular}{ccccccc}
\topline
系统时刻/s & SKETCH-N\footnote{使用基于非线性迭代策略的CMFD方法,网格10cm$\times$10cm$\times$5cm,时间步长0.25s,
             数据来源\onlinecite{zimin1998nodal}。}
         & PANTHER\footnote{使用基于非线性迭代策略的解析节块法,网格10cm$\times$10cm$\times$5cm,时间步长0.25s,
             数据来源\onlinecite{sutton1996diffusion}。}
         & NGFMN-K\footnote{使用节块格林函数法,网格10cm$\times$10cm$\times$5cm,计算方法DIRK(5,4)-E,
             计算精度1e-5,数据来源\onlinecite{zhaowenbo}。}
         & SPANDEX\footnote{使用变时间步节块展开法,网格5cm$\times$5cm$\times$2.5cm,计算精度5e-2,
             数据来源\onlinecite{aviles1993development,sutton1996diffusion}。}
         & NLSANMT\footnote{使用基于非线性迭代策略的CMFD方法,网格10cm$\times$10cm$\times$5cm,时间步长0.25s,
             数据来源\onlinecite{liaochengkui,zhaowenbo}。}
         & \ProgramName \footnote{本文工作,细网有限差分法,
             网格划分1cm$\times$1cm$\times$1cm,时间步长$1/12$s。} 
         \\
\midline
10 & 1.338 & 1.347 & 1.341 & 1.341 & 1.339 & 1.341\\
20 & 1.705 & 1.726 & 1.720 & 1.713 & 1.706 & 1.705\\
30 & 1.369 & 1.382 & 1.381 & 1.373 & 1.368 & 1.362\\
40 & 0.809 & 1.813 & 0.814 & 0.809 & 0.808 & 0.804\\
50 & 0.502 & 0.505 & 0.504 & 0.503 & 0.502 & 0.501\\
60 & 0.385 & 0.387 & 0.387 & 0.386 & 0.385 & 0.385\\ 
\bottomline
\end{tabular}
\end{minipage}
\end{sidewaystable}


\FloatBarrier
\section{计算时间与加速效果分析}

\subsection{静态计算}

\begin{comment}

由于\ProgramName 和Citation均使用迭代算法进行求解,
而迭代算法的计算时间和通量的收敛程度有直接关系,
一般来说计算时间和迭代次数成正比。
但无论是计算本征值的外迭代还是求解线性方程的内迭代,
\ProgramName 和Citation都使用了不同的迭代方式,
两者对误差的估计方式也有所差别,不宜直接进行比较。

\begin{figure}
\centering
\includegraphics{testresult_iaea_1cm}
\caption{\label{fig:testresult.iaea}静态IAEA三维基准题1cm网格\ProgramName 与Citation的计算时间-通量最大偏差图}
\end{figure}

在实际应用中,相同误差时的计算时间才能反映程序间的速度差异,
这里分别选取不同的收敛精度使用\ProgramName 和Citation进行计算,
绘制两者的计算时间--偏差图,
这里取Citation在$\epsilon_\phi=5\times10^{-7}$时的计算结果作为参考值,
对于静态IAEA PWR 三维基准题1cm网格划分,
\ProgramName 与Citation在不同精度下的计算时间见\floatref{fig:testresult.iaea},
可见不同的算法的计算时间和结果精度的关系差异很大,对结果进行拟合得
\begin{align}
  &T_\mathrm{CG-SG\ ML} = \exp\pb{-0.235151\log \epsilon_\phi-146.097\epsilon_\phi+1.87266}\\
  &T_\mathrm{CG-SG} = \exp\pb{-0.307056\log \epsilon_\phi+2.02687} \\
  &T_\mathrm{Citation} = \exp\pb{-414.444\epsilon_\phi+10.7569} 
\end{align}
取结果偏差为$\epsilon_\phi=0.01,0.003,0.001$进行比较得

\begin{table}
\centering
\begin{tabular}{crrr}
\topline
$\epsilon_\phi$ & $T_\mathrm{Citation}$ & $T_\mathrm{CG-SG\ ML}$ & $T_\mathrm{CG-SG}$\\
\midline
0.01  &   744.31 &  4.46 & 31.22 \\
0.003 & 13542.20 & 16.45 & 45.18 \\
0.001 & 31022.10 & 28.53 & 63.30 \\
\bottomline
\end{tabular}
\end{table}

\end{comment}

以静态IAEA PWR 三维基准题为例比较\ProgramName 和Citation程序的结果,
使用Citation程序在$\epsilon_\phi=5\times10^{-7}$时的计算结果作为参考值。
对于1cm网格,不同通量收敛精度下\ProgramName 和Citation程序的结果见\floatref{tab:testresult.iaea},
可见,相同的收敛精度下,CG-SG MultiLevel和CG-SG算法都明显快于Citation程序,
CG-SG MultiLevel的加速效果在52-272倍,CG-SG的加速效果在21-130倍,
$k_\mathrm{eff}$的收敛效果差于Citation程序,但也在$2\times10^{-5}$的范围内,
对于最大通量偏差,CG-SG MultiLevel和CG-SG算法的结果略好于Citation程序。

\begin{table}
\centering
\caption{\label{tab:testresult.iaea}静态IAEA三维基准题1cm网格\ProgramName 与Citation计算结果}
\begin{minipage}{\textwidth}
\centering
\begin{tabular}{cccccc}
\topline
算法 & 通量收敛精度 & $T/s$ & $\epsilon_{k_\mathrm{eff}}$ & $\epsilon_{\phi_{\bm{k},g}}$ & 加速比\\
\midline
Citation & $10^{-4}$ & 1338.96 & 8e-07\footnote{Citation只给出8位有效数字。} & 1.336767e-02 & --\\
Citation & $10^{-5}$ & 2143.08 & 5e-07 & 7.224095e-03 & --\\
Citation & $10^{-6}$ & 21437.00 & 0e-07 & 1.999625e-03 & --\\
CS-SG ML & $10^{-4}$ & 18.299 & 2.080200e-05 & 1.126408e-02 & 73.2\\
CS-SG ML & $10^{-5}$ & 41.215 & 1.003682e-05 & 1.322194e-03 & 52.0\\
CS-SG ML & $10^{-6}$ & 78.749 & 8.253441e-06 & 8.773718e-04 & 272.2\\
CS-SG & $10^{-4}$ &  48.376 & 1.939363e-05 & 2.183641e-02 & 27.7\\
CS-SG & $10^{-5}$ & 101.104 & 7.378584e-06 & 2.226546e-03 & 21.2\\
CS-SG & $10^{-6}$ & 164.877 & 8.031013e-06 & 1.035618e-03 & 130.0\\
\bottomline
\end{tabular}
\end{minipage}
\end{table}


\subsection{时空动力学}

上一节中的时空动力学程序大多缺少和时间步对应的计算时间结果,
无法进行直接比较,这里只分析\ProgramName 程序自己的计算结果。

\subsubsection{动态TWIGL二维基准问题}

动态TWIGL二维基准题阶跃反应性计算结果见
\floatref{tab:testresult.twigl.1.1-2}和
\floatref{tab:testresult.twigl.1.4-8},
从结果中可以看出,网格大小放大为4cm时,
给定时刻的堆芯总功率偏差只有0.001左右,
网格大小进一步放大为8cm时,总功率偏差大约为0.02。
网格大小固定为4cm时,时间步长对总功率影响主要在第一个0.1s内,
步长从0.01放大到0.1的过程中,
0.1s时的堆芯总功率偏差逐渐达到0.09左右,
而0.2s时的堆芯总功率偏差最大只有0.09左右。
网格大小为4cm,时间步长取0.05s时,计算时间约等于模型时间,
基本达到实时模拟,此时最大总功率偏差出现在0.1s,偏差值约为0.03。

线性反应性计算结果见
\floatref{tab:testresult.twigl.2.1-2}和
\floatref{tab:testresult.twigl.2.4-8},
当网格大小放大到8cm时,总功率偏差值约有0.02。
固定网格大小为8cm后,调整时间步长,
总功率偏差主要出现在0.2s,偏差值最大达到0.04。
网格大小为2cm时间步长取0.10s时,
以及网格大小为4cm时间步长取0.05s时,能够实现实时模拟。


\begin{table}
\centering
\caption{动态TWIGL二维基准题阶跃反应性计算结果1\label{tab:testresult.twigl.1.1-2}}
\subcaptionbox{网格大小1cm 时间步长0.01s}
{
\small
\begin{tabular}{cccc}
\topline
模型 & 堆芯 & 计算 & 累计计算\\
时间/s & 功率 & 时间/s & 时间/s\\
\midline
init & 1.000 & -- & 1.825\\
0.1 & 2.060 & 1.700 & 3.525\\
0.2 & 2.079 & 1.482 & 5.007\\
0.3 & 2.096 & 1.513 & 6.520\\
0.4 & 2.114 & 1.544 & 8.064\\
0.5 & 2.131 & 1.496 & 9.560\\
\bottomline
\end{tabular}
}
\subcaptionbox{网格大小1cm 时间步长0.02s}
{
\small
\begin{tabular}{cccc}
\topline
模型 & 堆芯 & 计算 & 累计计算\\
时间/s & 功率 & 时间/s & 时间/s\\
\midline
init & 1.000 & -- & 1.935\\
0.1 & 2.057 & 0.934 & 2.869\\
0.2 & 2.079 & 0.842 & 3.711\\
0.3 & 2.096 & 0.843 & 4.554\\
0.4 & 2.114 & 0.858 & 5.412\\
0.5 & 2.131 & 0.858 & 6.270\\
\bottomline
\end{tabular}
}

\subcaptionbox{网格大小1cm 时间步长0.05s}
{
\small
\begin{tabular}{cccc}
\topline
模型 & 堆芯 & 计算 & 累计计算\\
时间/s & 功率 & 时间/s & 时间/s\\
\midline
init & 1.000 & -- & 1.887\\
0.1 & 2.031 & 0.390 & 2.277\\
0.2 & 2.078 & 0.359 & 2.636\\
0.3 & 2.096 & 0.343 & 2.979\\
0.4 & 2.114 & 0.343 & 3.322\\
0.5 & 2.131 & 0.358 & 3.680\\
\bottomline
\end{tabular}
}
\subcaptionbox{网格大小1cm 时间步长0.10s}
{
\small
\begin{tabular}{cccc}
\topline
模型 & 堆芯 & 计算 & 累计计算\\
时间/s & 功率 & 时间/s & 时间/s\\
\midline
init & 1.000 & -- & 2.356\\
0.1 & 1.965 & 0.312 & 2.668\\
0.2 & 2.070 & 0.266 & 2.934\\
0.3 & 2.095 & 0.250 & 3.184\\
0.4 & 2.114 & 0.249 & 3.433\\
0.5 & 2.132 & 0.250 & 3.683\\
\bottomline
\end{tabular}
}


\subcaptionbox{网格大小2cm 时间步长0.01s}
{
\small
\begin{tabular}{cccc}
\topline
模型 & 堆芯 & 计算 & 累计计算\\
时间/s & 功率 & 时间/s & 时间/s\\
\midline
init & 1.000 & -- & 1.497\\
0.1 & 2.060 & 0.811 & 2.308\\
0.2 & 2.078 & 0.700 & 3.008\\
0.3 & 2.096 & 0.684 & 3.692\\
0.4 & 2.113 & 0.732 & 4.424\\
0.5 & 2.131 & 0.764 & 5.188\\
\bottomline
\end{tabular}
}
\subcaptionbox{网格大小2cm 时间步长0.02s}
{
\small
\begin{tabular}{cccc}
\topline
模型 & 堆芯 & 计算 & 累计计算\\
时间/s & 功率 & 时间/s & 时间/s\\
\midline
init & 1.000 & -- & 1.482\\
0.1 & 2.056 & 0.406 & 1.888\\
0.2 & 2.078 & 0.330 & 2.218\\
0.3 & 2.096 & 0.375 & 2.593\\
0.4 & 2.113 & 0.326 & 2.919\\
0.5 & 2.131 & 0.390 & 3.309\\
\bottomline
\end{tabular}
}
\subcaptionbox{网格大小2cm 时间步长0.05s}
{
\small
\begin{tabular}{cccc}
\topline
模型 & 堆芯 & 计算 & 累计计算\\
时间/s & 功率 & 时间/s & 时间/s\\
\midline
init & 1.000 & -- & 1.482\\
0.1 & 2.031 & 0.172 & 1.654\\
0.2 & 2.077 & 0.156 & 1.810\\
0.3 & 2.096 & 0.124 & 1.934\\
0.4 & 2.113 & 0.156 & 2.090\\
0.5 & 2.131 & 0.156 & 2.246\\
\bottomline
\end{tabular}
}
\subcaptionbox{网格大小2cm 时间步长0.10s}
{
\small
\begin{tabular}{cccc}
\topline
模型 & 堆芯 & 计算 & 累计计算\\
时间/s & 功率 & 时间/s & 时间/s\\
\midline
init & 1.000 & -- & 1.482\\
0.1 & 1.964 & 0.093 & 1.575\\
0.2 & 2.069 & 0.078 & 1.653\\
0.3 & 2.095 & 0.078 & 1.731\\
0.4 & 2.113 & 0.078 & 1.809\\
0.5 & 2.131 & 0.078 & 1.887\\
\bottomline
\end{tabular}
}
\end{table}

\begin{table}
\centering
\caption{动态TWIGL二维基准题阶跃反应性计算结果2\label{tab:testresult.twigl.1.4-8}}
\subcaptionbox{网格大小4cm 时间步长0.01s}
{
\small
\begin{tabular}{cccc}
\topline
模型 & 堆芯 & 计算 & 累计计算\\
时间/s & 功率 & 时间/s & 时间/s\\
\midline
init & 1.000 & -- & 1.092\\
0.1 & 2.061 & 0.437 & 1.529\\
0.2 & 2.079 & 0.358 & 1.887\\
0.3 & 2.096 & 0.360 & 2.247\\
0.4 & 2.114 & 0.374 & 2.621\\
0.5 & 2.132 & 0.404 & 3.025\\
\bottomline
\end{tabular}
}
\subcaptionbox{网格大小4cm 时间步长0.02s}
{
\small
\begin{tabular}{cccc}
\topline
模型 & 堆芯 & 计算 & 累计计算\\
时间/s & 功率 & 时间/s & 时间/s\\
\midline
init & 1.000 & -- & 1.169\\
0.1 & 2.057 & 0.265 & 1.434\\
0.2 & 2.079 & 0.249 & 1.683\\
0.3 & 2.096 & 0.219 & 1.902\\
0.4 & 2.114 & 0.234 & 2.136\\
0.5 & 2.132 & 0.250 & 2.386\\
\bottomline
\end{tabular}
}

\subcaptionbox{网格大小4cm 时间步长0.05s}
{
\small
\begin{tabular}{cccc}
\topline
模型 & 堆芯 & 计算 & 累计计算\\
时间/s & 功率 & 时间/s & 时间/s\\
\midline
init & 1.000 & -- & 1.092\\
0.1 & 2.031 & 0.109 & 1.201\\
0.2 & 2.078 & 0.094 & 1.295\\
0.3 & 2.097 & 0.062 & 1.357\\
0.4 & 2.114 & 0.078 & 1.435\\
0.5 & 2.132 & 0.078 & 1.513\\
\bottomline
\end{tabular}
}
\subcaptionbox{网格大小4cm 时间步长0.10s}
{
\small
\begin{tabular}{cccc}
\topline
模型 & 堆芯 & 计算 & 累计计算\\
时间/s & 功率 & 时间/s & 时间/s\\
\midline
init & 1.000 & -- & 1.232\\
0.1 & 1.965 & 0.062 & 1.294\\
0.2 & 2.070 & 0.047 & 1.341\\
0.3 & 2.096 & 0.047 & 1.388\\
0.4 & 2.114 & 0.031 & 1.419\\
0.5 & 2.132 & 0.031 & 1.450\\
\bottomline
\end{tabular}
}


\subcaptionbox{网格大小8cm 时间步长0.01s}
{
\small
\begin{tabular}{cccc}
\topline
模型 & 堆芯 & 计算 & 累计计算\\
时间/s & 功率 & 时间/s & 时间/s\\
\midline
init & 1.000 & -- & 0.920\\
0.1 & 2.080 & 0.233 & 1.153\\
0.2 & 2.099 & 0.204 & 1.357\\
0.3 & 2.117 & 0.218 & 1.575\\
0.4 & 2.135 & 0.204 & 1.779\\
0.5 & 2.153 & 0.204 & 1.983\\
\bottomline
\end{tabular}
}
\subcaptionbox{网格大小8cm 时间步长0.02s}
{
\small
\begin{tabular}{cccc}
\topline
模型 & 堆芯 & 计算 & 累计计算\\
时间/s & 功率 & 时间/s & 时间/s\\
\midline
init & 1.000 & -- & 0.889\\
0.1 & 2.076 & 0.126 & 1.015\\
0.2 & 2.099 & 0.110 & 1.125\\
0.3 & 2.117 & 0.110 & 1.235\\
0.4 & 2.135 & 0.095 & 1.330\\
0.5 & 2.153 & 0.095 & 1.425\\
\bottomline
\end{tabular}
}
\subcaptionbox{网格大小8cm 时间步长0.05s}
{
\small
\begin{tabular}{cccc}
\topline
模型 & 堆芯 & 计算 & 累计计算\\
时间/s & 功率 & 时间/s & 时间/s\\
\midline
init & 1.000 & -- & 0.780\\
0.1 & 2.050 & 0.063 & 0.843\\
0.2 & 2.098 & 0.062 & 0.905\\
0.3 & 2.117 & 0.032 & 0.937\\
0.4 & 2.135 & 0.047 & 0.984\\
0.5 & 2.153 & 0.032 & 1.016\\
\bottomline
\end{tabular}
}
\subcaptionbox{网格大小8cm 时间步长0.10s}
{
\small
\begin{tabular}{cccc}
\topline
模型 & 堆芯 & 计算 & 累计计算\\
时间/s & 功率 & 时间/s & 时间/s\\
\midline
init & 1.000 & -- & 0.795\\
0.1 & 1.982 & 0.031 & 0.826\\
0.2 & 2.089 & 0.031 & 0.857\\
0.3 & 2.116 & 0.016 & 0.873\\
0.4 & 2.135 & 0.015 & 0.888\\
0.5 & 2.153 & 0.015 & 0.903\\
\bottomline
\end{tabular}
}
\end{table}

%-----------------------TWIGL 2-------------------------------------

\begin{table}
\centering
\caption{动态TWIGL二维基准题线性反应性计算结果1\label{tab:testresult.twigl.2.1-2}}
\subcaptionbox{网格大小1cm 时间步长0.005s}
{
\small
\begin{tabular}{cccc}
\topline
模型 & 堆芯 & 计算 & 累计计算\\
时间/s & 功率 & 时间/s & 时间/s\\
\midline
init & 1.000 & -- & 1.919\\
0.1 & 1.309 & 3.823 & 5.742\\
0.2 & 1.962 & 3.900 & 9.642\\
0.3 & 2.076 & 3.384 & 13.026\\
0.4 & 2.093 & 3.370 & 16.396\\
0.5 & 2.111 & 3.292 & 19.688\\
\bottomline
\end{tabular}
}
\subcaptionbox{网格大小1cm 时间步长0.02s}
{
\small
\begin{tabular}{cccc}
\topline
模型 & 堆芯 & 计算 & 累计计算\\
时间/s & 功率 & 时间/s & 时间/s\\
\midline
init & 1.000 & -- & 2.137\\
0.1 & 1.311 & 0.982 & 3.119\\
0.2 & 1.969 & 1.279 & 4.398\\
0.3 & 2.077 & 0.890 & 5.288\\
0.4 & 2.095 & 0.795 & 6.083\\
0.5 & 2.112 & 0.843 & 6.926\\
\bottomline
\end{tabular}
}

\subcaptionbox{网格大小1cm 时间步长0.05s}
{
\small
\begin{tabular}{cccc}
\topline
模型 & 堆芯 & 计算 & 累计计算\\
时间/s & 功率 & 时间/s & 时间/s\\
\midline
init & 1.000 & -- & 1.794\\
0.1 & 1.313 & 0.375 & 2.169\\
0.2 & 1.982 & 0.390 & 2.559\\
0.3 & 2.078 & 0.344 & 2.903\\
0.4 & 2.097 & 0.312 & 3.215\\
0.5 & 2.115 & 0.328 & 3.543\\
\bottomline
\end{tabular}
}
\subcaptionbox{网格大小1cm 时间步长0.10s}
{
\small
\begin{tabular}{cccc}
\topline
模型 & 堆芯 & 计算 & 累计计算\\
时间/s & 功率 & 时间/s & 时间/s\\
\midline
init & 1.000 & -- & 1.825\\
0.1 & 1.318 & 0.218 & 2.043\\
0.2 & 2.000 & 0.203 & 2.246\\
0.3 & 2.079 & 0.172 & 2.418\\
0.4 & 2.102 & 0.156 & 2.574\\
0.5 & 2.120 & 0.156 & 2.730\\
\bottomline
\end{tabular}
}


\subcaptionbox{网格大小2cm 时间步长0.005s}
{
\small
\begin{tabular}{cccc}
\topline
模型 & 堆芯 & 计算 & 累计计算\\
时间/s & 功率 & 时间/s & 时间/s\\
\midline
init & 1.000 & -- & 1.497\\
0.1 & 1.309 & 1.623 & 3.120\\
0.2 & 1.961 & 1.608 & 4.728\\
0.3 & 2.075 & 1.405 & 6.133\\
0.4 & 2.093 & 1.342 & 7.475\\
0.5 & 2.110 & 1.356 & 8.831\\
\bottomline
\end{tabular}
}
\subcaptionbox{网格大小2cm 时间步长0.02s}
{
\small
\begin{tabular}{cccc}
\topline
模型 & 堆芯 & 计算 & 累计计算\\
时间/s & 功率 & 时间/s & 时间/s\\
\midline
init & 1.000 & -- & 1.513\\
0.1 & 1.310 & 0.421 & 1.934\\
0.2 & 1.968 & 0.437 & 2.371\\
0.3 & 2.076 & 0.390 & 2.761\\
0.4 & 2.094 & 0.390 & 3.151\\
0.5 & 2.112 & 0.342 & 3.493\\
\bottomline
\end{tabular}
}
\subcaptionbox{网格大小2cm 时间步长0.05s}
{
\small
\begin{tabular}{cccc}
\topline
模型 & 堆芯 & 计算 & 累计计算\\
时间/s & 功率 & 时间/s & 时间/s\\
\midline
init & 1.000 & -- & 2.278\\
0.1 & 1.313 & 0.281 & 2.559\\
0.2 & 1.981 & 0.281 & 2.840\\
0.3 & 2.077 & 0.233 & 3.073\\
0.4 & 2.097 & 0.234 & 3.307\\
0.5 & 2.115 & 0.171 & 3.478\\
\bottomline
\end{tabular}
}
\subcaptionbox{网格大小2cm 时间步长0.10s}
{
\small
\begin{tabular}{cccc}
\topline
模型 & 堆芯 & 计算 & 累计计算\\
时间/s & 功率 & 时间/s & 时间/s\\
\midline
init & 1.000 & -- & 1.576\\
0.1 & 1.318 & 0.094 & 1.670\\
0.2 & 1.999 & 0.093 & 1.763\\
0.3 & 2.078 & 0.094 & 1.857\\
0.4 & 2.102 & 0.078 & 1.935\\
0.5 & 2.120 & 0.078 & 2.013\\
\bottomline
\end{tabular}
}
\end{table}

\begin{table}
\centering
\caption{动态TWIGL二维基准题线性反应性计算结果2\label{tab:testresult.twigl.2.4-8}}
\subcaptionbox{网格大小4cm 时间步长0.005s}
{
\small
\begin{tabular}{cccc}
\topline
模型 & 堆芯 & 计算 & 累计计算\\
时间/s & 功率 & 时间/s & 时间/s\\
\midline
init & 1.000 & -- & 1.217\\
0.1 & 1.309 & 0.998 & 2.215\\
0.2 & 1.962 & 1.122 & 3.337\\
0.3 & 2.076 & 0.857 & 4.194\\
0.4 & 2.094 & 0.716 & 4.910\\
0.5 & 2.111 & 0.826 & 5.736\\
\bottomline
\end{tabular}
}
\subcaptionbox{网格大小4cm 时间步长0.02s}
{
\small
\begin{tabular}{cccc}
\topline
模型 & 堆芯 & 计算 & 累计计算\\
时间/s & 功率 & 时间/s & 时间/s\\
\midline
init & 1.000 & -- & 1.139\\
0.1 & 1.310 & 0.234 & 1.373\\
0.2 & 1.969 & 0.202 & 1.575\\
0.3 & 2.077 & 0.157 & 1.732\\
0.4 & 2.095 & 0.158 & 1.890\\
0.5 & 2.113 & 0.172 & 2.062\\
\bottomline
\end{tabular}
}

\subcaptionbox{网格大小4cm 时间步长0.05s}
{
\small
\begin{tabular}{cccc}
\topline
模型 & 堆芯 & 计算 & 累计计算\\
时间/s & 功率 & 时间/s & 时间/s\\
\midline
init & 1.000 & -- & 1.107\\
0.1 & 1.313 & 0.094 & 1.201\\
0.2 & 1.982 & 0.094 & 1.295\\
0.3 & 2.078 & 0.063 & 1.358\\
0.4 & 2.098 & 0.078 & 1.436\\
0.5 & 2.115 & 0.063 & 1.499\\
\bottomline
\end{tabular}
}
\subcaptionbox{网格大小4cm 时间步长0.10s}
{
\small
\begin{tabular}{cccc}
\topline
模型 & 堆芯 & 计算 & 累计计算\\
时间/s & 功率 & 时间/s & 时间/s\\
\midline
init & 1.000 & -- & 1.076\\
0.1 & 1.318 & 0.047 & 1.123\\
0.2 & 2.000 & 0.046 & 1.169\\
0.3 & 2.079 & 0.047 & 1.216\\
0.4 & 2.103 & 0.031 & 1.247\\
0.5 & 2.121 & 0.031 & 1.278\\
\bottomline
\end{tabular}
}


\subcaptionbox{网格大小8cm 时间步长0.005s}
{
\small
\begin{tabular}{cccc}
\topline
模型 & 堆芯 & 计算 & 累计计算\\
时间/s & 功率 & 时间/s & 时间/s\\
\midline
init & 1.000 & -- & 0.780\\
0.1 & 1.312 & 0.359 & 1.139\\
0.2 & 1.978 & 0.390 & 1.529\\
0.3 & 2.096 & 0.374 & 1.903\\
0.4 & 2.114 & 0.359 & 2.262\\
0.5 & 2.132 & 0.328 & 2.590\\
\bottomline
\end{tabular}
}
\subcaptionbox{网格大小8cm 时间步长0.02s}
{
\small
\begin{tabular}{cccc}
\topline
模型 & 堆芯 & 计算 & 累计计算\\
时间/s & 功率 & 时间/s & 时间/s\\
\midline
init & 1.000 & -- & 0.873\\
0.1 & 1.314 & 0.141 & 1.014\\
0.2 & 1.985 & 0.078 & 1.092\\
0.3 & 2.097 & 0.093 & 1.185\\
0.4 & 2.115 & 0.110 & 1.295\\
0.5 & 2.133 & 0.093 & 1.388\\
\bottomline
\end{tabular}
}
\subcaptionbox{网格大小8cm 时间步长0.05s}
{
\small
\begin{tabular}{cccc}
\topline
模型 & 堆芯 & 计算 & 累计计算\\
时间/s & 功率 & 时间/s & 时间/s\\
\midline
init & 1.000 & -- & 0.857\\
0.1 & 1.316 & 0.063 & 0.920\\
0.2 & 1.998 & 0.062 & 0.982\\
0.3 & 2.098 & 0.047 & 1.029\\
0.4 & 2.118 & 0.047 & 1.076\\
0.5 & 2.136 & 0.032 & 1.108\\
\bottomline
\end{tabular}
}
\subcaptionbox{网格大小8cm 时间步长0.10s}
{
\small
\begin{tabular}{cccc}
\topline
模型 & 堆芯 & 计算 & 累计计算\\
时间/s & 功率 & 时间/s & 时间/s\\
\midline
init & 1.000 & -- & 0.873\\
0.1 & 1.321 & 0.031 & 0.904\\
0.2 & 2.017 & 0.031 & 0.935\\
0.3 & 2.099 & 0.016 & 0.951\\
0.4 & 2.123 & 0.016 & 0.967\\
0.5 & 2.141 & 0.031 & 0.998\\
\bottomline
\end{tabular}
}
\end{table}



\subsubsection{动态LMW三维基准问题}

动态LMW三维基准题计算结果见
\floatref{tab:testresult.lmw.size1}、
\floatref{tab:testresult.lmw.size2}、
\floatref{tab:testresult.lmw.size25}和
\floatref{tab:testresult.lmw.size5},
从结果可见,把空间网格放大到5cm时,
堆芯总功率的偏差值最大有0.03左右。
固定网格大小为5cm后,放大时间步长至1s,
总功率最大偏差放大到0.04左右,
时刻为30s。


\begin{table}
\centering
\caption{动态LMW三维基准题计算结果1\label{tab:testresult.lmw.size1}}
\subcaptionbox{网格大小1cm 时间步长1/12s}
{
\small
\begin{tabular}{cccc}
\topline
模型 & 堆芯 & 计算 & 累计计算\\
时间/s & 功率 & 时间/s & 时间/s\\
\midline
init & 1.000 & -- & 17.909\\
10.0 & 1.341 & 677.245 & 695.154\\
20.0 & 1.705 & 723.421 & 1418.575\\
30.0 & 1.362 & 714.762 & 2133.337\\
40.0 & 0.804 & 682.797 & 2816.134\\
50.0 & 0.501 & 658.147 & 3474.281\\
60.0 & 0.385 & 606.188 & 4080.469\\
\bottomline
\end{tabular}
}
\subcaptionbox{网格大小1cm 时间步长1/6s}
{
\small
\begin{tabular}{cccc}
\topline
模型 & 堆芯 & 计算 & 累计计算\\
时间/s & 功率 & 时间/s & 时间/s\\
\midline
init & 1.000 & -- & 17.690\\
10.0 & 1.342 & 354.043 & 371.733\\
20.0 & 1.705 & 375.402 & 747.135\\
30.0 & 1.360 & 367.315 & 1114.450\\
40.0 & 0.803 & 353.246 & 1467.696\\
50.0 & 0.501 & 340.503 & 1808.199\\
60.0 & 0.385 & 306.380 & 2114.579\\
\bottomline
\end{tabular}
}

\subcaptionbox{网格大小1cm 时间步长0.5s}
{
\small
\begin{tabular}{cccc}
\topline
模型 & 堆芯 & 计算 & 累计计算\\
时间/s & 功率 & 时间/s & 时间/s\\
\midline
init & 1.000 & -- & 17.815\\
10.0 & 1.349 & 131.836 & 149.651\\
20.0 & 1.711 & 133.583 & 283.234\\
30.0 & 1.356 & 133.971 & 417.205\\
40.0 & 0.799 & 132.957 & 550.162\\
50.0 & 0.500 & 124.378 & 674.540\\
60.0 & 0.384 & 104.364 & 778.904\\
\bottomline
\end{tabular}
}
\subcaptionbox{网格大小1cm 时间步长1s}
{
\begin{tabular}{cccc}
\topline
模型 & 堆芯 & 计算 & 累计计算\\
时间/s & 功率 & 时间/s & 时间/s\\
\midline
init & 1.000 & -- & 18.080\\
10.0 & 1.353 & 67.923 & 86.003\\
20.0 & 1.710 & 69.591 & 155.594\\
30.0 & 1.344 & 68.671 & 224.265\\
40.0 & 0.790 & 67.079 & 291.344\\
50.0 & 0.496 & 65.100 & 356.444\\
60.0 & 0.382 & 56.755 & 413.199\\
\bottomline
\end{tabular}
}
\end{table}

\begin{table}
\centering
\caption{动态LMW三维基准题计算结果2\label{tab:testresult.lmw.size2}}
\subcaptionbox{网格大小2cm 时间步长1/12s}
{
\small
\begin{tabular}{cccc}
\topline
模型 & 堆芯 & 计算 & 累计计算\\
时间/s & 功率 & 时间/s & 时间/s\\
\midline
init & 1.000 & -- & 7.816\\
10.0 & 1.341 & 64.663 & 72.479\\
20.0 & 1.703 & 69.853 & 142.332\\
30.0 & 1.360 & 67.894 & 210.226\\
40.0 & 0.803 & 67.125 & 277.351\\
50.0 & 0.501 & 65.412 & 342.763\\
60.0 & 0.385 & 62.287 & 405.050\\
\bottomline
\end{tabular}
}
\subcaptionbox{网格大小2cm 时间步长1/6s}
{
\small
\begin{tabular}{cccc}
\topline
模型 & 堆芯 & 计算 & 累计计算\\
时间/s & 功率 & 时间/s & 时间/s\\
\midline
init & 1.000 & -- & 7.815\\
10.0 & 1.342 & 33.745 & 41.560\\
20.0 & 1.706 & 36.989 & 78.549\\
30.0 & 1.360 & 35.520 & 114.069\\
40.0 & 0.802 & 35.101 & 149.170\\
50.0 & 0.500 & 34.196 & 183.366\\
60.0 & 0.384 & 31.747 & 215.113\\
\bottomline
\end{tabular}
}

\subcaptionbox{网格大小2cm 时间步长0.5s}
{
\small
\begin{tabular}{cccc}
\topline
模型 & 堆芯 & 计算 & 累计计算\\
时间/s & 功率 & 时间/s & 时间/s\\
\midline
init & 1.000 & -- & 7.612\\
10.0 & 1.348 & 12.260 & 19.872\\
20.0 & 1.711 & 12.825 & 32.697\\
30.0 & 1.356 & 12.371 & 45.068\\
40.0 & 0.799 & 12.246 & 57.314\\
50.0 & 0.499 & 11.637 & 68.951\\
60.0 & 0.384 & 10.325 & 79.276\\
\bottomline
\end{tabular}
}
\subcaptionbox{网格大小2cm 时间步长1s}
{
\begin{tabular}{cccc}
\topline
模型 & 堆芯 & 计算 & 累计计算\\
时间/s & 功率 & 时间/s & 时间/s\\
\midline
init & 1.000 & -- & 7.722\\
10.0 & 1.359 & 6.614 & 14.336\\
20.0 & 1.725 & 6.598 & 20.934\\
30.0 & 1.361 & 6.457 & 27.391\\
40.0 & 0.802 & 6.351 & 33.742\\
50.0 & 0.503 & 6.159 & 39.901\\
60.0 & 0.387 & 5.520 & 45.421\\
\bottomline
\end{tabular}
}
\end{table}

\begin{table}
\centering
\caption{动态LMW三维基准题计算结果3\label{tab:testresult.lmw.size25}}
\subcaptionbox{网格大小2.5cm 时间步长1/12s}
{
\small
\begin{tabular}{cccc}
\topline
模型 & 堆芯 & 计算 & 累计计算\\
时间/s & 功率 & 时间/s & 时间/s\\
\midline
init & 1.000 & -- & 3.978\\
10.0 & 1.341 & 34.352 & 38.330\\
20.0 & 1.707 & 36.424 & 74.754\\
30.0 & 1.365 & 35.536 & 110.290\\
40.0 & 0.805 & 35.666 & 145.956\\
50.0 & 0.501 & 34.881 & 180.837\\
60.0 & 0.384 & 33.381 & 214.218\\
\bottomline
\end{tabular}
}
\subcaptionbox{网格大小2.5cm 时间步长1/6s}
{
\small
\begin{tabular}{cccc}
\topline
模型 & 堆芯 & 计算 & 累计计算\\
时间/s & 功率 & 时间/s & 时间/s\\
\midline
init & 1.000 & -- & 4.165\\
10.0 & 1.342 & 19.022 & 23.187\\
20.0 & 1.708 & 20.409 & 43.596\\
30.0 & 1.364 & 20.045 & 63.641\\
40.0 & 0.805 & 19.791 & 83.432\\
50.0 & 0.501 & 19.469 & 102.901\\
60.0 & 0.384 & 18.248 & 121.149\\
\bottomline
\end{tabular}
}

\subcaptionbox{网格大小2.5cm 时间步长0.5s}
{
\small
\begin{tabular}{cccc}
\topline
模型 & 堆芯 & 计算 & 累计计算\\
时间/s & 功率 & 时间/s & 时间/s\\
\midline
init & 1.000 & -- & 3.932\\
10.0 & 1.346 & 6.707 & 10.639\\
20.0 & 1.714 & 6.865 & 17.504\\
30.0 & 1.364 & 6.895 & 24.399\\
40.0 & 0.804 & 6.862 & 31.261\\
50.0 & 0.501 & 6.599 & 37.860\\
60.0 & 0.385 & 5.928 & 43.788\\
\bottomline
\end{tabular}
}
\subcaptionbox{网格大小2.5cm 时间步长1s}
{
\begin{tabular}{cccc}
\topline
模型 & 堆芯 & 计算 & 累计计算\\
时间/s & 功率 & 时间/s & 时间/s\\
\midline
init & 1.000 & -- & 3.900\\
10.0 & 1.353 & 3.573 & 7.473\\
20.0 & 1.718 & 3.665 & 11.138\\
30.0 & 1.358 & 3.683 & 14.821\\
40.0 & 0.800 & 3.540 & 18.361\\
50.0 & 0.500 & 3.542 & 21.903\\
60.0 & 0.385 & 3.056 & 24.959\\
\bottomline
\end{tabular}
}
\end{table}

\begin{table}
\centering
\caption{动态LMW三维基准题计算结果4\label{tab:testresult.lmw.size5}}
\subcaptionbox{网格大小5cm 时间步长1/12s}
{
\small
\begin{tabular}{cccc}
\topline
模型 & 堆芯 & 计算 & 累计计算\\
时间/s & 功率 & 时间/s & 时间/s\\
\midline
init & 1.000 & -- & 2.933\\
10.0 & 1.337 & 9.435 & 12.368\\
20.0 & 1.686 & 10.111 & 22.479\\
30.0 & 1.342 & 9.909 & 32.388\\
40.0 & 0.795 & 10.015 & 42.403\\
50.0 & 0.501 & 9.625 & 52.028\\
60.0 & 0.385 & 8.782 & 60.810\\
\bottomline
\end{tabular}
}
\subcaptionbox{网格大小5cm 时间步长1/6s}
{
\small
\begin{tabular}{cccc}
\topline
模型 & 堆芯 & 计算 & 累计计算\\
时间/s & 功率 & 时间/s & 时间/s\\
\midline
init & 1.000 & -- & 2.979\\
10.0 & 1.339 & 4.708 & 7.687\\
20.0 & 1.685 & 5.179 & 12.866\\
30.0 & 1.339 & 5.051 & 17.917\\
40.0 & 0.793 & 4.963 & 22.880\\
50.0 & 0.501 & 4.899 & 27.779\\
60.0 & 0.385 & 4.492 & 32.271\\
\bottomline
\end{tabular}
}

\subcaptionbox{网格大小5cm 时间步长0.5s}
{
\small
\begin{tabular}{cccc}
\topline
模型 & 堆芯 & 计算 & 累计计算\\
时间/s & 功率 & 时间/s & 时间/s\\
\midline
init & 1.000 & -- & 2.949\\
10.0 & 1.344 & 1.733 & 4.682\\
20.0 & 1.686 & 1.887 & 6.569\\
30.0 & 1.331 & 1.811 & 8.380\\
40.0 & 0.789 & 1.779 & 10.159\\
50.0 & 0.500 & 1.652 & 11.811\\
60.0 & 0.385 & 1.638 & 13.449\\
\bottomline
\end{tabular}
}
\subcaptionbox{网格大小5cm 时间步长1s}
{
\begin{tabular}{cccc}
\topline
模型 & 堆芯 & 计算 & 累计计算\\
时间/s & 功率 & 时间/s & 时间/s\\
\midline
init & 1.000 & -- & 2.965\\
10.0 & 1.346 & 0.920 & 3.885\\
20.0 & 1.683 & 0.996 & 4.881\\
30.0 & 1.318 & 0.983 & 5.864\\
40.0 & 0.780 & 0.983 & 6.847\\
50.0 & 0.496 & 0.921 & 7.768\\
60.0 & 0.382 & 0.874 & 8.642\\
\bottomline
\end{tabular}
}
\end{table}




\chapter{大型扩散方程GPU求解方法研究}

本课题主要使用中子扩散方程作为实际研究问题,
中子扩散方程在实际求解中大多使用有限差分方式进行空间离散,
得到的线性方程组为7对角实对称对角占优矩阵(三维、单一能群内)。
稳态和动力学扩散方程的求解主要涉及到大型7对角线矩阵的最大特征值计算
和大型线性方程组求解。
本章则主要研究如何在GPU上高效地求解这类问题。

\section{稀疏矩阵格式选择}

为了比较各种存储格式的速度,使用三维扩散临界计算程序来进行测试,
测试算例为三维IAEA基准题(见\sectionref{sec:result.test.iaea}),
各存储格式的计算时间见\floatref{tab:equsolve.spformat}及
\floatref{fig:equsolve.spformat},图中DM表示双层网格加速(见\sectionref{sec:equsolve.multimesh}),
SP表示单精度、DP表示双精度,每种情况计算5次,时间取最小值。

结果很清楚地显示:对于三维扩散方程,DIA和ELL格式明显优于CSR和COO格式,其中DIA性能最高。
这主要是因为DIA和ELL较为适合向量机型处理器的运算和内存访问方式,
而且DIA格式占用的空间最少(因为矩阵正好是7对角线矩阵)。
在CG和BiCGStab计算中,矩阵参与的部分是稀疏矩阵-向量乘法(以下简称为SpMV),
SpMV在GPU上的主要瓶颈是显存带宽\cite{bell2008spmv,baskaran2008optimizing}\footnote{现在的GPU运算能力太强,使得显存带宽成为瓶颈。},
所以DIA性能最好是预料之中。

%\begin{sidewaystable}
%\pdfrotate
\begin{table}
\centering
\begin{minipage}{\linewidth}
\centering
\caption[不同稀疏矩阵格式求解三维临界扩散的时间表]
{\label{tab:equsolve.spformat}%
不同稀疏矩阵格式求解三维临界扩散的时间表(单位:s)%
\footnote{不同的存储格式对非零元有着不同的求和顺序,由于浮点误差存在,%
不同的格式要达到收敛标准所需要的迭代次数可能有差别,并导致总计算时间的改变。}
}
\small
\begin{tabular}{ccccccccc}
\toprule
 \multirow {3}{*}{矩阵格式}  &
       \multicolumn{4}{c}{2.5cm $\times$ 2.5cm $\times$ 2.5cm}
       &\multicolumn{4}{c}{1.25cm $\times$ 1.25cm $\times$ 1.25cm} \\
 &\multicolumn{2}{c}{CG\footnote{内迭代每轮18次,下同。}}
 	   &\multicolumn{2}{c}{BiCGStab\footnote{内迭代每轮30次,下同。}}
       & \multicolumn{2}{c}{CG}& \multicolumn{2}{c}{BiCGStab}\\
 & SP& DP& SP& DP& SP& DP& SP& DP\\
\midrule
 DIA&  5.475&  7.956& 12.199& 19.391& 24.071& 43.415&  64.256& 116.922\\
 ELL&  6.255&  8.596& 14.711& 20.997& 29.141& 47.362&  81.261& 133.599\\
 CSR& 10.452& 13.042& 27.814& 34.960& 62.478& 82.929& 180.430& 238.462\\
 COO& 11.232& 13.432& 30.934& 36.629& 64.303& 82.384& 197.263& 248.618\\
 DIA DM\footnote{粗网内迭代每轮10次,下同。}
       &  3.838&  4.883&  6.349&  8.003&  9.266& 14.976&  14.165&  31.590\\
 ELL DM&  4.399&  5.351&  7.363&  8.626& 10.920& 16.224&  17.503&  35.322\\
 CSR DM&  5.975&  6.989& 11.513& 14.337& 19.734& 26.301&  38.438&  61.776\\
 COO DM&  6.474&  7.660& 11.840& 13.712& 20.733& 26.364&  42.400&  65.910\\
\bottomrule
\end{tabular}
\end{minipage}
%\end{sidewaystable}
\end{table}

\begin{figure}
\centering
\includegraphics{equsolve-spformat}
\caption{\label{fig:equsolve.spformat}不同稀疏矩阵格式求解三维临界扩散的时间}
\end{figure}


\section{迭代算法选择}

为了比较不同迭代求解方法和预处理器的效果,这里仍然选择用IAEA基准题进行测试,
网格大小分别取5cm、2.5cm、2cm、1cm进行测试。

选择的算法包括
\begin{enumerate}
\item Jacobi-SG,逐群使用Jacobi迭代进行求解。
%\item Jacobi-MG,使用Jacobi对所有能群统一求解。
\item CG-SG,逐群使用CG迭代进行求解。
\item BiCGStab-MG,使用BiCGStab对所有能群统一求解。
\item GMRES-MG,使用GMRES对所有能群统一求解。
\end{enumerate}
以上算法中,除Jacobi-SG外都使用对角线预处理算法。
由于对于三维扩散计算,单精度浮点计算已经足够,
所以这里的测试均使用单精度进行计算。
经测试如果改用双精度对最优参数的影响很小。

由于在临界计算的源迭代中一般在每步内迭代中精确求解方程组,
往往是迭代一个较少的次数,可以达到大幅减少计算时间的目的。
这个内迭代次数和问题的规模和迭代算法、预条件算法、初值的好坏都有关系,
如何根据问题规模选择最优的内迭代次数不在文本的讨论之内,
为了公平地比较各种算法,以下将分别寻找各种情况下最优的内迭代次数。

\subsection{Jacobi-SG}
\label{sec:equsolve.iter.jacobi-sg}

Jacobi-SG算法网格大小分别取5cm、2.5cm、2cm、1cm的计算结果
见\floatref{tab:equsolve.iter.jacobi-sg.5cm}、%
\floatref{tab:equsolve.iter.jacobi-sg.2.5cm}、%
\floatref{tab:equsolve.iter.jacobi-sg.2cm}和%
\floatref{tab:equsolve.iter.jacobi-sg.1cm},
从表中可见最优的内迭代次数分别为7、11、16、18。

\begin{datasheet}
\sectionref{sec:equsolve.iter.jacobi-sg}的数据表:
\floatref{tab:equsolve.iter.jacobi-sg.5cm}、
\floatref{tab:equsolve.iter.jacobi-sg.2.5cm}、
\floatref{tab:equsolve.iter.jacobi-sg.2cm}、
\floatref{tab:equsolve.iter.jacobi-sg.1cm}
。

\begin{table}
\centering
\caption{5cm 网格时 Jacobi-SG 不同内迭代次数的计算时间及总迭代次数}
\label{tab:equsolve.iter.jacobi-sg.5cm}
\begin{tabular}{cccc}
\toprule
内迭代次数 & 计算时间/s & 总内迭代次数 & 外迭代次数\\
\midrule
%1 & 2.855 & 4138 & 2069\\
2 & 1.965 & 4180 & 1045\\
3 & 1.701 & 4266 & 711\\
4 & 1.576 & 4368 & 546\\
5 & 1.513 & 4500 & 450\\
6 & 1.497 & 4644 & 387\\
7 & 1.420 & 4802 & 343\\
8 & 1.435 & 4976 & 311\\
9 & 1.482 & 5220 & 290\\
10 & 1.607 & 5500 & 275\\
20 & 2.012 & 8440 & 211\\
30 & 2.683 & 11580 & 193\\
40 & 3.463 & 14960 & 187\\
50 & 4.181 & 18400 & 184\\
\bottomrule
\end{tabular}
\end{table}

\begin{table}
\centering
\caption{2.5cm 网格时 Jacobi-SG 不同内迭代次数的计算时间及总迭代次数}
\label{tab:equsolve.iter.jacobi-sg.2.5cm}
\begin{tabular}{cccc}
\toprule
内迭代次数 & 计算时间/s & 总内迭代次数 & 外迭代次数\\
\midrule
2 & 12.870 & 15780 & 3945\\
3 & 11.466 & 15810 & 2635\\
4 & 10.889 & 15832 & 1979\\
5 & 9.906 & 15880 & 1588\\
6 & 9.547 & 15936 & 1328\\
7 & 9.516 & 15988 & 1142\\
8 & 9.391 & 16064 & 1004\\
9 & 9.376 & 16146 & 897\\
10 & 9.297 & 16240 & 812\\
11 & 9.048 & 16324 & 742\\
12 & 9.141 & 16440 & 685\\
13 & 9.329 & 16536 & 636\\
14 & 9.453 & 16660 & 595\\
15 & 9.048 & 16770 & 559\\
16 & 9.204 & 16896 & 528\\
17 & 9.141 & 17034 & 501\\
18 & 9.641 & 17172 & 477\\
19 & 9.594 & 17328 & 456\\
20 & 9.469 & 17480 & 437\\
30 & 10.078 & 19080 & 318\\
40 & 11.373 & 21840 & 273\\
50 & 12.776 & 24700 & 247\\
\bottomrule
\end{tabular}
\end{table}

\begin{table}
\centering
\caption{2cm 网格时 Jacobi-SG 不同内迭代次数的计算时间及总迭代次数}
\label{tab:equsolve.iter.jacobi-sg.2cm}
\begin{tabular}{cccc}
\toprule
内迭代次数 & 计算时间/s & 总内迭代次数 & 外迭代次数\\
\midrule
11 & 23.415 & 25256 & 1148\\
12 & 23.072 & 25320 & 1055\\
13 & 23.073 & 25428 & 978\\
14 & 23.493 & 25480 & 910\\
15 & 23.181 & 25590 & 853\\
16 & 22.948 & 25664 & 802\\
17 & 23.119 & 25772 & 758\\
18 & 23.104 & 25884 & 719\\
19 & 23.135 & 25992 & 684\\
20 & 23.150 & 26120 & 653\\
30 & 23.868 & 27420 & 457\\
40 & 25.756 & 29040 & 363\\
50 & 26.333 & 30800 & 308\\
\bottomrule
\end{tabular}
\end{table}


\begin{table}
\centering
\caption{1cm 网格时 Jacobi-SG 不同内迭代次数的计算时间及总迭代次数}
\label{tab:equsolve.iter.jacobi-sg.1cm}
\begin{tabular}{cccc}
\toprule
内迭代次数 & 计算时间/s & 总内迭代次数 & 外迭代次数\\
\midrule
14 & 518.420 & 96768 & 3456\\
16 & 511.619 & 96832 & 3026\\
18 & 509.309 & 96804 & 2689\\
20 & 512.507 & 96920 & 2423\\
30 & 511.556 & 97260 & 1621\\
40 & 510.714 & 97840 & 1223\\
50 & 510.136 & 98600 & 986\\
60 & 513.568 & 99480 & 829\\
70 & 521.774 & 100520 & 718\\
80 & 526.875 & 101600 & 635\\
90 & 528.966 & 102780 & 571\\
100 & 534.348 & 104000 & 520\\
\bottomrule
\end{tabular}
\end{table}

\end{datasheet}


\subsection{Jacobi-MG}
\label{sec:equsolve.iter.jacobi-mg}

Jacobi-SG算法网格大小分别取5cm、2.5cm、2cm、1cm的计算结果
见\floatref{tab:equsolve.iter.jacobi-mg.5cm}、%
\floatref{tab:equsolve.iter.jacobi-mg.2.5cm}、%
\floatref{tab:equsolve.iter.jacobi-mg.2cm}和%
\floatref{tab:equsolve.iter.jacobi-mg.1cm},
从中可见最优的内迭代次数分别为4\footnote{各迭代次数时间接近,这里取总内迭代次数最小的。}、8、11、38。

\begin{datasheet}
\sectionref{sec:equsolve.iter.jacobi-mg}的数据表:
\floatref{tab:equsolve.iter.jacobi-mg.5cm}、
\floatref{tab:equsolve.iter.jacobi-mg.2.5cm}、
\floatref{tab:equsolve.iter.jacobi-mg.2cm}、
\floatref{tab:equsolve.iter.jacobi-mg.1cm}
。

\begin{table}
\centering
\caption{5cm 网格时 Jacobi-MG 不同内迭代次数的计算时间及总迭代次数}
\label{tab:equsolve.iter.jacobi-mg.5cm}
\begin{tabular}{cccc}
\toprule
内迭代次数 & 计算时间/s & 总内迭代次数 & 外迭代次数\\
\midrule
2-3 & \multicolumn{3}{c}{不收敛} \\ %Fail:PhiErr exceeds 10
4 & 1.295 & 2436 & 609\\
5 & 1.232 & 2510 & 502\\
6 & 1.217 & 2586 & 431\\
7 & 1.185 & 2660 & 380\\
8 & 1.170 & 2736 & 342\\
9 & 1.185 & 2817 & 313\\
10 & 1.185 & 2930 & 293\\
11 & 1.139 & 3069 & 279\\
12 & 1.170 & 3204 & 267\\
13 & 1.217 & 3341 & 257\\
14 & 1.217 & 3486 & 249\\
15 & 1.279 & 3615 & 241\\
16 & 1.280 & 3760 & 235\\
17 & 1.341 & 3910 & 230\\
18 & 1.373 & 4050 & 225\\
19 & 1.310 & 4180 & 220\\
20 & 1.404 & 4340 & 217\\
30 & 1.747 & 5850 & 195\\
40 & 2.137 & 7480 & 187\\
50 & 2.621 & 9200 & 184\\
\bottomrule
\end{tabular}
\end{table}

\begin{table}
\centering
\caption{2.5cm 网格时 Jacobi-MG 不同内迭代次数的计算时间及总迭代次数}
\label{tab:equsolve.iter.jacobi-mg.2.5cm}
\begin{tabular}{cccc}
\toprule
内迭代次数 & 计算时间/s & 总内迭代次数 & 外迭代次数\\
\midrule
2-5 & \multicolumn{3}{c}{不收敛} \\ %Fail:PhiErr exceeds 10
6 & 30.654 & 29070 & 4845\\
7 & 12.355 & 12166 & 1738\\
8 & 6.708 & 6512 & 814\\
9 & 8.627 & 8604 & 956\\
10 & 8.486 & 8690 & 869\\
11 & 8.439 & 8767 & 797\\
12 & 8.346 & 8820 & 735\\
13 & 8.377 & 8905 & 685\\
14 & 8.549 & 8974 & 641\\
15 & 8.658 & 9045 & 603\\
16 & 8.533 & 9120 & 570\\
17 & 8.517 & 9197 & 541\\
18 & 8.533 & 9270 & 515\\
19 & 8.736 & 9348 & 492\\
20 & 8.752 & 9420 & 471\\
30 & 9.282 & 10200 & 340\\
40 & 9.999 & 11360 & 284\\
50 & 11.232 & 12750 & 255\\
\bottomrule
\end{tabular}
\end{table}

\begin{table}
\centering
\caption{2cm 网格时 Jacobi-MG 不同内迭代次数的计算时间及总迭代次数}
\label{tab:equsolve.iter.jacobi-mg.2cm}
\begin{tabular}{cccc}
\toprule
内迭代次数 & 计算时间/s & 总内迭代次数 & 外迭代次数\\
\midrule
2-6 & \multicolumn{3}{c}{不收敛} \\ %Fail:PhiErr exceeds 10
7 & 174.721 & 98966 & 14138\\
8 & 63.461 & 36336 & 4542\\
9 & 37.596 & 21708 & 2412\\
10 & 25.397 & 14700 & 1470\\
11 & 17.238 & 10010 & 910\\
12 & 22.667 & 13272 & 1106\\
13 & 22.901 & 13468 & 1036\\
14 & 22.979 & 13566 & 969\\
15 & 23.228 & 13620 & 908\\
16 & 23.244 & 13696 & 856\\
17 & 23.010 & 13770 & 810\\
18 & 23.010 & 13824 & 768\\
19 & 23.337 & 13927 & 733\\
20 & 23.384 & 13980 & 699\\
30 & 23.993 & 14730 & 491\\
40 & 24.960 & 15520 & 388\\
50 & 26.535 & 16350 & 327\\
\bottomrule
\end{tabular}
\end{table}


\begin{table}
\centering
\caption{1cm 网格时 Jacobi-MG 不同内迭代次数的计算时间及总迭代次数}
\label{tab:equsolve.iter.jacobi-mg.1cm}
\begin{tabular}{cccc}
\toprule
内迭代次数 & 计算时间/s & 总内迭代次数 & 外迭代次数\\
\midrule
2-31(超时) & >600 & -- & -- \\ %Fail:Solve Time exceeds 600.000
32 & 560.212 & 52800 & 1650\\
33 & 512.289 & 48378 & 1466\\
34 & 467.221 & 44132 & 1298\\
35 & 427.035 & 40250 & 1150\\
36 & 391.748 & 36864 & 1024\\
37 & 360.011 & 33818 & 914\\
38 & 332.083 & 31388 & 826\\
39 & 530.947 & 50232 & 1288\\
40 & 537.717 & 50880 & 1272\\
41 & 538.803 & 51086 & 1246\\
42 & 541.852 & 51324 & 1222\\
43 & 542.958 & 51428 & 1196\\
44 & 545.269 & 51568 & 1172\\
45 & 542.681 & 51570 & 1146\\
46 & 543.505 & 51704 & 1124\\
47 & 546.406 & 51794 & 1102\\
48 & 547.090 & 51792 & 1079\\
49 & 547.062 & 51891 & 1059\\
50 & 549.698 & 51950 & 1039\\
\bottomrule
\end{tabular}
\end{table}

\end{datasheet}


\subsection{CG-SG}
\label{sec:equsolve.iter.cg-sg}

CG-SG算法网格大小分别取5cm、2.5cm、2cm、1cm的计算结果
见\floatref{fig:equsolve.iter.cg-sg},
原始数据表及各次结果的总内迭代次数、外迭代次数见
\floatref{tab:equsolve.iter.cg-sg.5cm}、%
\floatref{tab:equsolve.iter.cg-sg.2.5cm}、%
\floatref{tab:equsolve.iter.cg-sg.2cm}和%
\floatref{tab:equsolve.iter.cg-sg.1cm},
从中可见最优的内迭代次数分别为4、7、10、17。

\begin{figure}
\begin{subfigure}{.5\textwidth}
\centering
\begin{tikzpicture}
\datavisualization
  [scientific axes,
   visualize as line,
   x axis={ ticks={major={at={2,3,4,5,6,8,10,20}}},
            logarithmic, attribute=maxiter,
            label={内迭代次数} },
   y axis={ ticks={major={at={1,1.5,2,2.5}}},
            logarithmic, attribute=time,
            label={计算时间/s} }
   ]
  data {
  maxiter, time, inner, outter
  2 , 1.576 , 1960 , 490
  3 , 1.046 , 1464 , 244
  4 , 0.920 , 1472 , 184
  5 , 1.092 , 1840 , 184
  6 , 1.186 , 2196 , 183
  7 , 1.342 , 2562 , 183
  8 , 1.467 , 2912 , 182
  9 , 1.622 , 3276 , 182
  10 , 1.685 , 3640 , 182
  11 , 1.841 , 3988 , 182
  12 , 1.966 , 4305 , 182
  13 , 2.138 , 4595 , 182
  14 , 2.246 , 4869 , 182
  15 , 2.450 , 5124 , 182
  16 , 2.372 , 5362 , 182
  17 , 2.512 , 5576 , 182
  18 , 2.559 , 5746 , 182
  19 , 2.684 , 5888 , 182
  20 , 2.792 , 6009 , 182
  };
\end{tikzpicture}
\caption{5cm网格}
\end{subfigure}
\begin{subfigure}{.5\textwidth}
\centering
\begin{tikzpicture}
\datavisualization
  [scientific axes,
   visualize as line,
   x axis={ ticks={major={at={2,3,4,5,7,10,20}}},
            logarithmic, attribute=maxiter,
            label={内迭代次数} },
   y axis={ ticks={major={at={3,5,7,10}}},
            logarithmic, attribute=time,
            label={计算时间/s} }
   ]
  data {
  maxiter, time, inner, outter
  2, 9.999, 7472, 1868
  3, 5.694, 4740, 790
  4, 3.837, 3504, 438
  5, 3.120, 2910, 291
  6, 2.855, 2856, 238
  7, 2.761, 2800, 200
  8, 2.792, 2960, 185
  9, 3.042, 3312, 184
  10, 3.385, 3680, 184
  11, 3.572, 4026, 183
  12, 3.947, 4392, 183
  13, 4.181, 4758, 183
  14, 4.493, 5096, 182
  15, 4.727, 5460, 182
  16, 5.070, 5824, 182
  17, 5.460, 6188, 182
  18, 5.787, 6552, 182
  19, 5.912, 6916, 182
  20, 6.146, 7280, 182
  };
\end{tikzpicture}
\caption{2.5cm网格}
\end{subfigure}
\\[1cm]
\begin{subfigure}{.5\textwidth}
\centering
\begin{tikzpicture}
\datavisualization
  [scientific axes,
   visualize as line,
   x axis={ ticks={major={at={2,3,4,5,8,10,20}}},
            logarithmic, attribute=maxiter,
            label={内迭代次数} },
   y axis={ ticks={major={at={6,8,10,15,20}}},
            logarithmic, attribute=time,
            label={计算时间/s} }
   ]
  data {
  maxiter, time, inner, outter
  2, 23.384, 10940, 2735
  3, 13.260, 7410, 1235
  4, 9.048, 5528, 691
  5, 6.927, 4390, 439
  6, 6.068, 3864, 322
  7, 5.757, 3808, 272
  8, 5.367, 3696, 231
  9, 5.726, 3924, 218
  10, 5.476, 3700, 185
  11, 5.897, 4048, 184
  12, 6.396, 4416, 184
  13, 6.739, 4784, 184
  14, 7.067, 5124, 183
  15, 7.395, 5490, 183
  16, 7.831, 5856, 183
  17, 8.611, 6222, 183
  18, 8.939, 6588, 183
  19, 8.939, 6916, 182
  20, 9.376, 7280, 182
  };
\end{tikzpicture}
\caption{2cm网格}
\end{subfigure}
\begin{subfigure}{.5\textwidth}
\centering
\begin{tikzpicture}
\datavisualization
  [scientific axes,
   visualize as line,
   x axis={ ticks={major={at={2,3,4,5,10,17,40}}},
            logarithmic, attribute=maxiter,
            label={内迭代次数} },
   y axis={ ticks=many, ticks={major={at={50,100,200,400}}},
            logarithmic, attribute=time,
            label={计算时间/s} }
   ]
  data {
  maxiter, time, inner, outter
  2, 445.958, 43336, 10834
  3, 252.955, 28458, 4743
  4, 186.670, 22616, 2827
  5, 141.945, 18000, 1800
  6, 113.818, 14928, 1244
  7, 93.303, 12558, 897
  8, 78.983, 10784, 674
  9, 69.108, 9594, 533
  10, 60.591, 8480, 424
  11, 61.776, 8580, 390
  12, 53.102, 7536, 314
  13, 56.269, 8034, 309
  14, 50.981, 7336, 262
  15, 51.839, 7500, 250
  16, 50.357, 7200, 225
  17, 48.875, 7106, 209
  18, 52.806, 7740, 215
  19, 49.093, 7106, 187
  20, 50.216, 7400, 185
  21, 52.416, 7770, 185
  22, 55.785, 8140, 185
  23, 57.002, 8464, 184
  24, 59.521, 8832, 184
  25, 61.932, 9200, 184
  26, 64.849, 9568, 184
  27, 66.019, 9882, 183
  28, 68.266, 10248, 183
  29, 70.855, 10614, 183
  30, 73.710, 10980, 183
  31, 74.802, 11346, 183
  32, 77.703, 11712, 183
  33, 79.654, 12078, 183
  34, 83.226, 12444, 183
  35, 84.521, 12810, 183
  36, 87.953, 13176, 183
  37, 88.967, 13542, 183
  38, 92.149, 13832, 182
  39, 94.302, 14196, 182
  40, 96.642, 14560, 182
  };
\end{tikzpicture}
\caption{1cm网格}
\end{subfigure}
\caption{CG-SG 不同内迭代次数的计算时间\label{fig:equsolve.iter.cg-sg}}
\end{figure}

\begin{datasheet}
\sectionref{sec:equsolve.iter.cg-sg}的数据表:
\floatref{tab:equsolve.iter.cg-sg.5cm}、
\floatref{tab:equsolve.iter.cg-sg.2.5cm}、
\floatref{tab:equsolve.iter.cg-sg.2cm}、
\floatref{tab:equsolve.iter.cg-sg.1cm}
。

\begin{table}
\centering
\caption{5cm 网格时 CG-SG 不同内迭代次数的计算时间及总迭代次数}
\label{tab:equsolve.iter.cg-sg.5cm}
\begin{tabular}{cccc}
\toprule
内迭代次数 & 计算时间/s & 总内迭代次数 & 外迭代次数\\
\midrule
%1 & 4.540 & 4048 & 2024\\
2 & 1.576 & 1960 & 490\\
3 & 1.046 & 1464 & 244\\
4 & 0.920 & 1472 & 184\\
5 & 1.092 & 1840 & 184\\
6 & 1.186 & 2196 & 183\\
7 & 1.342 & 2562 & 183\\
8 & 1.467 & 2912 & 182\\
9 & 1.622 & 3276 & 182\\
10 & 1.685 & 3640 & 182\\
11 & 1.841 & 3988 & 182\\
12 & 1.966 & 4305 & 182\\
13 & 2.138 & 4595 & 182\\
14 & 2.246 & 4869 & 182\\
15 & 2.450 & 5124 & 182\\
16 & 2.372 & 5362 & 182\\
17 & 2.512 & 5576 & 182\\
18 & 2.559 & 5746 & 182\\
19 & 2.684 & 5888 & 182\\
20 & 2.792 & 6009 & 182\\
\bottomrule
\end{tabular}
\end{table}

\begin{table}
\centering
\caption{2.5cm 网格时 CG-SG 不同内迭代次数的计算时间及总迭代次数}
\label{tab:equsolve.iter.cg-sg.2.5cm}
\begin{tabular}{cccc}
\toprule
内迭代次数 & 计算时间/s & 总内迭代次数 & 外迭代次数\\
\midrule
%1 & 31.309 & 16256 & 8128\\
2 & 9.999 & 7472 & 1868\\
3 & 5.694 & 4740 & 790\\
4 & 3.837 & 3504 & 438\\
5 & 3.120 & 2910 & 291\\
6 & 2.855 & 2856 & 238\\
7 & 2.761 & 2800 & 200\\
8 & 2.792 & 2960 & 185\\
9 & 3.042 & 3312 & 184\\
10 & 3.385 & 3680 & 184\\
11 & 3.572 & 4026 & 183\\
12 & 3.947 & 4392 & 183\\
13 & 4.181 & 4758 & 183\\
14 & 4.493 & 5096 & 182\\
15 & 4.727 & 5460 & 182\\
16 & 5.070 & 5824 & 182\\
17 & 5.460 & 6188 & 182\\
18 & 5.787 & 6552 & 182\\
19 & 5.912 & 6916 & 182\\
20 & 6.146 & 7280 & 182\\
\bottomrule
\end{tabular}
\end{table}

\begin{table}
\centering
\caption{2cm 网格时 CG-SG 不同内迭代次数的计算时间及总迭代次数}
\label{tab:equsolve.iter.cg-sg.2cm}
\begin{tabular}{cccc}
\toprule
内迭代次数 & 计算时间/s & 总内迭代次数 & 外迭代次数\\
\midrule
%1 & 139.698 & 49450 & 24725\\
2 & 23.384 & 10940 & 2735\\
3 & 13.260 & 7410 & 1235\\
4 & 9.048 & 5528 & 691\\
5 & 6.927 & 4390 & 439\\
6 & 6.068 & 3864 & 322\\
7 & 5.757 & 3808 & 272\\
8 & 5.367 & 3696 & 231\\
9 & 5.726 & 3924 & 218\\
10 & 5.476 & 3700 & 185\\
11 & 5.897 & 4048 & 184\\
12 & 6.396 & 4416 & 184\\
13 & 6.739 & 4784 & 184\\
14 & 7.067 & 5124 & 183\\
15 & 7.395 & 5490 & 183\\
16 & 7.831 & 5856 & 183\\
17 & 8.611 & 6222 & 183\\
18 & 8.939 & 6588 & 183\\
19 & 8.939 & 6916 & 182\\
20 & 9.376 & 7280 & 182\\
\bottomrule
\end{tabular}
\end{table}


\begin{table}
\centering
\caption{1cm 网格时 CG-SG 不同内迭代次数的计算时间及总迭代次数}
\label{tab:equsolve.iter.cg-sg.1cm}
\small
\begin{tabular}{cccc}
\toprule
内迭代次数 & 计算时间/s & 总内迭代次数 & 外迭代次数\\
\midrule
2 & 445.958 & 43336 & 10834\\
3 & 252.955 & 28458 & 4743\\
4 & 186.670 & 22616 & 2827\\
5 & 141.945 & 18000 & 1800\\
6 & 113.818 & 14928 & 1244\\
7 & 93.303 & 12558 & 897\\
8 & 78.983 & 10784 & 674\\
9 & 69.108 & 9594 & 533\\
10 & 60.591 & 8480 & 424\\
11 & 61.776 & 8580 & 390\\
12 & 53.102 & 7536 & 314\\
13 & 56.269 & 8034 & 309\\
14 & 50.981 & 7336 & 262\\
15 & 51.839 & 7500 & 250\\
16 & 50.357 & 7200 & 225\\
17 & 48.875 & 7106 & 209\\
18 & 52.806 & 7740 & 215\\
19 & 49.093 & 7106 & 187\\
20 & 50.216 & 7400 & 185\\
21 & 52.416 & 7770 & 185\\
22 & 55.785 & 8140 & 185\\
23 & 57.002 & 8464 & 184\\
24 & 59.521 & 8832 & 184\\
25 & 61.932 & 9200 & 184\\
26 & 64.849 & 9568 & 184\\
27 & 66.019 & 9882 & 183\\
28 & 68.266 & 10248 & 183\\
29 & 70.855 & 10614 & 183\\
30 & 73.710 & 10980 & 183\\
31 & 74.802 & 11346 & 183\\
32 & 77.703 & 11712 & 183\\
33 & 79.654 & 12078 & 183\\
34 & 83.226 & 12444 & 183\\
35 & 84.521 & 12810 & 183\\
36 & 87.953 & 13176 & 183\\
37 & 88.967 & 13542 & 183\\
38 & 92.149 & 13832 & 182\\
39 & 94.302 & 14196 & 182\\
40 & 96.642 & 14560 & 182\\
\bottomrule
\end{tabular}
\end{table}

\end{datasheet}

\subsection{BiCGStab-MG}
\label{sec:equsolve.iter.bicgstab-mg}

BiCGStab-MG算法网格大小分别取5cm、2.5cm、2cm、1cm的计算结果
见\floatref{fig:equsolve.iter.bicgstab-mg},
原始数据表及各次结果的总内迭代次数、外迭代次数见
\floatref{tab:equsolve.iter.bicgstab-mg.5cm}、%
\floatref{tab:equsolve.iter.bicgstab-mg.2.5cm}、%
\floatref{tab:equsolve.iter.bicgstab-mg.2cm}和%
\floatref{tab:equsolve.iter.bicgstab-mg.1cm},
从中可见最优的内迭代次数分别为3、8、10、21。

\begin{figure}
\begin{subfigure}{.5\textwidth}
\centering
\begin{tikzpicture}
\datavisualization
  [scientific axes,
   visualize as line,
   x axis={ ticks={major={at={3,4,5,7,10,14,20}}},
            logarithmic, attribute=maxiter,
            label={内迭代次数} },
   y axis={ ticks=many, ticks={major={at={1,2,3,4}}},
            logarithmic, attribute=time,
            label={计算时间/s} }
   ]
  data {
  maxiter, time, inner, outter
  3, 0.998, 624, 208
  4, 1.108, 752, 188
  5, 1.264, 915, 183
  6, 1.419, 1086, 181
  7, 1.607, 1267, 181
  8, 1.763, 1448, 181
  9, 1.919, 1629, 181
  10, 2.091, 1820, 182
  11, 2.340, 2013, 183
  12, 2.434, 2160, 180
  13, 2.589, 2366, 182
  14, 2.777, 2548, 182
  15, 2.980, 2730, 182
  16, 3.214, 2912, 182
  17, 3.292, 3094, 182
  18, 3.448, 3276, 182
  19, 3.572, 3458, 182
  20, 3.869, 3640, 182
  };
\end{tikzpicture}
\caption{5cm网格}
\end{subfigure}
\begin{subfigure}{.5\textwidth}
\centering
\begin{tikzpicture}
\datavisualization
  [scientific axes,
   visualize as line,
   x axis={ ticks={major={at={7,8,10,14,20}}},
            logarithmic, attribute=maxiter,
            label={内迭代次数} },
   y axis={ ticks=many, ticks={major={at={4,5,6,8}}},
            logarithmic, attribute=time,
            label={计算时间/s} }
   ]
  data {
  maxiter, time, inner, outter
  7, 4.181, 1638, 234
  8, 4.056, 1600, 200
  9, 4.134, 1647, 183
  10, 4.446, 1800, 180
  11, 4.898, 1991, 181
  12, 5.132, 2124, 177
  13, 5.445, 2275, 175
  14, 5.834, 2450, 175
  15, 6.194, 2625, 175
  16, 6.521, 2800, 175
  17, 6.942, 2992, 176
  18, 7.332, 3186, 177
  19, 7.894, 3401, 179
  20, 8.222, 3600, 180
  };
\end{tikzpicture}
\caption{2.5cm网格}
\end{subfigure}
\\[1cm]
\begin{subfigure}{.5\textwidth}
\centering
\begin{tikzpicture}
\datavisualization
  [scientific axes,
   visualize as line,
   x axis={ ticks={major={at={9,10,12,14,17,20}}},
            logarithmic, attribute=maxiter,
            label={内迭代次数} },
   y axis={ ticks=many, ticks={major={at={9,10,12,14}}},
            logarithmic, attribute=time,
            label={计算时间/s} }
   ]
  data {
  maxiter, time, inner, outter
  9, 10.250, 2412, 268
  10, 8.954, 2120, 212
  11, 9.360, 2233, 203
  12, 9.313, 2232, 186
  13, 9.641, 2301, 177
  14, 10.452, 2534, 181
  15, 10.857, 2655, 177
  16, 11.388, 2784, 174
  17, 11.840, 2941, 173
  18, 12.527, 3114, 173
  19, 13.292, 3287, 173
  20, 13.900, 3460, 173
  };
\end{tikzpicture}
\caption{2cm网格}
\end{subfigure}
\begin{subfigure}{.5\textwidth}
\centering
\begin{tikzpicture}
\datavisualization
  [scientific axes,
   visualize as line,
   x axis={ ticks={major={at={19,21,24,27,30}}},
            logarithmic, attribute=maxiter,
            label={内迭代次数} },
   y axis={ ticks=many, ticks={major={at={100,150,200,250}}},
            logarithmic, attribute=time,
            label={计算时间/s} }
   ]
  data {
  maxiter, time, inner, outter
  19, 258.804, 11134, 586
  20, 154.456, 6580, 329
  21, 100.339, 4305, 205
  22, 105.627, 4554, 207
  23, 106.673, 4554, 198
  24, 107.936, 4656, 194
  25, 109.965, 4725, 189
  26, 115.472, 4992, 192
  27, 112.991, 4887, 181
  28, 120.354, 5180, 185
  29, 124.020, 5336, 184
  30, 129.511, 5580, 186
  };
\end{tikzpicture}
\caption{1cm网格}
\end{subfigure}
\caption{BiCGStab-MG 不同内迭代次数的计算时间\label{fig:equsolve.iter.bicgstab-mg}}
\end{figure}

\begin{datasheet}
\sectionref{sec:equsolve.iter.bicgstab-mg}的数据表:
\floatref{tab:equsolve.iter.bicgstab-mg.5cm}、
\floatref{tab:equsolve.iter.bicgstab-mg.2.5cm}、
\floatref{tab:equsolve.iter.bicgstab-mg.2cm}、
\floatref{tab:equsolve.iter.bicgstab-mg.1cm}
。

\begin{table}
\centering
\caption{5cm 网格时 BiCGStab-MG 不同内迭代次数的计算时间及总迭代次数}
\label{tab:equsolve.iter.bicgstab-mg.5cm}
\begin{tabular}{cccc}
\toprule
内迭代次数 & 计算时间/s & 总内迭代次数 & 外迭代次数\\
\midrule
%1 & \multicolumn{3}{c}{不收敛} \\ %\footnote{Fail:Nan: KeffErr, PhiErr, }
2 & \multicolumn{3}{c}{不收敛} \\ %\footnote{Fail:Nan: KeffErr, PhiErr, }
3 & 0.998 & 624 & 208\\
4 & 1.108 & 752 & 188\\
5 & 1.264 & 915 & 183\\
6 & 1.419 & 1086 & 181\\
7 & 1.607 & 1267 & 181\\
8 & 1.763 & 1448 & 181\\
9 & 1.919 & 1629 & 181\\
10 & 2.091 & 1820 & 182\\
11 & 2.340 & 2013 & 183\\
12 & 2.434 & 2160 & 180\\
13 & 2.589 & 2366 & 182\\
14 & 2.777 & 2548 & 182\\
15 & 2.980 & 2730 & 182\\
16 & 3.214 & 2912 & 182\\
17 & 3.292 & 3094 & 182\\
18 & 3.448 & 3276 & 182\\
19 & 3.572 & 3458 & 182\\
20 & 3.869 & 3640 & 182\\
\bottomrule
\end{tabular}
\end{table}

\begin{table}
\centering
\caption{2.5cm 网格时 BiCGStab-MG 不同内迭代次数的计算时间及总迭代次数}
\label{tab:equsolve.iter.bicgstab-mg.2.5cm}
\begin{tabular}{cccc}
\toprule
内迭代次数 & 计算时间/s & 总内迭代次数 & 外迭代次数\\
\midrule
%1 & 3.900 & 900 & 900\\
2-5 & \multicolumn{3}{c}{不收敛} \\ %\footnote{Fail:Nan: KeffErr, PhiErr, }
6(超时) & >600 & >241926 & >40321 \\ %\footnote{Fail:Solve Time exceeds 600.000}
7 & 4.181 & 1638 & 234\\
8 & 4.056 & 1600 & 200\\
9 & 4.134 & 1647 & 183\\
10 & 4.446 & 1800 & 180\\
11 & 4.898 & 1991 & 181\\
12 & 5.132 & 2124 & 177\\
13 & 5.445 & 2275 & 175\\
14 & 5.834 & 2450 & 175\\
15 & 6.194 & 2625 & 175\\
16 & 6.521 & 2800 & 175\\
17 & 6.942 & 2992 & 176\\
18 & 7.332 & 3186 & 177\\
19 & 7.894 & 3401 & 179\\
20 & 8.222 & 3600 & 180\\
\bottomrule
\end{tabular}
\end{table}

\begin{table}
\centering
\caption{2cm 网格时 BiCGStab-MG 不同内迭代次数的计算时间及总迭代次数}
\label{tab:equsolve.iter.bicgstab-mg.2cm}
\begin{tabular}{cccc}
\toprule
内迭代次数 & 计算时间/s & 总内迭代次数 & 外迭代次数\\
\midrule
%1 & 12.152 & 1766 & 1766\\
2-7 & \multicolumn{3}{c}{不收敛} \\ %\footnote{Fail:Nan: KeffErr, PhiErr, }
8(超时) & >600 & >142232 & >17779 \\ %\footnote{Fail:Solve Time exceeds 600.000}
9 & 10.250 & 2412 & 268\\
10 & 8.954 & 2120 & 212\\
11 & 9.360 & 2233 & 203\\
12 & 9.313 & 2232 & 186\\
13 & 9.641 & 2301 & 177\\
14 & 10.452 & 2534 & 181\\
15 & 10.857 & 2655 & 177\\
16 & 11.388 & 2784 & 174\\
17 & 11.840 & 2941 & 173\\
18 & 12.527 & 3114 & 173\\
19 & 13.292 & 3287 & 173\\
20 & 13.900 & 3460 & 173\\
\bottomrule
\end{tabular}
\end{table}


\begin{table}
\centering
\caption{1cm 网格时 BiCGStab-MG 不同内迭代次数的计算时间及总迭代次数}
\label{tab:equsolve.iter.bicgstab-mg.1cm}
\begin{tabular}{cccc}
\toprule
内迭代次数 & 计算时间/s & 总内迭代次数 & 外迭代次数\\
\midrule
%1 & 600.929 & 16239 & 16239 \\ %Fail:Solve Time exceeds 600.000
2-17 & \multicolumn{3}{c}{不收敛} \\ %Fail:Nan: KeffErr, PhiErr,
18(超时) & >600 & >25650 & >1425 \\ %Fail:Solve Time exceeds 600.000
19 & 258.804 & 11134 & 586\\
20 & 154.456 & 6580 & 329\\
21 & 100.339 & 4305 & 205\\
22 & 105.627 & 4554 & 207\\
23 & 106.673 & 4554 & 198\\
24 & 107.936 & 4656 & 194\\
25 & 109.965 & 4725 & 189\\
26 & 115.472 & 4992 & 192\\
27 & 112.991 & 4887 & 181\\
28 & 120.354 & 5180 & 185\\
29 & 124.020 & 5336 & 184\\
30 & 129.511 & 5580 & 186\\
\bottomrule
\end{tabular}
\end{table}

\end{datasheet}

\subsection{GMRES-MG}
GMRES在实际计算中往往需要Restart过程,
相对于前面的迭代算法增加了每隔多少次迭代进行Restart的参数(以下简记为Restart),
这里Restart分别取5、10、15进行计算,
计算结果见\floatref{tab:equsolve.iter.gmres}及
\floatref{fig:fig:equsolve.iter.gmres}。

\begin{figure}
\centering
\subcaptionbox{5cm网格}
{\includegraphics[scale=0.8]{equsolve-iter-gmres-5cm}}
\subcaptionbox{2.5cm网格}
{\includegraphics[scale=0.8]{equsolve-iter-gmres-2-5cm}}
\\[1cm]
\subcaptionbox{2cm网格}
{\includegraphics[scale=0.8]{equsolve-iter-gmres-2cm}}
\subcaptionbox{1cm网格}
{\includegraphics[scale=0.8]{equsolve-iter-gmres-1cm}}
\caption{\label{fig:fig:equsolve.iter.gmres}GMRES算法计算时间}
\end{figure}

\begin{table}
\centering
\caption{\label{tab:equsolve.iter.gmres}GMRES算法在不同参数下的最优内迭代次数及结果}
\begin{tabular}{cccccc}
\toprule
网格大小 & Restart & 最优内迭代次数 & 计算时间/s & 总内迭代次数 & 外迭代次数\\
\midrule
\multirow{3}{*}{5cm}
 & 5 & 6 & 1.076 & 1218 & 203\\
 & 10 & 6 & 1.061 & 1212 & 202\\
 & 15 & 6 & 1.123 & 1212 & 202\\
\multirow{3}{*}{2.5cm}
 & 5 & 10 & 4.852 & 2580 & 258\\
 & 10 & 11 & 5.647 & 2420 & 220\\
 & 15 & 11 & 5.663 & 2244 & 204\\
\multirow{3}{*}{2cm}
 & 5 & 14 & 12.324 & 4004 & 286\\
 & 10 & 14 & 12.277 & 3388 & 242\\
 & 15 & 14 & 13.416 & 2926 & 209\\
\multirow{3}{*}{1cm}
 & 5 & 28 & 224.079 & 12712 & 454\\
 & 10 & 27 & 171.382 & 7992 & 296\\
 & 15 & 27 & 161.804 & 6264 & 232\\
\bottomrule
\end{tabular}
\end{table}

%原始数据
\begin{comment}
1cm
restart 5
1-4 & \multicolumn{3}{c}{不收敛} \\ %Fail:PhiErr exceeds 10
5 & 1.232 & 1390 & 278\\
6 & 1.076 & 1218 & 203\\
7 & 1.201 & 1372 & 196\\
8 & 1.263 & 1512 & 189\\
9 & 1.357 & 1665 & 185\\
10 & 1.373 & 1810 & 181\\
11 & 1.513 & 1991 & 181\\
12 & 1.685 & 2196 & 183\\
13 & 1.872 & 2392 & 184\\
14 & 1.997 & 2590 & 185\\
15 & 2.059 & 2760 & 184\\
16 & 2.246 & 2944 & 184\\
17 & 2.325 & 3111 & 183\\
18 & 2.262 & 3276 & 182\\
19 & 2.574 & 3458 & 182\\
20 & 2.527 & 3620 & 181\\

1-4 & \multicolumn{3}{c}{不收敛} \\ %Fail:PhiErr exceeds 10
5 & 1.248 & 1390 & 278\\
6 & 1.061 & 1212 & 202\\
7 & 1.201 & 1309 & 187\\
8 & 1.404 & 1472 & 184\\
9 & 1.576 & 1647 & 183\\
10 & 1.841 & 1820 & 182\\
11 & 1.965 & 2002 & 182\\
12 & 2.028 & 2196 & 183\\
13 & 2.138 & 2379 & 183\\
14 & 2.215 & 2562 & 183\\
15 & 2.309 & 2745 & 183\\
16 & 2.496 & 2912 & 182\\
17 & 2.667 & 3094 & 182\\
18 & 2.840 & 3276 & 182\\
19 & 3.058 & 3458 & 182\\
20 & 3.338 & 3640 & 182\\

1-4 & \multicolumn{3}{c}{不收敛} \\ %Fail:PhiErr exceeds 10
5 & 1.280 & 1390 & 278\\
6 & 1.123 & 1212 & 202\\
7 & 1.202 & 1309 & 187\\
8 & 1.342 & 1472 & 184\\
9 & 1.544 & 1647 & 183\\
10 & 1.809 & 1820 & 182\\
11 & 2.060 & 2002 & 182\\
12 & 2.340 & 2184 & 182\\
13 & 2.606 & 2366 & 182\\
14 & 2.902 & 2548 & 182\\
15 & 3.292 & 2730 & 182\\
16 & 3.401 & 2912 & 182\\
17 & 3.432 & 3094 & 182\\
18 & 3.604 & 3276 & 182\\
19 & 3.666 & 3458 & 182\\
20 & 3.822 & 3640 & 182\\




res_gmres_size2.5_*restart5
1-8 & \multicolumn{3}{c}{不收敛} \\ %Fail:PhiErr exceeds 10
9 & 7.113 & 3942 & 438\\
10 & 4.852 & 2580 & 258\\
11 & 4.929 & 2618 & 238\\
12 & 5.070 & 2784 & 232\\
13 & 5.257 & 2873 & 221\\
14 & 5.616 & 3038 & 217\\
15 & 6.053 & 3300 & 220\\
16 & 6.115 & 3296 & 206\\
17 & 6.193 & 3502 & 206\\
18 & 6.505 & 3708 & 206\\
19 & 6.911 & 3933 & 207\\
20 & 7.238 & 4080 & 204\\

res_gmres_size2.5_*restart10
1-9 & \multicolumn{3}{c}{不收敛} \\ %Fail:PhiErr exceeds 10
10 & 7.098 & 3020 & 302\\
11 & 5.647 & 2420 & 220\\
12 & 5.710 & 2520 & 210\\
13 & 5.959 & 2652 & 204\\
14 & 6.225 & 2772 & 198\\
15 & 6.412 & 2895 & 193\\
16 & 6.645 & 3040 & 190\\
17 & 6.973 & 3179 & 187\\
18 & 7.551 & 3312 & 184\\
19 & 8.065 & 3458 & 182\\
20 & 8.502 & 3620 & 181\\

res_gmres_size2.5_*restart15
1-9 & \multicolumn{3}{c}{不收敛} \\ %Fail:PhiErr exceeds 10
10 & 7.160 & 3020 & 302\\
11 & 5.663 & 2244 & 204\\
12 & 6.287 & 2412 & 201\\
13 & 6.895 & 2509 & 193\\
14 & 7.675 & 2618 & 187\\
15 & 8.377 & 2775 & 185\\
16 & 8.455 & 2960 & 185\\
17 & 8.705 & 3145 & 185\\
18 & 8.970 & 3312 & 184\\
19 & 9.547 & 3496 & 184\\
20 & 9.937 & 3680 & 184\\



res_gmres_size2_*restart5
1-11 & \multicolumn{3}{c}{不收敛} \\ %Fail:PhiErr exceeds 10
12 & 16.598 & 5400 & 450\\
13 & 13.260 & 4264 & 328\\
14 & 12.324 & 4004 & 286\\
15 & 13.822 & 4545 & 303\\
16 & 12.277 & 4016 & 251\\
17 & 12.231 & 3944 & 232\\
18 & 12.620 & 4176 & 232\\
19 & 13.806 & 4465 & 235\\
20 & 13.884 & 4640 & 232\\

res_gmres_size2_*restart10
1-12 & \multicolumn{3}{c}{不收敛} \\ %Fail:PhiErr exceeds 10
13 & 18.595 & 5096 & 392\\
14 & 12.277 & 3388 & 242\\
15 & 11.638 & 3210 & 214\\
16 & 12.558 & 3328 & 208\\
17 & 12.480 & 3434 & 202\\
18 & 12.995 & 3546 & 197\\
19 & 13.650 & 3629 & 191\\
20 & 14.274 & 3760 & 188\\

res_gmres_size2_*restart15
1-12 & \multicolumn{3}{c}{不收敛} \\ %Fail:PhiErr exceeds 10
13 & 13.884 & 3146 & 242\\
14 & 13.416 & 2926 & 209\\
15 & 14.337 & 3015 & 201\\
16 & 15.210 & 3152 & 197\\
17 & 15.116 & 3332 & 196\\
18 & 15.241 & 3474 & 193\\
19 & 16.442 & 3629 & 191\\
20 & 16.240 & 3780 & 189\\


res_gmres_size1_*restart5
1 & 600.882 & 21444 & 21444 \\ %Fail:Solve Time exceeds 600.000
2 & 600.928 & 28732 & 14366 \\ %Fail:Solve Time exceeds 600.000
3 & 600.929 & 31359 & 10453 \\ %Fail:Solve Time exceeds 600.000
4 & 601.006 & 32168 & 8042 \\ %Fail:Solve Time exceeds 600.000
5-19 & \multicolumn{3}{c}{不收敛} \\ %Fail:PhiErr exceeds 10
20 & 600.913 & 33540 & 1677 \\ %Fail:Solve Time exceeds 600.000
21 & \multicolumn{3}{c}{不收敛} \\ %Fail:PhiErr exceeds 10
22 & 601.225 & 33616 & 1528 \\ %Fail:Solve Time exceeds 600.000
23 & 600.960 & 33810 & 1470 \\ %Fail:Solve Time exceeds 600.000
24 & 601.288 & 34104 & 1421 \\ %Fail:Solve Time exceeds 600.000
25 & 600.975 & 33950 & 1358 \\ %Fail:Solve Time exceeds 600.000
26 & 600.913 & 33410 & 1285 \\ %Fail:Solve Time exceeds 600.000
27 & 501.416 & 28377 & 1051\\
28 & 224.079 & 12712 & 454\\
29 & 231.770 & 12992 & 448\\
30 & 227.308 & 12840 & 428\\
31 & 246.465 & 13702 & 442\\
32 & 234.984 & 13184 & 412\\
33 & 252.986 & 14256 & 432\\
34 & 281.019 & 15844 & 466\\
35 & 302.750 & 17150 & 490\\
36 & 279.739 & 15624 & 434\\
37 & 248.134 & 13949 & 377\\
38 & 229.804 & 12958 & 341\\
39 & 242.065 & 13650 & 350\\
40 & 253.376 & 14240 & 356\\

res_gmres_size1_*restart10
1 & 600.866 & 20165 & 20165 \\ %Fail:Solve Time exceeds 600.000
2 & 600.991 & 27676 & 13838 \\ %Fail:Solve Time exceeds 600.000
3 & 600.944 & 30489 & 10163 \\ %Fail:Solve Time exceeds 600.000
4 & 600.975 & 31296 & 7824 \\ %Fail:Solve Time exceeds 600.000
5-23 & \multicolumn{3}{c}{不收敛} \\ %Fail:PhiErr exceeds 10
24 & 601.288 & 28248 & 1177 \\ %Fail:Solve Time exceeds 600.000
25 & 366.850 & 17275 & 691\\
26 & 398.144 & 18694 & 719\\
27 & 171.382 & 7992 & 296\\
28 & 174.986 & 8176 & 292\\
29 & 185.375 & 8497 & 293\\
30 & 200.242 & 9120 & 304\\
31 & 184.314 & 8432 & 272\\
32 & 180.430 & 8352 & 261\\
33 & 173.051 & 8085 & 245\\
34 & 171.553 & 8058 & 237\\
35 & 175.142 & 8225 & 235\\
36 & 180.586 & 8496 & 236\\
37 & 185.593 & 8695 & 235\\
38 & 191.319 & 8930 & 235\\
39 & 199.524 & 9243 & 237\\
40 & 207.137 & 9480 & 237\\

res_gmres_size1_*restart15
1 & 600.929 & 19030 & 19030 \\ %Fail:Solve Time exceeds 600.000
2 & 600.960 & 26496 & 13248 \\ %Fail:Solve Time exceeds 600.000
3 & 600.944 & 29478 & 9826 \\ %Fail:Solve Time exceeds 600.000
4 & 600.944 & 30732 & 7683 \\ %Fail:Solve Time exceeds 600.000
5-22 & \multicolumn{3}{c}{不收敛} \\ %Fail:PhiErr exceeds 10
23 & 601.194 & 24242 & 1054 \\ %Fail:Solve Time exceeds 600.000
24 & 601.350 & 24192 & 1008 \\ %Fail:Solve Time exceeds 600.000
25 & 169.182 & 6700 & 268\\
26 & 160.555 & 6292 & 242\\
27 & 161.804 & 6264 & 232\\
28 & 166.328 & 6412 & 229\\
29 & 175.844 & 6612 & 228\\
30 & 185.750 & 6930 & 231\\
31 & 184.314 & 6944 & 224\\
32 & 185.750 & 7104 & 222\\
33 & 187.013 & 7161 & 217\\
34 & 185.983 & 7242 & 213\\
35 & 192.192 & 7385 & 211\\
36 & 192.255 & 7524 & 209\\
37 & 192.816 & 7622 & 206\\
38 & 197.792 & 7752 & 204\\
39 & 198.604 & 7839 & 201\\
40 & 204.251 & 8000 & 200\\
\end{comment}

\subsection{不同迭代算法对比}

以上各种迭代算法的最优参数及计算结果见\floatref{tab:equsolve.iter.compare},
可见对于三维扩散临界问题,CG-SG算法最为适合。

\begin{table}
\centering
\caption{\label{tab:equsolve.iter.compare} 不同迭代算法对比}
\begin{tabular}{cccccc}
\toprule
\begin{tabular}{c}网格\\ 大小 \end{tabular}
  & 迭代算法 & 计算时间/s
  & \begin{tabular}{c}总内迭代\\ 次数 \end{tabular}
   & \begin{tabular}{c}外迭代\\ 次数 \end{tabular}
   & 参数\\
\midrule
\multirow{4}{*}{5cm}
  & Jacobi-SG   & 1.420 & 4802 & 343 & 内迭代7次\\
  & CG-SG       & 0.920 & 1472 & 184 & 内迭代4次\\
  & BiCGStab-MG & 0.998 & 624 & 208 & 内迭代3次\\
  & GMRES-MG    & 1.061 & 1212 & 202 & 
    \begin{tabular}{c}内迭代6次,\\ Restart10 \end{tabular}\\
\multirow{4}{*}{2.5cm}
  & Jacobi-SG   & 9.048 & 16324 & 742 & 内迭代11次\\
  & CG-SG       & 2.761 & 2800 & 200 & 内迭代7次\\
  & BiCGStab-MG & 4.056 & 1600 & 200 & 内迭代8次\\
  & GMRES-MG    & 4.852 & 2580 & 258 & 
    \begin{tabular}{c}内迭代10次,\\ Restart5 \end{tabular}\\
\multirow{4}{*}{2cm}
  & Jacobi-SG   & 22.948 & 25664 & 802 & 内迭代16次\\
  & CG-SG       & 5.476 & 3700 & 185 & 内迭代10次\\
  & BiCGStab-MG & 8.954 & 2120 & 212 & 内迭代10次\\
  & GMRES-MG    & 12.277 & 3388 & 242 & 
    \begin{tabular}{c}内迭代14次,\\ Restart10 \end{tabular}\\
\multirow{4}{*}{1cm}
  & Jacobi-SG   & 509.309 & 96804 & 2689 & 内迭代18次\\
  & CG-SG       & 48.875 & 7106 & 209 & 内迭代17次\\
  & BiCGStab-MG & 100.339 & 4305 & 205 & 内迭代21次\\
  & GMRES-MG    & 161.804 & 6264 & 232 & 
    \begin{tabular}{c}内迭代27次,\\ Restart15 \end{tabular}\\
\bottomrule
\end{tabular}

\end{table}

\section{双层网格加速}
\label{sec:equsolve.multimesh}

由于初值对于迭代算法的运行时间影响较大,
所以可以通过改善初值来实现总计算时间的缩减。
对于反应堆类问题,堆中的材料分布相对较简单,
尤其是在细网离散时往往会出现大片的网格材料相同的情况。
可以首先对问题进行粗网离散,可以用较低的开销进行求解,
获得一个较粗略的结果后,可以变换为一个较好的细网计算初值,
达到减少总计算时间的目的。

本课题采用如下方式计算通量初值:
产生原网格xyz方向网格数量均减半的空间网格划分,
使用如前所用的CG-SG方法进行求解,该阶段用户可以通过输入文件自定义
每轮内迭代次数、外迭代收敛标准等控制变量,
在粗网上求解后把粗网上的通量插值为细网通量,
细网上的初始$K_\mathrm{eff}$取为粗网$K_\mathrm{eff}$即可%
\footnote{同一个扩散问题采用不同网格大小进行离散后得到的$K_\mathrm{eff}$并不完全一致,
略有差异。}。

该方法的好处是实际使用范围较广,实现较简单,而效果显著,




\chapter{结论与展望}

\Closesolutionfile{datasheetfile}

%%% 其它部分
\backmatter



% 参考文献
\bibliographystyle{thubib}
\bibliography{ref/refs}


% 致谢
%

\begin{ack}
  衷心感谢导师余纲林老师和核能所所长王侃教授对本人的指导。
  余纲林老师在学术上和生活中的悉心指导让我终生受益,
  王侃教授的严格要求使我受益匪浅,
  在此我衷心地感谢他们。
  如果没有余纲林老师和王侃教授,我肯定取得不了这样的成绩。
  
  此外还要感谢工物系核能所的张鹏、李林森、李泽光、佘顶、
  徐琪、余健开等众位师兄和同学对我的帮助,
  在本课题的研究过程中,他们都或多或少的参与了讨论,
  给了我不少启发和帮助,在这里我也向他们表示感谢。
  
  最后,还要感谢 \thuthesis,它让我的论文写作过程轻松了很多。

\end{ack}


% 附录
\begin{appendix}

\chapter{原始数据表}

\input{datasheetfile}

\iffalse

程序性能结果应使用几何平均数。\cite{fleming1986not}

\chapter{课题程序使用说明}
\section{运行环境}
\section{输入文件格式}
\section{输出文件格式}
\subsection{HDF5简介}



\TODO 使用简介

\subsection{Python简介}
\subsection{后处理脚本示例}
\section{编译说明}

\fi

%%%% Local Variables: 
%%% mode: latex
%%% TeX-master: "../main"
%%% End: 

\chapter{程序辅助功能}

\section{程序外部依赖库简介}

\subsection{Thrust}

Thrust是一个并行算法库,提供了Map、Reduce等基础函数,简化了CUDA程序开发。
Thrust由NVidia公司的雇员Jared Hoberock、Nathan Bell开发并维护,
被CUDA开发环境CUDA Toolkit所包含,并在Github上基于Apache开源协议开源,
当前版本为Thrust 1.7.0,
项目主页\url{http://thrust.github.com}。

\subsection{CUSP}

CUSP是CUDA环境下的一个稀疏矩阵函数库,依赖于Thrust,
由NVidia公司的雇员Steven Dalton、Nathan Bell开发。
CUSP基于Apache开源协议开源,当前版本为CUSP 0.3.1,
项目地址\url{http://cusplibrary.github.com}。

\subsection{HDF5}

HDF5是一种分层的、带有元数据的、支持多种数据格式的、针对科学计算的数据文件存储格式及相关技术。
它最早由美国国家超级计算应用中心(NCSA, 
National Center for Supercomputing Applications)提出,
现在由非营利的HDF Group管理和维护。\cite{HDF5Wiki}

HDF5文件格式的读写库基于一种类似于BSD的协议开源,协议文本可以从
\url{http://www.hdfgroup.org/HDF5/doc/Copyright.html} 获取。

当前版本为HDF5 1.8.10 patch1,
可从\url{http://www.hdfgroup.org/downloads/index.html} 获取。

\subsection{Lua、LuaJIT }

Lua是一种快速、轻量级的嵌入式脚本语言,
在1993年由巴西里约热内卢天主教大学(Pontifical Catholic University of Rio de Janeiro)
计算机图形技术组(Computer Graphics Technology Group)的成员
Roberto Ierusalimschy、 Luiz Henrique de Figueiredo和 Waldemar Celes所开发。
\cite{LuaWiki}

Lua很容易嵌入到其他程序中作为内嵌的解释器使用,已在PC游戏界广泛使用。
当前版本5.2.1,基于MIT开源协议开源,
可从 \url{http://www.lua.org/download.html}获得。

LuaJIT是Lua语言的非官方JIT(Just-In-Time Compiler)实现,由Mike Pall开发。
LuaJIT基于MIT开源协议开源,当前版本LuaJIT 2.0.1,可以从\url{luajit.org/download.html}获取。
LuaJIT是当前世界上最快的动态语言实现之一。\cite{LuaJITHomepage}


\end{appendix}

% 个人简历
%\begin{resume}

  \resumeitem{个人简历}

  1988 年 5 月 11 日出生于黑龙江省省齐齐哈尔市。
  
  2007 年 9 月考入清华大学工程物理系工程物理专业,2011年 7 月本科毕业并获得工学学士学位。
  
  2011 年 9 月免试进入清华大学工程物理系攻读核科学与技术硕士学位至今。

  \resumeitem{发表的学术论文} % 发表的和录用的合在一起

  \begin{enumerate}[{[}1{]}]
  \item 孙嘉龙, 余纲林, 佘顶等. 堆用蒙特卡罗程序几何重复结构功能开发[J]. 强激光与粒子束, 2013,25(01): 219-222.
  (EI源刊,检索号20130315915132)
  \end{enumerate}

\end{resume}

\end{document}
