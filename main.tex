%%% Local Variables:
%%% mode: latex
%%% TeX-master: t
%%% End:

\documentclass[master,xetex]{thuthesis}
% \documentclass[%
%   bachelor|master|doctor|postdoctor, % mandatory option
%   xetex|pdftex|dvips|dvipdfm, % optional
%   secret,
%   openany|openright,
%   arialtoc,arialtitle]{thuthesis}

% 所有其它可能用到的包都统一放到这里了,可以根据自己的实际添加或者删除。
\usepackage[
addfootnotetoref
]{thutils}

\usepackage[xetex,hyperref]{xcolor}

\newcommand{\TODO}{ \textcolor{blue}{TODO} }
\colorlet{BLUE}{blue}

%--------------------------------------------------------------------------
% 自定义函数

%bracket系列
\newcommand{\bracket}[4]
{\ensuremath{%
\ifthenelse{\equal{#1}{n}}{#3 #2 #4}{}%
\ifthenelse{\equal{#1}{b}}{\bigl#3 #2 \bigr#4}{}%
\ifthenelse{\equal{#1}{B}}{\Bigl#3 #2 \Bigr#4}{}%
\ifthenelse{\equal{#1}{bg}}{\biggl#3 #2 \biggr#4}{}%
\ifthenelse{\equal{#1}{Bg}}{\Biggl#3 #2 \Biggr#4}{}%
}}

\newcommand{\pbracket}[2]{\bracket{#1}{#2}{(}{)}}
\newcommand{\Sbracket}[2]{\bracket{#1}{#2}{[}{]}}
\newcommand{\Bbracket}[2]{\bracket{#1}{#2}{\lbrace}{\rbrace}}
\newcommand{\vbracket}[2]{\bracket{#1}{#2}{|}{|}}

\newcommand{\getsize}[2]
{%
\ifthenelse{\equal{#1}{n}}{#2}{}%
\ifthenelse{\equal{#1}{b}}{\big#2}{}%
\ifthenelse{\equal{#1}{B}}{\Big#2}{}%
\ifthenelse{\equal{#1}{bg}}{\bigg#2}{}%
\ifthenelse{\equal{#1}{Bg}}{\Bigg#2}{}%
}

\newcommand{\pb}[2][n]{\pbracket{#1}{#2}}
\newcommand{\Sb}[2][n]{\Sbracket{#1}{#2}}
\newcommand{\Bb}[2][n]{\Bbracket{#1}{#2}}
\newcommand{\vb}[2][n]{\vbracket{#1}{#2}}

\newcommand{\diff}[1]{\mathrm{d}#1}

\usepackage{datetime}
\newdateformat{mydate}{\THEYEAR-\twodigit{\THEMONTH}-\twodigit{\THEDAY}}
\newtimeformat{mytime}{\twodigit{\THEHOUR}:\twodigit{\THEMINUTE}}
\settimeformat{mytime}

\usepackage[draft=true,allpages=true]{draftmark}
\draftmarksetup{angle=45,grayness=0.9,
mark={DRAFT \\ \huge 编译时间:\mydate\today\hspace{5pt} \currenttime}}

\begin{document}

% 定义所有的eps文件在 figures 子目录下
\graphicspath{{figures/}}


%%% 封面部分
\frontmatter

%%% Local Variables:
%%% mode: latex
%%% TeX-master: t
%%% End:
\secretlevel{绝密} \secretyear{2100}

\ctitle{清华大学学位论文 \LaTeX\ 模板\\使用示例文档}
% 根据自己的情况选,不用这样复杂
\makeatletter
\ifthu@bachelor\relax\else
  \ifthu@doctor
    \cdegree{工学博士}
  \else
    \ifthu@master
      \cdegree{工学硕士}
    \fi
  \fi
\fi
\makeatother


\cdepartment[计算机]{计算机科学与技术系}
\cmajor{计算机科学与技术}
\cauthor{薛瑞尼} 
\csupervisor{郑纬民教授}
% 如果没有副指导老师或者联合指导老师,把下面两行相应的删除即可。
\cassosupervisor{陈文光教授}
\ccosupervisor{某某某教授}
% 日期自动生成,如果你要自己写就改这个cdate
%\cdate{\CJKdigits{\the\year}年\CJKnumber{\the\month}月}

% 博士后部分
% \cfirstdiscipline{计算机科学与技术}
% \cseconddiscipline{系统结构}
% \postdoctordate{2009年7月——2011年7月}

\etitle{An Introduction to \LaTeX{} Thesis Template of Tsinghua University} 
% 这块比较复杂,需要分情况讨论:
% 1. 学术型硕士
%    \edegree:必须为Master of Arts或Master of Science(注意大小写)
%              “哲学、文学、历史学、法学、教育学、艺术学门类,公共管理学科
%               填写Master of Arts,其它填写Master of Science”
%    \emajor:“获得一级学科授权的学科填写一级学科名称,其它填写二级学科名称”
% 2. 专业型硕士
%    \edegree:“填写专业学位英文名称全称”
%    \emajor:“工程硕士填写工程领域,其它专业学位不填写此项”
% 3. 学术型博士
%    \edegree:Doctor of Philosophy(注意大小写)
%    \emajor:“获得一级学科授权的学科填写一级学科名称,其它填写二级学科名称”
% 4. 专业型博士
%    \edegree:“填写专业学位英文名称全称”
%    \emajor:不填写此项
\edegree{Doctor of Engineering} 
\emajor{Computer Science and Technology} 
\eauthor{Xue Ruini} 
\esupervisor{Professor Zheng Weimin} 
\eassosupervisor{Chen Wenguang} 
% 这个日期也会自动生成,你要改么?
% \edate{December, 2005}

% 定义中英文摘要和关键字
\begin{cabstract}
  论文的摘要是对论文研究内容和成果的高度概括。摘要应对论文所研究的问题及其研究目
  的进行描述,对研究方法和过程进行简单介绍,对研究成果和所得结论进行概括。摘要应
  具有独立性和自明性,其内容应包含与论文全文同等量的主要信息。使读者即使不阅读全
  文,通过摘要就能了解论文的总体内容和主要成果。

  论文摘要的书写应力求精确、简明。切忌写成对论文书写内容进行提要的形式,尤其要避
  免“第 1 章……;第 2 章……;……”这种或类似的陈述方式。

  本文介绍清华大学论文模板 \thuthesis{} 的使用方法。本模板符合学校的本科、硕士、
  博士论文格式要求。

  本文的创新点主要有:
  \begin{itemize}
    \item 用例子来解释模板的使用方法;
    \item 用废话来填充无关紧要的部分;
    \item 一边学习摸索一边编写新代码。
  \end{itemize}

  关键词是为了文献标引工作、用以表示全文主要内容信息的单词或术语。关键词不超过 5
  个,每个关键词中间用分号分隔。(模板作者注:关键词分隔符不用考虑,模板会自动处
  理。英文关键词同理。)
\end{cabstract}

\ckeywords{\TeX, \LaTeX, CJK, 模板, 论文}

\begin{eabstract} 
   An abstract of a dissertation is a summary and extraction of research work
   and contributions. Included in an abstract should be description of research
   topic and research objective, brief introduction to methodology and research
   process, and summarization of conclusion and contributions of the
   research. An abstract should be characterized by independence and clarity and
   carry identical information with the dissertation. It should be such that the
   general idea and major contributions of the dissertation are conveyed without
   reading the dissertation. 

   An abstract should be concise and to the point. It is a misunderstanding to
   make an abstract an outline of the dissertation and words ``the first
   chapter'', ``the second chapter'' and the like should be avoided in the
   abstract.

   Key words are terms used in a dissertation for indexing, reflecting core
   information of the dissertation. An abstract may contain a maximum of 5 key
   words, with semi-colons used in between to separate one another.
\end{eabstract}

\ekeywords{\TeX, \LaTeX, CJK, template, thesis}

%\makecover

% 目录
\tableofcontents

% 符号对照表
%\begin{denotation}

\item[HDF] Hierarchical Data Format
\item[GPU] Graphics Processing Unit
\item[CPU] Central processing unit
\item[CG] Conjugate Gradient
\item[\ProgramName] \ProgramFullName

\item[PWR] Pressurized Water Reactor
\item[API] Application Programming Interface
\item[CUDA] Compute Unified Device Architecture
\item[MPI] Message Passing Interface
\item[SIMD] Single instruction, multiple data
\item[MIMD] multiple instruction, multiple data
\item[ADI] Alternating direction implicit method
\item[SM] Streaming Multiprocessor
\item[SMX] Next Generation Streaming Multiprocessor
\item[PTX] Parallel Thread Execution
\item[ISA] Instruction Set Architecture
\item[JIT] Just-in-time Compiler
\item[GMRES] Generalized Minimal RESidual method

\end{denotation}



%%% 正文部分
\mainmatter

\chapter{引言}
\section{研究背景}
\section{国内外研究现状}
\section{研究内容和论文组织结构}

\chapter{GPU通用计算简介}
\section{GPU编程技术简介}

\subsection{CUDA}
\subsection{OpenCL}

\subsection{其他技术}
\subsubsection{AMP}
\subsubsection{OpenACC与OpenMP}

\section{GPU线性方程组求解算法}
\subsection{传统迭代算法回顾}
\subsection{Krylov子空间类算法}
\subsubsection{CG}
\subsubsection{BiCGStab}
\subsubsection{GMRES}
\subsection{预条件处理}
\subsubsection{多重网格}

\chapter{GPU大型中子扩散方程求解方法研究}
\section{迭代算法选择}
\section{预条件算法选择}
\section{双层网格加速}

\chapter{中子扩散时空动力学理论推导}
\section{稳态公式推导}
\section{动力学公式推导}

\chapter{数值结果验证及参数影响对比}



\chapter{结论与展望}


%%% 其它部分
\backmatter



% 参考文献
\bibliographystyle{thubib}
\bibliography{ref/refs}


% 致谢
%

\begin{ack}
  衷心感谢导师余纲林老师和核能所所长王侃教授对本人的指导。
  余纲林老师在学术上和生活中的悉心指导让我终生受益,
  王侃教授的严格要求使我受益匪浅,
  在此我衷心地感谢他们。
  如果没有余纲林老师和王侃教授,我肯定取得不了这样的成绩。
  
  此外还要感谢工物系核能所的张鹏、李林森、李泽光、佘顶、
  徐琪、余健开等众位师兄和同学对我的帮助,
  在本课题的研究过程中,他们都或多或少的参与了讨论,
  给了我不少启发和帮助,在这里我也向他们表示感谢。
  
  最后,还要感谢 \thuthesis,它让我的论文写作过程轻松了很多。

\end{ack}


% 附录
\begin{appendix}

\chapter{课题程序使用说明}
\section{运行环境}
\section{输入文件格式}
\subsection{Lua语法简介}
\section{输出文件格式}
\subsection{HDF5简介}
\subsection{Python简介}
\subsection{后处理脚本示例}

%\include{data/mathbasic}

%

\chapter{公式推导}

\section{扩散时空动力学偏微分方程}

多群扩散时空动力学方程为
\begin{align}
  \newcommand{\para}{\pb{\bm{x},t}}
  \left\{
  \begin{aligned}
    \frac{1}{v_g}\frac{\partial \phi_g\para}{\partial t}
    &=\nabla\cdot D_g\para \nabla\phi_g\para 
      -\Sigma_{t,g}\para \phi_g\para
      +\sum_{i=1}^I \chi_{i,g}\para \lambda_i C_i\para \\
          & \hspace{1cm}
      +\sum_{g'=1}^G\pb[B]{\chi_g\para \pb{1-\beta}\nu\Sigma_{f,g'}\para
                            +\Sigma_{g'\rightarrow g}\para}\phi_{g'}\para \\
    \frac{\partial C_i\para}{\partial t}
     &=\beta_i \sum_{g'=1}^G \nu\Sigma_{f,g'}\para \phi_{g'}\para
        -\lambda_i C_i\para \qquad i=1,2,\cdots,I
  \end{aligned}
  \right.
  \titlelabel{equ:pro.diff.equ}{多群扩散时空动力学方程}
\end{align}

如果初始条件为稳态
\begin{align}
  \newcommand{\para}{\pb{\bm{x},t}}
  \left\{
  \begin{aligned}
    \frac{\partial \phi_g\para}{\partial t}\Big|_{t=0} &=0 \\
    \frac{\partial C_i\para}{\partial t}\Big|_{t=0} &=0
  \end{aligned}
  \right.
  \label{equ:pro.diff.init.equ}
\end{align}

联立\aeqref{equ:pro.diff.equ}及\aeqref{equ:pro.diff.init.equ},
消去$C_i\pb{\bm{x},0}$可得
\begin{align}
  \newcommand{\para}{\pb{\bm{x},0}}
  \begin{aligned}
  &\nabla\cdot D_g\para \nabla\phi_g\para 
   -\Sigma_{t,g}\para \phi_g\para \\
  & \hspace{3cm}
   +\sum_{g'=1}^G\pb[B]{\chi_g\para \nu\Sigma_{f,g'}\para
                        +\Sigma_{g'\rightarrow g}\para}\phi_{g'}\para =0
  \end{aligned}
\end{align}

此为$k_\mathrm{eff}=1$时的稳态扩散方程,
由于问题的已知条件一般仅有$k_\mathrm{eff}\approx 1$,
所以一般求解普通的临界扩散方程作为通量初值。
\begin{align}
  \newcommand{\para}{\pb{\bm{x},0}}
  \begin{aligned}
  &\nabla\cdot D_g\para \nabla\phi_g\para 
   -\Sigma_{t,g}\para \phi_g\para \\
  & \hspace{3cm}
   +\sum_{g'=1}^G\pb[B]{\frac{1}{k_\mathrm{eff}}\chi_g\para \nu\Sigma_{f,g'}\para
                        +\Sigma_{g'\rightarrow g}\para}\phi_{g'}\para =0
  \end{aligned}
  \titlelabel{equ:pro.diff.init.diff.equ1}{扩散时空动力学问题的初始通量方程}
\end{align}

解出\aeqref{equ:pro.diff.init.diff.equ1}后
可由\aeqref{equ:pro.diff.init.equ}得到初始条件的$C_i\pb{\bm{x},0}$
\begin{align}
  \newcommand{\para}{\pb{\bm{x},0}}
  C_i\para = \frac{\beta_i}{\lambda_i}\sum_{g'=1}^G \nu\Sigma_{f,g'}\para\phi_{g'}\para
  \label{equ:pro.diff.init.c}
\end{align}

边界条件使用外推边界条件,即
\begin{align}
  \bm{n}\cdot\nabla\phi_g\pb{\bm{x},t} = -\frac{\phi_g\pb{\bm{x},t}}{\delta_g\pb{\bm{x},t}}
  \qquad \bm{x} \in \partial \mathcal{D}
  \titlelabel{equ:pro.diff.boundary.equ}{扩散方程外推边界条件}
\end{align}
其中$\partial \mathcal{D}$是待求解问题区域的边界面,
$\bm{n}$是边界面上的点$\bm{x}$在边界面上的法向量,
方向指向区域外。
这里不考虑中子从边界面离开又从另一处边界面进入的情况,即问题区域是一个凸空间。

外推长度取$\delta_g\pb{\bm{x},t}=2D_g\pb{\bm{x},t}$。

\section{几何空间网格划分}

\subsection{含外边界网格}
本文只考虑三维$xyz$坐标系,问题几何为立方体,
划分为$(K_x-2)\times(K_y-2)\times(K_z-2)$的结构网格,
每个方向的两个边界各增加一个边界网格,
则全部网格集合为$\mathcal{D}_{\bm{k}}=\big\{(k_x,k_y,k_z)\big|k_w = 0,1,\cdots,K_w-1 ; w=x,y,z\big\}$。

实际空间网格集记为$\underline{\mathcal{D}_{\bm{k}}}=\big\{(k_x,k_y,k_z)\big|k_w = 1,\cdots,K_w-2 ; w=x,y,z\big\}$。

边界网格集合记为$\partial \mathcal{D}_{\bm{k}}=\mathcal{D}_{\bm{k}} - \underline{\mathcal{D}_{\bm{k}}}$。

只在某一个方向上处于边界位置的网格集合记为
\begin{align}
  \begin{aligned}
  \underline{\partial \mathcal{D}_{\bm{k}}}
  &=\big\{(k_x,k_y,k_z)\big|k_w=0,K_w-1;k_v = 0,1,\cdots,K_v-1 ; v\in\{x,y,z\}-\{w\} ; w=x,y,z\big\}\\
  &=\big\{(k_x,k_y,k_z)\big|k_x=0,K_x-1;k_w = 0,1,\cdots,K_w-1 ; w=y,z\big\}
  \bigcup\\
  &\hspace{1cm}
  \big\{(k_x,k_y,k_z)\big|k_y=0,K_y-1;k_w = 0,1,\cdots,K_w-1 ; w=x,z\big\}
  \bigcup\\
  &\hspace{1cm}
  \big\{(k_x,k_y,k_z)\big|k_z=0,K_z-1;k_w = 0,1,\cdots,K_w-1 ; w=x,y\big\}
  \end{aligned}
\end{align}

$\underline{\partial \mathcal{D}_{\bm{k}}}$是实际参与计算的网格。
在$\underline{\partial \mathcal{D}_{\bm{k}}}$上定义边界网格的离散外法向量$\bm{n}_{\bm{k}}$,
对于某边界网格$\bm{k}=(k_x,k_y,k_z)$,若$k_w=0 \quad (w\in\{x,y,z\})$,则$\bm{n}_{\bm{k}}=-\hat{\bm{w}}$;
若$k_w=K_w-1$,则$\bm{n}_{\bm{k}}=\hat{\bm{w}}$,
其中$\hat{\bm{x}}=(1,0,0) \quad \hat{\bm{y}}=(0,1,0) \quad \hat{\bm{z}}=(0,0,1)$。

处于棱、角点上的边界网格集合记为
$\partial^2 \mathcal{D}_{\bm{k}} = \partial \mathcal{D}_{\bm{k}} - \underline{\partial \mathcal{D}_{\bm{k}}}$,
这部分边界网格不参与计算。

设$\vb{\cdot}$表示集合的元素个数,则有
\footnote{其中$\sum_{ \substack{w<v \\ w,v=x,y,z} } K_wK_v = K_xK_y+K_xK_z+K_yK_z$}
\begin{align}
  \begin{aligned}
    \vb[b]{\mathcal{D}_{\bm{k}}} &= \prod_{w=x,y,z}K_w\\
    \vb[b]{\underline{\mathcal{D}_{\bm{k}}}} &= \prod_{w=x,y,z}(K_w-2)
      = \prod_{w=x,y,z}K_w 
       -2\sum_{ \substack{w<v \\ w,v=x,y,z} } K_wK_v
       +4\sum_{w=x,y,z}K_w
       -8\\
    \vb[b]{\partial \mathcal{D}_{\bm{k}}} 
      &= \vb[b]{\mathcal{D}_{\bm{k}}} - \vb[b]{\underline{\mathcal{D}_{\bm{k}}}}
      = 2\sum_{ \substack{w<v \\ w,v=x,y,z} } K_wK_v
        -4\sum_{w=x,y,z}K_w
        +8  \\
    \vb[b]{\underline{\partial \mathcal{D}_{\bm{k}}}}
      &= 2\sum_{ \substack{w<v \\ w,v=x,y,z} } K_wK_v
        -8\sum_{w=x,y,z}K_w
        +24\\
    \vb[b]{\partial^2 \mathcal{D}_{\bm{k}}}
      &= \vb[b]{\partial \mathcal{D}_{\bm{k}}} - \vb[b]{\underline{\partial \mathcal{D}_{\bm{k}}}}
      =4\sum_{w=x,y,z}K_w - 16
  \end{aligned}
\end{align}


\subsection{不含外边界网格}

本文只考虑三维$xyz$坐标系,问题几何为立方体,
划分为$K_x\times K_y \times K_z$的结构网格,
则网格集合为$\mathcal{D}_{\bm{k}}=\big\{(k_x,k_y,k_z)\big|k_w = 0,1,\cdots,K_w-1 ; w=x,y,z\big\}$。
处于边界位置的网格集合记为
\begin{align}
  \begin{aligned}
  \underline{\partial \mathcal{D}_{\bm{k}}}
  &=\big\{(k_x,k_y,k_z)\big|k_x=0,K_x-1;k_w = 0,1,\cdots,K_w-1 ; w=y,z\big\}
  \bigcup\\
  &\hspace{1cm}
  \big\{(k_x,k_y,k_z)\big|k_y=0,K_y-1;k_w = 0,1,\cdots,K_w-1 ; w=x,z\big\}
  \bigcup\\
  &\hspace{1cm}
  \big\{(k_x,k_y,k_z)\big|k_z=0,K_z-1;k_w = 0,1,\cdots,K_w-1 ; w=x,y\big\}
  \end{aligned}
\end{align}

\TODO 在$\underline{\partial \mathcal{D}_{\bm{k}}}$上定义边界网格的离散外法向量$\bm{n}_{\bm{k}}$,
对于某边界网格$\bm{k}=(k_x,k_y,k_z)$,若$k_w=0 \quad (w\in\{x,y,z\})$,则$\bm{n}_{\bm{k}}=-\hat{\bm{w}}$;
若$k_w=K_w-1$,则$\bm{n}_{\bm{k}}=\hat{\bm{w}}$,
其中$\hat{\bm{x}}=(1,0,0) \quad \hat{\bm{y}}=(0,1,0) \quad \hat{\bm{z}}=(0,0,1)$。

设$\vb{\cdot}$表示集合的元素个数,则有
\footnote{其中$\sum_{ \substack{w<v \\ w,v=x,y,z} } K_wK_v = K_xK_y+K_xK_z+K_yK_z$}
\begin{align}
  \begin{aligned}
    \vb[b]{\mathcal{D}_{\bm{k}}} &= \prod_{w=x,y,z}K_w\\
    \vb[b]{\partial \mathcal{D}_{\bm{k}}} 
      &= \prod_{w=x,y,z}K_w - \prod_{w=x,y,z}\pb{K_w-2} \\
      &= 2\sum_{ \substack{w<v \\ w,v=x,y,z} } K_wK_v
        -4\sum_{w=x,y,z}K_w
        +8
  \end{aligned}
\end{align}




\section{直接解法}

\subsection{时间离散}


将\aeqref{equ:pro.diff.equ}对时间$t$进行离散。

\subsubsection{全隐式向后差分}

采用隐式向后差分格式
\begin{align}
  \newcommand{\para}[1][n]{\pb{\bm{x}}^{(#1)}}
  \left\{
  \begin{aligned}
    \frac{1}{v_g}\frac{\phi_g\para - \phi_g\para[n-1]}{\Delta t}
    &=\nabla\cdot D_g\para \nabla\phi_g\para 
      -\Sigma_{t,g}\para \phi_g\para \\
    & \hspace{1cm}
      +\sum_{g'=1}^G\pb[B]{\chi_g\para \pb{1-\beta}\nu\Sigma_{f,g'}\para
                           +\Sigma_{g'\rightarrow g}\para}\phi_{g'}\para \\
    &\hspace{1cm}
      +\sum_{i=1}^I \chi_{i,g}\para \lambda_i C_i\para \\
    \frac{C_i\para - C_i\para[n-1]}{\Delta t}
     &=\beta_i \sum_{g'=1}^G \nu\Sigma_{f,g'}\para \phi_{g'}\para
        -\lambda_i C_i\para
  \end{aligned}
  \right.
  \label{equ:pro.diff.dt.equ0}
\end{align}

解出$C_i\pb{\bm{x}}^{(n)}$可得
\begin{align}
  \newcommand{\para}[1][n]{\pb{\bm{x}}^{(#1)}}
  C_i\para = \frac{1}{1+\lambda_i\Delta t}
    \pb[B]{C_i\para[n-1]
    + \beta_i \Delta t \sum_{g'=1}^G \nu\Sigma_{f,g'}\para \phi_{g'}\para}
  \label{equ:pro.diff.dt.c}
\end{align}

代回\aeqref{equ:pro.diff.dt.equ0}得
\begin{align}
  \newcommand{\para}[1][n]{\pb{\bm{x}}^{(#1)}}
  \begin{aligned}
    &\quad \frac{1}{v_g}\frac{\phi_g\para - \phi_g\para[n-1]}{\Delta t} \\
    &=\nabla\cdot D_g\para \nabla\phi_g\para 
      -\Sigma_{t,g}\para \phi_g\para \\
    & \hspace{1cm}
      +\sum_{g'=1}^G\pb[B]{\chi_g\para \pb{1-\beta}\nu\Sigma_{f,g'}\para
                           +\Sigma_{g'\rightarrow g}\para}\phi_{g'}\para \\
    &\hspace{1cm}
      +\sum_{i=1}^I \frac{\chi_{i,g}\para \lambda_i}{1+\lambda_i\Delta t}
          \pb[B]{C_i\para[n-1] 
      + \beta_i \Delta t \sum_{g'=1}^G \nu\Sigma_{f,g'}\para \phi_{g'}\para}
  \end{aligned}
\end{align}

取$\chi_{i,g}=\chi_g$得
\begin{align}
  \newcommand{\para}[1][n]{\pb{\bm{x}}^{(#1)}}
  \begin{aligned}
    &\quad \frac{1}{v_g}\frac{\phi_g\para - \phi_g\para[n-1]}{\Delta t} \\
    &=\nabla\cdot D_g\para \nabla\phi_g\para 
      -\Sigma_{t,g}\para \phi_g\para 
      +\sum_{i=1}^I \frac{\chi_g\para \lambda_i}{1+\lambda_i\Delta t} C_i\para[n-1]\\
    & \hspace{1cm}
      +\sum_{g'=1}^G\pb[Bg]{\chi_g\para
        \pb[bg]{1-\beta 
          + \sum_{i=1}^I \frac{\lambda_i \beta_i \Delta t }{1+\lambda_i\Delta t}}
      \nu\Sigma_{f,g'}\para \\
    &\hspace{8cm}
         +\Sigma_{g'\rightarrow g}\para}\phi_{g'}\para
  \end{aligned}
\end{align}

记
\begin{align}
  \newcommand{\para}[1][n]{(\bm{x})^{(#1)}}
  S_C\para = \sum_{i=1}^I \frac{\lambda_i}{1+\lambda_i\Delta t} C_i\para[n-1]
  \titlelabel{equ:pro.diff.dt.sc}{离散扩散时空动力学中$S_C(\bm{x})^{(n)}$定义式} \\
  %
  B = 1-\beta + \sum_{i=1}^I \frac{\lambda_i \beta_i \Delta t }{1+\lambda_i\Delta t}
  \titlelabel{equ:pro.diff.dt.B}{离散扩散时空动力学中$B(\bm{x})^{(n)}$定义式}
\end{align}

则有
\begin{align}
  \newcommand{\para}[1][n]{\pb{\bm{x}}^{(#1)}}
  \begin{aligned}
    &\quad \frac{1}{v_g}\frac{\phi_g\para - \phi_g\para[n-1]}{\Delta t} \\
    &=\nabla\cdot D_g\para \nabla\phi_g\para 
      -\Sigma_{t,g}\para \phi_g\para + \chi_g\para S_C\para\\
    & \hspace{1cm}
      +\sum_{g'=1}^G\pb[B]{\chi_g\para
        B \nu\Sigma_{f,g'}\para
         +\Sigma_{g'\rightarrow g}\para}\phi_{g'}\para
  \end{aligned}
  \titlelabel{equ:pro.diff.dt.equ1}{时间$t$隐式向后差分离散后的扩散时空动力学通量$\phi$方程}
\end{align}


\subsubsection{裂变源显式}

采用隐式向后差分格式,但裂变源使用显式向前差分格式。
\begin{align}
  \newcommand{\para}[1][n]{\pb{\bm{x}}^{(#1)}}
  \left\{
  \begin{aligned}
    \frac{1}{v_g}\frac{\phi_g\para - \phi_g\para[n-1]}{\Delta t}
    &=\nabla\cdot D_g\para \nabla\phi_g\para 
      -\Sigma_{t,g}\para \phi_g\para \\
    & \hspace{1cm}
      +\sum_{g'=1}^G\pb[B]{\chi_g\para \pb{1-\beta}
                                \nu\Sigma_{f,g'}\para[n-1] \phi_{g'}\para[n-1]\\
    &\hspace{3cm}
                           +\Sigma_{g'\rightarrow g}\para \phi_{g'}\para} \\
    &\hspace{1cm}
      +\sum_{i=1}^I \chi_{i,g}\para \lambda_i C_i\para \\
    \frac{C_i\para - C_i\para[n-1]}{\Delta t}
     &=\beta_i \sum_{g'=1}^G \nu\Sigma_{f,g'}\para[n-1] \phi_{g'}\para[n-1]
        -\lambda_i C_i\para
  \end{aligned}
  \right.
  \label{equ:pro.diff.dt(f).equ0}
\end{align}

解出$C_i\pb{\bm{x}}^{(n)}$可得
\begin{align}
  \newcommand{\para}[1][n]{\pb{\bm{x}}^{(#1)}}
  C_i\para = \frac{1}{1+\lambda_i\Delta t}
    \pb[B]{C_i\para[n-1]
    + \beta_i \Delta t \sum_{g'=1}^G \nu\Sigma_{f,g'}\para[n-1] \phi_{g'}\para[n-1] }
  \label{equ:pro.diff.dt(f).c}
\end{align}


代回\aeqref{equ:pro.diff.dt(f).equ0},取$\chi_{i,g}=\chi_g$得
\begin{align}
  \newcommand{\para}[1][n]{\pb{\bm{x}}^{(#1)}}
  \begin{aligned}
    &\quad \frac{1}{v_g}\frac{\phi_g\para - \phi_g\para[n-1]}{\Delta t} \\
    &=\nabla\cdot D_g\para \nabla\phi_g\para 
      -\Sigma_{t,g}\para \phi_g\para 
      +\sum_{i=1}^I \frac{\chi_g\para \lambda_i}{1+\lambda_i\Delta t} C_i\para[n-1]\\
    & \hspace{1cm}
      +\sum_{g'=1}^G\pb[Bg]{\chi_g\para
        \pb[bg]{1-\beta 
          + \sum_{i=1}^I \frac{\lambda_i \beta_i \Delta t }{1+\lambda_i\Delta t}}
      \nu\Sigma_{f,g'}\para[n-1] \phi_{g'}\para[n-1]\\
    &\hspace{8cm}
         +\Sigma_{g'\rightarrow g}\para \phi_{g'}\para}
  \end{aligned}
\end{align}

代入\aeqref{equ:pro.diff.dt.sc}及\aeqref{equ:pro.diff.dt.B},并令
\begin{align}
  \newcommand{\para}[1][n]{\pb{\bm{x}}^{(#1)}}
  S_F\para = \sum_{g'=1}^G B \nu\Sigma_{f,g'}\para[n-1] \phi_{g'}\para[n-1]
\end{align}

可得
\begin{align}
  \newcommand{\para}[1][n]{\pb{\bm{x}}^{(#1)}}
  \begin{aligned}
    \frac{1}{v_g}\frac{\phi_g\para - \phi_g\para[n-1]}{\Delta t}
    &=\nabla\cdot D_g\para \nabla\phi_g\para 
      -\Sigma_{t,g}\para \phi_g\para  \\
    & \hspace{1cm}
       + \chi_g\para \pb[b]{S_C\para + S_F\para}\\
    & \hspace{1cm}
      +\sum_{g'=1}^G \Sigma_{g'\rightarrow g}\para \phi_{g'}\para
  \end{aligned}
  \titlelabel{equ:pro.diff.dt(f).equ1}{时间$t$裂变源显式离散后的扩散时空动力学通量$\phi$方程}
\end{align}



\subsection{空间离散}


在$xyz$坐标系中有一般离散关系
\begin{align}
  \phi(\bm{x}) &\rightarrow \phi_{\bm{k}}\\
  \Sigma(\bm{x}) &\rightarrow \Sigma_{\bm{k}}\\
  C_i(\bm{x}) &\rightarrow C_{i,\bm{k}}
\end{align}

其中$\bm{k}=(k_x,k_y,k_z)$为$xyz$空间离散后的网格坐标,
为方便起见这里暂时省略能群$g$,时间步长$n$等下标上标,下同。

离散的主要问题是微分项$\nabla\cdot D(\bm{x})\nabla\phi(\bm{x})$
和边界条件\aeqref{equ:pro.diff.boundary.equ}的离散方式。

首先考虑微分项,设$\nabla\cdot D(\bm{x})\nabla\phi(\bm{x})$对应的离散项为
$\pb[b]{\nabla\cdot D\nabla\phi}_{\bm{k}}$,
在$xyz$坐标系中,可取
\begin{align}
  \begin{aligned}
  \pb[b]{\nabla\cdot D\nabla\phi}_{\bm{k}}
    &=\sum_{w=x,y,z} \Sb[bg]{
      \frac{2D_{\bm{k}}D_{\bm{k}+\hat{\bm{w}}}\pb{\phi_{\bm{k}+\hat{\bm{w}}} - \phi_{\bm{k}}}}
           {\Delta w_{\bm{k}}\pb{D_{\bm{k}}\Delta w_{\bm{k}+\hat{\bm{w}}}+D_{\bm{k}+\hat{\bm{w}}}\Delta w_{\bm{k}}}}
           \\
    &\hspace{4cm} -\frac{2D_{\bm{k}}D_{\bm{k}-\hat{\bm{w}}}\pb{\phi_{\bm{k}} - \phi_{\bm{k}-\hat{\bm{w}}}}}
           {\Delta w_{\bm{k}}\pb{D_{\bm{k}}\Delta w_{\bm{k}-\hat{\bm{w}}}+D_{\bm{k}-\hat{\bm{w}}}\Delta w_{\bm{k}}}}
     }
  \end{aligned}
  \label{equ:dnabla2.equ0}
\end{align}

下面考虑边界条件\aeqref{equ:pro.diff.boundary.equ}的离散方式,
为方便起见本文只考虑方形问题区域,即问题区域的边界$\partial \mathcal{D}$仅包括与$xyz$坐标轴垂直的平面。

则边界条件\aeqref{equ:pro.diff.boundary.equ}可离散为\TODO
\begin{align}
  \frac{\phi_{\bm{k}}-\phi_{\bm{k}-\bm{n}_{\bm{k}}}}{|\Delta_{\bm{k}}\cdot \bm{n}_{\bm{k}}|}
   = -\frac{\phi_{\bm{k}}}{\delta_{\bm{k}}}
  \qquad \bm{k} \in \underline{\partial \mathcal{D}_{\bm{k}}}
\end{align}

其中
\begin{align}
  \Delta_{\bm{k}} = (\Delta x, \Delta y, \Delta z)
\end{align}

\subsubsection{初始通量方程}

\aeqref{equ:pro.diff.init.diff.equ1}的离散形式为
\begin{align}
  \pb[b]{\nabla\cdot D_g^{(0)} \nabla\phi_g^{(0)}}_{\bm{k}}
   -\Sigma_{t,g,\bm{k}}^{(0)} \phi_{g,\bm{k}}^{(0)}
   +\sum_{g'=1}^G\pb[B]{\frac{1}{k_\mathrm{eff}^{(0)}}\chi_{g,\bm{k}}^{(0)} \nu\Sigma_{f,g',\bm{k}}^{(0)}
                        +\Sigma_{g'\rightarrow g,\bm{k}}^{(0)}}\phi_{g',\bm{k}}^{(0)} =0
  \quad \bm{k} \in \underline{\mathcal{D}_{\bm{k}}}
\end{align}

\aeqref{equ:pro.diff.init.c}的离散形式为
\begin{align}
  C_{i,\bm{k}}^{(0)} = \frac{\beta_i}{\lambda_i}
    \sum_{g'=1}^G \nu\Sigma_{f,g',\bm{k}}^{(0)}\phi_{g',\bm{k}}^{(0)}
  \qquad \bm{k} \in \underline{\mathcal{D}_{\bm{k}}}
\end{align}


\subsubsection{全隐式向后差分}


则\aeqref{equ:pro.diff.dt.equ1}的离散形式为
\begin{align}
  \begin{aligned}
    \frac{1}{v_g}\frac{\phi_{g,\bm{k}}^{(n)} - \phi_{g,\bm{k}}^{(n-1)}}{\Delta t} 
    &=\pb[b]{\nabla\cdot D_{g}^{(n)} \nabla\phi_{g}^{(n)}}_{\bm{k}}
      -\Sigma_{t,g,\bm{k}}^{(n)} \phi_{g,\bm{k}}^{(n)} + \chi_{g,\bm{k}}^{(n)} S_{C,\bm{k}}^{(n)}\\
    & \hspace{1cm}
      +\sum_{g'=1}^G\pb[B]{\chi_{g,\bm{k}}^{(n)}
        B \nu\Sigma_{f,g',\bm{k}}^{(n)}
         +\Sigma_{g'\rightarrow g,\bm{k}}^{(n)}}\phi_{g',\bm{k}}^{(n)}
  \end{aligned}
  \qquad \bm{k} \in \underline{\mathcal{D}_{\bm{k}}}
  \label{equ:pro.diff.dt.dx.equ1}
\end{align}

其中
\begin{align}
  S_{C,\bm{k}}^{(n)} &= \sum_{i=1}^I \frac{\lambda_i}{1+\lambda_i\Delta t} C_{i,\bm{k}}^{(n-1)}
  \qquad \bm{k} \in \underline{\mathcal{D}_{\bm{k}}}
  \titlelabel{equ:pro.diff.dt.dx.sc}{离散扩散时空动力学中$S_{C,\bm{k}}^{(n)}$定义式}
\end{align}

\aeqref{equ:pro.diff.dt.c}的离散形式为
\begin{align}
  C_{i,\bm{k}}^{(n)} = \frac{1}{1+\lambda_i\Delta t}
    \pb[B]{C_{i,\bm{k}}^{(n-1)}
    + \beta_i \Delta t \sum_{g'=1}^G \nu\Sigma_{f,g',\bm{k}}^{(n)} \phi_{g',\bm{k}}^{(n)}}
  \qquad \bm{k} \in \underline{\mathcal{D}_{\bm{k}}}
\end{align}




\subsubsection{裂变源显式差分}


则\ateqref{equ:pro.diff.dt(f).equ1}的离散形式为
\begin{align}
  \begin{aligned}
    \frac{1}{v_g}\frac{\phi_{g,\bm{k}}^{(n)} - \phi_{g,\bm{k}}^{(n-1)}}{\Delta t} 
    &=\pb[b]{\nabla\cdot D_{g}^{(n)} \nabla\phi_{g}^{(n)}}_{\bm{k}}
      -\Sigma_{t,g,\bm{k}}^{(n)} \phi_{g,\bm{k}}^{(n)} 
      + \chi_{g,\bm{k}}^{(n)} \pb[b]{S_{C,\bm{k}}^{(n)} + S_{F,\bm{k}}^{(n)}}\\
    & \hspace{1cm}
      +\sum_{g'=1}^G \Sigma_{g'\rightarrow g,\bm{k}}^{(n)}\phi_{g',\bm{k}}^{(n)}
  \end{aligned}
  \qquad \bm{k} \in \underline{\mathcal{D}_{\bm{k}}}
  \label{equ:pro.diff.dt.dx.equ1}
\end{align}

其中$S_{C,\bm{k}}^{(n)}$定义同\aeqref{equ:pro.diff.dt.dx.sc},且
\begin{align}
  S_{F,\bm{k}}^{(n)} = \sum_{g'=1}^G B \nu\Sigma_{f,g',\bm{k}}^{(n-1)} \phi_{g',\bm{k}}^{(n-1)}
  \qquad \bm{k} \in \underline{\mathcal{D}_{\bm{k}}}
  \titlelabel{equ:pro.diff.dt.dx.sf}{离散扩散时空动力学中$S_{F,\bm{k}}^{(n)}$定义式}
\end{align}

\ateqref{equ:pro.diff.dt(f).c}的离散形式为
\begin{align}
  C_{i,\bm{k}}^{(n)} = \frac{1}{1+\lambda_i\Delta t}
    \pb[B]{C_{i,\bm{k}}^{(n-1)}
    + \beta_i \Delta t \sum_{g'=1}^G \nu\Sigma_{f,g',\bm{k}}^{(n-1)} \phi_{g',\bm{k}}^{(n-1)}}
  \qquad \bm{k} \in \underline{\mathcal{D}_{\bm{k}}}
\end{align}



\section{准静态}

考虑到反应堆物理计算中,一般总体通量幅度变化较快,
而相对功率分布即形状函数变化较慢,所以可以将中子通量分布分解为两部分
\begin{align}
  \phi\pb{\bm{x},E,t}=n(t) \psi\pb{\bm{x},E,t}
\end{align}
其中$n(t)$是幅度因子,$\psi\pb{\bm{x},E,t}$是形状因子。

将上式代入\ateqref{equ:pro.diff.equ},
并在方程两边作用算符$\int_D \diff{\bm{x}} \int \diff{E} \ w(\bm{r},E) $,
其中
\begin{align}
  \newcommand{\para}{\pb{\bm{x},t}}
  \begin{split}
    &\int_D \diff{\bm{x}} \sum_{g=1}^G
        w_g(\bm{x}) \frac{1}{v_g}\frac{\partial \phi_g\para}{\partial t} \\
    & \hspace{2cm}
      =\frac{\partial n(t)}{\partial t}\int_D \diff{\bm{x}} \sum_{g=1}^G \frac{w_g(\bm{x})}{v_g} \psi_g\para
      + n(t)\int_D \diff{\bm{x}} \sum_{g=1}^G \frac{w_g(\bm{x})}{v_g} \frac{\partial \psi_g\para}{\partial t}
  \end{split}
\end{align}
最后有
\begin{align}
  \newcommand{\para}{\pb{\bm{x},t}}
  \left\{
  \begin{aligned}
    \begin{split}
    &\frac{\partial n(t)}{n(t)\partial t}\int_D \diff{\bm{x}} \sum_{g=1}^G \frac{w_g(\bm{x})}{v_g} \psi_g\para
      + \int_D \diff{\bm{x}} \sum_{g=1}^G \frac{w_g(\bm{x})}{v_g} \frac{\partial \psi_g\para}{\partial t} \\
     & \hspace{1.5cm}
     =\int_D \diff{\bm{x}} \sum_{g=1}^G w_g(\bm{x})\pb[bg]{
       \sum_{i=1}^I \chi_{i,g}\para \lambda_i C_i\para \\
     &\hspace{2cm}
       +\nabla\cdot D_g\para \nabla\psi_g\para 
        -\Sigma_{t,g}\para \psi_g\para\\
     & \hspace{3cm}
       +\sum_{g'=1}^G\pb[B]{\chi_g\para \pb{1-\beta}\nu\Sigma_{f,g'}\para
                   +\Sigma_{g'\rightarrow g}\para}\psi_{g'}\para
       }
   \end{split}\\
   &\int_D \diff{\bm{x}} \sum_{g=1}^G w_g(\bm{x})\frac{\partial C_i\para}{\partial t}\\
   & \hspace{2cm}
    =\int_D \diff{\bm{x}} \sum_{g=1}^G w_g(\bm{x}) \pb[bg]{
      \beta_i \sum_{g'=1}^G \nu\Sigma_{f,g'}\para n(t)\psi_{g'}\para
       -\lambda_i C_i\para}
  \end{aligned}
  \right.
\end{align}
\TODO

\section{能群耦合\TODO}

本文不考虑向上散射,向下散射只散射到邻近的能群,则以上离散后的扩散方程可以写成如下形式
\begin{align}
  \begin{pmatrix}
  A_{11} & D_{12} & \cdots & D_{1G}\\
  D_{21} & A_{22} & &\\
   & \ddots & \ddots &\\
   & & D_{G-1,G} & A_{GG}
  \end{pmatrix}
  \begin{pmatrix}
  \phi_1 \\ \phi_2 \\ \vdots \\ \phi_G
  \end{pmatrix}
  =
  \begin{pmatrix}
  S_1 \\ S_2 \\ \vdots \\ S_G
  \end{pmatrix}
\end{align}

其中 $A_{gg}$是7对角对称阵,$D_{g_1g_2}$ 是对角阵。

两群情况则简化为
\begin{align}
  \begin{pmatrix}
  A_{11} & D_{12} \\
  D_{21} & A_{22}
  \end{pmatrix}
  \begin{pmatrix}
  \phi_1 \\ \phi_2
  \end{pmatrix}
  =
  \begin{pmatrix}
  S_1 \\ S_2
  \end{pmatrix}
\end{align}

\TODO

\section{多重网格法\TODO}

\onlinecite{trottenberg2001multigrid}
\onlinecite{briggs2000multigrid}

%%%% Local Variables: 
%%% mode: latex
%%% TeX-master: "../main"
%%% End: 

\chapter{程序辅助功能}

\section{程序外部依赖库简介}

\subsection{Thrust}

Thrust是一个并行算法库,提供了Map、Reduce等基础函数,简化了CUDA程序开发。
Thrust由NVidia公司的雇员Jared Hoberock、Nathan Bell开发并维护,
被CUDA开发环境CUDA Toolkit所包含,并在Github上基于Apache开源协议开源,
当前版本为Thrust 1.7.0,
项目主页\url{http://thrust.github.com}。

\subsection{CUSP}

CUSP是CUDA环境下的一个稀疏矩阵函数库,依赖于Thrust,
由NVidia公司的雇员Steven Dalton、Nathan Bell开发。
CUSP基于Apache开源协议开源,当前版本为CUSP 0.3.1,
项目地址\url{http://cusplibrary.github.com}。

\subsection{HDF5}

HDF5是一种分层的、带有元数据的、支持多种数据格式的、针对科学计算的数据文件存储格式及相关技术。
它最早由美国国家超级计算应用中心(NCSA, 
National Center for Supercomputing Applications)提出,
现在由非营利的HDF Group管理和维护。\cite{HDF5Wiki}

HDF5文件格式的读写库基于一种类似于BSD的协议开源,协议文本可以从
\url{http://www.hdfgroup.org/HDF5/doc/Copyright.html} 获取。

当前版本为HDF5 1.8.10 patch1,
可从\url{http://www.hdfgroup.org/downloads/index.html} 获取。

\subsection{Lua、LuaJIT }

Lua是一种快速、轻量级的嵌入式脚本语言,
在1993年由巴西里约热内卢天主教大学(Pontifical Catholic University of Rio de Janeiro)
计算机图形技术组(Computer Graphics Technology Group)的成员
Roberto Ierusalimschy、 Luiz Henrique de Figueiredo和 Waldemar Celes所开发。
\cite{LuaWiki}

Lua很容易嵌入到其他程序中作为内嵌的解释器使用,已在PC游戏界广泛使用。
当前版本5.2.1,基于MIT开源协议开源,
可从 \url{http://www.lua.org/download.html}获得。

LuaJIT是Lua语言的非官方JIT(Just-In-Time Compiler)实现,由Mike Pall开发。
LuaJIT基于MIT开源协议开源,当前版本LuaJIT 2.0.1,可以从\url{luajit.org/download.html}获取。
LuaJIT是当前世界上最快的动态语言实现之一。\cite{LuaJITHomepage}


\end{appendix}

% 个人简历
%\begin{resume}

  \resumeitem{个人简历}

  1988 年 5 月 11 日出生于黑龙江省省齐齐哈尔市。
  
  2007 年 9 月考入清华大学工程物理系工程物理专业,2011年 7 月本科毕业并获得工学学士学位。
  
  2011 年 9 月免试进入清华大学工程物理系攻读核科学与技术硕士学位至今。

  \resumeitem{发表的学术论文} % 发表的和录用的合在一起

  \begin{enumerate}[{[}1{]}]
  \item 孙嘉龙, 余纲林, 佘顶等. 堆用蒙特卡罗程序几何重复结构功能开发[J]. 强激光与粒子束, 2013,25(01): 219-222.
  (EI源刊,检索号20130315915132)
  \end{enumerate}

\end{resume}

\end{document}
